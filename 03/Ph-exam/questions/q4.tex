\subsection{Закон сохранения импульса. Упругое и неупругое взаимодействие}

\begin{definition}
    Закон сохранения импульса — векторная сумма импульсов всех тел системы есть величина постоянная, 
    если векторная сумма внешних сил, действующих на систему тел, равна нулю
\end{definition}

Выводится из 2-го закона Ньютона в импульсной форме.

\begin{definition}
    Абсолютно упругий удар — модель соударения, при которой полная кинетическая энергия системы сохраняется 
    (шарики сталкиваются и летят в разные стороны), т.е. не было деформации, тела не нагрелись - столкнулись и разлетелись.
\end{definition}

$$m_1\vec\upsilon_1+m_2\vec\upsilon_2=m_1\vec\upsilon_1'+m_2\vec\upsilon_2'$$

\begin{remark}
    Импульсы складываются векторно (смотреть на направления скоростей)
\end{remark}

При абсолютно упругом ударе выполняется закон сохранения энергии.
Если сталкиваются тела одинаковой массы, они просто обмениваются скоростями.

\begin{definition}
    Абсолютно неупругий удар — удар, в результате которого тела соединяются и продолжают движение как единое тело.
\end{definition}

$$m_a\vec\upsilon_a+m_b\vec\upsilon_b=(m_a+m_b)\vec\upsilon$$

$$\vec\upsilon=\frac{m_a\vec\upsilon_a+m_b\vec\upsilon_b}{m_a+m_b}$$

При абсолютно упругом ударе не выполняется закон сохранения энергии.