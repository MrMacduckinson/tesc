\subsection{Закон всемирного тяготения. Зависимость ускорения свободного падения от высоты. Первая космическая скорость}

\begin{definition}
    Закон всемирного тяготения — сила $F$ гравитационного притяжения между двумя материальными точками с массами $m_1$ и $m_2$, 
    разделёнными расстоянием $f$, 
    действует вдоль соединяющей их прямой, пропорциональна обеим массами и обратно пропорциональна квадрату расстояния.
\end{definition}

Закон всемирного тяготения — любые две точечные массы притягиваются друг к другу с силой, прямо пропорциональной произведению масс 
и обратно пропорциональной квадрату расстояния между ними:

$$F=G\cdot\frac{m_1\cdot m_2}{r^2}$$

$G\approx 6,67\cdot 10^{-11} \frac{Н \cdot м^2}{кг^2}$ - гравитационная постоянная

\begin{definition}
     Сила тяжести  — сила гравитационного взаимодействия тела с Землёй

     Сила тяжести  — сила, с которой Земля притягивает к себе тела.
\end{definition}

\begin{remark}
    $R_З\approx 6,4\cdot 10^6 м$ - радиус Земли
\end{remark}

\begin{remark}
    Сила тяготения вблизи поверхности Земли:

\end{remark}

$$\vec F=-G\vec r\frac{mM}{r^3},$$

значит, сила направлена к центру Земли, а $\vec r$ - от центра.

Для тел вблизи земли $r\approx R_З$ $\displaystyle(\vec g=-G\frac{M}{R^2_З}\vec l_r)$:

$$\vec F=-m\vec g$$

Гравитационное ускорение на высоте $h$ над поверхностью космического тела можно вычислить по формуле ($M$ - масса планеты):

$$g(h)=\frac{GM}{(r+h)^2}$$

Сила тяжести на высоте $h$:

$$F=\frac{GmM}{(r+h)^2}$$

Чем выше тело, тем меньше ускорение свободного падения и тем меньше сила тяжести.

\begin{definition}
    Вес тела — сила, с которой тело давит на опору или подвес.

    В инерциальных системах отсчёта вес тела численно равен силе тяжести:

    $$\vec P=m\vec g$$

    В неинерциальной системе отсчёта вес зависит от ускорения системы отсчёта, в которой находится тело:

    $$
    \vec P=m(\vec g\pm\vec a)
    $$

    Состояние невесомости: $P=0$
\end{definition}

\begin{definition}
    Первая космическая скорость (круговая скорость) — минимальная (для заданной высоты над поверхностью планеты) горизонтальная скорость, 
    которую необходимо придать объекту, чтобы он совершал движение по круговой орбите вокруг планеты и не начал падать ($7,91\frac{км}{с}$ для Земли)
\end{definition}

\begin{remark}
    Вычисление.

    На орбите на объект действует только сила тяготения земли: $G \frac{m\cdot M}{R^2}$, 
    однако тело движется по окружности с постоянной скоростью. Тогда центростремительное ускорение равно $\frac{U^2}{R}$. 
    Подставим его вместо ускорения:

    $$m\ \frac{\upsilon^2_1}{R+h}=G\ \frac{m\cdot M}{(R+h)^2}$$

    Отсюда находим $\upsilon_1$ - первую космическую скорость:
    $$\upsilon_1=\sqrt{\frac{G\cdot M}{R+h}}$$
\end{remark}

\begin{remark}
    $h << R_З$, поэтому $v_1 = \sqrt{\frac{G\cdot M}{R}}$
\end{remark}