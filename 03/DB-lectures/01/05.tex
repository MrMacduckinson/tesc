\subsection{%
  Лекция \texttt{23.09.29}.%
}

\begin{definition}
  \textit{Реляционная алгебра} --- это теоретический язык операций, позволяющий на 
  основе одного или нескольких отношений создавать другие отношения без изменения самих
  исходных отношений.
\end{definition}

Реляционная алгебра замкнута. В ней существуют следующие операции:
\begin{enumerate}
\item
  Проекция \(\Pi_{a_1 \dotsc a_n} (R)\)

  Результатом проекции является новое отношение, содержащее вертикальное
  подмножество исходного отношения, создаваемое посредством извлечения указанных
  атрибутов и исключения из результата атрибутов-дубликатов.

\item
  Выборка \(\sigma_{\text{предикат}} (R)\)

  Результатом выборки является отношение, которое содержит только те кортежи из
  исходного отношения, которые удовлетворяют заданному условию (предикату).

\item
  Объединение \(R \cup S\)

  Объединение двух отношений \(R\) и \(S\) определяет новое отношение, которое
  включает все кортежи, содержащиеся только в \(R\), все кортежи, содержащиеся
  только в \(S\) и кортежи, содержащиеся и в \(R\), и в \(S\) с исключением
  дубликатов.

  Объединять можно не все отношения, а только совместные по объединению
  отношения. Отношения совместны по объединению, когда они состоят их
  одинакового количества атрибутов и каждая соответствующая пара атрибутов имеет
  одинаковый домен.

\item
  Разность \(R - S\)

  Разность состоит из кортежей, которые есть в \(R\), но отсутствуют в \(S\).
  Разность двух отношений определена только если они совместны по объединению.

\item
  Пересечение \(R \cap S\)

  Операция пересечения определяет отношение, содержащее кортежи, находящиеся как
  в \(R\), так и в \(S\). Пересечение двух отношений определено только если они
  совместны по объединению.

\item
  Декартово произведение \(R \times S\)

  Декартово произведение определяет новое отношение, которое является
  результатом конкатенации каждого кортежа из отношения \(R\) с каждым кортежем
  из отношения \(S\).

\item
  Тета-соединение \(R \bowtie_F S\)

  Определяет отношение содержащее кортежи из декартового произведения \(R \times
  S\), удовлетворяющие предикату \(F = R_{a_i} \Theta S_{b_i}\), где \(\Theta\)
  это одна из операций сравнения \(\set{>, <, =, \dotsc}\)

\item
  Экви-соединение

  Это тета-соединение, где \(\Theta\) это \quote{\(=\)}.

\item
  Естественное соединение \(R \bowtie S\)

  Соединение по эквивалентности двух отношений, выполненное по всем общим
  атрибутам, из результатов которого исключили по одному экземпляру каждого
  атрибута.

\item
  Левое внешнее соединение \(R \leftouterjoin S\)

  Это естественное соединение, при котором в результирующее отношение включаются
  также кортежи отношения \(R\), не имеющие совпадающих значений в общих
  атрибутах отношения \(S\).

\item
  Полусоединение \(R \mathrel{\rhd_F} S\)

  Это отношение, содержащее кортежи \(R\), которые входят в тета-соединение
  \(R\) и \(S\).

\item
  Деление.

  Эту операцию мы рассматривать не будем.
\end{enumerate}

Таким образом общий вид \sqlinline{SELECT} запроса будет следующим

\begin{sqlcode}
  SELECT [DISTINCT | ALL] { * | [ColumnExpr [AS NewName]] [, ...]}
  FROM Table [AS NewName]
  [{ INNER | LEFT OUTER | FULL } JOIN Table [AS NewName] [, ...]]
  [WHERE condition]
  [GROUP BY ColumnList [HAVING condition]]
  [ORDER BY ColumnName [ASC | DESC]]
\end{sqlcode}

Порядок выполнения \texttt{SQL}-запроса будет таким:
\begin{enumerate}
\item
  \texttt{FROM ... ON ... JOIN}

\item
  \texttt{WHERE}

\item
  \texttt{GROUP BY}

\item
  \texttt{HAVING}

\item
  \texttt{SELECT}

\item
  \texttt{DISTINCT}

\item
  \texttt{ORDER BY}
\end{enumerate}

