\subsection{%
  Лекция \texttt{23.09.01}.%
}

Информация бывает трех видов

\begin{enumerate}
\item
  Сигнал.

  Например, для человека, который не знает какой-либо язык, текст на этом языке
  будет сигналом: т.е. можно понять, что какая-то информация была передана, но
  принять (понять) эту информацию нельзя.

\item
  Знание.

  Студент, слушающий лектора и пишущий конспект, получает знания.

\item
  Данные.

  Данные это формализованные знания. В отличие от знаний данные не искажаются в
  процессе передачи. Если лектор передаст студентам конспект лекции, который он
  написал сам, то он передаст именно данные.
\end{enumerate}

\begin{definition}
  Данные это поддающиеся многократной интерпретации представления информации в
  формализованном виде пригодном для передачи, интерпретации и обработки.
\end{definition}

Все данные проходят путь вида

\begin{center}
  Сбор \(\to\)
  Обработка \(\to\)
  Передача \(\to\)
  Хранение \(\to\)
  Представление
\end{center}
