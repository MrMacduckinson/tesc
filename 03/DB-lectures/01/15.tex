\subsection{%
  Лекция \texttt{23.12.08}.%
}

\subsubheader{}{Графовые базы данных}

\begin{enumerate}
\item
  Можем красить вершины. Цвет вершины определяет ее тип (т.е. ограничения и
  атрибуты, присущие этой вершине).

\item
  Можем красить дуги. Цвет дуги определяет тип связи и ограничения,
  накладываемые на эту связь.

\item
  Дуги и вершины храним как \(\tuple{key : value}\) пары.
\end{enumerate}

\begin{definition}
  База знаний это совокупность единиц знаний, которые представляют собой
  формализованные с помощью некоторого метода представление знаний, отражение
  объектов предметной области и их взаимосвязей, действий над объектами и,
  возможно, неопределенностей с которыми эти действия осуществляются.
\end{definition}

\begin{remark}
  У базы знаний, в отличие от базы данных, есть механизм вывода новых знаний.
\end{remark}

Базы знаний делятся на открытые (можно добавлять/изменять данные) и закрытые
(данные добавляются только один раз и далее их нельзя менять).

Варианты модели знаний

\begin{enumerate}
\item
  Логическая \(M = \tuple{T, P, A, B}\)

  \begin{enumerate}
  \item
    \(T\)~--- множество базовых знаний (алфавит)

  \item
    \(P\)~--- множество синтаксических правил

  \item
    \(A\)~--- аксиомы

  \item
    \(B\)~--- множество правил вывода (применяя их к аксиомам можно получать
    новые синтаксические правила)
  \end{enumerate}

\item
  Сетевая \(H = \tuple{I, C_1, \dotsc, C_n, G}\)

  \begin{enumerate}
  \item
    \(I\)~--- множество информационных единиц

  \item
    \(C_1, \dotsc, C_n\)~--- множество типов связей

  \item
    \(G\)~--- граф
  \end{enumerate}

\item
  Фреймовая (\(\approx\) ООП подход)

  Фрейм это совокупность слотов. Каждый слот это свойство или метод для его
  получения. Существуют фасеточные слоты~--- слоты, содержащие множество
  значений.

\item
  Продукционнная \(\tuple{I, Q, P, A \to B, N}\)

  \begin{enumerate}
  \item
    \(I\)~--- уникальное имея продукции

  \item
    \(Q\)~--- область применения продукции

  \item
    \(P\)~--- условие, которое должно вызывать эту продукцию

  \item
    \(A \to B\)~--- ядро продукции

  \item
    \(N\)~--- постусловие, последствия, реализуемые только если ядро продукции
    выполнено
  \end{enumerate}
\end{enumerate}
