\subsection{%
  Лекция \texttt{23.09.22}.%
}

\begin{definition}
  \textit{Схема отношения} --- это строка заголовков.
\end{definition}

\begin{definition}
  Одна строка называется \textit{кортежем}.
\end{definition}

\begin{definition}
  Один столбец называется \textit{атрибутом}.
  Заголовок столбца определяет его имя.
\end{definition}

\begin{definition}
  \textit{Домен} --- это множество допустимых значений атрибута.
\end{definition}

\begin{definition}
  \textit{Степень отношения} --- это количество его атрибутов.
\end{definition}

\begin{definition}
  \textit{Кардинальность отношения} --- это количество его кортежей.
\end{definition}

\begin{definition}
  \textit{Отношение} --- это множество упорядоченных кортежей,
  где каждое значение берется из соответствующего домена.

  \begin{equation*}
    R = \set{d_1, \dotsc, d_n} \qquad d_i \in D_i
  \end{equation*}
\end{definition}

Свойства отношений (по Кодду):
\begin{enumerate}
  \item
        Каждая ячейка содержит одно неделимое значение.

  \item
        Каждый кортеж уникален.

  \item
        Уникальность имени отношения в реляционной схеме.

  \item
        Уникальность имени атрибута в пределах отношения.

  \item
        Значения атрибута берутся из одного и того же домена.

  \item
        Порядок следования атрибутов и кортежей не имеет значения.
\end{enumerate}

\begin{definition}
  \textit{Суперключ} --- это атрибут (множество атрибутов),
  который единственным образом идентифицирует кортеж.
\end{definition}

\begin{remark}
  Схема отношения всегда является суперключом (см. второе свойство отношений).
\end{remark}

\begin{definition}
  \textit{Потенциальный ключ} --- это суперключ, который не содержит подмножества,
  также являющегося суперключом этого отношения.
\end{definition}

\begin{definition}
  \textit{Первичный ключ} --- это потенциальный ключ, который выбран для
  идентификации кортежей внутри отношения.
\end{definition}

\begin{remark}
  Иногда используют технический первичный ключ (например поле \texttt{id}),
  который не несет никакой семантики, а служит лишь для определения уникальности
  кортежа.
\end{remark}

\begin{definition}
  \textit{Внешний ключ} --- это атрибут (множество атрибутов) внутри отношения,
  который соответствует потенциальному ключу некоторого (возможно того же самого)
  отношения.
\end{definition}

\begin{definition}
  \textit{Целостность сущностей} означает, что в отношении первичный ключ
  не может содержать \texttt{NULL} значения.
\end{definition}

\begin{definition}
  \textit{Ссылочная целостность} означает, что если в отношении существует
  внешний ключ, то его значение либо соответствует значениям потенциального ключа,
  либо полностью состоит из \texttt{NULL} значений.
\end{definition}
