\subsection{%
  Лекция \texttt{23.09.15}.%
}

\begin{definition}[по Коннолли и Бегг]
  \textit{База данных} --- это совместно используемый набор логически связанных 
  данных и описание этих данных, предназначенный для удовлетворения информационных
  потребностей предприятия.
\end{definition}

\begin{definition}[по Дейту]
  \textit{База данных} --- это набор постоянно хранимых данных, используемых 
  прикладными системами предприятия.
\end{definition}

\begin{definition}[по Хомоненко]
  \textit{База данных} --- это совокупность специальным образом организованных данных,
  хранимых в памяти вычислительной системы и отображающих состояние объектов и
  их взаимосвязей в рассматриваемой предметной области.
\end{definition}

Рассмотрим некоторые модели данных.

\subheader{Иерархическая модель данных}

Эта модель удобная для восприятия человеком, т.к. он часто сталкивается с
разного рода иерархией в реальном мире. У нас есть некоторые поля данных,
которые являются неделимыми атомами, а некоторые объединение этих полей
называется \textit{сегментом данных}.
Каждая запись в БД --- это экземпляр сегмента данных.

К минусам этой модели можно отнести дублирование данных, вследствие чего
возникает некоторая сложность поддержки их целостности.

\subheader{Сетевая модель данных}

Если иерархическая модель данных по сути являлась деревом, то сетевая модель
представляет собой граф, в котором вершины представляют собой некоторые
экземпляры объектов.

\subheader{Реляционная модель данных}

В отличие от иерархической и сетевой модели элементы модели сразу двумерные
(т.е. таблицы) и при этом мы не храним связь явно~--- мы выделяем некоторый
атрибут, который наделяем семантикой связи. Это может привести к некоторым
проблемам~--- семантика может быть неправильна трактована, в результате чего
запрос к БД либо не выполнится, либо выполнится некорректно. Проблему поддержки
целостности данных будем решать нормализацией (об этом будет рассказано в
последующих лекциях). Одним из плюсов этой модели является то, что мы может
оценить время работы запроса к БД.

\subheader{Постреляционная модель данных}

Берем реляционную модель и снимаем запрет на неделимость поля, т.е. в качестве
атрибута теперь можно хранить массив строк через запятую или сразу JSON с
какими-либо данными.

Эта модель была сделана для того, чтобы решить проблему очень большой
персонификации данных. Допустим, есть таблица с деталями и у некоторых
деталей есть какие-то атрибуты, которых в принципе не может быть у других. Если
добавить все возможные атрибуты каждой детали, то получим очень разряженную
таблицу почти полностью состоящую из null-значений \(\implies\) проблемы с
памятью.

С другой стороны мы усложняем себе поиск по такой базе (сложно проверять детали,
которые имеют разный набор атрибутов) и поддержку целостности~---  поле
JSON-данными при желании можно записать некорректное значение (либо нужно жестко
валидировать все записываемое в каждую такую ячейку \(\implies\) проблемы со
скоростью работы).

\subheader{Многомерная модель данных}

Пусть есть таблица со столбцами \quote{Продавец}, \quote{Товар}, \quote{Регион},
\quote{Квартал} и \quote{Количество}. Мы хотим уметь быстро строить сводки по
продавцу, товару, региону, кварталу (или по комбинации параметром), находить
наибольшее и наименьшее значения и т.п.

Можно представить наши данные в 4-ёх мерном пространстве, где первые четыре
колонки будут задавать координаты, а пятая (количество товара)~--- значение.
Теперь, чтобы получать различные метрики, мы будем рассекать полученный гиперкуб
гиперплоскостями, искать в них минимум/максимум и т.п.

Данная модель имеет большие затраты по памяти, но зато с её помощью можно быстро
выполнять запросы. Её можно использовать в следующем ключе: храним данные в
реляционной модели. Ночью строим по ней многомерную модель, а днем пользуемся ей
не внося никаких изменений (будем считать, что продажи товаров за день не сильно
влияют на глобальную ситуацию \(\implies\) ими можно пренебречь). Следующей
ночью заново строим многомерную модель по уже новым данным и так далее. Таким
образом в течение дня мы получаем возможность быстро делать разные запросы и
узнавать разные метрики.

\subheader{Объектно-ориентированная модель данных}

Похожа на постреляционную модель: мы сериализуем все объекты в системе и храним
пары вида \texttt{id: [serialized]}. Это усложняет поиск, т.к. нужно
десериализовывать данные.
