\subsection{Плотность тока. Уравнение непрерывности для плотности тока. Постоянный электрический ток}

\begin{definition}
    Электрический ток — упорядоченное (направленное) движение электрических зарядов
\end{definition}

\begin{definition}
    Сила тока $[I, А]$ — количественная мера электрического тока. Скалярная величина, определяемая электрическим зарядом, 
    проходящим через поперечное сечение проводника в единицу времени

    $$
    I=\frac{dq}{dt}
    $$
\end{definition}

Электрический ток может быть обусловлен движением как положительных, так и отрицательных носителей. Если в проводнике движутся носители обоих знаков, то:

$$
I=\frac{dq^+}{dt}+\frac{dq^-}{dt}
$$

За направление тока условно принимают направление движения положительных зарядов.

\begin{definition}
Плотность тока — физическая величина, определяемая силой тока, проходящего через единицу площади поперечного сечения проводника, перпендикулярного направлению тока.

$$
j=\frac{dI}{dS_\perp}
$$    
\end{definition}

В общем случае плотность тока не будет одинаковой по всему сечению проводника, поэтому:

$$
j=lim_{dS\to0}\frac{dI}{dS}
$$

$$
dI=jdS
$$

Сила тока сквозь произвольную поверхность $S$ определяется как поток вектора $j$, т.е.:

$$
I=\int_Sj_ndS
$$

Ток, сила и направление которого не изменяются со временем, называется постоянным.

\begin{definition}
    Уравнение непрерывности.

    Рассмотрим в некоторой среде, в которой течет ток, некоторую замкнутую поверхность $S$:

    Заряд, выходящий в единицу времени из объема $V$, ограниченного поверхностью $S$:  $\displaystyle\int_S(\vec j, d\vec S)$.

    В силу сохранения заряда эта величина должна быть равна скорости убывания заряда, содержащегося в данном объеме (уравнение непрерывности):

    $$
    \int_S(\vec j,d\vec S)=-\frac{\partial q}{\partial t}
    $$

    Знак минус означает то, что заряды вытекают из объема $V$.
\end{definition}

Воспользовавшись теоремой Остроградского-Гаусса, получим уравнение непрерывности в дифференциальной форме:

$$
(\vec\nabla,\ \vec j)=-\frac{\partial\rho}{\partial t}
$$

В точках, являющихся источниками вектора плотности тока, происходит убывание заряда.

Выражает закон сохранения электрического заряда: полный заряд системы не может изменяться, если через 
ее границу не проходят электрически заряженные частицы.

В случае постоянного тока ($\varphi, \rho = const)$:

$$
(\vec\nabla,\ \vec j)=0
$$

\begin{itemize}
    \item вектор плотности тока не имеет источников в случае постоянного тока
    \item линии тока нигде не начинаются и не заканчиваются — линии постоянного тока замкнуты
\end{itemize}
