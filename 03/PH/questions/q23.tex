\subsection{Дивергенция градиента. Оператор Лапласа. Уравнения Пуассона и Лапласа для ЭС поля}

\begin{definition}
    Дивергенция от градиента есть лапласиан:

    $$
    div\ grad\varphi=\nabla^2\varphi
    $$
\end{definition}

Оператор Лапласа (Лапласиан):

$$
\bigg(\frac{\partial^2}{\partial x^2_1}+\cdots+\frac{\partial^2}{\partial x_n^2}\bigg)F
$$

Уравнение Пуассона

$$
\frac{\partial^2 U}{\partial x^2}+\frac{\partial^2 U}{\partial y^2}+\frac{\partial^2 U}{\partial z^2}=-\frac{\rho}{\varepsilon}
$$

Уравнения Пуассона и Лапласа являются основными дифференциальными уравнениями электростатики. Они вытекают из теоремы Гаусса в дифференциальной
форме.
\begin{definition}
    Уравнение Лапласа (частный вид ур-я Пуассона).
    $$
    \frac{\partial^2 U}{\partial x^2}+\frac{\partial^2 U}{\partial y^2}+\frac{\partial^2 U}{\partial z^2}=0
    $$
\end{definition}

$$U = \frac{1}{4 \pi \varepsilon} \int\limits_V \frac{p dV}{r}$$  является решением уравнения Пуассона в случае, когда заряды распределены в конечной области пространства.

Если рассматриваемой области пространства отсутствуют объемные электрические заряды¸ то уравнение Пуассона станет уравнением Лапласа.

Даёт возможность определить потенциал поля объемных зарядов, если известно их расположение:

$$
\varphi=\int\frac{\rho dV}{R}
$$