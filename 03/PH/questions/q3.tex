\subsection{Инерциальные системы отсчета и первый закон Ньютона. Второй закон Ньютона. Масса, импульс, сила}

\begin{definition}
    Первый закон Ньютона (закон инерции).

    Существуют такие системы отсчёта, в которых свободное движение тел выглядит как прямолинейное и равномерное.
\end{definition}

\begin{definition}
    Свободное движение — движение, при котором на тело не действуют другие тела (отсутствует внешнее воздействие)
\end{definition}

\begin{definition}
    Равномерное и прямолинейное движение тела при отсутствии внешних воздействий называется движением по инерции (поэтому и закон инерции)
\end{definition}

$$
\begin{aligned}
\vec F_{рез}=0\ \implies\ &\vec\upsilon=const,\\
&\vec a=0
\end{aligned}
$$

\begin{definition}
    Система отсчёта — система координат и часы, связанные с телом отсчёта. 

    Движение тела зависит не только от воздействия на него других тел, но и 
    от свойств системы отсчёта, относительно которой рассматривается поведение этого тела.
\end{definition}

\begin{example}
    Поведение светофора в системе отсчёта, связанной с Землёй (светофор покоится), 
    и в системе отсчёта, связанной с тормозящим автомобилем (светофор движется с отрицательным ускорением)
\end{example}

\begin{definition}
    Инерциальная система отсчёта — система отсчёта, в которой все тела движутся прямолинейно и равномерно, либо находятся в состоянии покоя, 
    если на них не действуют никакие силы.
\end{definition}

\begin{definition}
    Неинерциальная система отсчёта — система отсчёта, которая двигается с ускорением по отношению к инерциальной.
\end{definition}

\begin{remark}
    Земля — инерциальная система (это на самом деле не так, однако вычисленное ускорение поверхности земли пренебрежительно мало)
\end{remark}

\begin{definition}
    Второй закон Ньютона.

    В инерциальных системах отсчёта ускорение, приобретаемое материальной точкой, прямо пропорционально вызывающей его силе, 
    совпадает с ней по направлению, и обратно пропорционально массе материальной точки.
\end{definition}

$$\vec F=\frac{d\vec p}{dt}$$, где $p$ - импульс, $t$ - время;

$$a=\frac{\upsilon}{t},\  m\upsilon=p \implies F=\frac{\Delta p}{\Delta t}=ma$$

\begin{remark}
    Это основной закон динамики, устанавливающий связь между силой и ускорением.
\end{remark}

\begin{definition}
    Сила $[\vec F, Н]$ — мера воздействия одного тела на другое, под действием которой тело получает ускорение или изменяет форму.
\end{definition}

Сила — векторная величина, характеризующаяся величиной, направлением и точкой приложения.

Две силы равны, если они равны по величине, имеют одинаковое направление и действуют по одной линии.

Если на тело действует несколько сил, то их можно сложить и заменить одной силой, которую называют равнодействующей. Она находится по правилу сложения векторов:

$$|\vec{F_р}| = \sqrt{F_1^2 + F_2^2 + 2F_1F_2\cos\alpha}$$

Или $F_р = \sqrt{F_{рx}^2 + F_{рy}^2}$, где $\vec{F_{рx}} = \sum \vec{F_{ix}}$ и $\vec{F_{рy}} = \sum \vec{F_{iy}}$

\begin{definition}
    Инертность — свойство тела “сопротивляться” воздействию (изменению скорости). Мерой инертности является инертная масса.
\end{definition}

\begin{definition}
    Масса $[m, кг]$ — скалярная величина, определяющая инерционные и гравитационные свойства тел в ситуациях, когда их скорость намного меньше скорости света.
\end{definition}

Под действием разных сил $\vec F_i$ одно и то же тело будет получать разные ускорения $\vec a_i$. Однако соотношение $\displaystyle\frac{\vec F_i}{\vec a_i}=const$. Именно эта константа принимается за инертную массу тела, она не зависит от сил и ускорения и является свойством самого тела.

Инертная масса является коэффициентом пропорциональности между результирующей силой и ускорением.

$$\vec F=m\vec a,\ m=const$$

\begin{remark}
    Ускорение, с которым движется тело, прямо пропорционально силе, действующей на тело, и обратно пропорционально массе тела
\end{remark}

\begin{definition}
    Импульс (количество движения) $[P,кг\cdot\frac{м}{с}]$ — произведение массы тела на его скорость
\end{definition}

$$\vec p=m\vec\upsilon=\vec F\Delta t$$

Изменение импульса в единицу времени равно результирующей силе, которая действует на тело.

$$\vec F_р=\frac{\Delta\vec p}{\Delta t}\implies\vec F_р\cdot\Delta t=\Delta\vec p$$

\begin{definition}
    Величину $\vec F\cdot\Delta t$ называют импульсом силы. Импульс силы равен изменению импульса тела.
\end{definition}

\begin{remark}
    В общем случае ($m\ne const$) 2й закон Ньютона записывается как: $\sum\limits_i\vec F_i=\frac{d\vec p}{dt}$
\end{remark}