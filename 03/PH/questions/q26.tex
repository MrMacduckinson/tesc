\subsection{Диэлектрики в электрическом поле. Полярные и неполярные диэлектрики. Индуцированный дипольный момент. Поляризация}

\begin{definition}
    Диэлектрики — вещества, не способные проводить электрический ток.

    В отличие от проводников, в идеальных диэлектриках отсутствуют свободные заряды. Заряды, входящие в состав диэлектрика тесно 
    связаны между собой и могут освободиться только под действием очень сильных полей. Такие заряды называются связанными.

    Внесённые во внешнее электрическое поле, диэлектрики испытывают изменение, называемое поляризацией
\end{definition}

\begin{definition}
    Поляризация диэлектрика — явление возникновения зарядов противоположного знака на противоположных концах диэлектрика при 
    внесении его во внешнее электрическое поле.
\end{definition}

\begin{remark}
    Механизм поляризации диэлектриков определяется строением их молекул. Известно, что атом или молекула любого вещества 
    являются электрически нейтральными. Однако центры распределения положительных и отрицательных зарядов в них могут или совпадать, 
    или не совпадать. В первом случае молекулы называются неполярными, а во втором — полярными.

    Полярные диэлектрики состоят из молекул, в которых центры распределения положительных и отрицательных зарядов не совпадают. 
    Такие молекулы можно представить в виде двух одинаковых по модулю разноименных точечных зарядов, находящихся на некотором расстоянии 
    друг от друга, называемых диполем.

    Неполярные диэлектрики состоят из атомов и молекул, у которых центры распределения положительных и отрицательных зарядов совпадают.
\end{remark}

\begin{definition}
    Индуцированный дипольный момент.

    Векторная величина, характеризующая, наряду с суммарным зарядом, электрические свойства системы заряженных частиц (распределения зарядов) 
    в смысле создаваемого ими поля и действия на неё внешних полей.

    $$
    \vec p=q\vec l
    $$
    где $q$ - величина положительного заряда,
    $l$ - вектор с началом в отрицательном заряде и концом в положительном.
\end{definition}

Для системы из $n$ частиц:
$$
\vec p=\sum_{i=1}^nq_i\vec r_i
$$
где $q_i$ - заряд частицы с номером $i$,
$r_i$ - её радиус-вектор.

Или, если суммировать отдельно по положительным и отрицательным зарядам:
$$
\vec p=\sum_{i=1}^{N^+}q_i^+\vec r_i-\sum_{i=1}^{N^-}|q_i^-|\vec r_i=Q^+\vec R^+-|Q^-|\vec R^-
$$

где $N^\pm$ - число положительно/отрицательно заряженных частиц

\begin{definition}
    Поляризованность — векторная величина, равная отношению суммы электрических моментов молекул, заключенных в физически малом элементе 
    диэлектрика, содержащем данную точку, к объему этого элемента:

    $$
    \overline P=\frac{\sum_{\Delta V}\overline p_i}{\Delta V}
    $$
    где $\overline p_i$ - дипольный момент $i$-той молекулы.
\end{definition}