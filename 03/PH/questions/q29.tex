\subsection{Понятия проводимости и сопротивления. Теория электропроводности Друда-Лоренца, ее ограничения}

\begin{definition}
    Электропроводность или электрическая проводимость — способность тела/среды проводить электрический ток.
\end{definition}

\begin{definition}
    Электропроводность — свойство тела/среды, определяющее возникновение в них электрического тока под действием электрического поля.

    $$
    G=\frac{I}{U}
    $$
    или ($\rho$, $Ом\cdot м$ - удельное электрическое сопротивление):
    $$
    \lambda=\frac{1}{\rho}
    $$
\end{definition}

\begin{definition}
    Электрическое сопротивление $[R,\ Ом]$ — физическая величина, характеризующая свойство проводника препятствовать прохождению эл. тока 
    и равная отношению напряжения на концах проводника к силе тока, протекающего по нему.
    $$
    I=\frac{U}{R}
    $$
    где $R$ - электрическое сопротивление проводника
\end{definition}

Для однородного линейного проводника сопротивление $R$ прямо пропорционально его длине $l$ и обратно пропорционально площади 
его поперечного сечения $S$ ($\rho$, $Ом\cdot м$ - удельное электрическое сопротивление):

$$
R=\rho\frac{l}{S}
$$

\begin{theorem}
Теория электропроводности Друда-Лоренца.
\begin{enumerate}
    \item Электроны проводимости в металле ведут себя подобно молекулам идеального газа.
    \item Между соударениями электроны движутся свободно, пробегая в среднем путь.
    \item Электроны сталкиваются в основном с ионами, образующими кристаллическую решетку, а не между собой.
    \item Столкновения электронов с ионами приводят к установлению теплового равновесия между электронным газом и кристаллической решеткой
\end{enumerate}

Ограничения:
\begin{itemize}
    \item Взаимодействие электрона с другими электронами и ионами не учитывается между столкновениями.
    \item Столкновения являются мгновенными событиями, внезапно меняющими скорость электрона
    \item Вероятность для электрона испытать столкновение за единицу времени равна $\frac{1}{r}$
    \item Состояние термодинамического равновесия достигается благодаря столкновениям.
\end{itemize}
\end{theorem}