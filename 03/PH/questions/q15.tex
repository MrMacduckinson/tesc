\subsection{Электрическое (ЭС) поле. Силовое и энергетическое описание. Закон Кулона}

\begin{definition}
    Электрическое поле — вид материи, посредством которого 
    осуществляется силовое воздействие на электрические заряды, находящиеся в этом поле.
\end{definition}

\begin{definition}
    Электростатическое поле — электрическое поле, 
    созданное системой неподвижных зарядов (частный случай электрического)
\end{definition}

Электростатическое поле не изменяется во времени.

\begin{definition}
    Основное свойство электрического поля: на всякий заряд, помещённый в это поле, действует сила.
\end{definition}

\begin{definition}
    Силовой характеристикой электрического поля является напряжённость.

    Напряжённость электрического поля в данной точке $[E,\frac{В}{м}]$ — векторная величина, численно равная
    силе, действующей на единичный положительный заряд, помещённый в данную точку поля.
\end{definition}

$$\vec E=\frac{\vec F}{q}$$

\begin{definition}
    Энергетической характеристикой электрического поля является потенциал.

    Потенциал $[\varphi, В]$ — скалярная величина, равная отношению потенциальной энергии, 
    которой обладает электрический заряд в данной точке электрического поля, к величине этого заряда.

    $$\varphi=\frac{E_{пот.q}}{q}$$
    
    где $E_{пот.q}$ - потенциальная энергия заряда в данной точке поля.
\end{definition}

\begin{definition}
    Закон Кулона.

    Два точечных заряда действуют друг на друга с силой, которая обратно пропорциональна квадрату расстояния 
    между ними и прямо пропорциональна произведению их зарядов (без учета знаков зарядов):
\end{definition}

$$F=k\frac{|q_1|\cdot|q_2|}{r^2} = \frac{1}{4\pi\varepsilon_0}\frac{q_1q_2}{r^2}$$

$k = 1,38 \cdot 10^{-23} \frac{Дж}{К}$ — постоянная Больцмана

$\varepsilon_0=8,85\cdot10^{-12} \frac{Ф}{м}$ — электрическая постоянная 