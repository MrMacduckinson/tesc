\subsection{Кинетическая энергия вращающегося твердого тела}

\begin{definition}
    Кинетическая энергия вращающегося тела — алгебраическая сумма кинетических энергий отдельных точек тела, 
    масса которых $\Delta m_i$
\end{definition}

\begin{definition}
    Кинетическая энергия – величина аддитивная, поэтому кинетическая энергия тела, движущегося произвольным образом, 
    равна сумме кинетических энергий всех n материальных точек, на которое это тело можно мысленно разбить.
\end{definition}

$$E_K=\sum_iE_{Ki}=\frac{1}{2}\sum_i\Delta m_i\upsilon_i^2$$

Учитывая, что $\upsilon_i=\omega\cdot r_i$:

$$E_K=\frac{1}{2}\omega^2\sum_i\Delta m_ir_i^2$$

, где $I=\sum_i\Delta m_ir_i^2$ - момент инерции твёрдого тела. Следовательно,

$$E_K=\frac{I\omega^2}{2}$$

Если тело вращается вокруг неподвижной оси $Oz$ с угловой скоростью $\overline\omega$, 
то линейная скорость $i$-й точки равна

$$\overline\upsilon_i=\overline\omega R_i$$

\begin{definition}
    Общий случай движения твёрдого тела.

Можно представить в виде суммы поступательного движения со скоростью $\upsilon_c$ и вращательного движения с 
угловой скоростью $\omega$ вокруг мгновенной оси, проходящей через центр инерции.
\end{definition}

Полная кинетическая энергия этого тела:
$$E_{K\ полн}=\frac{m\upsilon^2_c}{2}+\frac{I_c\omega^2}{2}$$

, где $I_c$ - момент инерции относительно мгновенной оси вращения, проходящей через центр инерции.