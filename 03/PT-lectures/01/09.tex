\subsection{%
  Лекция \texttt{23.10.31}.%
}

\subheader{Сингулярное распределение}

\begin{definition}
  Случайная величина \(\xi\) имеет сингулярное распределение, если существует
  борелевское множество \(B \in \mathcal{B}\) с нулевой мерой Лебега такое, что
  \(\prob{\xi \in B} = 1\), но \(\forall x \in B \given \prob{\xi = x} = 0\).
\end{definition}

Заметим, что данное множество \(B\)~--- несчетное, т.к. в противном случае по
свойству счетной аддитивности получили бы \(\prob{\xi \in B} = 0\). Таким
образом при сингулярном распределении случайная величина \(\xi\) распределена на
несчетном множестве лебеговой меры ноль. Т.к. \(\prob{\xi = x} = 0\), то функция
сингулярного распределения является непрерывной.

\begin{example}
  Функция распределения случайной величины \(\xi\)~--- лестница Кантора.

  \begin{equation*}
    F_{\xi} (x) = \begin{cases}
      0,                            & x \le 0 \\
      \frac{1}{2} F_{\xi} (3 x)     & 0 < x \le \frac{1}{3} \\
      \frac{1}{2}                   & \frac{1}{3} < x \le \frac{2}{3} \\
      \frac{1}{2} + \frac{1}{2} F_{\xi} (3 x - 2) & \frac{2}{3} < x < 1 \\
      1,                            & x \ge 1
    \end{cases}
  \end{equation*}
\end{example}

\begin{theorem}[Лебега]
  Пусть \(F_{\xi} (x)\)~--- функция распределения некоторой случайной величины
  \(\xi\), то ее можно представить в виде линейной комбинации

  \begin{equation*}
    F_{\xi} (x) = p_1 F_1 (x) + p_2 F_2 (x) + p_3 F_3 (x)
  \end{equation*}

  где \(p_1 + p_2 + p_3 = 1\) и \(F_1 (x)\), \(F_2 (x)\), \(F_3 (x)\)~--- это
  функции дискретного, абсолютно непрерывного и сингулярного распределения
  соответственно.
\end{theorem}

Таким образом существуют только дискретные, абсолютно непрерывные, сингулярные
распределения и их смеси.
