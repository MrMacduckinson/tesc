\subsection{%
  Лекция \texttt{23.11.28}.%
}

\subheader{Пространство случайных величин}

\begin{remark}
  Если случайные величины \(\xi\) и \(\eta\) равны \quote{почти наверное}
  (\(\prob{\xi = \eta} = 1 \)), то будем считать, что \(\xi = \eta\).
\end{remark}

Пусть дано вероятностное пространство \(\tuple{\Omega, \euF, \probP}\). Введем
пространство \(L_2\) как множество случайных величин с конечным вторым моментом.

\begin{equation*}
  L_2 (\Omega, \euF, \probP) = \set{\xi \given \variance{\xi} < \infty}
\end{equation*}

Ясно, что это линейное пространство.

\begin{definition}
  Скалярным произведением случайных величин \(\xi, \eta \in L_2\) называется
  \(\dotpdt{\xi}{\eta} = \expected{\xi \eta}\).
\end{definition}

\begin{remark}
  Если \(\tuple{\xi, \eta}\) имеет дискретное распределение, то \(\expected{\xi
  \eta} = \sum_{i, j} x_i y_j p_{i, j}\).

  Если \(\tuple{\xi, \eta}\) имеет абсолютно непрерывное распределение с
  плотностью \(f_{\xi, \eta} (x, y)\), то \(\expected{\xi \eta} = \iint_{\RR^2}
  x y f_{\xi, \eta} (x, y) \dd x \dd y\).
\end{remark}

\begin{remark}
  Т.к. дисперсии конечны, то данное скалярное произведение существует и конечно.
\end{remark}

\begin{equation*}
  \begin{rcases}
    \dotpdt{\xi}{\eta} = \dotpdt{\eta}{\xi}
  \\
    \dotpdt{C \xi}{\eta} = C \dotpdt{\xi}{\eta}
  \\
    \dotpdt{\xi_1 + \xi_2}{\eta} = \dotpdt{\xi_1}{\eta} + \dotpdt{\xi_2}{\eta}
  \\
    \dotpdt{\xi}{\xi} = \expected{\xi^2} \ge 0
  \\
    \dotpdt{\xi}{\xi} = 0 \implies \xi = 0 \text{ п.н.}
  \end{rcases}
  \implies
  \expected{\xi \eta} \text{ действительно скалярное произведение}
\end{equation*}

\begin{definition}
  Норма случайной величины определяется стандартно как корень из скалярного
  квадрата: \(\norm{\xi} = \sqrt{\dotpdt{\xi}{\xi}}\).
\end{definition}

\begin{remark}
  Расстояние между случайными величинами (метрику) определим как норму разности:
  \(\norm{\xi - \eta}\).
\end{remark}

\begin{remark}
  Таким образом рассматриваемое пространство является Гильбертовым (и даже
  Банаховым).
\end{remark}

\begin{theorem}[Неравенство Коши-Буняковского-Шварца] \label{thr:C-B-inequality}
  Пусть случайные величины \(\xi\) и \(\eta\) имеют конечные вторые моменты,
  тогда

  \begin{equation*}
    \abs{\expected{\xi \eta}} \le \sqrt{\expected{\xi^2} \expected{\eta^2}}
  \end{equation*}

  Причем равенство достигается тогда и только тогда, когда \(\xi = C \eta\).
\end{theorem}

\begin{proof}
  Доказательство аналогично приведенному ранее в курсе линейной алгебры.
  Рассмотрим квадратный трехчлен:

  \begin{equation*}
    P_2 (x)
    = \expected{(x \xi - \eta)^2}
    = x^2 \expected{\xi^2} - 2 x \expected{\xi \eta} + \expected{\eta^2}
  \end{equation*}

  Т.к. \(\forall x\) это трехчлен неотрицательный, то его дискриминант
  неположительный. Имеем

  \begin{equation*}
    D = 4 \expected{\xi \eta}^2 - 4 \expected{\xi^2} \expected{\eta^2} \le 0
    \implies
    \abs{\expected{\xi \eta}} \le \sqrt{\expected{\xi^2} \expected{\eta^2}}
  \end{equation*}

  Далее покажем, когда достигается равенство.

  \begin{equation*}
    \abs{\expected{\xi \eta}} = \sqrt{\expected{\xi^2} \expected{\eta^2}}
    \iff
    D = 0
    \implies
    \exists C = const \given
    \expected{(C \xi - \eta)^2} = 0
    \implies
    C \xi = \eta \text{ п.н.}
  \end{equation*}
\end{proof}

\subheader{Условное математическое ожидание}

Рассмотрим линейное подпространство, образованное множеством случайных величин
\(g(\eta)\) (функция \(g\)~--- борелевская) с конечным вторым моментом.

\begin{equation*}
  L
  = L (\eta)
  = \set{g(\eta) \given \variance{g(\eta)} < \infty}
  \subset L_2 (\Omega, \euF, \probP)
\end{equation*}

\begin{definition}
  Условным математическим ожиданием (УМО) \(\expected{\xi \mid \eta}\)
  называется ортогональная проекция случайной величины \(\xi\) на
  подпространство \(L(\eta)\).
\end{definition}

Далее будем обозначать \(\hat{\xi} = \expected{\xi \mid \eta}\).

\begin{lemma}[Тождество ортопроекции] \label{lem:ortoproj}
  Пусть \(\hat{\xi} \in L(\eta)\), тогда

  \begin{equation*}
    \hat{\xi} = \expected{\xi \mid \eta}
    \iff
    \expected{\xi g(\eta)} = \expected{\hat{\xi} g(\eta)}
    \qquad
    \forall g(\eta) \in L(\eta)
  \end{equation*}
\end{lemma}

\begin{proof}
  \begin{equation*}
    \begin{aligned}
      \hat{\xi} = \expected{\xi \mid \eta}
      \iff
      \xi - \hat{\xi} \perp g(\eta) \in L(\eta)
      \qquad
      \forall g(\eta)
    \\
      \dotpdt{\xi - \hat{\xi}}{g(\eta)}
      = \expected{(\xi - \hat{\xi}) g(\eta)}
      = 0
      \iff
      \expected{\xi g(\eta)} = \expected{\hat{\xi} g(\eta)}
      \qquad
      \forall g(\eta) \in L(\eta)
    \end{aligned}
  \end{equation*}
\end{proof}

\begin{lemma}[Формула полного математического ожидания]
  \begin{equation*}
    \expected{\xi}
    = \expected{\expected{\xi \mid \eta}}
    = \expected{\hat{\xi}}
  \end{equation*}
\end{lemma}

\begin{proof}
  Искомое равенство можно легко получить, если в \ref{lem:ortoproj} положить
  \(g(\eta) \equiv 1\).
\end{proof}

\begin{remark}
  В случае распределения Бернулли эта формула является уже знакомой формулой
  полной вероятности.
\end{remark}

\begin{lemma}[Линейность]
  \begin{equation*}
    \expected{C_1 \xi_1 + C_2 \xi_2 \mid \eta}
    = C_1 \expected{\xi_1 \mid \eta} + C_2 \expected{\xi_2 \mid \eta}
  \end{equation*}
\end{lemma}

\begin{lemma}
  Если случайные величины \(\xi\) и \(\eta\) независимы, то

  \begin{equation*}
    \expected{\xi \mid \eta} = \expected{\xi}
  \end{equation*}  
\end{lemma}

\begin{proof}
  Проверим тождество ортопроекции.

  \begin{equation*}
    \expected{\expected{\xi} g(\eta)}
    = \expected{\xi} \expected{g(\eta)}
    \eqby{независимость}
    \expected{\xi g(\eta)}
  \end{equation*}

  Тождество выполняется, значит \(\expected{\xi} = \expected{\xi \mid \eta}\).
\end{proof}

\begin{lemma}
  Если случайные величины \(\xi\) и \(\eta\) независимы, то \(\xi -
  \expected{\xi} \perp L(\eta)\) и, в частности, \(\xi - \expected{\xi} \perp
  \eta\).
\end{lemma}

Покажем, что данное определение УМО согласуется с введенным ранее понятием
математического ожидания условного распределения, т.е. \(\expected{\xi \mid
\eta} = h(\eta)\), где \(h(y) = \expected{\xi \mid \eta = y}\). Рассмотрим
случай абсолютно непрерывной системы случайных величин \(\tuple{\xi, \eta}\) с
плотностью \(f_{\xi, \eta} (x, y)\). Тогда

\begin{equation*}
  h(y) = \int_{-\infty}^{\infty} x f(x \mid y) \dd x
  \qquad
  f(x \mid y) = \frac{f_{\xi, \eta} (x, y)}{f_{\eta} (y)}
\end{equation*}

Следует проверить, что \(h(y)\) удовлетворяет тождеству ортопроекции, т.е. то,
что \(\expected{\xi g(\eta)} = \expected{h(\eta) g(\eta)}\).

\begin{equation*}
  \begin{aligned}
    \expected{\xi g(\eta)}
    = \iint_{\RR^2} x g(y) f_{\xi, \eta} (x, y) \dd x \dd y
  \\
    \expected{h(\eta) g(\eta)}
    = \int_{-\infty}^{\infty} h(y) g(y) f_{\eta} (y) \dd y
    = \int_{-\infty}^{\infty} \prh{
        \int_{-\infty}^{\infty}
        x \frac{f_{\xi, \eta} (x, y)}{f_{\eta} (y)} \dd x
      } g(y) f_{\eta} (y) \dd y
    = \iint_{\RR^2} x g(y) f_{\xi, \eta} (x, y) \dd x \dd y
  \end{aligned}
\end{equation*}

\subheader{Условная дисперсия}

\begin{definition}
  Условной дисперсией случайной величины \(\xi\) относительно случайной величины
  \(\eta\) называется случайная величина

  \begin{equation*}
    \variance{\xi \mid \eta}
    = \expected{\prh{\xi - \expected{\xi \mid \eta}}^2 \mid \eta}
  \end{equation*}

  Т.е. это дисперсия соответствующего условного распределения.
\end{definition}

\begin{lemma} \label{lem:variance-formula}
  \begin{equation*}
    \variance{\xi \mid \eta}
    = \expected{\xi^2 \mid \eta} - \prh{\expected{\xi \mid \eta}}^2
  \end{equation*}
\end{lemma}

\begin{proof}
  Доказательство аналогично доказательству такой же формулы для обычной
  дисперсии.
\end{proof}

\begin{theorem}[Закон полной дисперсии]
  \begin{equation*}
    \variance{\xi}
    = \expected{\variance{\xi \mid \eta}} + \variance{\expected{\xi \mid \eta}}
  \end{equation*}
\end{theorem}

\begin{proof}
  Известно, что \(\variance{\xi} = \expected{\xi^2} - \prh{\expected{\xi}}^2\).
  Применим к каждого слагаемому формулу полного математического ожидания.

  \begin{equation*} \label{eq:full-variance-law-1} \tag{1}
    \variance{\xi}
    = \expected{\expected{\xi^2 \mid \eta}}
      - \prh{\expected{\expected{\xi \mid \eta}}}^2
  \end{equation*}

  Из \ref{lem:variance-formula} можно получить, что

  \begin{equation*} \label{eq:full-variance-law-2} \tag{2}
    \expected{\xi^2 \mid \eta}
    = \variance{\xi \mid \eta} + \prh{\expected{\xi \mid \eta}}^2
  \end{equation*}

  Подставим \eqref{eq:full-variance-law-2} в \eqref{eq:full-variance-law-1}.

  \begin{equation*}
    \begin{aligned}
      \variance{\xi}
      & = \expected{\variance{\xi \mid \eta} + \prh{\expected{\xi \mid \eta}}^2}
      - \prh{\expected{\expected{\xi \mid \eta}}}^2
    \\
      & = \expected{\variance{\xi \mid \eta}} + \under{
        \expected{\prh{\expected{\xi \mid \eta}}^2}
          - \prh{\expected{\expected{\xi \mid \eta}}}^2
      }{\variance{\expected{\xi \mid \eta}}}
    \\
      & = \expected{\variance{\xi \mid \eta}}
        + \variance{\expected{\xi \mid \eta}}
    \end{aligned}
  \end{equation*}
\end{proof}

\begin{remark}
  Пусть \(\xi\) и \(\eta\) независимы, тогда

  \begin{equation*}
    \variance{\expected{\xi \mid \eta}}
    = \variance{\expected{\xi}}
    = 0
    \implies
    \variance{\xi} = \expected{\variance{\xi \mid \eta}}
  \end{equation*}
\end{remark}

\begin{remark}
  Если имеется функциональная зависимость \(\xi = g(\eta)\), то

  \begin{equation*}
    \variance{\expected{\xi \mid \eta}}
    = \variance{\expected{g(\eta) \mid \eta}}
    = \variance{g(\eta)}
    = \variance{\xi}
  \end{equation*}
\end{remark}

Таким образом доля

\begin{equation*}
  0 \le R^2 = \frac{\variance{\expected{\xi \mid \eta}}}{\variance{\xi}} \le 1
\end{equation*}

отражает уровень зависимости случайной величины \(\xi\) от случайной величины
\(\eta\).

\begin{definition}
  Величина \(R\) называется корреляционным отношением.
\end{definition}

\subheader{Числовые характеристики зависимости случайных величин}

Если \(\xi\) и \(\eta\) независимы, то \(\expected{\xi \eta} = \expected{\xi}
\expected{\eta}\), поэтому в качестве индикатора наличия связи логично взять
величину \(\expected{\xi \eta} - \expected{\xi} \expected{\eta} =
\cov{\xi}{\eta}\).

\begin{definition}
  Ковариацией случайных величин \(\xi\) и \(\eta\) называется число

  \begin{equation*}
    \cov{\xi}{\eta}
    = \expected{\prh{\xi - \expected{\xi}} \prh{\eta - \expected{\eta}}}
  \end{equation*}
\end{definition}

\begin{lemma}[Равносильность определений]
  \begin{equation*}
    \cov{\xi}{\eta}
    = \expected{\prh{\xi - \expected{\xi}} \prh{\eta - \expected{\eta}}}
    = \expected{\xi \eta} - \expected{\xi} \expected{\eta}
  \end{equation*}
\end{lemma}

\begin{proof}
  \begin{equation*}
    \begin{aligned}
      \cov{\xi}{\eta}
      & = \expected{\prh{\xi - \expected{\xi}} \prh{\eta - \expected{\eta}}}
    \\
      & = \expected{
        \xi \eta
        - \eta \expected{\xi}
        - \xi \expected{\eta}
        + \expected{\xi} \expected{\eta}
      }
    \\
      & = \expected{\xi \eta}
        - \expected{\xi} \expected{\eta}
        - \expected{\xi} \expected{\eta}
        + \expected{\xi} \expected{\eta}
    \\
      & = \expected{\xi \eta} - \expected{\xi} \expected{\eta}
    \end{aligned}
  \end{equation*}
\end{proof}

\subsubheader{}{Свойства ковариации}

\begin{enumerate}
\item
  \(\display{
    \cov{\xi}{\xi}
    = \expected{\xi^2} - \prh{\expected{\xi}}^2
    = \variance{\xi}
  }\)

\item
  \(\display{
    \cov{\xi}{\eta} = \cov{\eta}{\xi}
  }\)

\item
  \(\display{
    \cov{C \xi}{\eta} = C \cov{\xi}{\eta}
  }\)

\item
  \(\display{
    \cov{\xi_1 + \xi_2}{\eta} = \cov{\xi_1}{\eta} + \cov{\xi_2}{\eta}
  }\)

\item
  \(\display{
    \variance{\xi_1 + \dotsc + \xi_n}
    = \variance{\xi_1} + \dots + \variance{\xi_n}
      + 2 \sum_{i < j} \cov{\xi_i}{\xi_j}
    = \sum_{i, j} \cov{\xi_i}{\xi_j}
  }\)

\item
  Если \(\xi\) и \(\eta\) независимы, то \(\cov{\xi}{\eta} = 0\).

\item
  Если \(\cov{\xi}{\eta} \neq 0\), то \(\xi\) и \(\eta\)~--- зависимы.

\item
  Если \(\cov{\xi}{\eta} = 0\), то нельзя сделать никаких выводов.

\item
  Если \(\cov{\xi}{\eta} > 0\), то зависимость прямая.

\item
  Если \(\cov{\xi}{\eta} < 0\), то зависимость обратная.
\end{enumerate}

\begin{remark}
  По величине ковариации нельзя судить о силе связи, т.к. она зависит от единиц
  измерения. Ее следует \quote{нормировать}.
\end{remark}

\subheader{Коэффициент линейной корреляции}

\begin{definition}
  Коэффициентом линейной корреляции \(\rho_{\xi, \eta}\) случайных величин
  \(\xi\) и \(\eta\) с конечным вторым моментом называется число

  \begin{equation*}
    \rho_{\xi, \eta}
    = \frac{\cov{\xi}{\eta}}{\sqrt{\variance{\xi}} \sqrt{\variance{\eta}}}
    = \frac{\expected{\xi \eta} - \expected{\xi} \expected{\eta}}
      {\stder{\xi} \stder{\eta}}
  \end{equation*}
\end{definition}

\begin{remark}
  Эта безразмерная величина, поэтому по ней можно судить о силе связи.
\end{remark}

\begin{remark}
  \begin{equation*}
    \rho_{\xi, \eta}
    = \frac{\expected{\prh{\xi - \expected{\xi}} \prh{\eta - \expected{\eta}}}}
      {\sqrt{\expected{\xi - \expected{\xi}^2}}
        \sqrt{\expected{\eta - \expected{\eta}^2}}}
    = \frac{\dotpdt{\xi - \expected{\xi}}{\eta - \expected{\eta}}}
        {\norm{\xi - \expected{\xi}} \cdot \norm{\eta - \expected{\eta}}}
    = \cos \angle \prh[\Big]{\xi - \expected{\xi}, \eta - \expected{\eta}}
  \end{equation*}
\end{remark}

\subsubheader{}{Свойства корреляции}

\begin{enumerate}
\item
  \(\display{
    \rho_{\xi, \eta} = \rho_{\eta, \xi}
  }\)

\item
  Если \(\xi\) и \(\eta\) независимы, то \(\rho_{\xi, \eta} = 0\).

\item
  Если \(\rho_{\xi, \eta} \neq 0\), то \(\xi\) и \(\eta\)~--- зависимы.

\item
  Если \(\rho_{\xi, \eta} = 0\), то нельзя сделать никаких выводов.
\end{enumerate}

\begin{lemma}
  \begin{equation*}
    \abs{\rho_{\xi, \eta}} \le 1
  \end{equation*}
\end{lemma}

\begin{proof}
  По \ref{thr:C-B-inequality} имеем

  \begin{equation*}
    \abs{\expected{\prh{\xi - \expected{\xi}} \prh{\eta - \expected{\eta}}}}
    \le \sqrt{\prh{\xi - \expected{\xi}}^2 \prh{\eta - \expected{\eta}}^2}
  \end{equation*}

  Подставив это в определение корреляции получаем искомую оценку.
\end{proof}

\begin{lemma} \label{lem:correlation-dependency}
  \begin{equation*}
    \abs{\rho_{\xi, \eta}} = 1
    \iff
    \eta = C \xi + b
  \end{equation*}
\end{lemma}

\begin{proof}
  По \ref{thr:C-B-inequality} имеем

  \begin{equation*}
    \begin{aligned}
      \abs{\rho_{\xi, \eta}} = 1
      \iff
      \abs{\expected{\prh{\xi - \expected{\xi}} \prh{\eta - \expected{\eta}}}}
      = \sqrt{\prh{\xi - \expected{\xi}}^2 \prh{\eta - \expected{\eta}}^2}
    \\
      \eta - \expected{\eta} = C \prh{\xi - \expected{\xi}}
      \iff
      \eta = C \xi + \under{\expected{\eta} - C \expected{\xi}}{b}
    \end{aligned}
  \end{equation*}
\end{proof}

\begin{lemma}
  \begin{equation*}
    \begin{aligned}
      \rho_{\xi, \eta} = 1
      \implies
      \eta = a \xi + b
      \qquad
      a > 0 \text{ (прямая зависимость)}
    \\
      \rho_{\xi, \eta} = -1
      \implies
      \eta = a \xi + b
      \qquad
      a < 0 \text{ (обратная зависимость)}
    \end{aligned}
  \end{equation*}
\end{lemma}

\begin{proof}
  По \ref{lem:correlation-dependency} получаем, что \(\eta = a \xi + b\).
  Подставим это в определение корреляции.

  \begin{equation*}
    \begin{aligned}
      \rho_{\xi, \eta}
      & = \frac{\expected{\xi \eta} - \expected{\xi} \expected{\eta}}
        {\sqrt{\variance{\xi}} \sqrt{\variance{\eta}}}
    \\
      & = \frac{\expected{\xi \prh{a \xi + b}}
          - \expected{\xi} \expected{a \xi + b}}
        {\sqrt{\variance{\xi}} \sqrt{\variance{a \xi + b}}}
    \\
      & = \frac{a \expected{\xi^2} + b \expected{\xi}
          - a \prh{\expected{\xi}}^2 - b \expected{\xi}}
        {\sqrt{\variance{\xi}} \sqrt{a^2 \variance{\xi}}}
    \\
      & = \frac{a \prh{\expected{\xi^2} - \prh{\expected{\xi}}^2}}
        {\abs{a} \variance{\xi}}
    \\
      & = \frac{a \variance{\xi}}
        {\abs{a} \variance{\xi}}
    \\
      & = \operatorname{sign} a
    \end{aligned}
  \end{equation*}  
\end{proof}

\begin{definition}
  Если \(\rho_{\xi, \eta} \neq 0\), то говорят, что случайные величины \(\xi\) и
  \(\eta\) коррелированны друг с другом. Если \(\rho_{\xi, \eta}\) больше нуля,
  то имеется прямая корреляция между случайными величинами \(\xi\) и \(\eta\), а
  если меньше нуля, то обратная корреляция.
\end{definition}

\begin{remark}
  Коэффициент корреляции отражает преимущественно силу линейной части связи, а
  саму силу лучше отражает корреляционное отношение.
\end{remark}

\begin{theorem}
  \begin{equation*}
    R \ge \rho_{\xi, \eta}
  \end{equation*}
\end{theorem}

\begin{remark}
  \(R = \rho_{\xi, \eta}\) тогда и только тогда, когда уравнение регрессии
  является линейным.
\end{remark}

\begin{example}
  Пусть между случайными величинами \(\xi_1\) и \(\xi_2\) прямая корреляция,
  между \(\xi_2\) и \(\xi_3\) также прямая корреляция. Однако из этого
  \textbf{не следует}, что между \(\xi_1\) и \(\xi_3\) будет прямая корреляция.
\end{example}
