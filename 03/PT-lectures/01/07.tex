\subsection{%
  Лекция \texttt{23.10.17}.%
}

\subheader{Стандартное дискретное распределение}

\subsubheader{I.}{Распределение Бернулли \(B_p\) (с параметром \(0 < p < 1\))}

Пусть случайная величина \(\xi\)~--- число успехов при одном испытании, где
\(p\)~--- вероятность успеха. Закон распределения имеет вид:

\begin{ttable}{0.4 \linewidth}{X|X|X}
  \(\xi\) & \(0\)         & \(1\)
  \\ \hline
  \(p\)   & \(q = 1 - p\) & \(p\)
\end{ttable}

Вычислим его числовые характеристики.

\begin{equation*}
  \begin{aligned}
    \expected{\xi}
    = 0 \cdot (1 - p) + 1 \cdot p
    = p
  \\
    \variance{\xi}
    = 0^2 \cdot (1 - p) + 1^2 \cdot p - p^2
    = p - p^2
    = p (1 - p)
    = p q
  \end{aligned}
\end{equation*}

Индикатор случайной величины имеет распределение Бернулли.

\subsubheader{II.}{Биномиальное распределение \(\bindst{n}{p}\)}

Пусть случайная величина \(\xi\) это число успехов в серии из \(n\) независимых
испытаний, где \(p\) это вероятность успеха при одном испытании.

\begin{equation*}
  \xi \in \bindst{n}{p}
  \iff
  \forall 0 \le k \le n, k \in \ZZ \given
  \prob{\xi = k} = \comb{n}{k} p^k q^{n - k}
\end{equation*}

Закон распределения имеет вид:

\begin{ttable}{0.6 \linewidth}{X|X|X|X|X|X|X}
  \(\xi\) & \(0\)   & \(1\)                       & \(\dotsc\)
    & \(k\)           & \(\dotsc\) & \(n\)
  \\ \hline
  \(p\)   & \(q^n\) & \(\comb{n}{1} p q^{n - 1}\) & \(\dotsc\)
	& \(\comb{n}{k} p^k q^{n - k} \) & \(\dotsc\) & \(p^n\)
\end{ttable}

Заметим, что \(\xi\) складывается из числа успехов в каждом испытании, поэтому
\(\xi = \xi_1 + \dotsc + \xi_n\), где \(\xi_i \in B_p\)~--- число успехов при
\(i\)-ом испытании. Получаем

\begin{equation*}
  \begin{aligned}
    \expected{\xi}
    = \expected{\xi_1} + \dotsc + \expected{\xi_n}
    = n p
  \\
    \variance{\xi}
    = \variance{\xi_1} + \dotsc + \variance{\xi_n}
    = n p q
  \\
    \stder{\xi}
    = \sqrt{n p q}
  \end{aligned}
\end{equation*}

\subsubheader{III.}{Геометрическое распределение \(\geodst{p}\) (\(0 < p < 1\))}

Пусть случайная величина \(\xi\) это номер первого успешного испытания, где
\(p\)~--- вероятность успеха при одном испытании.

\begin{equation*}
  \xi \in \geodst{p}
  \iff
  \forall k \in \NN \given \prob{\xi = k} = q^{k - 1} p
\end{equation*}

Закон распределения будет иметь вид:

\begin{ttable}{0.6 \linewidth}{X|X|X|X|X|X}
  \(\xi\) & \(1\) & \(2\)   & \(\dotsc\) & \(k\)           & \(\dotsc\)
  \\ \hline
  \(p\)   & \(p\) & \(p q\) & \(\dotsc\) & \(q^{k - 1} p\) & \(\dotsc\)
\end{ttable}

Найдем числовые характеристики.

\begin{equation*}
  \begin{aligned}
    \expected{\xi}
    = \sum_{k = 1}^{\infty} k \cdot q^{k - 1} p
    = p \sum_{k = 1}^{\infty} \prh{q^k}'
    = p \prh{\sum_{k = 1}^{\infty} q^k}'
    = p \prh{\frac{1}{1 - q}}'
    = p \cdot \frac{1}{(1 - q)^2}
    = \frac{p}{p^2}
    = \frac{1}{p}
  \\
    \expected{\xi^2}
    = \sum_{k = 1}^{\infty} k^2 \cdot q^{k - 1} p
    = \sum_{k = 1}^{\infty} \prh{k (k - 1) + k} \cdot q^{k - 1} p
    = \under{\sum_{k = 1}^{\infty} k (k - 1) \cdot q^{k - 1} p}{\text{ниже}}
      + \under{\sum_{k = 1}^{\infty} k \cdot q^{k - 1} p}{\expected{\xi}}
  \\
    \sum_{k = 1}^{\infty} k (k - 1) \cdot q^{k - 1} p
    = p q \sum_{k = 1}^{\infty} \prh{q^k}''
    = p q \prh{\sum_{k = 1}^{\infty} q^k}''
    = p q \prh{\frac{1}{1 - q}}''
    = p q \cdot \frac{2}{(1 - q)^3}
    = \frac{2 q}{p^2}
  \\
    \expected{\xi^2}
    = \frac{2 q}{p^2} + \frac{1}{p}
    = \frac{2 q + p}{p^2}
    = \frac{q + 1}{p^2}
  \\
    \variance{\xi}
    = \expected{\xi^2} - \prh{\expected{\xi}}^2
    = \frac{q + 1}{p^2} - \prh{\frac{1}{p}}^2
    = \frac{q}{p^2}
  \end{aligned}
\end{equation*}

\subsubheader{IV.}{Распределение Пуассона \(\Pi_{\lambda}\) (c параметром
\(\lambda > 0\))}

\begin{definition}
  Случайная величина \(\xi\) имеет распределение Пуассона \(\Pi_{\lambda}\) с
  параметром \(\lambda > 0\), если

  \begin{equation*}
    \forall k \ge 0, k \in \ZZ \given
    \prob{\xi = k} = \frac{\lambda^k}{k!} e^{-\lambda}
  \end{equation*}
\end{definition}

\begin{ttable}{0.6 \linewidth}{X|X|X|X|X|X}
  \(\xi\) & \(0\)            & \(1\)                    & \(\dotsc\)
    & \(k\)                                 & \(\dotsc\)
  \\ \hline
  \(p\)   & \(e^{-\lambda}\) & \(\lambda e^{-\lambda}\) & \(\dotsc\)
    & \(\frac{\lambda^k}{k!} e^{-\lambda}\) & \(\dotsc\)
\end{ttable}

Вычислим числовые характеристики.

\begin{equation*}
  \begin{aligned}
    \expected{\xi}
    = \sum_{k = 0}^{\infty} k \frac{\lambda^k}{k!} e^{-\lambda}
    = \lambda e^{-\lambda}
      \sum_{k = 1}^{\infty} \frac{\lambda^{k - 1}}{(k - 1)!} 
    = \lambda e^{-\lambda} \sum_{n = 0}^{\infty} \frac{\lambda^n}{n!}
    = \lambda e^{-\lambda} \cdot e^{\lambda}
    = \lambda
  \\
    \expected{\xi^2}
    = \sum_{k = 0}^{\infty} k^2 \frac{\lambda^k}{k!} e^{-\lambda}
    = \sum_{k = 0}^{\infty}
      \frac{k (k - 1) + k}{k!} \cdot \lambda^k e^{-\lambda}
    = \lambda^2 e^{-\lambda}
        \sum_{k = 2}^{\infty} \frac{\lambda^{k - 2}}{(k - 2)!}
      + \under{
        \sum_{k = 0}^{\infty} \frac{k}{k!} \cdot \lambda^k e^{-\lambda}
      }{\expected{\xi}}
    = \lambda^2 e^{-\lambda} e^{\lambda} + \lambda
    = \lambda^2 + \lambda
  \\
    \variance{\xi}
    = \expected{\xi^2} - \prh{\expected{\xi}}^2
    = \lambda^2 + \lambda - \lambda^2
    = \lambda
  \end{aligned}
\end{equation*}

\subheader{Функция распределения}

\begin{definition}
  Функцией распределения (\figref{01_07_01}) \(F_{\xi} (x)\) случайной величины
  \(\xi\) называется функция

  \begin{equation*}
    F_{\xi} (x) = \prob{\xi < x}, x \in \RR
  \end{equation*}
\end{definition}

\gallerydouble
  {01_07_01}{Функция распределения}
  {01_07_02}{Функция распределения Бернулли}

\begin{example}
  Функция распределения Бернулли \(B_p\) (\figref{01_07_02}).

  \begin{equation*}
    F(x) = \begin{cases}
      0,     & x \le 0     \\
      1 - p, & 0 < x \le 1 \\
      1,     & x > 1       \\
    \end{cases}
  \end{equation*}
\end{example}

\subheader{Свойства функции распределения}

\begin{lemma} \label{lem:dist-func-lim}
  \(F(x)\)~--- ограниченная функция. \(0 \le F(x) \le 1\)
\end{lemma}

\begin{proof}
  \(F(x)\) это вероятность \(\implies\) по определению вероятности она лежит в
  отрезке \(\segment{0}{1}\).
\end{proof}

\begin{lemma} \label{lem:dist-func-nondecr}
  \(F(x)\)~--- неубывающая функция. \(x_1 < x_2 \implies F(x_1) \le F(x_2)\)
\end{lemma}

\begin{proof}
  \begin{equation*}
    x_1 < x_2 \implies
    \set{\xi \in X_1} \subset \set{\xi \in X_2} \implies
    F(x_1) = \prob{\xi < x_1} \le \prob{\xi < x_2} = F(x_2)
  \end{equation*}
\end{proof}

\begin{lemma} \label{lem:dist-func-interval}
  \begin{equation*}
    \prob{\alpha \le \xi < \beta} = F(\beta) - F(\alpha)
  \end{equation*}
\end{lemma}

\begin{proof}
  Воспользуемся свойством аддитивности.

  \begin{equation*}
    \begin{aligned}
      \prob{\xi < \beta}
      = \prob{\xi < \alpha} + \prob{\alpha \le \xi < \beta} 
    \\
      F(\beta) = F(\alpha) + \prob{\alpha \le \xi < \beta} 
      \implies
      \prob{\alpha \le \xi < \beta} = F(\beta) - F(\alpha)
    \end{aligned}
  \end{equation*}
\end{proof}

\begin{remark}
  Т.к. борелевская \(\sigma\)-алгебра порождается интервалами
  \(\interval{-\infty}{x}\), то зная функцию распределения, можем найти
  вероятности попадания случайной величины в любое борелевское множество.
  Следовательно, функция распределения полностью задает распределение.
\end{remark}

\begin{lemma}
  \begin{equation*}
    \lim_{n \to -\infty} F(x) = 0
    \qquad
    \lim_{n \to +\infty} F(x) = 1
  \end{equation*}
\end{lemma}

\begin{proof}
  Т.к. функция распределения монотонна (\ref{lem:dist-func-nondecr}) и
  ограничена (\ref{lem:dist-func-lim}), то данные пределы существуют и конечны.
  Значит, достаточно найти эти пределы для последовательности \(\seq{x_n} \to
  \pm \infty\). Рассмотрим \(A_n = \set{n - 1 \le \xi_n < n}, n \in \ZZ\)~---
  несовместные события при разных \(n\). По свойству счетной аддитивности
  получаем

  \begin{equation*}
    1
    = \prob{\Omega}
    = \prob{\sum_{n = -\infty}^{\infty} A_n}
    = \sum_{n = -\infty}^{\infty} \prob{A_n}
    \eqby{\ref{lem:dist-func-interval}}
    \sum_{n = -\infty}^{\infty} \prh{F(n) - F(n - 1)}
    = \lim_{N \to \infty} \sum_{n = -N}^{N} \prh{F(n) - F(n - 1)}
  \end{equation*}

  Заметим, что в сумме все слагаемые кроме первого и последнего сокращаются,
  поэтому получаем

  \begin{equation*}
    1
    = \lim_{N \to \infty} \prh{F(N) - F(-N - 1)}
    = \lim_{N \to \infty} F(N) - \lim_{N \to \infty} F(-N - 1)
    \implies
    \lim_{N \to \infty} F(N) = 1 + \lim_{N \to \infty} F(-N - 1)
  \end{equation*}

  Левый предел не превосходит единицы, а правый предел не меньше нуля, значит
  получаем, что левый предел в точности равен единице, а правый предел в
  точности равен нулю.
\end{proof}

\begin{lemma}
  \(F(x)\) непрерывная слева, т.е.

  \begin{equation*}
    \forall x_0 \in \RR \given F(x_0 - 0) = F(x_0)
  \end{equation*}
\end{lemma}

\begin{proof}
  Т.к. \(F(x)\) монотонна (\ref{lem:dist-func-nondecr}) и ограничена сверху
  \(F(x_0)\), то данный предел существует. Пусть \(B_n = \set{x_0 - \frac{1}{n}
  \le \xi < x_0}\)~--- убывающая цепочка событий, т.е. \(B_1 \supset B_2 \supset
  \dotsc \supset B_n \supset \dotsc\) и \(\bigcup_{n = 1}^{\infty} B_n =
  \varnothing\). Значит по аксиоме непрерывности \(\prob{B_n} \Rarr{n \to
  \infty} 0\), получаем

  \begin{equation*}
    \begin{aligned}
      0
      = \lim_{n \to \infty} \prob{B_n}
      = \lim_{n \to \infty} \prob{x_0 - \frac{1}{n} \le \xi < x_0}
      = \lim_{n \to \infty} \prh{F(x_0) - F \prh{x_0 - \frac{1}{n}}}
      = F(x_0) - \lim_{n \to \infty} F \prh{x_0 - \frac{1}{n}}
      = 0
    \\
      F(x_0) = \lim_{n \to \infty} F \prh{x_0 - \frac{1}{n}}
    \end{aligned}
  \end{equation*}
\end{proof}

\begin{lemma} \label{lem:jump-dot}
  Скачок в точке \(x_0\) равен вероятности случайной величины попасть в данную
  точку.

  \begin{equation*}
    F(x_0 + 0)
    = F(x_0) + \prob{\xi = x_0} 
    = \prob{\xi \le x_0} 
  \end{equation*}
\end{lemma}

\begin{proof}
  Т.к. \(F(x)\) монотонна (\ref{lem:dist-func-nondecr}) и ограничена снизу, то
  данный предел существует. Пусть \(C_n = \set{x_0 < \xi < x_0 +
  \frac{1}{n}}\)~--- убывающая цепочка событий,  т.е. \(C_1 \supset C_2 \supset
  \dotsc \supset C_n \supset \dotsc\) и \(\bigcup_{n = 1}^{\infty} C_n =
  \varnothing\). Значит по аксиоме непрерывности \(\prob{C_n} \Rarr{n \to
  \infty} 0\), получаем

  \begin{equation*}
    \begin{aligned}
      \prob{x_0 < \xi < x_0 + \frac{1}{n}} + \prob{\xi = x_0}
      & \Rarr{n \to \infty} &
      0 + \prob{\xi = x_0}
    \\
      \prob{x_0 \le \xi < x_0 + \frac{1}{n}}
      & \Rarr{n \to \infty} &
      \prob{\xi = x_0}
    \\
      F \prh{x_0 + \frac{1}{n}} - F(x_0)
      & \Rarr{n \to \infty} &
      \prob{\xi = x_0}
    \end{aligned}
  \end{equation*}
\end{proof}

\begin{lemma}
  Если \(F(x)\)~--- непрерывна в точке \(x = x_0\), то \(\prob{\xi = x_0} = 0\).
\end{lemma}

\begin{lemma}
  Если \(F(x)\)~--- непрерывна, то

  \begin{equation*}
    \prob{\alpha \le \xi < \beta}
    = \prob{\alpha < \xi < \beta}
    = \prob{\alpha \le \xi \le \beta}
    F(\beta) - F(\alpha)
  \end{equation*}
\end{lemma}

\begin{theorem}
  Случайная величина \(\xi\) имеет дискретное распределение \(\iff\) ее функция
  распределения имеет ступенчатый вид.
\end{theorem}

\subheader{Абсолютно непрерывные случайные величины}

\begin{definition}
  Случайная величина \(\xi\) имеем абсолютно непрерывное распределение, если
  существует \(f_{\xi} (x)\) такая, что

  \begin{equation*}
    \forall B \in \mathcal{B} (\RR) \given
    \prob{\xi \in B} = \int_{B} f_{\xi} (x) \dd x
  \end{equation*}

  Функция \(f_{\xi} (x)\) называется плотностью распределения.
\end{definition}

\subheader{Свойства плотности и функции распределения абсолютно непрерывного
распределения}

\begin{equation*}
  B = \segment{\alpha}{\beta}
  \implies
  \prob{\alpha \le \xi \le \beta} = \int_{\alpha}^{\beta} f_{\xi} (x) \dd x
\end{equation*}

Используя геометрический смысл определенного интеграла, получаем, что
\(\prob{\alpha \le \xi \le \beta}\) это площадь под графиком \(f_{\xi} (x)\) на
отрезке \(\segment{\alpha}{\beta}\).

\begin{lemma}
  Условие нормировки: площадь всей фигуры под графиком \(f_{\xi} (x)\) должна
  равняться единице, т.е.

  \begin{equation*}
    \int_{-\infty}^{\infty} f_{\xi} (x) \dd x = 1
  \end{equation*}
\end{lemma}

\begin{proof}
  Искомое равенство получается из определения при \(B = \RR\)

  \begin{equation*}
    \prob{\xi \in \RR}
    = \int_{-\infty}^{\infty} f_{\xi} (x) \dd x
    = 1
  \end{equation*}
\end{proof}

\begin{lemma}
  \begin{equation*}
    F_{\xi} (x) = \int_{-\infty}^{x} f_{\xi} (x) \dd x
  \end{equation*}
\end{lemma}

\begin{proof} \label{}
  Искомое равенство получается из определения при \(B = \interval{-\infty}{x}\).

  \begin{equation*}
    F_{\xi} (x)
    = \prob{\xi < x} 
    = \int_{-\infty}^{x} f_{\xi} (x) \dd x
  \end{equation*}
\end{proof}

\begin{lemma}
  \(F_{\xi} (x)\)~--- абсолютно непрерывная функция.
\end{lemma}

\begin{proof}
  Это свойство непрерывности интеграла с переменным верхним пределом.
\end{proof}

\begin{lemma}
  \(F_{\xi} (x)\)~--- дифференцируема почти всюду и \(F'_{\xi} (x) = f_{\xi}
  (x)\) для почти всех \(x\).
\end{lemma}

\begin{proof}
  Это следует из теоремы Барроу.
\end{proof}

\begin{lemma}
  \(f_{\xi} (x) \ge 0\)
\end{lemma}

\begin{proof}
  Это справедливо, т.к. функция распределения неубывающая функция.
\end{proof}

\begin{lemma} \label{lem:cont-prob-dot}
  \begin{equation*}
    \forall x_0 \in \RR \given
    \prob{\xi = x_0} = 0
  \end{equation*}
\end{lemma}

\begin{proof}
  Т.к. \(F_{\xi} (x)\)~--- непрерывная функция, т.е. скачка в данной точке нет,
  то по \ref{lem:jump-dot} вероятность попадания в точку равна величине скачка,
  т.е. нулю.
\end{proof}

\begin{lemma}
  \begin{equation*}
    \prob{\alpha \le \xi < \beta}
    = \prob{\alpha \le \xi \le \beta}
    = \prob{\alpha < \xi < \beta}
    = \prob{\alpha < \xi \le \beta}
    F(\beta) - F(\alpha)
  \end{equation*}
\end{lemma}

\begin{proof}
  Это следствие из \ref{lem:cont-prob-dot}.
\end{proof}

\begin{theorem}
  \begin{equation*}
    \begin{rcases}
      \forall x \in \RR \given f(x) \ge 0 \\
      \int_{-\infty}^{\infty} f(x) \dd x = 1
    \end{rcases}
    \implies
    f(x) \text{ это плотность некоторого распределения}
  \end{equation*}
\end{theorem}

\subheader{Числовые характеристики абсолютно непрерывного распределения}

\subsubheader{I.}{Математическое ожидание}

\begin{definition}
  Математическим ожиданием абсолютно непрерывной случайной величины \(\xi\)
  называется число

  \begin{equation*}
    \expected{\xi} = \int_{-\infty}^{\infty} x f_{\xi} (x) \dd x
  \end{equation*}

  при условии, что данный интеграл сходится абсолютно.
\end{definition}

\begin{remark}
  \begin{equation*}
    \int_{-\infty}^{\infty} \abs{x} f_{\xi} (x) \dd x
    \implies
    \text{математическое ожидание не существует}
  \end{equation*}
\end{remark}

\subsubheader{II.}{Дисперсия}

\begin{definition}
  Дисперсией абсолютно непрерывной случайной величины \(\xi\) называется число

  \begin{equation*}
    \variance{\xi}
    = \expected{\prh{\xi - \expected{\xi}}^2}
    = \int_{-\infty}^{\infty} \prh{x - \expected{\xi}}^2 f_{\xi} (x) \dd x
  \end{equation*}

  при условии, что данный интеграл сходится.
\end{definition}

\begin{remark}
  Для вычисления дисперсии удобнее формула 

  \begin{equation*}
    \variance{\xi}
    = \expected{\xi^2} - \prh{\expected{\xi}}^2
    = \int_{-\infty}^{\infty} x^2 f_{\xi} (x) \dd x - \prh{\expected{\xi}}^2
  \end{equation*}
\end{remark}

\subsubheader{III.}{Среднее квадратичное отклонение}

\begin{definition}
  \begin{equation*}
    \stder{\xi} = \sqrt{\variance{\xi}}
  \end{equation*}
\end{definition}

\begin{remark}
  Смысл и свойства этих числовых характеристик полностью идентичны случаю
  дискретной случайной величины.
\end{remark}

\subheader{Другие числовые характеристики}

\subsubheader{IV}{Моменты старших порядков}

\begin{definition}
  Момент \(k\)-ого порядка это

  \begin{equation*}
    m_k
    = \expected{\xi^k}
    = \int_{-\infty}^{\infty} x^k f_{\xi} (x) \dd x
  \end{equation*}
\end{definition}

\begin{definition}
  Центральный момент \(k\)-ого порядка это

  \begin{equation*}
    \mu_k
    = \expected{\prh{\xi - \expected{\xi}}^k} 
    = \int_{-\infty}^{\infty} \prh{x - \expected{\xi}}^k f_{\xi} (x) \dd x
  \end{equation*}
\end{definition}

\subsubheader{V}{Медиана}

\begin{definition}
  Медианой \(\Me\) называется значение случайной величины \(\xi\) такое, что

  \begin{equation*}
    \prob{\xi < \Me} = \prob{\xi > \Me} 
  \end{equation*}
\end{definition}

\subsubheader{VI}{Мода}

\begin{definition}
  Модой \(\Mo\) абсолютно непрерывного распределения называется точка локального
  максимума плотности.
\end{definition}
