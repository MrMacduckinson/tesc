\subsection{%
  Лекция \texttt{23.11.14}.%
}

\subheader{Совместное распределение случайных величин}

\begin{definition}
  Случайным вектором \(\tuple{\xi_1, \dotsc, \xi_n}\) называется упорядоченный
  набор случайных величин, заданных на одном вероятностном пространстве
  \(\tuple{\Omega, \euF, \probP}\).
\end{definition}

Случайный вектор задает отображение \(\tuple{\xi_1, \dotsc, \xi_n} (\omega)
\colon \Omega \to \RR^n\), поэтому его еще называют многомерной случайной
величиной, а соответствующее распределение называют многомерным распределением.

\begin{equation*}
  \prob{B}
  = \prob{\omega \in \Omega \given \tuple{\xi_1, \dotsc, \xi_n} (\omega) \in B}
  \qquad
  B \in \mathcal{B} \prh{\RR^n}
\end{equation*}

Таким образом мы получаем новое вероятностное пространство \(\tuple{\RR^n,
\mathcal{B}, \prob{B}}\).

\subheader{Функция распределения}

\begin{definition}
  Функцией совместного распределения случайных величин \(\xi_1, \dotsc, \xi_n\)
  (или случайного вектора) называется функция

  \begin{equation*}
    F_{\xi_1, \dotsc, \xi_n} (x_1, \dotsc, x_n)
    = \prob{\xi_1 < x_1, \dotsc, \xi_n < x_n} 
  \end{equation*}
\end{definition}

\begin{remark}
  Зная функцию распределения можно найти вероятность попадания случайного
  вектора в любое борелевское множество, а значит функция распределения
  полностью задает распределение.
\end{remark}

\galleryone{01_11_01}{Функция совместного распределения двух случайных величин}

\begin{remark}
  Далее рассматриваем только системы из двух случайных величин.

  \begin{equation*}
     F_{\xi, \eta} (x, y) = \prob{\xi < x, \eta < y}
  \end{equation*}

  Иллюстрация приведена на \figref{01_11_01}.
\end{remark}

\subheader{Свойства функции распределения}

\begin{equation*}
  \begin{aligned}
    0 \le F_{\xi, \eta} (x, y) \le 1
  \\
    F_{\xi, \eta} (x, y) \text{ неубывающая по каждому аргументу}
  \\
    \lim_{x \to -\infty} F_{\xi, \eta} (x, y) = 0
  \\
    \lim_{y \to -\infty} F_{\xi, \eta} (x, y) = 0
  \\
    \lim_{\substack{x \to \infty \\ y \to \infty}} F_{\xi, \eta} (x, y) = 1
  \\
    F_{\xi, \eta} (x, y) \text{ непрерывна слева по каждому аргументу}
  \\
    \lim_{y \to +\infty} F_{\xi, \eta} (x, y) = F_{\xi} (x)
  \\
    \lim_{x \to +\infty} F_{\xi, \eta} (x, y) = F_{\eta} (y)
  \\
    \prob{x_1 \le \xi < x_2, y_1 \le \eta < y_2} =
      F(x_2, y_2) - F(x_1, y_2) - F(x_2, y_1) + F(x_1, y_1)
  \end{aligned}
\end{equation*}

\subheader{Независимость случайных величин}

\begin{definition}
  Случайные величины \(\xi_1, \dotsc, \xi_n\) независимы в совокупности, если

  \begin{equation*}
    \prob{\xi_1 \in B_1, \dotsc, \xi_n \in B_n}
    = \prob{\xi_1 \in B_1} \dotsc \prob{\xi_n \in B_n}
    \qquad
    \forall B_1, \dotsc, B_n \in \mathcal{B} \prh{\RR^n}
  \end{equation*}

  Т.е. случайные величины принимают значения независимо друг от друга.
\end{definition}

\begin{definition}
  Случайные величины \(\xi_1, \dotsc, \xi_n\) попарно независимы, если
  независимы любые две из них.
\end{definition}

\begin{remark}
  Из независимости в совокупности следует попарная независимость. Пусть \(\xi_1,
  \dotsc, \xi_n\) независимы в совокупности, тогда \(\forall i, j\) возьмем
  \(B_k = \RR\), \(k \neq i, k \neq j\). Получим

  \begin{equation*}
    \prob{\xi_i \in B_i, \xi_j \in B_j}
    = 1 \cdot \dotsc \cdot
      \prob{\xi_i \in B_i} \cdot \dotsc \cdot
      \prob{\xi_j \in B_j} \cdot \dotsc \cdot
      1
    = \prob{\xi_i \in B_i} \prob{\xi_j \in B_j}
  \end{equation*}

  Обратное в общем случае неверно
\end{remark}

Т.к. вероятности \(\prob{\xi_1 \in B_1, \dotsc, \xi_n \in B_n}\) полностью
определяются функцией совместного распределения, то получаем эквивалентное
определение:

\begin{definition}
  \(\xi_1, \dotsc, \xi_n\) независимы в совокупности, если

  \begin{equation*}
    F_{\xi_1, \dotsc, \xi_n} (x_1, \dotsc, \xi_n) =
    F_{\xi_1} (x) \dotsc F_{\xi_n} (x)
  \end{equation*}
\end{definition}

\begin{remark}
  В дальнейшем под \quote{независимостью} будем понимать независимость в
  совокупности.
\end{remark}

\subheader{Дискретная система двух случайных величин}

\begin{definition}
  Случайные величины \(\xi\) и \(\eta\) имеют дискретное совместное
  распределение, если случайный вектор \(\tuple{\xi, \eta}\) принимает не более
  чем счетное число значений. Т.е. существует конечный или счетный набор точек
  \(x_i, y_j\) таких, что

  \begin{equation*}
    \forall i, j \given \prob{\xi = x_i, \eta = y_j} > 0
    \qquad
    \sum_{i, j} \prob{\xi = x_i, \eta = y_j} = 1
  \end{equation*}
\end{definition}

Таким образом двумерная дискретная случайная величина задается законом
распределения~--- таблицей вероятностей \(\probP_{i, j} = \prob{\xi = x_i, \eta
= y_j}\).

\begin{table}[h]
  \centering

  \begin{tabular}{c|c|c|c|c}
    \diagbox{\(\xi\)}{\(\eta\)}
               & \(y_1\)     & \(y_2\)     & \(\dotsc\) & \(y_m\)     \\ \hline
    \(x_1\)    & \(p_{1,1}\) & \(p_{1,2}\) & \(\dotsc\) & \(p_{1,m}\) \\ \hline
    \(x_2\)    & \(p_{2,1}\) & \(p_{2,2}\) & \(\dotsc\) & \(p_{2,m}\) \\ \hline
    \(\vdots\) & \(\vdots\)  & \(\vdots\)  & \(\ddots\) & \(\vdots\)  \\ \hline
    \(x_n\)    & \(p_{n,1}\) & \(p_{n,2}\) & \(\dotsc\) & \(p_{n,m}\)
  \end{tabular}
\end{table}

\(x_i\)~--- значения случайной величины \(\xi\), \(y_j\)~--- значения случайной
величины \(\eta\), \(p_{i,j}\)~--- вероятность появления точки \((x_i, y_j)\).
По условию нормировки \(\sum_{i, j} p_{i, j} = 1\).

Зная общий закон распределения можно найти частные законы распределения \(\xi\)
и \(\eta\) по формулам

\begin{equation*}
  p_i = \prob{\xi = x_i} = \sum_{j = 1}^m p_{i, j} 
  \qquad
  p_j = \prob{\eta = y_j} = \sum_{i = 1}^n p_{i, j} 
\end{equation*}

\begin{definition}
  Дискретные случайные величины \(\xi_1, \dotsc, \xi_n\) независимы, если

  \begin{equation*}
    \prob{\xi_1 = x_1, \dotsc \xi_n = x_n} =
    \prob{\xi_1 = x_1} \dotsc \prob{\xi_n = x_n}
    \qquad
    \forall x_i \in \RR
  \end{equation*}
\end{definition}

\begin{remark}
  Т.е. в случае двух случайных величин \(\xi\) и \(\eta\) независимы, если
  \(\forall i, j \given p_{i, j} = p_i q_j\).
\end{remark}

\todo На лекции 23.11.21 был небольшой пример, можно его сюда добавить

\subheader{Абсолютно непрерывная система двух случайных величин}

\begin{definition}
  Случайные величины \(\xi\) и \(\eta\) имеют абсолютно непрерывное совместное
  распределение, если

  \begin{equation*}
    \exists f_{\xi, \eta} (x, y) \given
    \prob{\tuple{\xi, \eta} \in B}
    = \iint_{B} f_{\xi, \eta} (x, y) \dd x \dd y
  \end{equation*}
\end{definition}

\begin{definition}
  Функция \(f_{\xi, \eta} (x, y)\) называется плотностью совместного
  распределения случайных величин \(\xi\) и \(\eta\).
\end{definition}

\begin{remark}
  В \(n\)-мерном случае определение плотности аналогично:

  \begin{equation*}
    \prob{\tuple{\xi_1, \dotsc, \xi_n} \in B}
    = \idotsint_{B} f_{\xi_1, \dotsc, \xi_n} (x_1, \dotsc, x_n)
      \dd x_1 \dotsc \dd x_n
  \end{equation*}
\end{remark}

\subheader{Свойства плотности}

\begin{equation*}
  \begin{aligned}
    f_{\xi, \eta} (x, y) \ge 0
  \\
    \text{Условие нормировки }
    \iint_{\RR^2} f_{\xi, \eta} (x, y) \dd x \dd y = 1
  \\
    F_{\xi, \eta} (x, y)
    = \int_{-\infty}^{x} \int_{-\infty}^{y} f_{\xi, \eta} (x, y) \dd x \dd y
  \\
    f_{\xi, \eta} (x, y)
    = \frac{\partial F_{\xi, \eta} (x, y)}{\partial x \partial y}
  \end{aligned}
\end{equation*}

\begin{lemma}
  Если случайные величины \(\xi\) и \(\eta\) имеют абсолютные непрерывные
  распределения с плотностью \(f_{\xi, \eta} (x, y)\), то маргинальные
  распределения случайных величин \(\xi\) и \(\eta\) также абсолютно непрерывны
  с плотностями

  \begin{equation*}
    f_{\xi} (x) = \int_{-\infty}^{\infty} f_{\xi, \eta} (x, y) \dd y
    \qquad
    f_{\eta} (y) = \int_{-\infty}^{\infty} f_{\xi, \eta} (x, y) \dd x
  \end{equation*}
\end{lemma}

\begin{proof}
  \begin{equation*}
    \begin{rcases}
      \displaystyle F_{\xi} (x)
      = \lim_{y \to \infty} F_{\xi, \eta} (x, y)
      = \int_{-\infty}^{x}
        \prh{\int_{-\infty}^{\infty} f_{\xi, \eta} (x, y) \dd y} \dd x
      \\
      \displaystyle F_{\xi} (x)
      = \int_{-\infty}^{x} f_{\xi} (x) \dd x
    \end{rcases}
    \implies
    f_{\xi} (x) = \int_{-\infty}^{\infty} f_{\xi, \eta} (x, y) \dd y
  \end{equation*}

  Доказательство для \(F_{\eta} (y)\) аналогично.
\end{proof}

\begin{theorem}
  Абсолютно непрерывные случайные величины \(\xi_1, \dotsc, \xi_n\) независимы в
  совокупности тогда и только тогда, когда

  \begin{equation*}
    f_{\xi_1, \dotsc, \xi_n} (x_1, \dotsc, x_n)
    = f_{\xi_1} (x) \dotsc f_{\xi_n} (x)
  \end{equation*}
\end{theorem}

\begin{proof}
  \textit{(только при \(n = 2\))}

  Пусть случайные величины \(\xi\) и \(\eta\) независимы, тогда

  \begin{equation*}
    \begin{rcases}
      \displaystyle F_{\xi, \eta} (x, y)
      = F_{\xi} (x) F_{\eta} (y)
      = \int_{-\infty}^{x} f_{\xi} (x) \dd x
        \int_{-\infty}^{y} f_{\eta} (y) \dd y
      = \int_{-\infty}^{x} \int_{-\infty}^{y}
        f_{\xi} (x) f_{\eta} (y) \dd x \dd y 
    \\
      \displaystyle F_{\xi, \eta} (x, y)
      = \int_{-\infty}^{x} \int_{-\infty}^{y}
        f_{\xi, \eta} (x, y) \dd x \dd y 
    \end{rcases}
    \implies
    f_{\xi} (x) f_{\eta} (y) = f_{\xi, \eta} (x, y)
  \end{equation*}
\end{proof}

\begin{remark}
  Совместное распределение абсолютно непрерывных случайных величин не обязано
  быть абсолютно непрерывным. Оно может быть сингулярным.
\end{remark}

\begin{example}
  Бросаем наугад точку на отрезок прямой \(y = 2 x\), \(0 \le x \le 1\). Пусть
  случайная величина \(\xi \in \evenly{0}{1}\) это первая координата точки, а
  \(\eta \in \evenly{0}{2}\)~--- вторая. Обе случайные величины имеют абсолютно
  непрерывное равномерное распределение, однако случайный вектор \(\tuple{\xi,
  \eta}\) не имеет абсолютно непрерывного распределения, т.к. мера Лебега
  отрезка равна нулю и случайный вектор распределен на несчетном множестве
  нулевой меры. Таким образом имеем сингулярное распределение.
\end{example}

\begin{example}
  Случайному вектору \(\tuple{\xi, \eta}\) сопоставляем его направление.
  Направление задается углом, т.е. значение случайного вектора~--- точки на
  единичной окружности, которая имеет меру ноль.
\end{example}

\subheader{Многомерное равномерное распределение}

\begin{definition}
  Пусть область \(D \subset \RR^n\) это борелевское множество с конечной
  лебеговой мерой \(\lambda (D) > 0\). Случайный вектор \(\tuple{\xi_1, \dotsc,
  \xi_n}\) имеет равномерное распределение в области \(D\), если плотность
  совместного распределения постоянна в области \(D\) и равна нулю вне этой
  области.

  \begin{equation*}
    f_{\xi_1, \dotsc, \xi_n} (x_1, \dotsc, x_n)
    = \begin{cases}
      0, & (x_1, \dotsc, x_n) \notin D \\
      \frac{1}{\lambda (D)}, & (x_1, \dotsc, x_n) \in D
    \end{cases}
  \end{equation*}
\end{definition}

Смысл: случайный вектор \(\tuple{\xi_1, \dotsc, \xi_n}\) это координаты наугад
брошенной точки в области \(D\). В этом и только в этом случае применима формула
геометрической вероятности.

\subheader{Многомерное нормальное распределение (кратко)}

\begin{definition}
  Случайный вектор \(\tuple{\xi_1, \dotsc, \xi_n}\) имеет многомерное нормальное
  распределение, если каждая его координата имеет нормальное распределение (но
  их параметры могут отличаться) и существует плотность совместного
  распределения.
\end{definition}

Многомерное нормальное распределение задается через вектор математических
ожиданий и матрицу ковариаций.

\begin{equation*}
  \expected{\tuple{\xi_1, \dotsc, \xi_n}}
  = \mtxp{\expected{\xi_1} \\ \vdots \\ \expected{\xi_n}}
  \qquad
  \cov{\xi_i}{\xi_j} = \mtxp{
    \variance{\xi_1}   & \cov{\xi_1}{\xi_2} & \dots  & \cov{\xi_1}{\xi_n} \\
    \cov{\xi_1}{\xi_2} & \variance{\xi_2}   & \dots  & \cov{\xi_2}{\xi_n} \\
    \vdots             & \vdots             & \ddots & \vdots             \\
    \cov{\xi_1}{\xi_n} & \cov{\xi_2}{\xi_n} & \dots  & \variance{\xi_n}
  }
\end{equation*}
