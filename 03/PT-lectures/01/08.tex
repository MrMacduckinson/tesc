\subsection{%
  Лекция \texttt{23.10.24}.%
}

\subheader{Стандартное абсолютно непрерывное распределение}

\subsubheader{I}{Равномерное распределение}

\begin{definition}
  Случайная величина \(\xi\) равномерно распределена на отрезке
  \(\segment{a}{b}\) (обозначается \(\xi \in \evenly{a}{b}\)), если ее плотность
  на данном отрезке постоянна.
\end{definition}

\gallerydouble
  {01_08_01}{Плотность равномерного распределения}
  {01_08_02}{Функция распределения}

Учитывая условие нормировки (\figref{01_08_01}) получаем, что плотность \(\xi\)
будет иметь вид

\begin{equation*}
  f_{\xi} (x) = \begin{cases}
    0,               & x < a \\
    \frac{1}{b - a}, & a \le x \le b \\
    0,               & x > b \\
  \end{cases}
\end{equation*}

Найдем функцию распределения (\figref{01_08_02}).

\begin{equation*}
  \begin{aligned}
    F_{\xi} (x) & = \int_{-\infty}^{\infty} f_{\xi} (x) \dd x
  \\
    x < a & \implies
    F_{\xi} (x) = 0
  \\
    a \le x \le b & \implies
    F_{\xi} (x)
    = \int_a^x \frac{\dd x}{b - a}
    = \frac{x - a}{b - a}
  \\
    x > b & \implies
    F_{\xi} (x) = 1
  \\
    F_{\xi} (x) & = \begin{cases}
      0,                   & x < a \\
      \frac{x - a}{b - a}, & a \le x \le b \\
      1,                   & x > b
    \end{cases}
  \end{aligned}
\end{equation*}

Вычислим числовые характеристики.

\begin{equation*}
  \begin{aligned}
    \expected{\xi}
    = \int_{-\infty}^{\infty} x f(x) \dd x
    = \int_a^b \frac{x \dd x}{b - a}
    = \frac{1}{b - a} \cdot \frac{x^2}{2} \Big\vert_a^b
    = \frac{a + b}{2}
  \\
    \expected{\xi^2}
    = \int_{-\infty}^{\infty} x^2 f(x) \dd x
    = \int_a^b \frac{x^2 \dd x}{b - a}
    = \frac{1}{b - a} \cdot \frac{x^3}{3} \Big\vert_a^b
    = \frac{a^2 + a b + b^2}{3}
  \\
    \variance{\xi}
    = \expected{\xi^2} - \prh{\expected{\xi}}^2
    = \frac{a^2 + a b + b^2}{3} - \frac{a^2 + 2 a b + b^2}{4}
    = \frac{(b - a)^2}{12}
  \\
    \stder{\xi}
    = \sqrt{\variance{\xi}}
    = \frac{b - a}{2 \sqrt{3}}
  \end{aligned}
\end{equation*}

Вычислим вероятность попадания в интервал.

\begin{equation*}
  \prob{\alpha < \xi < \beta} = \frac{\beta - \alpha}{b - a}
  \qquad
  \alpha, \beta \in \segment{a}{b}
\end{equation*}

Равномерное распределение возникают в задачах, связанных со временем, а также в
задачах, связанных с генерацией псевдослучайных чисел
(\(\xi \in \evenly{0}{1}\)).

\subsubheader{II}{Показательное (экспоненциальное) распределение}

\begin{definition}
  Случайная величина \(\xi\) имеет показательное распределение с параметром
  \(\alpha > 0\) (обозначается \(\xi \in E_{\alpha}\)), если ее плотность имеет
  вид (\figref{01_08_03}):

  \begin{equation*}
    f_{\xi} (x) = \begin{cases}
      0,                    & x < 0 \\ 
      \alpha e^{-\alpha x}, & x \ge 0
    \end{cases}
  \end{equation*}
\end{definition}

\gallerydouble
  {01_08_03}{Плотность показательного распределения}
  {01_08_04}{Функция распределения}

Найдем функцию распределения (\figref{01_08_04}).

\begin{equation*}
  \begin{aligned}
    F_{\xi} (x) & = \int_{-\infty}^{\infty} f_{\xi} (x) \dd x
  \\
    x < 0 & \implies
    F_{\xi} (x) = 0
  \\
    x \ge 0 & \implies
    F_{\xi} (x)
    = \int_0^x \alpha e^{-\alpha x}
    = -e^{-\alpha x} \Big\vert_0^x
    = 1 - e^{-\alpha x}
  \\i
    F_{\xi} (x) & = \begin{cases}
      0,                   & x < 0  \\
      1 - e^{-\alpha x},   & x \ge 0
    \end{cases}
  \end{aligned}
\end{equation*}

Вычислим числовые характеристики.

\begin{equation*}
  \begin{aligned}
    \expected{\xi}
    = \int_{-\infty}^{\infty} x f(x) \dd x
    = \int_0^{\infty} x \alpha e^{-\alpha x} \dd x
    = \mtxb{
      u = x & \dd u = \dd x \\
      \dd v = \alpha e^{-\alpha x} \dd x & v = -e^{-\alpha x}
    }
  \\
    = -x e^{-\alpha x} \Big\vert_0^{\infty}
      + \int_0^{\infty} e^{-\alpha x} \dd x
    = -\lim_{x \to \infty} \frac{x}{e^{\alpha x}}
      - \frac{1}{\alpha} e^{-\alpha x} \Big\vert_0^{\infty}
    = -\frac{1}{\alpha} \prh{\lim_{x \to \infty} e^{-\alpha x} - e^{0}}
    = \frac{1}{\alpha}
  \\ \\
    \expected{\xi^2}
    = \int_{-\infty}^{\infty} x^2 f(x) \dd x
    = \int_0^{\infty} x^2 \alpha e^{-\alpha x} \dd x
    = \mtxb{
      u = x^2 & \dd u = 2 x \dd x \\
      \dd v = \alpha e^{-\alpha x} \dd x & v = -e^{-\alpha x}
    }
  \\
    = -x^2 e^{-\alpha x} \Big\vert_0^{\infty}
      + 2 \int_0^{\infty} e^{-\alpha x} \dd x
    = \frac{2}{\alpha} \int_0^{\infty} x \alpha e^{-\alpha x} \dd x
    = \frac{2}{\alpha} \expected{\xi}
    = \frac{2}{\alpha^2}
  \\ \\
    \variance{\xi}
    = \expected{\xi^2} - \prh{\expected{\xi}}^2
    = \frac{2}{\alpha^2} - \prh{\frac{1}{\alpha}}^2
    = \frac{1}{\alpha^2}
  \\
    \stder{\xi}
    = \sqrt{\variance{\xi}}
    = \frac{1}{\alpha}
  \end{aligned}
\end{equation*}

Используя функцию распределения, вычислим вероятность попадания в интервал.

\begin{equation*}
  \prob{\alpha < \xi < \beta} = e^{-a \alpha} - e^{-b \alpha}
\end{equation*}

Из непрерывных распределений только показательное обладает свойством нестарения.

\begin{theorem}
  \begin{equation*}
    \prob{\xi > x + y \given \xi > x} = \prob{\xi > y}
  \end{equation*}
\end{theorem}

\begin{proof}
  \begin{equation*}
    \begin{aligned}
      \prob{\xi > x + y \given \xi > x}
    = \frac{\prob{\xi > x + y, \xi > x}}{\prob{\xi > x}}
    = \frac{1 - \prob{\xi < x + y}}{1 - \prob{\xi < x}}
    = \frac{1 - F(x + y)}{1 - F(x)}
    = \frac{1 - 1 + e^{-\alpha (x + y)}}{1 - 1 + e^{-\alpha x}}
  \\
    = e^{-\alpha y}
    = 1 - (1 - e^{-\alpha y})
    = 1 - F(y)
    = \prob{\xi > y}
    \end{aligned}
  \end{equation*}  
\end{proof}

Показательное распределение возникает, (например) если нужно рассчитать время
работы надежного прибора до поломки. Также показательное распределение можно
трактовать как время между появлениями двух соседних редких событий. Возникает в
системах массового обслуживания, теории надежности, теории катастроф и ряде
областей физики.

\subsubheader{III}{Нормальное (Гауссовское) распределение}

\begin{definition}
  Случайное величина \(\xi\) имеет нормальное распределение с параметрами \(a\)
  и \(\sigma^2 > 0\) (обозначается \(\xi \in N(a; \sigma^2)\)), если ее
  плотность имеет вид

  \begin{equation*}
    f(x) = \frac{1}{\sigma \sqrt{2 \pi}} e^{-\frac{(x - a)^2}{2 \sigma^2}}
    \qquad
    x \in \interval{-\infty}{\infty}
  \end{equation*}
\end{definition}

\gallerydouble
  {01_08_05}{Нормальное распределение}
  {01_08_06}{Функция распределения}

Смысл параметров распределения.

\begin{equation*}
  \expected{\xi} = a
  \qquad
  \variance{\xi} = \sigma^2
  \qquad
  \stder{\xi} = \sigma
\end{equation*}

Функция распределения будет иметь вид

\begin{equation*}
  F_{\xi} (x) = \int_{-\infty}^{x} f_{\xi} (t) \dd t
\end{equation*}

\subheader{Стандартное нормальное распределение}

\begin{definition}
  Стандартным нормальным распределением называется нормальное распределение с
  параметрами \(a = 0\), \(\sigma^2 = 1\).
\end{definition}

Получаем функцию плотности (функцию Гаусса, \figref{01_08_07}):

\begin{equation*}
  \phi(x) = \frac{1}{\sqrt{2 \pi}} e^{\frac{-x^2}{2}}
\end{equation*}

\galleryone{01_08_07}{Функция Гаусса}

Функция распределения \(F_0 (x)\) будет иметь вид
\begin{equation*}
  F_0 (x) = \frac{1}{\sqrt{2 \pi}} \int_{-\infty}^{x} e^{-z^2 / 2} \dd z
\end{equation*}

\begin{remark}
  По формуле Пуассона

  \begin{equation*}
    \int_{-\infty}^{\infty} e^{-z^2 / 2} \dd z = \sqrt{2 \pi}
  \end{equation*}

  Значит условие нормировки выполнено \(\implies \phi(x)\) это действительно
  плотность некоторого распределения.
\end{remark}

\begin{remark}
  \begin{equation*}
    \begin{aligned}
      F_0 (x)
      = \frac{1}{\sqrt{2 \pi}} \int_{-\infty}^0 e^{-z^2 / 2} \dd z
        + \frac{1}{\sqrt{2 \pi}} \int_0^x e^{-z^2 / 2} \dd z
      = \frac{1}{2} + \Phi(x)
    \\
      \Phi(x) = \frac{1}{\sqrt{2 \pi}} \int_0^x e^{-z^2 / 2} \dd z
    \end{aligned}
  \end{equation*}

  где \(\Phi(x)\) это функция Лапласа.
\end{remark}

Вычислим характеристики.

\begin{equation*}
  \begin{aligned}
    \expected{\xi}
    = \int_{-\infty}^{\infty} x \phi(x) \dd x
    = \frac{1}{\sqrt{2 \pi}} \int_{-\infty}^{\infty} x e^{-x^2 / 2} \dd x
    = \frac{1}{\sqrt{2 \pi}}
      \int_{-\infty}^{\infty} e^{-x^2 / 2} \dd \prh{\frac{x^2}{2}}
  \\
    = -\frac{1}{\sqrt{2 \pi}} e^{-x^2 / 2} \Big\vert_{-\infty}^{\infty}
    = -\frac{1}{\sqrt{2 \pi}} \prh{
      \lim_{x \to \infty} e^{-x^2 / 2} - \lim_{x \to -\infty} e^{-x^2 / 2}
    }
    = 0
  \\ \\
    \variance{\xi}
    = \int_{-\infty}^{\infty} x^2 \phi(x) \dd x
      - \under{\prh{\expected{\xi}}^2}{0}
    = \frac{1}{\sqrt{2 \pi}} \int_{-\infty}^{\infty} x^2 e^{-x^2 / 2} \dd x
    = \mtxb{
      u = x & \dd u = \dd x \\
      \dd v = x e^{-x^2 / 2} \dd x & v = -e^{-x^2 / 2}
    }
  \\
    = \frac{1}{\sqrt{2 \pi}} \prh{
      \under{-x e^{-x^2 / 2} \Big\vert_{-\infty}^{\infty}}{0}
      + \int_{-\infty}^{\infty} e^{-x^2 / 2} \dd x
    }
    = \frac{1}{\sqrt{2 \pi}} \int_{-\infty}^{\infty} e^{-x^2 / 2} \dd x
    = 1
  \end{aligned}
\end{equation*}

\subheader{Связь между нормальным и стандартным нормальным распределением}

\begin{lemma}
  \begin{equation*}
    \xi \in N(a; \sigma^2)
    \implies
    F_{\xi} (x) = F_0 \prh{\frac{x - a}{\sigma}}
  \end{equation*}
\end{lemma}

\begin{proof}
  \begin{equation*}
    \begin{aligned}
      F_{\xi} (x)
      = \frac{1}{\sigma \sqrt{2 \pi}}
          \int_{-\infty}^{x} e^{-\frac{(t - a)^2}{2 \sigma^2}}\dd t
      = \mtxb{
        z = \frac{t - a}{\sigma} & t = \sigma z + a & \dd t = \sigma \dd z \\
        z(-\infty) = -\infty & z(x) = \frac{x - a}{\sigma}
      }
    \\
      = \frac{1}{\sigma \sqrt{2 \pi}}
        \int_{-\infty}^{\frac{x - a}{\sigma}} e^{-z^2 / 2} \sigma \dd z
      = \frac{1}{\sqrt{2 \pi}}
        \int_{-\infty}^{\frac{x - a}{\sigma}} e^{-z^2 / 2} \dd z
      = F_0 \prh{\frac{x - a}{\sigma}}
    \end{aligned}
  \end{equation*}
\end{proof}

\begin{lemma}
  \begin{equation*}
    \xi \in N (a; \sigma^2)
    \implies
    \eta = \frac{\xi - a}{\sigma} \in N(0; 1)
  \end{equation*}
\end{lemma}

\begin{proof}
  \begin{equation*}
    F_{\eta} (x)
    = \prob{\eta < x}
    = \prob{\frac{\xi - a}{\sigma} < x}
    = \prob{\xi < \sigma x + a}
    = F_{\xi} (\sigma x + a)
    = F_0 \prh{\frac{\sigma x + a - a}{\sigma}}
    = F_0 (x)
    \implies
    \eta \in N(0; 1)
  \end{equation*}
\end{proof}

\begin{lemma}
  \begin{equation*}
    \xi \in N(a; \sigma^2)
    \implies
    \begin{cases}
      \expected{\xi} = a \\
      \variance{\xi} = \sigma^2
    \end{cases}
  \end{equation*}
\end{lemma}

\begin{proof}
  \begin{equation*}
    \begin{aligned}
      \eta = \frac{\xi - a}{\sigma} \in N(0; 1)
      \implies
      \begin{cases}
        \expected{\eta} = 0 \\
        \variance{\eta} = 1
      \end{cases}
    \\
      \xi = \sigma \eta + a
      \implies
      \begin{cases}
        \expected{\xi} = \sigma \under{\expected{\eta}}{0} + a = a \\
        \variance{\xi} = \sigma^2 \under{\variance{\eta}}{1} = \sigma^2
      \end{cases}
    \end{aligned}
  \end{equation*}
\end{proof}

\begin{lemma}
  \begin{equation*}
    \prob{\alpha < \xi < \beta}
    = \Phi \prh{\frac{\beta - a}{\sigma}}
      - \Phi \prh{\frac{\alpha - a}{\sigma}}
  \end{equation*}
\end{lemma}

\begin{lemma}
  \begin{equation*}
    \prob{\alpha < \xi < \beta}
    = F_{\xi} (\beta) - F_{\xi} (\alpha)
    = F_0 \prh{\frac{\beta - a}{\sigma}}
      - F_0 \prh{\frac{\alpha - a}{\sigma}}
    = \Phi \prh{\frac{\beta - a}{\sigma}}
      - \Phi \prh{\frac{\alpha - a}{\sigma}}
  \end{equation*}
\end{lemma}

\begin{lemma} \label{lem:diff-from-expected}
  Вероятность отклонения нормальной случайной величины от ее среднего значения
  (или попадания в симметричный интервал относительно среднего) равна

  \begin{equation*}
    \prob{\abs{\xi - a} < t} = 2 \Phi \prh{\frac{t}{\sigma}}
  \end{equation*}
\end{lemma}

\begin{proof}
  \begin{equation*}
    \prob{\abs{\xi - a} < t}
    = \prob{a - t < \xi < a + t}
    = \Phi \prh{\frac{a + t - a}{\sigma}} - \Phi \prh{\frac{a - t - a}{\sigma}}
    = 2 \Phi \prh{\frac{t}{\sigma}}
  \end{equation*}
\end{proof}

\begin{remark}
  Если \(\Phi(x)\) заменить на \(F_0 (x)\), то формула
  \ref{lem:diff-from-expected} приобретет вид

  \begin{equation*}
    \prob{\abs{\xi - a} < t} = 2 F_0 \prh{\frac{t}{\sigma}} - 1
  \end{equation*}
\end{remark}

\begin{lemma}[Правило \quote{трех сигм}]
  \begin{equation*}
    \prob{\abs{\xi - a} < 3 \sigma} \approx 0.9973
  \end{equation*}
\end{lemma}

\begin{proof}
  \begin{equation*}
    \prob{\abs{\xi - a} < 3 \sigma}
    = 2 \Phi \prh{\frac{3 \sigma}{\sigma}}
    = 2 \Phi (3)
    \approx 0.9973
  \end{equation*}
\end{proof}

\begin{remark}
  Как правило, при нормальном распределении случайной величины, мы почти
  гарантированно попадаем в интервал \(\segment{a - 3 \sigma}{a + 3 \sigma}\).
\end{remark}

\begin{lemma}[Устойчивость относительно суммирования]
  Если \(\xi_1\) и \(\xi_2\) независимые случайные величины, то

  \begin{equation*}
    \begin{rcases}
      \xi_1 \in N(a_1; \sigma_1^2) \\
      \xi_2 \in N(a_2; \sigma_2^2)
    \end{rcases}
    \implies
    \xi_1 + \xi_2 \in N(a_1 + a_2; \sigma_1^2 + \sigma_2^2)
  \end{equation*}
\end{lemma}

\subheader{Коэффициенты асимметрии и эксцесса}

\begin{definition}
  Асимметрией распределения называется число

  \begin{equation*}
    A_S
    = \expected{\prh{\frac{\xi - a}{\sigma}}^3}
    = \frac{\mu_3}{\sigma^3}
  \end{equation*}
\end{definition}

\begin{remark}
  Если распределение симметрично относительно точки \(a\), то \(A_S = 0\). Если
  \(A_S > 0\), то график плотности имеет более крутой спуск слева и наоборот.
\end{remark}

\begin{definition}
  Эксцессом распределения называется число

  \begin{equation*}
    E_k
    = \expected{\prh{\frac{\xi - a}{\sigma}}^4} - 3
    = \frac{\mu^4}{\sigma^4} - 3
  \end{equation*}
\end{definition}

\begin{remark}
  Если случайная величина имеет нормальное распределение \(E_k = 0\). При \(E_k
  > 0\) имеет более \quote{острую} вершину, чем у нормального распределения, а
  при \(E_k < 0\)~--- более \quote{тупую}.
\end{remark}

\subheader{Стандартизация случайной величины}

\begin{definition}
  Пусть имеется случайная величина \(\xi\). Соответствующей ей стандартной
  случайной величиной называется случайная величина

  \begin{equation*}
    \eta = \frac{\xi - \expected{\xi}}{\sigma_{\xi}}
  \end{equation*}
\end{definition}

\begin{lemma}
  \begin{equation*}
    \expected{\eta} = 0
    \qquad
    \variance{\eta} = 1
  \end{equation*}
\end{lemma}

\begin{proof}
  \begin{equation*}
    \begin{aligned}
      \expected{\eta}
      = \expected{\frac{\xi - \expected{\xi}}{\xi}}
      = \frac{1}{\sigma_{\xi}} \prh{\expected{\xi} - \expected{\xi}}
      = 0
    \\
      \variance{\eta}
      = \variance{\frac{\xi - \expected{\xi}}{\sigma_{\xi}}}
      = \frac{1}{\sigma_{\xi}^2} \variance{\xi - \expected{\xi}}
      = \frac{1}{\sigma_{\xi}^2} \variance{\xi}
      = \frac{1}{\sigma_{\xi}^2} \sigma_{\xi}^2
      = 1
    \end{aligned}
  \end{equation*}
\end{proof}

\begin{remark}
  Стандартизованная случайная величина не зависит от единиц измерения.
\end{remark}

\begin{remark}
  При операции стандартизации в общем случае тип распределения может поменяться.
\end{remark}