\subsection{%
  Лекция \texttt{23.10.03}.%
}

\subheader{Схемы испытаний и соответствующие распределения}

Обозначим \(n\)~--- число испытаний, \(p\)~--- вероятность успеха при одном
испытании, \(q = 1 - p\)~--- вероятность неудачи при одном испытании.

\subsubheader{Схема I.}{Схема Бернулли}

Пусть \(Y_n\)~--- число успехов при \(n\) испытаниях. По формуле Бернулли

\begin{equation*}
  \probP_n (Y_n = k) = \comb{n}{k} p^k q^{n - k}
  \qquad
  0 \le k \le n
\end{equation*}

\begin{definition}
  Соответствие \(k \to \comb{n}{k} p^k q^{n - k}\), \(0 \le k \le n\) называется
  биномиальным распределением вероятностей и обозначается \(\bindst{n}{p}\) или
  \(\bindstB \prh{n, p}\).
\end{definition}

\subsubheader{Схема II.}{Схема до первого успешного испытания}

Пусть проводится бесконечная серия независимых испытаний, которая заканчивается
после первого успешного испытания под номером \(\tau\).

\begin{lemma}
  \begin{equation*}
    \prob{\tau = k} = q^{k - 1} p
    \qquad
    k = 0, 1, \dotsc
  \end{equation*}
\end{lemma}

\begin{proof}
  \begin{equation*}
    \prob{\tau = k}
    = \prob{\under{N \dotsc N}{k - 1} Y} 
    = q^{k - 1} p
  \end{equation*}
\end{proof}

\begin{definition}
  Соответствие \(k \to q^{k - 1} p\), \(k \in \NN\) называется геометрическим
  распределением и обозначается \(\geodst{p}\).
\end{definition}

Геометрическое распределение обладает свойством \quote{не старения} или
\quote{отсутствия последействия}.

\begin{theorem}
  \begin{equation*}
    \prob{\tau = k} = q^{k - 1} p, k \in \NN
    \implies
    \forall n, k \ge 0 \given
    \condprob{\tau > n + k}{\tau > n} = \prob{\tau > k}
  \end{equation*}
\end{theorem}

\begin{proof}
  Заметим, что \(\prob{\tau > m} = q^m\), т.к. нас интересуют ситуации, в
  которых каким минимум первые \(m\) испытаний были неудачными. Применим
  определение и получим

  \begin{equation*}
    \condprob{\tau > n + k}{\tau > n}
    = \frac{\prob{\tau > n + k, \tau > n} }{\prob{\tau > n}}
    = \frac{q^{n + k}}{q^n}
    = q^k
    = \prob{\tau > k} 
  \end{equation*}
\end{proof}

\subsubheader{Схема III.}{Испытания с несколькими исходами (полиномиальная
схема)}

Пусть при \(n\) независимых испытаниях могут произойти \(m\) несовместных
исходов, вероятности которых равны \(p_i\) при одном отдельном испытании.

\begin{theorem}
  Вероятность того, что при \(n\) испытаниях первый исход произойдет \(n_1\)
  раз, второй~--- \(n_2\) раз, \(\dotsc\), \(m\)-ый~--- \(n_m\) раз равна

  \begin{equation*}
    \probP_n (n_1, \dotsc, n_m)
    = \frac{n!}{n_1! \cdot \dotsc n_m!}
      \cdot p_1^{n_1} \cdot \dotsc \cdot p_m^{n_m}
  \end{equation*}
\end{theorem}

\begin{proof}
  Рассмотрим один элементарный исход благоприятный данному событию.

  \begin{equation*}
    A_1
    = \under{1 \dotsc 1}{n_1} \under{2 \dotsc 2}{n_2}
      \dotsc \under{m \dotsc m}{n_m}
    = p_1^{n_1} \cdot \dotsc \cdot p_m^{n_m}
  \end{equation*}

  Остальные благоприятные исходы имеют те же вероятности и отличаются лишь
  расстановкой \(i\)-ых исходов по \(n\) местам. Число таких исходов равно

  \begin{equation*}
    \comb{n}{n_1} \cdot \comb{n - n_1}{n_2} \dotsc \cdot \comb{n_m}{n_m}
    = \under{1 \dotsc 1}{n_1} \under{2 \dotsc 2}{n_2}
      \dotsc \under{m \dotsc m}{n_m}
  \end{equation*}

  Это формула перестановок с повторениями. В итоге получаем требуемую формулу.
\end{proof}

\begin{remark}
  При \(n = 2\) доказанная формула превращается в уже рассмотренную формулу
  Бернулли.
\end{remark}

\begin{example}
  Два одинаковых по силе шахматиста играют матч из шести партий. Вероятность
  ничьи при одной партии \(0.5\). Какова вероятность того, что второй игрок
  выиграет две партии, а еще три сведет к ничьей.

  \solution{} Обозначим \(n = 6\), \(m = 3\) (победа, поражение, ничья), \(p_3 =
  0.5\), \(p_1 = p_2 = 0.25\). По формуле получаем

  \begin{equation*}
    \probP_6 (1, 2, 3)
    = \frac{6!}{1! \cdot 2! \cdot 3!}
      \cdot \prh{\frac{1}{4}}^1
      \cdot \prh{\frac{1}{4}}^2
      \cdot \prh{\frac{1}{2}}^3
    = \frac{15}{128}
    \approx 0.12
  \end{equation*}
\end{example}

\subsubheader{Схема IV.}{\quote{Урновая} схема}

В урне \(N\) шаров, из которых \(K\)~--- белые, \(N - K\)~--- черные. Выбираем
\(n\) шаров без учета порядка, обозначим \(k\)~--- число белых шаров в выборке.

\subsubheader{Схема IV. (a)}{С возвратом}

Пусть после вытаскивания шара мы сразу же возвращаем его в урну. Вероятность
выбрать белый шар не меняется и равна \(p = \frac{K}{N}\). Таким образом
получается обычная схема Бернулли.

\begin{equation*}
  \probP_n (k)
  = \comb{n}{k} \cdot \prh{\frac{K}{N}}^k \cdot \prh{1 - \frac{K}{N}}^{n - k}
\end{equation*}

\subsubheader{Схема IV. (b)}{Без возврата}

\begin{equation*}
  \probP_{N, K} (n, k)
  = \frac{\comb{K}{k} \cdot \comb{N - K}{n - k}}{\comb{N}{n}}
  \qquad 0 \le k \le K
\end{equation*}

\begin{definition}
  Соответствие \(\display{k \to \frac{\comb{K}{k} \cdot \comb{N - K}{n -
  k}}{\comb{N}{n}}}\), \(0 \le k \le K\) называется гипергеометрическим
  распределением.
\end{definition}

Интуитивно ясно, что если число шаров \(N\) велико, а \(n\)~--- нет, то
извлечение без возврата белых шаров не сильно влияет на пропорцию шаров в урне и
поэтому результате по схеме (b) будут стремиться к схеме (a), а
гипергеометрическое распределение будет приближаться к биномиальному.

\begin{lemma} \label{lem:comb-equiv}
  \begin{equation*}
    \comb{K}{k} \sim \frac{K^k}{k!}
    \qquad
    K \to \infty, k = const
  \end{equation*}
\end{lemma}

\begin{proof}
  \begin{equation*}
    \begin{aligned}
      \comb{K}{k}
     & = \frac{K!}{k! (K - k)!}
    \\
      & = \frac{K \cdot (K - 1) \cdot \dotsc \cdot (K - k + 1)}{k!}
    \\
      & = \frac{K \cdot (K - 1) \cdot \dotsc \cdot (K - k + 1)}{K^k}
      \cdot \frac{K^k}{k!}
    \\
      & = \under{
        \prh{1 - \frac{1}{K}}
        \prh{1 - \frac{2}{K}}
        \cdot \dotsc \cdot
        \prh{1 - \frac{k - 1}{K}}
      }{\to 1} \cdot \frac{K^k}{k!}
      \sim \frac{K^k}{k!}
    \end{aligned}
  \end{equation*}
 
\end{proof}

\begin{theorem}
  Если \(N \to \infty\) и \(K \to \infty\), таким образом что \(\frac{K}{N} \to
  p \in \interval{0}{1}\), \(n\) и \(0 \le k \le n\) фиксированны, то

  \begin{equation*}
    \probP_{N, K} (n, k)
    = \frac{\comb{K}{k} \cdot \comb{N - K}{n - k}}{\comb{N}{n}}
    \Rarr{n \to \infty}
    \comb{n}{k} \cdot p^k \cdot \prh{1 - p}^{n - k}
  \end{equation*}
\end{theorem}

\begin{proof}
  К каждому сочетанию применим лемму \ref{lem:comb-equiv}.

  \begin{equation*}
    \frac{\comb{K}{k} \cdot \comb{N - K}{n - k}}{\comb{N}{n}}
    \sim \frac{K^k}{k!} \cdot \frac{(N - K)^{n - k}}{(n - k)!}
      \cdot \frac{n!}{N^n}
    = \comb{k}{n} \cdot \frac{K^k \cdot (N - K)^{n - k}}{N^n}
    = \comb{k}{n} \cdot \frac{K^k}{N^k} \cdot \frac{(N - K)^{n - k}}{N^{n - k}}
    = \comb{k}{n} \prh{\frac{K}{N}}^k \cdot \prh{1 - \frac{K}{N}}^{n - k}
  \end{equation*}

  Используя обозначение \(p = \frac{K}{N}\) получаем искомую формулу.
\end{proof}

\subsubheader{Схема V.}{Схема Пуассона}

Вероятность успеха \(p_n\) зависит от числа испытаний \(n\) таким образом, что
их произведение \(n p_n \approx \lambda = const\). Под \(\lambda\) можно
понимать интенсивность появления редких событий в заданную единицу времени.

\begin{theorem}[Формула Пуассона]
  Пусть \(n \to \infty\), тогда \(p_n \to 0\) так, что \(n \cdot p_n \to \lambda
  > 0\). Тогда вероятность \(k\) успехов при \(n\) испытаниях будет равна

  \begin{equation*}
    \prob{Y_n = k}
    = \comb{k}{n} \cdot p_n^k \cdot (1 - p_n)^{n - k}
  \Rarr{n \to \infty}
    \frac{\lambda^k}{k!} e^{-\lambda}
  \end{equation*}
\end{theorem}

\begin{proof}
  Положим \(\lambda_n = n \cdot p_n\), тогда \(\lambda_n \to \lambda\), \(p_n =
  \frac{\lambda_n}{n}\). Имеем

  \begin{equation*}
    \prob{Y_n = k}
    = \comb{k}{n} \cdot p_n^k \cdot (1 - p_n)^{n - k}
    = \comb{k}{n} \cdot \prh{\frac{\lambda_n}{n}}^k
      \cdot \prh{1 - \frac{\lambda_n}{n}} ^{n - k}
  \end{equation*}

  Далее применим лемму \ref{lem:comb-equiv}.

  \begin{equation*}
    \prob{Y_n = k}
    \Rarr{n \to \infty}
    \frac{n^k}{k!} \cdot \frac{\lambda_n^k}{n^k}
      \cdot \prh{1 - \frac{\lambda_n}{n}}^{n - k}
    = \frac{\lambda_n^k}{k!} \cdot \prh{1 - \frac{\lambda_n}{n}}^n
      \cdot \under{\prh{1 - \frac{\lambda_n}{n}}^{-k}}{\to 1}
  \end{equation*}

  Т.к. \(\frac{\lambda_n}{n} = p_n \to 0\) при \(n \to \infty\), а \(k =
  const\), то вторая скобка стремится к единице. Первую же скобку в пределе
  можно раскрыть с помощью второго замечательного предела. Итого

  \begin{equation*}
    \begin{aligned}
      \prh{1 - \frac{\lambda_n}{n}}^n
      = \prh{1 - \frac{\lambda_n}{n}}^{\frac{-n}{\lambda_n} \cdot (-\lambda_n)}
      \Rarr{n \to \infty}
      e^{-\lambda_n}
    \\
      \prob{Y_n = k}
      \Rarr{n \to \infty}
      \frac{\lambda_n^k}{k!} \cdot e^{-\lambda_n}
      \Rarr{n \to \infty}
      \frac{\lambda^k}{k!} \cdot e^{-\lambda}
    \end{aligned}
  \end{equation*}
\end{proof}

\begin{theorem}[Оценка погрешности в формуле Пуассона] \label{thr:P-err}
  Пусть \(Y_n\) это число успехов при \(n\) испытаниях в схеме Бернулли с
  вероятностью успеха \(p\). Обозначим \(\lambda = n p\) и \(A \subset \set{0,
  1, \dots, n}\) произвольное подмножество целых чисел. Тогда

  \begin{equation*}
    \abs{\prob{Y_n = k} - \sum_{k \in A} \frac{\lambda^k}{k!} \cdot
      e^{-\lambda}} \le \min \prh{p, \lambda p}
  \end{equation*}
\end{theorem}

\begin{definition}
  Соответствие \(\display{k \to \frac{\lambda^k}{k!} \cdot e^{-\lambda}}\), \(k
  \in \NN_0\) называется распределением Пуассона с параметром \(\lambda > 0\) и
  обозначается \(\Pi_{\lambda}\).
\end{definition}

\begin{example}
  Прибор состоит из \(1000\) элементов. Вероятность отказа каждого элемента
  \(0.001\). Какова вероятность отказа больше двух элементов?

  \solution{} Обозначим \(n = 1000\), \(p = 0.001\), \(\lambda = n p = 1\), \(k
  > 2\). Применим формулу Пуассона.

  \begin{equation*}
    \prob{k > 2}
    = 1 - \prob{k \le 2}
    = 1 - \prob{0} - \prob{1} - \prob{2}
    = 1 - \frac{1^0}{0!} \cdot e^{-1}
      - \frac{1^1}{1!} \cdot e^{-1}
      - \frac{1^2}{2!} \cdot e^{-1}
    = 1 - e^{-1} \cdot \prh{1 + 1 + \frac{1}{2}}
    \approx 0.0803
  \end{equation*}

  Применим \ref{thr:P-err}, чтобы оценить погрешность.

  \begin{equation*}
    \epsilon \le \min \prh{p, \lambda p} = 0.001
  \end{equation*}
\end{example}
