\subsection{%
  Лекция \texttt{23.12.05}.%
}

\subheader{Характеристические функции}

Пусть \(\xi + i \eta\)~--- комплексная случайная величина, причем \(\xi\) и
\(\eta\) это обычные случайные величины с конечным первым моментом.

\begin{definition}
  Математическое ожидание комплекснозначной случайной величины определим как

  \begin{equation*}
    \expected{\xi + i \eta} = \expected{\xi} + i \expected{\eta}
  \end{equation*}
\end{definition}

\begin{definition}
  Характеристической функцией случайной величины \(\xi\) называется функция

  \begin{equation*}
    \phi_{\xi} (t) = \expected{e^{i \xi t}}
    \qquad
    t \in \RR
  \end{equation*}
\end{definition}

\begin{lemma}
  Характеристическая функция существует для любой случайной величины \(\xi\),
  причем \(\abs{\phi_{\xi} (t)} \le 1\).  
\end{lemma}

\begin{proof}
  Заметим, что

  \begin{equation*} \label{eq:char-func-lim-1} \tag{1}
    \variance{\xi} = \expected{\xi^2} - \prh{\expected{\xi}}^2 \ge 0
    \implies
    \prh{\expected{\xi}}^2 \le \expected{\xi^2}
  \end{equation*}

  Оценим модуль характеристической функции.

  \begin{equation*} \label{eq:char-func-lim-2} \tag{2}
    \begin{aligned}
      \abs{\phi_{\xi} (t)}^2
      & = \abs{\expected{e^{i \xi t}}}^2
    \\
      & = \abs{\expected{\cos (\xi t) + i \sin (\xi t)}}^2
    \\
      & = \abs{\expected{\cos (\xi t)} + i \expected{\sin (\xi t)}}^2
    \\
      & = \abs{\expected{\cos (\xi t)}}^2 + \abs{\expected{\sin (\xi t)}}^2
    \end{aligned}
  \end{equation*}

  Применим к полученному в \eqref{eq:char-func-lim-2} неравенство из
  \eqref{eq:char-func-lim-1}, получим

  \begin{equation*}
    \abs{\phi_{\xi} (t)}^2
    \le \expected{\cos^2 (\xi t)} + \expected{\sin^2 (\xi t)}
    = \expected{\cos^2 (\xi t) + \sin^2 (\xi t)}
    = \expected{1}
    = 1
  \end{equation*}
\end{proof}

\begin{lemma} \label{lem:char-func-lin-transform}
  Пусть \(\phi_{\xi} (t)\) это характеристическая функция случайной величины
  \(\xi\). Тогда характеристическая функция случайной величины \(\eta = a + b
  \xi\) будет иметь вид

  \begin{equation*}
    \phi_{a + b \xi} (t) = e^{i t a} \cdot \phi_{\xi} (b t)
  \end{equation*}
\end{lemma}

\begin{proof}
  \begin{equation*}
    \phi_{a + b \xi} (t)
    = \expected{e^{i (a + b \xi) t}}
    = \expected{e^{i a t} \cdot e^{i b \xi t}}
    = e^{i a t} \expected{e^{i (b t) \xi}}
    = e^{i t a} \cdot \phi_{\xi} (b t)
  \end{equation*}  
\end{proof}

\begin{lemma} \label{lem:char-func-sum-ind}
  Характеристическая функция суммы независимых случайных величин равна
  произведению их характеристических функций.
\end{lemma}

\begin{proof}
  Пусть случайные величины \(\xi\) и \(\eta\) независимы, тогда

  \begin{equation*}
    \phi_{\xi + \eta} (t)
    = \expected{e^{i (\xi + \eta) t}}
    = \expected{e^{i \xi t} \cdot e^{i \eta t}}
    = \expected{e^{i \xi t}} \cdot \expected{e^{i \eta t}}
    = \phi_{\xi} (t) \phi_{\eta} (t)
  \end{equation*}
\end{proof}

\begin{lemma} \label{lem:char-func-series}
  Пусть \(\expected{\xi^k} < \infty\). Тогда

  \begin{equation*}
    \phi_{\xi} (t)
    = 1 + i t \cdot \expected{\xi} - \frac{t^2}{2} \expected{\xi^2}
      + \dotsc + \frac{i^k \expected{\xi^k}}{k!} t^k
      + \smallo \prh{\abs{t}^k}
  \end{equation*}
\end{lemma}

\begin{proof}
  Разложим экспоненту в ряд Маклорена и воспользуемся свойствами математического
  ожидания.

  \begin{equation*}
    \phi_{\xi} (t)
    = \expected{e^{i \xi t}}
    = \expected{
      1 + i \xi t + \frac{(i \xi t)^2}{2!}
      + \dotsc + \frac{(i \xi t)^k}{k!}
      + \smallo \prh{\abs{t}^k}
    }
    = 1 + i t \cdot \expected{\xi} - \frac{t^2}{2} \expected{\xi^2}
      + \dotsc + \frac{i^k \expected{\xi^k}}{k!} t^k
      + \smallo \prh{\abs{t}^k}
  \end{equation*}
\end{proof}

\begin{lemma}
  Пусть \(\expected{\xi^k} < \infty\). Тогда характеристическая функция
  непрерывно дифференцируема \(k\) раз, причем

  \begin{equation*}
    \phi_{\xi}^{(k)} (0) = i^k \expected{\xi^k}
  \end{equation*}
\end{lemma}

\begin{proof}
  Это следует из существования \(k\)-ого члена разложения характеристической
  функции в ряд Маклорена и равенства коэффициентов разложения:

  \begin{equation*}
    \frac{\phi_{\xi}^{(k)} (0)}{k!}
    = \frac{i^k \expected{\xi^k}}{k!}
    \implies
    \phi_{\xi}^{(k)} (0) = i^k \expected{\xi^k}
  \end{equation*}
\end{proof}

\begin{remark}
  Существует взаимнооднозначное соответствие между распределениями и
  характеристическими функциями. По характеристической функции можно
  восстановить распределение. В частности, если \(\xi\) абсолютно непрерывная
  случайная величина, то плотность можно найти по формуле

  \begin{equation*}
    f_{\xi} (x) = \frac{1}{\sqrt{2 \pi}} \int_{-\infty}^{\infty}
      e^{-i t \xi} \phi_{\xi} (t) \dd t
  \end{equation*}

  Это обратное преобразование Фурье.
\end{remark}

\begin{theorem}[О непрерывном соответствии] \label{thr:cont-match}
  Последовательность случайных величин \(\xi_n\) слабо сходится к случайной
  величине \(\xi\) тогда и только тогда, когда соответствующая
  последовательность характеристических функций поточечно сходится к
  характеристической функции случайной величины \(\xi\).

  \begin{equation*}
    \xi_n \rightrightarrows \xi
    \iff
    \phi_{\xi_n} (t) \to \phi_{\xi} (t)
    \qquad
    \forall t \in \RR
  \end{equation*}
\end{theorem}

\subheader{Характеристические функции стандартных распределений}

\subsubheader{I.}{Распределение Бернулли}

Пусть \(\xi \in B_p\), тогда

\begin{equation*}
  \phi_{\xi} (t)
  = \expected{e^{i \xi t}}
  = (1 - p) \expected{e^{i t \cdot 0}}
    + p \expected{e^{i t \cdot 1}}
  = 1 - p + p e^{i t}
\end{equation*}

\subsubheader{II.}{Биномиальное распределение}

Пусть \(\xi \in B_{n, p}\). Напомним, что

\begin{equation*}
  \prob{\xi = k} = \comb{n}{k} p^k q^{n - k}
  \qquad
  k = 0, 1, \dotsc, n
\end{equation*}

Тогда воспользуемся \ref{lem:char-func-sum-ind} и получим

\begin{equation*}
  \begin{aligned}
    \xi = \xi_1 + \dotsc + \xi_n
    \qquad
    \xi_i \in B_p
  \\
    \phi_{\xi} (t)
    = \prh{\phi_{\xi_1} (t)}^n
    = \prh{1 - p + p e^{i t}}^n
  \end{aligned}
\end{equation*}

\subsubheader{III.}{Распределение Пуассона}

Пусть \(\xi \in \Pi_{\lambda}\). Напомним, что

\begin{equation*}
  \prob{\xi = k} = \frac{\lambda^k}{k!} e^{-\lambda}
  \qquad
  k = 0, 1, \dotsc
\end{equation*}

Тогда получим

\begin{equation*}
  \phi_{\xi} (t)
  = \expected{e^{i \xi t}}
  = \sum_{k = 0}^{\infty} e^{i t k} \frac{\lambda^k}{k!} e^{-\lambda}
  = e^{-\lambda} \sum_{k = 0}^{\infty} \frac{\prh{\lambda e^{i t}}^k}{k!}
  = e^{-\lambda} \cdot e^{\lambda e^{i t}}
  = \exp \prh{\lambda \prh{e^{i t} - 1}}
\end{equation*}

\begin{lemma}
  Распределение Пуассона устойчиво относительно суммирования.
\end{lemma}

\begin{proof}
  Пусть даны независимые случайные величины \(\xi \in \Pi_{\lambda}\) и \(\eta
  \in \Pi_{\mu}\). Тогда по \ref{lem:char-func-sum-ind}

  \begin{equation*}
    \phi_{\xi + \eta} (t)
    = \phi_{\xi} (t) \phi_{\eta} (t)
    = \exp \prh{\lambda \prh{e^{i t} - 1}}
      \exp \prh{\mu \prh{e^{i t} - 1}}
    = \exp \prh{(\lambda + \mu) \prh{e^{i t} - 1}}
  \end{equation*}

  Получили характеристическую функцию распределения \(\Pi_{\xi + \eta}\).
\end{proof}

\subsubheader{IV.}{Показательное распределение}

Пусть \(\xi \in E_{\alpha}\). Напомним, что

\begin{equation*}
  f(x) = \begin{cases}
    0, & x < 0 \\
    \alpha e^{-\alpha x}, & x \ge 0
  \end{cases}
\end{equation*}

Тогда получим

\begin{equation*}
  \phi_{\xi} (t)
  = \expected{e^{i \xi t}}
  = \int_0^{+\infty} e^{i x t} \alpha e^{-\alpha x} \dd x
  = \alpha \int_0^{+\infty} e^{(i t - \alpha) x} \dd x
  = \frac{\alpha}{i t - \alpha} e^{(i t - \alpha) x} \Big\vert_0^{+\infty}
  = \frac{\alpha}{i t - \alpha} (0 - 1)
  = \frac{\alpha}{\alpha - i t}
\end{equation*}

\subsubheader{V.}{Стандартное нормальное распределение}

Пусть \(\xi \in N(0; 1)\). Напомним, что

\begin{equation*}
  f_{\xi} (x) = \frac{1}{\sqrt{2 \pi}} \exp \prh{-\frac{x^2}{2}}
  \qquad
  x \in \RR
\end{equation*}

Тогда получим

\begin{equation*}
  \begin{aligned}
    \phi_{\xi} (t)
    & = \expected{e^{i \xi t}}
  \\
    & = \int_{-\infty}^{\infty} e^{i t x}
      \frac{1}{\sqrt{2 \pi}} \exp \prh{-\frac{x^2}{2}} \dd x
  \\
    & = \frac{1}{\sqrt{2 \pi}} \int_{-\infty}^{\infty}
      \exp \prh{-\frac{1}{2} (x^2 - 2 i t x)} \dd x
  \\
    & = \frac{1}{\sqrt{2 \pi}} \int_{-\infty}^{\infty}
      \exp \prh{-\frac{1}{2} (x^2 - 2 i t x - t^2)}
      \exp \prh{-\frac{t^2}{2}} \dd x
  \\
    & = \frac{1}{\sqrt{2 \pi}} \exp \prh{-\frac{t^2}{2}}
      \under{
        \int_{-\infty}^{\infty} \exp \prh{-\frac{1}{2} (x - i t)^2}
        \dd (x - i t)
      }{\text{Интеграл Пуассона}}
  \\
    & = \frac{1}{\sqrt{2 \pi}} \exp \prh{-\frac{t^2}{2}} \sqrt{2 \pi}
  \\
    & = \exp \prh{-\frac{t^2}{2}}
  \end{aligned}
\end{equation*}
  
\subsubheader{VI.}{Нормальное распределение}

Пусть \(\xi \in N(a; \sigma^2)\). Нормируем ее и обозначим \(\eta = \frac{\xi -
a}{\sigma} \in N(0; 1)\). Характеристическая функция для нее уже найдена ранее:

\begin{equation*}
  \phi_{\eta} (t) = \exp \prh{-\frac{t^2}{2}}
\end{equation*}

Т.к. \(\xi = a + \eta \sigma\), то по \ref{lem:char-func-lin-transform}
получаем, что

\begin{equation*}
  \phi_{\xi} (t)
  = e^{i a t} \exp \prh{-\frac{(\sigma t)^2}{2}}
  = e^{i a t} \exp \prh{-\frac{1}{2} t^2 \sigma^2}
\end{equation*}

\begin{lemma}
  \begin{equation*}
    \begin{rcases}
      \xi \in N(a_1, \sigma_1^2) \\
      \eta \in N(a_2, \sigma_2^2) \\
      \xi \text{ и } \eta \text{ независимы}
    \end{rcases}
    \implies
    \xi + \eta \in N(a_1 + a_2, \sigma_1^2 + \sigma_2^2)
  \end{equation*}
\end{lemma}

\begin{proof}
  \begin{equation*}
    \phi_{\xi + \eta} (t)
    = \phi_{\xi} (t) \phi_{\eta} (t)
    = e^{i a_1 t} \exp \prh{-\frac{1}{2} t^2 \sigma_1^2}
      \cdot e^{i a_2 t} \exp \prh{-\frac{1}{2} t^2 \sigma_2^2}
    = e^{i (a_1 + a_2) t} \exp \prh{-\frac{1}{2} t^2 (\sigma_1^2 + \sigma_2^2)}
  \end{equation*}

  Получили характеристическую функцию нормального распределения с параметрами
  \(a = a_1 + a_2\) и \(\sigma^2 = \sigma_1^2 + \sigma_2^2\).
\end{proof}

\begin{lemma} \label{lem:second-wond-limit}
  \begin{equation*}
    \prh{1 + \frac{x}{n} + \smallo \prh{\frac{1}{n}}}^n
    \Rarr{n \to \infty}
    e^x
  \end{equation*}  
\end{lemma}

\begin{proof}
  \begin{equation*}
    \begin{aligned}
      \prh{1 + \frac{x}{n} + \smallo \prh{\frac{1}{n}}}^n
      & = \exp \prh{n \ln \prh{1 + \frac{x}{n} + \smallo \prh{\frac{1}{n}}}}
    \\
      & = \exp \prh{
        n \prh{
          \frac{x}{n}
          + \smallo \prh{\frac{1}{n}}
          + \smallo \prh{\frac{x}{n} + \smallo \prh{\frac{1}{n}}}
        }
      }
    \\
      & = \exp \prh{n \prh{\frac{x}{n} + \smallo \prh{\frac{1}{n}}}}
    \\
      & = \exp \prh{x + n \cdot \smallo \prh{\frac{1}{n}}}
      \Rarr{n \to \infty} e^x
    \end{aligned}
  \end{equation*}
\end{proof}

\begin{theorem}[Закон больших чисел Хинчина]
  Пусть \(\xi_1, \dotsc, \xi_n\) независимые одинаково распределенные случайные
  величины с конечным первым моментом \(\expected{\xi} = a < \infty\). Тогда

  \begin{equation*}
    \frac{S_n}{n} = \frac{\xi_1 + \dotsc + \xi_n}{n} \Rarr{\probP} a
  \end{equation*}
\end{theorem}

\begin{proof}
  Сходимость по вероятности к константе эквивалентна слабой сходимости
  (\ref{thr:weak-conv-to-const}), поэтому достаточно доказать, что
  \(\frac{S_n}{n} \rightrightarrows a\). По \ref{thr:cont-match} эта сходимость
  имеет место, если

  \begin{equation*}
    \phi_{\frac{S_n}{n}} (t)
    \to
    \phi_a (t)
    = \expected{e^{i t a}}
    = e^{i t a}
    \qquad \forall t \in \RR
  \end{equation*}

  По \ref{lem:char-func-lin-transform} и \ref{lem:char-func-sum-ind} имеем

  \begin{equation*}
    \phi_{\frac{S_n}{n}} (t)
    = \phi_{S_n} \prh{\frac{t}{n}}
    = \prh{\phi_{\xi_1} \prh{\frac{t}{n}}}^n
  \end{equation*}

  Т.к. первый момент существует, то по \ref{lem:char-func-series} получаем

  \begin{equation*}
    \phi_{\xi_1}
    = 1 + i t \expected{\xi_1} + \smallo (t)
    = 1 + i t a + \smallo (t)
  \end{equation*}

  Собираем все полученные формулы и применяем \ref{lem:second-wond-limit}.

  \begin{equation*}
    \phi_{\frac{S_n}{n}} (t)
    = \prh{1 + i \frac{t}{n} a + \smallo \prh{\frac{t}{n}}}^n
    = \prh{1 + \frac{i t a}{n} + \smallo \prh{\frac{1}{n}}}^n
    \Rarr{n \to \infty}
    e^{i t a}
  \end{equation*}

  Получили характеристическую функцию случайной величины \(a\).
\end{proof}

\begin{theorem}[Центральная предельная теорема]
  \label{thr:central-limit-theorem}
  Пусть \(\xi_1, \dotsc, \xi_n\) последовательность независимых одинаково
  распределенных случайных величин с конечным вторым моментом. Обозначим
  \(\expected{\xi_i} = a\) и \(\variance{\xi_i} = \sigma^2\). Тогда

  \begin{equation*}
    \frac{S_n - n a}{\sigma \sqrt{n}}
    \rightrightarrows
    N(0; 1)
  \end{equation*}
\end{theorem}

\begin{proof}
  Пусть \(\eta_i = \frac{\xi_i - a}{\sigma}\)~--- последовательность
  соответствующих стандартизованных случайных величин, причем
  \(\expected{\eta_i} = 0\) и \(\variance{\eta_i} = 1\). Тогда

  \begin{equation*}
    z_n
    = \eta_1 + \dotsc + \eta_n
    = \frac{\sum \xi_i - n a}{\sigma}
    = \frac{S_n - n a}{\sigma}
  \end{equation*}

  Теперь надо доказать, что

  \begin{equation*}
    \frac{z_n}{\sqrt{n}}
    = \frac{S_n - n a}{\sigma \sqrt{n}}
    \rightrightarrows
    N(0; 1)
  \end{equation*}

  По \ref{lem:char-func-lin-transform} и \ref{lem:char-func-sum-ind} имеем

  \begin{equation*}
    \phi_{\frac{z_n}{\sqrt{n}}} (t)
    = \phi_{z_n} \prh{\frac{t}{\sqrt{n}}}
    = \prh{\phi_{\eta_1} \prh{\frac{t}{\sqrt{n}}}}^n
  \end{equation*}

  Т.к. второй момент существует, то по \ref{lem:char-func-series} получаем

  \begin{equation*}
    \phi_{\eta_1}
    = 1 + i t \under{\expected{\eta_1}}{= 0}
      - \frac{t^2}{2} \under{\expected{\eta_1^2}}{= 1}
      + \smallo \prh{t^2}
    = 1 - \frac{t^2}{2} + \smallo \prh{t^2}
  \end{equation*}

  Подставим это в предыдущую формулу

  \begin{equation*}
    \phi_{\frac{z_n}{\sqrt{n}}} (t)
    = \prh{1 - \frac{t^2}{2 n} + \smallo \prh{\frac{t^2}{n}}}^n
    = \prh{1 - \frac{t^2}{2 n} + \smallo \prh{\frac{1}{n}}}^n
  \end{equation*}

  Применяем \ref{lem:second-wond-limit}.

  \begin{equation*}
    \phi_{\frac{z_n}{\sqrt{n}}} (t)
    \Rarr{n \to \infty}
    \exp \prh{-\frac{t^2}{2}}
  \end{equation*}

  Получили характеристическую функцию стандартного нормального распределения и
  по \ref{thr:cont-match} имеем

  \begin{equation*}
    \frac{z_n}{\sqrt{n}}
    = \frac{S_n - n a}{\sigma \sqrt{n}}
    \rightrightarrows
    N(0; 1)
  \end{equation*}
\end{proof}

\begin{theorem}[Предельная теорема Муавра---Лапласа] \label{thr:lim-M-L}
  Пусть \(v_n (A)\)~--- число появления события \(A\) при \(n\) независимых
  испытаниях, \(p\)~--- вероятность успеха при одном испытании, а \(q = 1 - p\).
  Тогда

  \begin{equation*}
    \frac{v_n (A) - n p}{\sqrt{n p q}}
    \rightrightarrows
    N(0; 1)
    \qquad
    (n \to \infty)
  \end{equation*}
\end{theorem}

\begin{proof}
  \begin{equation*}
    \begin{aligned}
      v_n (A) = \xi_1 + \dotsc + \xi_n
      \qquad
      \xi_i \in B_p \text{ это число успехов при \(i\)-ом испытании}
    \\
      \expected{\xi_1} = p
      \qquad
      \variance{\xi_1} = p q
      \qquad
      \stder{\xi_1} = \sqrt{p q}
    \end{aligned}
  \end{equation*}

  По \ref{thr:central-limit-theorem} имеем

  \begin{equation*}
    \frac{v_n (A) - n p}{\sqrt{p q} \sqrt{n}}
    = \frac{v_n (A) - n p}{\sqrt{n p q}}
    \rightrightarrows
    N(0; 1)
  \end{equation*}
\end{proof}

\begin{theorem}[Интегральная формула Муавра---Лапласа]
  \begin{equation*}
    \probP (k_1 \le v_n (A) \le k_2) \approx \Phi(x_2) - \Phi(x_1)
  \end{equation*}  
\end{theorem}

\begin{proof}
  \begin{equation*}
    \probP (k_1 \le v_n (A) \le k_2)
    = \probP \prh{
      \frac{k_1 - n p}{\sqrt{n p q}}
      \le \frac{v_n (A) - n p}{\sqrt{n p q}}
      \le \frac{k_2 - n p}{\sqrt{n p q}}
    }
    = F_{\eta} \prh{\frac{k_2 - n p}{\sqrt{n p q}}}
      - F_{\eta} \prh{\frac{k_1 - n p}{\sqrt{n p q}}}
    \qquad
    \eta = \frac{v_n (A) - n p}{\sqrt{n p q}}
  \end{equation*}

  По \ref{thr:lim-M-L} получаем, что

  \begin{equation*}
    F_{\eta} \prh{\frac{k_2 - n p}{\sqrt{n p q}}}
      - F_{\eta} \prh{\frac{k_1 - n p}{\sqrt{n p q}}}
    \rightrightarrows
    F_0 \prh{\frac{k_2 - n p}{\sqrt{n p q}}}
      - F_0 \prh{\frac{k_1 - n p}{\sqrt{n p q}}}
  \end{equation*}

  где

  \begin{equation*}
    F_0 = \frac{1}{\sqrt{2 \pi}} \int_{-\infty}^{x}
      \exp \prh{-\frac{t^2}{2}} \dd t
  \end{equation*}

  это функция стандартного нормального распределения.
\end{proof}

\begin{remark}
  Аналогичным образом Центральную Предельную Теорему применяют для приближенного
  вычисления вероятностей связанных с суммами большого числа независимых
  одинаково распределенных случайных величин, заменяя стандартизованную сумму на
  стандартное нормальное распределение. Но какова ошибка этого приближения? Для
  этого используют следующую теорему.
\end{remark}

\begin{theorem}[Неравенство Берри---Эссеена]
  В условиях Центральной Предельной Теоремы для случайных величин с конечным
  третьим моментом справедливо

  \begin{equation*}
    \abs{
      \probP \prh{
        \frac{S_n - n \expected{\xi_1}}
        {\sqrt{n \variance{\xi_1}}} < x
      } - F_0 (x)
    } \le C \frac{\expected{\abs{\xi_1 - \expected{\xi_1}}^3}}
      {\sqrt{n} \prh{\sqrt{\variance{\xi_1}}}^3}
    \qquad
    \forall x \in \RR
  \end{equation*}
\end{theorem}

\begin{remark}
  По теории \(C < 0.9\), однако на практике обычно можно брать \(0.4\).
\end{remark}
