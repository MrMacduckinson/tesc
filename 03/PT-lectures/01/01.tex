\subsection{%
  Лекция \texttt{23.09.05}.%
}

Пусть проводится \(n\) реальных экспериментов, в которых событие \(A\) появилось
\(n_A\) раз. Отношение \(\display{\prob{A} = \frac{n_A}{n}}\) называется
частотой события \(A\). Эксперименты показывают, что частота
\quote{стабилизируется} около некоторого числа, под которым и подразумевается
статистическая вероятность.

\begin{definition}
  Пространством элементарных исходов \(\Omega\) называется множество, содержащее
  все возможные результаты данного эксперимента, из которых при испытании
  происходит ровно один. Элементы этого множества называются элементарными
  исходами и обозначаются \(\omega\).
\end{definition}

\begin{definition}
  Случайными событиями называются подмножества \(A \subseteq \Omega\). Говорят,
  что в ходе эксперимента событие \(A\) наступило, если произошел один из
  элементарных исходов, принадлежащий \(A\).
\end{definition}

\begin{example}
  \begin{enumerate}
  \item
    Подбрасывание монеты. \(\Omega = \set{\Gamma, P}\)
  
  \item
    Подбрасывание кубика. \(\Omega = \set{1, 2, 3, 4, 5, 6}\), тогда, например,
    \(2\)~--- это элементарный исход, а \(A = \set{2, 4, 6}\) это событие
    \quote{выпало четное число}.

  \item
    Подбрасывание монеты дважды. Если учитывать порядок бросков, то \(\Omega =
    \set{\Gamma \Gamma, \Gamma P, P\Gamma, PP}\), если не учитывать порядок
    бросков, то \(\Omega = \set{\Gamma \Gamma, \Gamma P, PP}\).

  \item 
    Подбрасывание кубика дважды. \(\Omega = \set{(i, j) \given 1 \le i, j \le
    6}\). Событию \(A\) \quote{разница выпавших очков делится на \(3\)}
    благоприятствуют исходы \((1, 4), (4, 1), (2, 5), (5, 2), (3, 6), (6, 3),
    (1, 1), (2, 2), (3, 3), (4, 4), (5, 5), (6, 6)\).
  
  \item
    Монета подбрасывается до выпадения герба. \(\Omega = \set{\Gamma, P\Gamma,
    PP\Gamma, PPP\Gamma, \dotsc}\)~--- пространство элементарных исходов
    бесконечно, но счетно.

  \item
    Монета бросается на координатную плоскость. \(\Omega = \set{(x, y) \given x,
    y \in \RR}\).
  \end{enumerate}
\end{example}

\subheader{Операции над событиями}

Выделим два специальных события. \(\Omega\)~--- достоверное (универсальное)
событие, которое наступает всегда, т.к. содержит все элементарные исходы. Пустое
(невозможное) событие \(\varnothing\), которое никогда не наступает.

\begin{definition}
  Суммой \(A + B\) называется событие, состоящее в том, что произошло событие
  \(A\) или \(B\).
\end{definition}

\begin{definition}
  Произведением \(A \cdot B\) называется событие, состоящее в том, что произошло
  событие \(A\) и событие \(B\) (т.е. оба).
\end{definition}

\begin{remark}
  Сумма \(A_1 + \dotsc + A_n\)~--- произошло хотя бы одно из этих событий.
  Произведение \(A_1 \cdot \dotsc \cdot A_n\)~--- произошли все события.
\end{remark}

\begin{definition}
  Противоположным к \(A\) называется событие \(\bar{A}\), состоящее в том, что
  событие \(A\) не произошло.
\end{definition}

\begin{definition}
  Дополнением \(A \setminus B\) называется событие, состоящее в том, что событие
  \(A\) произошло, а событие \(B\)~--- нет, т.е. \(A \setminus B = A \cdot
  \bar{B}\).
\end{definition}

\begin{definition}
  События \(A\) и \(B\) называются несовместными, если из произведение равно
  \(\varnothing\), т.е. при одном эксперименте может произойти только одно из
  этих них.
\end{definition}

\begin{definition}
  Событие \(A\) влечет событие \(B\), если \(A \subseteq B\).
\end{definition}

\subheader{Подходы к определению вероятности}

Каждому случайному событию \(A\) хотим приписать числовую характеристику,
отражающую частоту события \(A\): \(0 \le \prob{A} \le 1\)~--- вероятность
события \(A\).

\subsubheader{Подход I.}{Классическое определение вероятности}

Пусть \(\Omega\) содержит конечное число равновозможных исходов, тогда применимо
классическое определение вероятности.

\begin{equation*}
  \prob{A} = \frac{\abs{A}}{\abs{\Omega}} = \frac{m}{n}
\end{equation*}

где \(m\) это число исходов, которые благоприятствуют событию \(A\), а \(n\)~---
число всех возможных исходов. В частности, если \(A_i\)~--- элементарный исход,
то \(\display{\prob{A_i} = \frac{1}{n}}\). Выделим некоторые свойства.

\begin{enumerate}
\item
  \(0 \le \prob{A} \le 1\)

\item
  \(\display{\prob{\Omega} = \frac{n}{n} = 1}\)

\item
  \(\display{\prob{\varnothing} = \frac{0}{n} = 0}\)

\item
  Если \(A\) и \(B\)~--- несовместные события, то \(\prob{A + B} = \prob{A} +
  \prob{B}\).
\end{enumerate}

Приведем несложное доказательство \(4^{\circ}\) свойства.

\begin{proof}
  Если \(A\) и \(B\)~--- несовместные события, то \(\abs{A + B} = \abs{A} +
  \abs{B}\), значит

  \begin{equation*}
    \prob{A + B}
    \frac{\abs{A + B}}{\abs{\Omega}}
    = \frac{\abs{A} + \abs{B}}{\abs{\Omega}}
    = \prob{A} + \prob{B}
  \end{equation*}
\end{proof}

\begin{example}
  Найти вероятность того, что на кубике выпадет четное число при одном броске.

  \begin{equation*}
    \begin{rcases}
      \Omega = \set{1, 2, 3, 4, 5, 6} \\
      A = \set{2, 4, 6}
    \end{rcases}
    \implies
    \prob{A}
    = \frac{\abs{A}}{\abs{\Omega}}
    = \frac{3}{6}
    = 0.5
  \end{equation*}
\end{example}

\subsubheader{Подход II.}{Геометрическое определение вероятности}

\begin{remark}
  Определение \quote{меры} не вводим, ограничимся тем, что мерой отрезкой
  является его длина, плоскости~--- площадь, а пространства~--- объем.
\end{remark}

Пусть \(\Omega\) можно изобразить в виде замкнутой ограниченной области \(\Omega
\subseteq \RR^n\) и мера \(\mu(\Omega)\) конечна. В эту область наугад бросается
точка. Здесь \quote{наугад} означает, что вероятность \(A\) зависит лишь от меры
\(A\) и не зависит от расположения \(A\) (грубо говоря, попадание в любую точку
равновозможно). При выполнении этих условий применимо геометрическое определение
вероятности.

\begin{equation*}
  \prob{A} = \frac{\mu(A)}{\mu(\Omega)}
\end{equation*}

Свойства геометрического определения вероятности полностью аналогичны свойствам
классического определения вероятности.

\begin{remark}
  Т.к. мера точки равна нулю, то вероятность попадания в нее также равна нулю,
  хотя попасть в точку мы можем.
\end{remark}

\galleryone{01_01_01}{Иллюстрация к \ref{ex:prob-geom-1}}

\begin{example} \label{ex:prob-geom-1}
  Монета диаметром \(6\)~см наудачу бросается на пол, вымощенный квадратной
  плиткой со стороной \(20\)~см. Какова вероятность того, что монета окажется
  целиком на одной плитке?

  \solution{} положение монеты определяется положением ее центра. Сделаем
  рисунок (\figref{01_01_01}), из которого видно, что монета полностью окажется
  на одной из плиток, если ее центра окажется во внутреннем (зеленом) квадрате
  со стороной \(14\)~см. Итого имеем

  \begin{equation*}
    \prob{A} = \frac{14^2}{20^2} = \frac{196}{400} = 0.49
  \end{equation*}
\end{example}

\gallerytwo{01_01_02}{Иллюстрация к \ref{ex:prob-geom-2}}
  {Схематичное изображение иголки на ламинате}
  {Применение геометрического определения вероятности}

\begin{example} \label{ex:prob-geom-2}
  Пол застелен ламинатом, на который бросается игла длиной равной длине одной
  доски ламината. Требуется найти вероятность того, что игла пересечет стык.

  \solution{} положение иглы определяется ее центром и углом поворота. Стоит
  отметить, что эти две величины независимы. Пусть игла имеет длину \(2 l\),
  сделаем рисунок (\figref{01_01_02a}). Обозначим через \(x \in \segment{0}{l}\)
  расстояние от середины иглы до ближайшего края доски, а через \(\phi \in
  \segment{0}{\pi}\)~--- угол между направлением иглы и доской. Таким образом
  \(\Omega = \segment{0}{\pi} \times \segment{0}{l}\). Игла пересечет доску,
  если \(x \le l \sin \phi\). Изобразим это (\figref{01_01_02b}) и вычислим
  площадь области \(A\).

  \begin{equation*}
    S_A
    = \int_{0}^{\pi} l \sin \phi
    =  -l \cos \phi \Big\vert_{0}^{\phi}
    = 2 l
  \end{equation*}

  Учитывая, что \(S_{\Omega} = \pi l\), получаем, что искомая вероятность будет
  равна \(\display{\prob{A} = \frac{2 l}{\pi l} = \frac{2}{\pi}}\).
\end{example}

\subheader{Статистика Максвелла--Больцмана}

Поделим прямоугольную область, в которой \quote{летает} \(r\) частиц, на \(n\)
ячеек. Пусть частицы различимы, в одной ячейки может одновременно находится
несколько частиц и все размещения равновероятны. Итого вероятность попадания
частицы в конкретную область будет равна \(\display{\frac{1}{n^r}}\).

\subheader{Статистика Бозе--Эйнштейна}

Пусть теперь частицы неразличимы, но в одной ячейке все еще может одновременно
находится несколько частиц. Получаем следующую формулу для вероятности попадания
частицы в конкретную область \(\display{\frac{1}{\comb{n + r - 1}{r}}}\).

\subheader{Статистика Ферми--Дирака}

Пусть частицы неразличимы и в одной ячейки может находится равно одна частица.
Значит формула для вероятности будет иметь вид
\(\display{\frac{1}{\comb{n}{r}}}\).
