\subsection{%
  Лекция \texttt{23.11.21}.%
}

\subheader{Функции от двух случайных величин}

Если известно совместное распределение двух или нескольких случайных величин, то
можно найти распределение их суммы, произведения и иных функций от этих
случайных величин.

Пусть \(\xi_1\) и \(\xi_2\)~--- случайные величины с плотностью совместного
распределения \(f_{\xi_1, \xi_2} (x, y)\). Дана \(g(x, y) \colon \RR^2 \to
\RR\)~--- борелевская функция. Требуется найти функцию распределения случайной
величины \(\eta = g(\xi_1, \xi_2)\).

\begin{theorem} \label{thr:dst-func-joint-dst}
  Функция распределения случайной величины \(\eta\)

  \begin{equation*}
    F_{\eta} (z) = \iint_{D_z} f_{\xi_1, \xi_2} (x, y) \dd x \dd y
    \qquad
    D_z = \set{\pair{x, y} \given g(x, y) < z}
  \end{equation*}
\end{theorem}

\begin{proof}
  \begin{equation*}
    F_{\eta} (z)
    = \prob{\eta < z}
    = \prob{g(\xi_1, \xi_2) < z}
    = \prob{(x, y) \in D_z}
    = \iint_{D_z} f_{\xi_1, \xi_2} (x, y) \dd x \dd y
  \end{equation*}  
\end{proof}

\begin{example}
  Двое договариваются встретиться между \texttt{12:00} и \texttt{13:00}.
  Случайная величина \(\eta\)~--- время ожидания. Найти ее функцию
  распределения.

  \solution{} Пусть случайные величины \(\xi_1, \xi_2\)~--- это время прихода
  первого и второго человека соответственно. Тогда \(\eta = \abs{\xi_1 -
  \xi_2}\). Имеем

  \begin{equation*}
    \xi_1, \xi_2 \in \evenly{0}{1}
    \implies
    \begin{cases}
      \displaystyle f_{\xi_1} (x) =
      \begin{cases}
        1, & x \in \segment{0}{1} \\
        0, & x \notin \segment{0}{1}
      \end{cases}
    \\
      \displaystyle f_{\xi_2} (y) =
      \begin{cases}
        1, & y \in \segment{0}{1} \\
        0, & y \notin \segment{0}{1}
      \end{cases}
    \end{cases}
  \end{equation*}

  Т.к. \(\xi_1\) и \(\xi_2\) независимы, то

  \begin{equation*}
    f_{\xi_1, \xi_2} (x, y)
    = f_{\xi_1} (x) f_{\xi_2} (y)
    = 1
    \qquad 0 \le x \le 1, 0 \le y \le 1
  \end{equation*}

  Далее применяем теорему \ref{thr:dst-func-joint-dst} и получаем, что

  \begin{equation*}
    \begin{aligned}
      F_{\eta} (z)
      = \iint_{D_z} f_{\xi_1, \xi_2} (x, y) \dd x \dd y
      \qquad
      D_z = \set{\pair{x, y} \given \abs{x - y} < z}
    \\
      F_{\eta} (z)
      = \iint_{D_z} \dd x \dd y
      = S_{D_z}
      \eqby{\figref{01_12_01}}
      1 - 2 \cdot \frac{1}{2} (1 - z)^2
      = 2 z - z^2
      \qquad
      z \in \segment{0}{1}
    \end{aligned}
  \end{equation*}
\end{example}

\galleryone{01_12_01}{Иллюстрация к задаче о встрече}

\subheader{Формула свертки}

\begin{theorem}
  Пусть \(\xi_1\) и \(\xi_2\) независимые абсолютно непрерывные случайные
  величины с плотностями \(f_{\xi_1} (x)\) и \(f_{\xi_2} (y)\) тогда плотность
  суммы \(\xi_1 + \xi_2\) существует и равна свертке плотностей \(f_{\xi_1}
  (x)\) и \(f_{\xi_2} (y)\):

  \begin{equation*}
    f_{\xi_1 + \xi_2} (t)
    = \int_{-\infty}^{\infty} f_{\xi_1} (x) f_{\xi_2} (t - x) \dd x
  \end{equation*}
\end{theorem}

\todo \href{https://www.youtube.com/watch?v=hUjvwyM2HUg&list=PLd7QXkfmSY7ZOTP3bhPT3jGoIlXql8kkX&index=12}{Таймкод}
