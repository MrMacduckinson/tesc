\subsection{%
  Лекция \texttt{23.12.22}.%
}

\begin{remark}
  Рассматриваем только изолированные точки однозначного характера (не
  ветвления).
\end{remark}

\begin{example}
  \begin{equation*}
    \begin{aligned}
      f(z) = \frac{\sin z}{z}
      & \qquad &
      z_0 = 0
      & \qquad &
      \lim_{z \to z_0} \frac{\sin z}{z} = 1
      \implies
      z_0 \text{ устранимая}
    \\
      f(z) = \frac{z}{z - 1}
      & \qquad &
      z_0 = 1
      & \qquad &
      \lim_{z \to z_0} \frac{z}{z - 1} = \infty
      \implies
      z_0 \text{ полюс}
    \\
      f(z) = e^{-\frac{1}{z^2}}
      & \qquad &
      z_0 = 0
      & \qquad &
      \begin{rcases}
        \lim_{z = x \to z_0} e^{-\frac{1}{x^2}} = 0 \\
        \lim_{z = i y \to z_0} e^{\frac{1}{y^2}} = \infty
      \end{rcases}
      \implies
      z_0 \text{ существенно особая точка}
    \end{aligned}
  \end{equation*}
\end{example}

\begin{remark}
  Особые точки находятся только на границе области аналитичности. В случае, если
  особая точка находится в области аналитичности, то мы окружаем ее бесконечно
  малым контуром, который является одной из границ области аналитичности.
\end{remark}

\subheader{Ряды Лорана в особых точках}

Если \(z = z_0 \in \CC\), то

\begin{equation*}
  f(z)
  = \under{\sum_{n = 0}^{\infty} c_n (z - z_0)^n}{\text{правильная часть}}
    + \under{\sum_{n = 1}^{\infty} \frac{c_{-n}}{(z - z_0)^n}}
      {\text{главная часть}}
\end{equation*}

Если \(z = \infty\), то

\begin{equation*}
  f(z)
  = \under{\sum_{n = 1}^{\infty} c_n z^n}{\text{главная часть}}
    + \under{\sum_{n = 0}^{\infty} \frac{c_{-n}}{z^n}}{\text{правильная часть}}
  \eqby{\(z = \frac{1}{\omega}\)}
  \sum_{n = 1}^{\infty} \frac{c_n}{\omega^n} + \sum_{n = 0}^{c_{-n} \omega^n}
\end{equation*}

Причем правильная часть стремится к нулю, а главная часть~--- к бесконечности.

\begin{theorem}
  \(z_o \in \CC\)~--- устранимая особая точка для \(f(z) \iff\) главная часть
  ряда Лорана равна нулю.
\end{theorem}

\begin{proof}
  \ness{} Пусть \(\display{\lim_{z \to z_0} f(z) = A \in \CC}\). Разложим
  \(f(z)\) в ряд Лорана, получим

  \begin{equation*}
    f(z) = \sum_{-\infty}^{\infty} c_n (z - z_0)^n
    \qquad
    c_n = \frac{1}{2 \pi i} \int_{\gamma_{\rho}}
      \frac{f(\zeta)}{(\zeta - z_0)^{n + 1}} \dd \zeta
  \end{equation*}

  Вычетом \(f(z)\) в точке \(z_0\) называется

  \begin{equation*}
    \res f(z_0) \bydef c_{-1}
    = \frac{1}{2 \pi i} \int_{\gamma} f(\zeta) \dd \zeta
  \end{equation*}

  Т.к. в \(z_0\) предел конечен, то

  \begin{equation*}
    \exists \nearo{z_0}{\rho} \colon
    \abs{f(z)} < M
    \qquad
    \forall z \in \nearo{z_0}{\rho}
    \qquad
    0 < \abs{z - z_0} < \rho
  \end{equation*}

  Рассмотрим \(\abs{c_n}\).

  \begin{equation*}
    \begin{aligned}
      c_n = \frac{1}{2 \pi i} \int_{\gamma_{\rho}}
        \frac{f(\zeta) \dd \prh{z_0 + \rho e^{i \phi}}}
          {\prh{\rho e^{i \phi}}^{n + 1}}
    \\
      \abs{c_n}
      = \abs{
        \frac{1}{2 \pi} \int_0^{2 \pi}
        \frac{f(\zeta) \rho e^{i \phi} \dd \phi}{\prh{\rho e^{i \phi}}^{n + 1}}
      }
      = \mtxb{
        \abs{e^{i \phi}} = 1 \\
        \abs{f(\zeta)} < M
      }
      \le \frac{1}{2 \pi} \int_0^{2 \pi} \frac{M \rho}{\rho^{n + 1}} \dd \phi
      = \frac{1}{2 \pi} \cdot \frac{M}{\rho^n} \cdot 2 \pi
      = \frac{M}{\rho^n}
    \end{aligned}
  \end{equation*}

  Т.к. \(n < 0\), то \(\frac{M}{\rho^n} \Rarr{\rho \to 0} 0\). Таким образом
  \(\forall c_{-n} = 0\), где \(n = 1, 2, \dotsc\). Т.е. главная часть равна
  нулю.

  \suff{} Если главная часть ряда Лорана равна нулю, то

  \begin{equation*}
    f(z)
    = \sum_{n = 0}^{\infty} c_n (z - z_0)^n
    = c_0 + c_1 (z - z_0) + \dotsc
  \end{equation*}

  При \(z \to z_0\) эта сумма стремится к \(c_0\). Таким образом

  \begin{equation*}
    \lim_{z \to z_0} f(z) = c_0 \in \CC
  \end{equation*}
\end{proof}

\begin{remark}
  Таким образом функцию \(f(z)\) можно доопределить в устранимой особой точке
  \(z_0\) значением \(f(z_0) = c_0\), тогда \(f(z)\) будет аналитична в круге
  \(\abs{z - z_0} < R\) (включая центр круга \(z_0\)).
\end{remark}

\begin{definition}
  Полюсом порядка \(m\) называется особая точка функции \(f(z)\) такая, что
  \(g(z) = \frac{1}{f(z)}\) число в этой точке нуль порядка \(m\), т.е. \(g(z)\)
  представима в виде \(\display{g(z) = (z - z_0)^m h(z)}\), где \(h(z)\)
  аналитична в \(z_0\) и \(h(z_0) \neq 0\).
\end{definition}

\begin{remark}
  Справедливость представления можно доказать, разложив \(g(z)\) в ряд.
\end{remark}

\begin{theorem}
  \(z_0 \in \CC\)~--- полюс \(\iff\) главная часть содержит \(m \in \NN\)
  ненулевых членов, причем \(\forall n \ge m + 1 \given c_{-n} = 0\), но
  остальные \(c_{-n} \neq 0\). Говорят, что в этом случае \(f(z)\) имеет полюс
  порядка \(m\). Если \(m = 1\), то полюс называется простым.
\end{theorem}

\begin{proof}
  \ness{} Т.к. \(h(z)\) аналитична в \(z_0\) и \(h(z_0) \neq 0\), то разложим в
  ряд функцию

  \begin{equation*}
    \frac{1}{h(z)} = \sum_{n = 0}^{\infty} b_n (z - z_0)^n
  \end{equation*}

  Имеем

  \begin{equation*}
    f(z)
    = \frac{1}{g(z)}
    = \frac{1}{(z - z_0)^m h(z)}
    = \frac{1}{(z - z_0)^m} \sum_{n = 0}^{\infty} b_n (z - z_0)^n
    = \under{b_0 (z - z_0)^{-m} + b_1 (z - z_0)^{-m + 1} + \dotsc}
      {\text{главная часть}}
      + \under{b_m + b_{m + 1} (z - z_0) + \dotsc}
      {\text{правильная часть}}
  \end{equation*}

  Переобозначим и получим

  \begin{equation*}
    f(z) = \frac{c_{-m}}{(z - z_0)^m} + \dotsc + \frac{c_{-1}}{z - z_0}
      + \sum_{n = 0}^{\infty} c_n (z - z_0)^n
  \end{equation*}

  Причем

  \begin{equation*}
    c_{-1} = b_0 = \lim_{z \to z_0} \frac{1}{h(z)} \neq 0
  \end{equation*}

  Т.к. \(h(z_0) \neq \infty\) в силу аналитичности.

  \suff{} Рассмотрим разложение в ряд Лорана.

  \begin{equation*}
    f(z) = \frac{c_{-m}}{(z - z_0)^m} + \dotsc + \frac{c_{-1}}{z - z_0}
      + \sum_{n = 0}^{\infty} c_n (z - z_0)^n
  \end{equation*}

  Умножив на \((z - z_0)^m\) получим аналитичную функцию \(\frac{1}{h(z)}\).
  Тогда \(g(z) = \frac{1}{f(z)}\), которая представима в виде \((z - z_0)^m
  f(z)\). Таким образом \(z = z_0\)~--- это ноль порядка \(m\), т.е. по
  определению является полюсом \(m\)-ого порядка для \(f(z)\).
\end{proof}

\begin{theorem}
  \(z_0 \in \CC\) является существенно особой точкой \(\iff\) главная часть ряда
  Лорана содержит бесконечное число ненулевых членов.
\end{theorem}

\begin{proof}
  Это следствие из двух предыдущих теорем.
\end{proof}

\begin{remark}
  Таким образом, если \(z_0\) это устранимая особая точка, то \(\res f(z_0) =
  0\), в остальных случаях \(\res f(z_0) \neq 0\).
\end{remark}

\begin{remark}
  Определение и критерии особых точек сохраняются для \(z = \infty\), но \(\res
  f(\infty) = -c_{-1}\).
\end{remark}

\begin{example}
  \begin{equation*}
    \begin{aligned}
      f(z)
      = \frac{\prh{e^z - 1}^2}{1 - \cos z}
      \sim
      \frac{z^2}{\frac{z^2}{2}} = 2
      \implies \text{ устранимая }
      \implies c_{-1} = 0
    \\
      f(z) = \frac{1}{\sin \frac{1}{z}}
      \Rarr{\frac{1}{z} = \pi k} \infty
      \implies
      z_k = \frac{1}{\pi k}
      \text{ полюсы}
    \\
      z_0 = \infty
      \qquad
      \frac{1}{\sin \frac{1}{z}}
      = \mtxb{
        z = \frac{1}{\omega} \\
        \omega \to 0
      }
      = \frac{1}{\sin \omega}
      \Rarr{\omega \to 0} \infty
      \implies z_0 = \infty \text{ простой полюс}
    \end{aligned}
  \end{equation*}
\end{example}

\begin{remark}
  Вычет всегда можно найти разложением в ряд Лорана, но чаще всего это бывает
  неудобно, поэтому вычеты ищут с помощью специальных формул и теорем о вычетах.
\end{remark}

\subheader{Теоремы о вычетах}

\begin{theorem}[Основная]
  Если \(f(z)\) аналитична в односвязной области \(D\) кроме конечного числа
  точек \(z_1, \dotsc, z_N \in \CC\), то

  \begin{equation*}
    \oint_{\gamma_{\rho}} f(\zeta) \dd \zeta
    = 2 \pi i \sum_{k = 1}^N \res f(z_k)
  \end{equation*}

  при условии, что \(\forall z_k\) внутри контура \(\gamma_{\rho}\).
\end{theorem}

\begin{proof}
  По теореме Коши для многосвязной области

  \begin{equation*}
    \int_{\text{внеш}} f(\zeta) \dd \zeta
    = \sum \int_{\text{внут}} f(\zeta) \dd \zeta
    = \int \frac{f(\zeta)}{(\zeta - z)^0} \dd \zeta
    = 2 \pi i c_{-1}
    = 2 \pi i \res f(z_0)
  \end{equation*}
\end{proof}

\begin{theorem}
  Пусть \(z_1, \dotsc, z_N\)~--- особые точки \(f(z)\), причем \(z_k \in
  \bar{\CC}\). В области \(\bar{\CC} \setminus \set{z_k}\) функция аналитична.
  Тогда

  \begin{equation*}
    \sum_{p = 1}^N \res f(z_p)
    = \sum_{k = 1}^{N - 1} \res f(z_k) + \res f(\infty)
    = 0
    \qquad
    z_k \in \CC
  \end{equation*}
\end{theorem}

\begin{proof}
  Пусть контур \(C\) это окружность с центром в нуле, которая охватывает все
  конечные особые точки и имеет радиус \(R\). Тогда по предыдущей теореме

  \begin{equation*}
    \int_C f(\zeta) \dd \zeta
     = 2 \pi i \sum_{k = 1}^{N - 1} \res f(z_k)
  \end{equation*}

  Контур \(C\), пройденный в обратном направлении, охватывает \(z = \infty\),
  поэтому

  \begin{equation*}
    \int_C f(\zeta) \dd \zeta
    = -\int_{C^-} f(\zeta) \dd \zeta
    = -\res f(\infty)
  \end{equation*}
\end{proof}

\begin{remark}
  Таким образом, если вычеты в конечных точках посчитать проще, чем разложить
  функцию в ряд Лорана, то оставшийся вычет получается как минус сумма первых.
\end{remark}

\subheader{Вычисление вычетов в полюсе}

\subsubheader{I.}{\(m = 1\)}

\begin{equation*}
  f(z) = c_{-1} (z - z_0)^{-1} + c_0 + c_1 (z - z_0) + c_2 (z - z_0)^2 + \dotsc
\end{equation*}

Умножим это на \((z - z_0)\) и устремим \(z \to z_0\). Получим

\begin{equation*}
  (z - z_0) f(z) = c_1 + \under{\dotsc}{\to 0}
\end{equation*}

Итого

\begin{equation*}
  c_{-1} = \lim_{z \to z_0} (z - z_0) f(z)
\end{equation*}

\subsubheader{I.}{\(m > 1\)}

\begin{equation*}
  (z - z_0)^m f(z) = c_{-m} + \dotsc + c_{-1} (z - z_0)^{m - 1} + \dotsc
\end{equation*}

Продифференцируем это \(m - 1\) раз, тогда

\begin{equation*}
  \lim_{z \to z_0} \prh{(z - z_0)^m f(z)}^{(m - 1)}_z = c_{-1} m!
\end{equation*}

\subheader{Вычисление несобственных вещественных интегралов}

\begin{lemma}
  Пусть \(\display{I = \int_{-\infty}^{\infty} f(x) \dd x \converge}\). Построим
  контур \(C\), который будет представлять их себя полуокружность радиуса \(R\)
  в верхней полуплоскости и будет охватывать все особые точки. \(f(z)\)
  аналитична в \(\Im z > 0\) и  \(\display{\lim_{R \to \infty} \int_C f(\zeta)
  \dd \zeta = 0}\). Тогда

  \begin{equation*}
    I = 2 \pi i \sum_{k = 1}^N \res f(z_0) 
  \end{equation*}
\end{lemma}

\begin{example}
  Дан интеграл

  \begin{equation*}
    I = \int_{-\infty}^{\infty} \frac{\dd x}{\prh{x^2 + 1}^4}
  \end{equation*}

  Рассмотрим

  \begin{equation*}
    \int_C \frac{\dd \zeta}{(\zeta^2 + 1)^4}
    = \int_C \frac{\dd \zeta}{(\zeta - i)^4 (\zeta + i)^4}
  \end{equation*}

  где \(C \colon \abs{z} = 2\). Получаем, что

  \begin{equation*}
    I
    = 2 \pi i \res f(i)
    = 2 \pi i c_{-1}
    = 2 \pi i \cdot \frac{1}{3!} \lim_{z \to z_0}
      \frac{\dd^3 \prh{\frac{1}{(z + i)^4}}}{\dd z^3}
    = \dotsc
    = \frac{-5 i}{32}
  \end{equation*}

  т.к. \(i\) это полюс.
\end{example}