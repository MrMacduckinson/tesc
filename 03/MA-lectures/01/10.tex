\subsection{%
  Лекция \texttt{23.11.03}.%
}

\subheader{Теория функции комплексного переменного (элементы)}

\subsubheader{1.}{Основные понятия}

\(z \in CC\)~--- комплексное число. \(z = \pair{x, y}\)~--- упорядоченная пара
вещественных чисел \(x\) и \(y\).

\begin{equation*}
  0_{\CC} = \pair{0, 0}
  \qquad
  1_{\CC} = \pair{1, 0}
  \qquad
  \pair{0, 1} \bydef i
  \qquad
  i^2 = -1
\end{equation*}

Алгебраическая форма \(z = x + i y\), \(x = \Re z\), \(y = \Im z\). Базовые
операции определяем как

\begin{equation*}
  z_1 + z_2 = \pair{x_1 + x_2, y_1 + y_2}
  \qquad
  z_1 z_2 = \pair{x_1 x_2 - y_1 y_2, x_1 y_2 + x_2 y_1}
\end{equation*}

Если \(z = x + i y\), то \(\bar{z} = x - i y\) называется комплексным
сопряженным.

\begin{equation*}
  \frac{z_1}{z_2} = \frac{z_1 \bar{z_2}}{\abs{z_2}^2}
  \qquad
  \abs{z} = \sqrt{x^2 + y^2}
\end{equation*}

\galleryone{01_10_01}{Геометрическая интерпретация}

Комплексные числа можно изображать на плоскости (\figref{01_10_01}) Тогда
аргументом (главным значением аргумента) называется \(\arg z = \phi\) при
условии, что \(0 \le \phi < 2 \pi\). При этом \(\Arg z = \arg z + 2 \pi k\).

\begin{equation*}
  \begin{cases}
    \cos \phi = \frac{x}{\rho} \\
    \sin \phi = \frac{y}{\rho} \\
  \end{cases}
  \qquad
  \rho = \abs{z}
\end{equation*}

Комплексное число также имеем тригонометрическую форму \(z = \rho (\cos \phi + i
\sin \phi)\) и показательную форму \(z = \rho e^{i \phi} = \rho e^{i (\arg z + 2
\pi k)}\).

\subsubheader{2.}{Множества}

\begin{definition}
  Окрестностью радиуса \(\delta\) с центром \(z_0\) называется

  \begin{equation*}
    \near{z_0}{\delta} \bydef \set{z \in \CC \given \abs{z - z_0} < \delta}
  \end{equation*}
\end{definition}

\begin{remark}
  \begin{equation*}
    \nearo{z_0}{\delta} \bydef \near{z_0}{\delta} \setminus \set{z_0}
  \end{equation*}
\end{remark}

\begin{definition}
  Окрестностью бесконечно удаленной точки называется

  \begin{equation*}
    \near{\infty}{\delta} \bydef \set{z \in \CC \given \abs{z} > \delta}
  \end{equation*}
\end{definition}

\begin{remark}
  Почему \quote{точка}? Введем понятие стереографической проекции (сферы
  Римана). Это взаимнооднозначное соответствие между точками сферы и точками
  комплексной плоскости (кроме одной, которую и называем бесконечно удаленной)
\end{remark}

\begin{remark}
  Операции сложения и умножения вида \(z + \infty\), \(z \cdot \infty\),
  \(\infty + \infty\) не определены, поскольку в общем случае про \(\zeta =
  \infty\) нельзя сказать каковы \(\Re z\), \(\Im z\), \(\Arg z\).
\end{remark}

\begin{remark}
  Хотя мы не можем явно обозначить координаты бесконечно удаленной точки, мы
  можем задавать направления стремления к ней. Например, \(\pair{-\infty,
  \infty}\) это \(\Re z\), а \(\pair{-i \infty, i \infty}\)~--- \(\Im z\).
\end{remark}

\begin{definition}
  Точка \(z_0 \in E\) называется внутренней точкой множества \(E\), если

  \begin{equation*}
    \exists \near{z_0}{\delta} \given \near{z_0}{\delta} \subset E
  \end{equation*}
\end{definition}

\begin{definition}
  Точка \(\hat{z} \in \CC\) называется граничной точкой множества \(E\), если

  \begin{equation*}
    \forall \near{\hat{z}}{\delta} \qquad \begin{cases}
      \exists z \in E \given z \in \near{\hat{z}}{\delta} \\
      \exists z' \notin E \given z' \in \near{\hat{z}}{\delta}
    \end{cases}
  \end{equation*}
\end{definition}

\begin{definition}
  Множество граничных точек называется границей \(\Gamma_E\) множества \(E\).
\end{definition}

\begin{definition}
  Открытое множество это множество, у которого все точки внутренние.
\end{definition}

\begin{definition}
  Замкнутое множество это множество, включающее в себя все свои граничные точки.
\end{definition}

\begin{definition}
  Множество \(D \subset \CC\) называется областью, если

  \begin{enumerate}
  \item
    \(D\) это открытое множество

  \item
    Любые две точки множества \(D\) можно соединить ломаной с конечным числом
    звеньев, которая вся лежит в этой области.
  \end{enumerate}

  Таким образом область это открытое и связное множество.
\end{definition}

\begin{definition}[Кривая на комплексной плоскости (комплексная переменная)]
  \(z = \phi(t) + i \psi(t)\)~--- переменная \(\CC\)-величина, \(t \in
  \segment{t_1}{t_2} \in \RR\). Если \(t\) меняется от \(t_1\) до \(t_2\), то
  \(z\) движется ориентированно. Тогда если \(\phi(t)\), \(\phi(t)\) непрерывны,
  то \(z = \phi(t) + i \psi(t)\) задает параметрически заданную непрерывную
  ориентированную кривую.
\end{definition}

\begin{remark}
  Если \(\phi(t)\) и \(\psi(t)\) имеют непрерывные производные, то кривая
  гладкая.
\end{remark}

\begin{definition}
  Замкнутая простая кривая называется контуром.
\end{definition}

\begin{definition}
  Область \(D\) называется односвязной, если всякий контур, лежащий в ней, может
  быть непрерывно стянут в точку области \(D\). В противном случае область
  называется многосвязной.
\end{definition}

\begin{example}
  Область \(D \colon 0 < \abs{z - a} < \epsilon\) не будет односвязной. Точка
  \(a\) в данном примере будет изолированной.
\end{example}

\begin{example}
  Пусть \(D \colon \abs{z} < 1, 0 < \arg z < 2 \pi\). Тогда \(\Gamma_D\) это
  окружность радиуса \(1\) и отрезок \(\segment{0}{1}\) проходится в двух
  направлениях (при этом  область является односвязной).
\end{example}

\begin{remark}
  Далее будем рассматривать области с кусочно-гладкой границей \(+\) с
  изолированные точки.
\end{remark}

\begin{definition}
  \(\bar{\CC} = \CC \cup \set{\infty}\)~--- расширенная \(\CC\)-плоскость.
\end{definition}

\begin{remark}
  В расширенной \(\CC\)-плоскости определение односвязной сохраняется, но
  стягивание контура к бесконечно удаленной точке следует рассматривать на сфере
  Римана.
\end{remark}

\begin{example}
  Пусть \(D \subset \bar{\CC} \colon z \neq a\). Данная область будет
  односвязной, т.к. любой контур можно стянуть в \(\infty\) (бесконечно
  удаленную точку).
\end{example}

\begin{remark}
  Любой контур стягивается в точку может быть равную бесконечности.
\end{remark}

\begin{theorem}[Жордана]
  Всякая непрерывная замкнутая кривая разбивает расширенную \(\CC\)-плоскость на
  две односвязные области.
\end{theorem}

\subsubheader{3.}{Предел последовательности}

\begin{equation*}
  z_n = \set{z_1, z_2, \dotsc, z_n, \dotsc},
  \qquad
  z_n = x_n + i y_n
\end{equation*}

Таким образом последовательность \(z_n\) определена парой вещественных
последовательностей \(\set{x_n}\) и \(\set{y_n}\).

\begin{equation*}
  L \in CC \bydef \lim_{n \to \infty} z_n
  \iff
  \forall \epsilon > 0 \given
  \exists N(\epsilon) \in \NN \given
  \forall n > N \colon
  \abs{z_n - L} < \epsilon
\end{equation*}

\begin{remark}
  Пусть \(z_n\) такова, что

  \begin{equation*}
    \forall \epsilon > 0 \given
    \exists N(\epsilon) \in \NN \given
    \forall n > N \colon
    \abs{z_n} > \epsilon
  \end{equation*}

  Т.е. \(\lim_{n \to \infty} z_n = \infty\)~--- определение бесконечно удаленной
  точки.
\end{remark}
