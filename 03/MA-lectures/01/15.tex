\subsection{%
  Лекция \texttt{23.12.08}.%
}

\subheader{3. Ряды}

\subsubheader{3.1}{Числовые ряды}

\begin{definition}
   \(\display{\sum c_n}\), где \(c_n \in \CC\) называется числовым рядом.
\end{definition}

\begin{definition}
  \(\display{S_n = \sum_{k = 1}^n}\)~--- частичная сумма ряда
\end{definition}

\begin{definition}
  \(\display{S = \lim_{n \to \infty} S_n \in \CC}\)~--- сумма ряда.
\end{definition}

\begin{remark}
  Для исследования можно использовать те же условия сходимости, что и в
  вещественном случае

  \begin{enumerate}
  \item
    Необходимое условие сходимости \(\display{\lim_{n \to \infty} c_n = 0}\)

  \item
    Признак сравнения (по модулю)

  \item
    Абсолютная сходимость

  \item
    Признаки Даламбера, Коши и т.п. (везде считаем по модулю)
  \end{enumerate}
\end{remark}

\begin{remark}[Критерий Коши]
  Ряд \(\display{\sum c_n}\) сходится к \(S \in \CC\) тогда и только тогда,
  когда

  \begin{equation*}
    \forall \epsilon > 0 \given
    \exists N(\epsilon) \in \NN \given
    \forall n > N \colon
    \abs{S_n - S} < \epsilon
  \end{equation*}

  Т.е. остаток ряда \(r_{n + 1} = S - S_n\) стремится к нулю (\(\abs{r_{n + 1}}
  < \epsilon\)).
\end{remark}

\subsubheader{3.2}{Функциональные ряды}

\begin{definition}
  \(\display{\sum u_n (z)}\)~--- функциональный ряд.
\end{definition}

\begin{definition}
  \(\display{f(z) = \sum u_n (z)}\)~--- сумма ряда, т.к. \(\forall z \in D\)
  определена сумма соответствующего числового ряда.
\end{definition}

\begin{definition}
  Сходящийся числовой ряд \(\sum c_n\) называется мажорирующим, если \(\forall z
  \in D \colon \abs{u_n (z)} < \abs{c_n}\).
\end{definition}

\begin{lemma}[Признак Вейерштрасса]
  Если ряд мажорируем в области \(D\), то он равномерно сходится в области \(D\)
  и \(f(z)\) непрерывна в \(D\).
\end{lemma}

\begin{remark}
  Сходимость функционального ряда означает, что

  \begin{equation*}
    \forall \epsilon > 0 \given
    \exists N(\epsilon, z) \in \NN \given
    \forall n > N \colon
    \abs{f(z) - \sum_{k = 1}^n u_k (z)} < \epsilon
  \end{equation*}
\end{remark}

\begin{remark}
  В случае равномерной сходимости \(N = N(\epsilon)\) и

  \begin{equation*}
    \abs{r_{n + 1} (z)}
    = \abs{f(z) - \sum_{k = 1}^n u_k (z)} < \epsilon
    < \epsilon
  \end{equation*}
\end{remark}

\begin{theorem}[Почленное интегрирование равномерносходящегося ряда]
  Пусть \(\display{f(z) = \sum u_n (z)}\) сходится равномерно в области \(D\),
  тогда

  \begin{equation*}
    \int_C f(\zeta) \dd \zeta
    = \sum_{n = 1}^{\infty}  \int_C f(\zeta) \dd \zeta
  \end{equation*}

  где \(C\) это кривая в области \(D\).
\end{theorem}

\begin{proof}
  Т.к. ряд сходится равномерно, то \(\abs{r_{n + 1} (z)} < \epsilon' =
  \frac{\epsilon}{l}\), где \(l\) это длина кривой \(C\). Рассмотрим

  \begin{equation*}
    \begin{aligned}
      \abs{\int_C f(\zeta) \dd \zeta
        - \sum_{k = 1}^n \int_C u_k (\zeta) \dd \zeta}
      & = \abs{\int_C \prh{f(\zeta) - \sum_{k = 1}^n u_k (\zeta)} \dd \zeta}
    \\
      & \le \int_C \abs{f(\zeta) - \sum_{k = 1}^n u_k (\zeta)} \dd \zeta
    \\
      & = \int_C \abs{r_{n + 1} (\zeta)} \dd \zeta
    \\
      & < \int_C \frac{\epsilon}{l} \dd \zeta
    \\
      & = \epsilon
    \end{aligned}
  \end{equation*}
\end{proof}

\subsubheader{3.3}{Степенные ряды}

\begin{definition}
  \(\display{\sum c_n (z - a)^n}\)~--- степенной ряд в точке \(z = a\).
\end{definition}

\begin{remark}
  Обозначив \(\zeta = z - a\), получим более удобный ряд \(\display{\sum c_n
  \zeta^n}\) (ряд в \(\zeta = 0\)).
\end{remark}

\begin{theorem}[Абеля]
  Если ряд \(\sum c_n z^n\) сходится в точке \(z_1\), то он сходится
  равномерно и абсолютно в точках \(z\) в круге радиуса \(\abs{z_1}\). Если ряд
  \(\sum c_n z^n\) расходится в точке \(z_2\), то он расходится \(\forall z\) за
  пределами круга радиуса \(\abs{z_2}\).
\end{theorem}

\begin{remark}
  Существует \(R \in \RR\)~--- радиус сходимости такой, что в круге радиуса
  \(R\) степенной ряд сходится.
\end{remark}

\begin{definition}
  Функция \(f(z)\) представима степенным рядом \(\sum c_n (z - a)^n\), то она
  называется регулярной в точке \(a\).
\end{definition}

\begin{remark}
  Далее докажем, что регулярность функции в области равносильна аналитичности в
  этой области.
\end{remark}

\begin{theorem}
  Регулярная в области \(D\) функция \(f(z)\) дифференцируема в \(D\) и

  \begin{equation*}
    f'(z) = \prh{\sum c_n z^n}' = \sum \prh{c_n z^n}'
  \end{equation*}
\end{theorem}

\begin{proof}
  Рассмотрим функцию \(S(z) = \sum_{n = 1}^{\infty} n c_n z^{n - 1}\). Ряд
  равномерно сходится в радиусе \(\rho\). Тогда \(S(z)\) непрерывна и определен
  интеграл \(\int_C S(\zeta) \dd \zeta\), где \(C\) это кривая в круге радиуса
  \(\rho\).
  
  Проинтегрируем функции \(\zeta^k\), где \(k \in \NN_0\). Отметим, что
  \(\zeta^k\) аналитическая в \(\CC\), поэтому по \ref{thr:C-loop-int}
  \(\display{\oint_K \zeta^k \dd \zeta = 0}\), таким образом интеграл \(\int_C
  \zeta^k \dd \zeta\) не зависит от пути, поэтому

  \begin{equation*}
    \int_C \zeta^k \dd \zeta
    = \frac{\zeta^{k + 1}}{k + 1} \bigg\rvert_0^z
  \end{equation*}

  где кривая \(C\) связывает точки \(0\) и \(z\). Получаем, что

  \begin{equation*}
    \sum_{k = 0}^{\infty} \int_C \zeta^k \dd \zeta
    = \sum_{k = 0}^{\infty} \frac{z^{k + 1}}{k + 1}
    \eqby{k + 1 = n}
    \sum_{n = 1}^{\infty} \frac{z^n}{n}   
  \end{equation*}

  Используем полученное равенство и получим

  \begin{equation*}
    \int_C S(\zeta) \dd \zeta
    = \int_C \sum_{n = 1}^{\infty} n c_n \zeta^{n - 1} \dd \zeta
    = \sum_{n = 1}^{\infty} \int_C n c_n \zeta^{n - 1} \dd \zeta
    = \sum_{n = 1}^{\infty} c_n z^n
    = \sum_{n = 0}^{\infty} c_n z^n - c_0
    = f(z) - c_0
  \end{equation*}

  Таким образом \(\display{f(z) = \int_0^z S(\zeta) \dd \zeta + c_0}\), т.е.
  \(f(z)\) является первообразной для \(S(z)\) и \(f'(z) = S(z)\).
\end{proof}

\begin{remark}
  Очевидно, что регулярную функцию можно продифференцировать еще раз и сколько
  угодно раз (т.к. производная также степенной ряд), значит она является
  аналитической.
\end{remark}

\begin{remark}
  Запишем

  \begin{equation*}
    \begin{aligned}
      f^{(n)} (z) & = \prh{\sum c_k z^k}^{(n)}
    \\
      f(z)        & = c_0    & + & c_1 z                 & + & \dotsc
    \\
      f'(z)       & = c_1    & + & 2 c_2 z               & + & \dotsc
    \\
      f''(z)      & = 2 c_2  & + & 3 \cdot 2 \cdot c_3 z & + & \dotsc
    \\
                  & \vdots   &   & \vdots                &   & \vdots
    \\
      f^{(n)} (z) & = n! c_n & + & (n + 1)! c_{n + 1} z  & + & \dotsc
    \end{aligned}
  \end{equation*}

  Отсюда получаем, что

  \begin{equation*}
    \begin{aligned}
      \begin{rcases}
        f(0)        & = c_0   \\
        f'(0)       & = c_1   \\
        f''(0)      & = 2 c_2 \\
        f^{(n)} (0) & = n! c_n
      \end{rcases}
      \implies
      c_n = \frac{f^{(n)} (0)}{n!}
    \end{aligned}
  \end{equation*}

  Для ряда \(\sum c_n (z - a)^n\) имеем \(\display{c_n =
  \frac{f^{(n)} (a)}{n!}}\) Таким образом получаем ряд Тейлора

  \begin{equation*}
    f(z) = \sum_{n = 0}^{\infty} \frac{f^{(n)} (a)}{n!} (z - a)^n
  \end{equation*}
\end{remark}

\begin{remark}
  Ряд Тейлора единственный. Доказательство аналогично вещественному случаю.
\end{remark}