\subsection{%
  Лекция \texttt{23.09.15}.%
}

\begin{theorem}[Радикальный признак Коши]
  \begin{equation*}
    \lim_{n \to \infty} \sqrt[n]{u_n}
    = K \in \RR
    = \begin{cases}
      0 \le K < 1 & \implies \converge \\
      K = 1 & \implies \text{ необходимо дополнительное исследование} \\
      K > 1 & \implies \diverge
    \end{cases}
  \end{equation*}
\end{theorem}

\begin{proof}
  Распишем предел по определению.

  \begin{equation*}
    \lim_{n \to \infty} \sqrt[n]{u_n} = K \iff
    \forall \epsilon > 0 \given
    \exists n_0 \in \NN \given
    \forall n > n_0 \colon
    \abs{\sqrt[n]{u_n} - K} < \epsilon
    \implies
    K - \epsilon< \sqrt[n]{u_n} < K + \epsilon
  \end{equation*}

  Рассмотрим случай \(0 \le K < 1\). Тогда из правой части неравенства получаем

  \begin{equation*}
    \exists r \given K < r < 1 \; (\epsilon = r - K)
    \implies \sqrt[n]{u_n} < r
    \implies u_n < r^n
  \end{equation*}

  Таким образом ряд \(\sum u_n\) почленно меньше ряда \(\sum r^n (0 < r < 1)\),
  который сходится. Значит по признаку сравнения он тоже сходится. Аналогично
  рассмотрим случай \(K > 1\). Тогда из левой части неравенства получаем

  \begin{equation*}
    \exists r \given K > r > 1 \; (\epsilon = K - r)
    \implies r < \sqrt[n]{u_n}
    \implies u_n > r^n
  \end{equation*}

  Таким образом ряд \(\sum u_n\) почленно больше расходящегося ряда \(\sum r^n
  \; (r > 1)\), значит он тоже расходится.
\end{proof}

\gallerytwo{01_03_01}{Иллюстрация к \ref{thr:С-attr}}
  {Нижние столбцы}
  {Верхние столбцы}

\begin{theorem}[Интегральный признак Коши] \label{thr:С-attr}
  Пусть есть ряд \(\sum_{n = a}^{\infty} u_n\) и интеграл \(\int_a^{\infty} f(x)
  \dd x\). Тогда если \(f(x) \ge 0\), \(f(x)\) монотонно убывает и \(\forall n
  \in \NN \given f(n) = u_n\), то ряд и интеграл имеют одинаковую сходимость.
\end{theorem}

\begin{proof}
  Пользуясь иллюстрацией \figref{01_03_01} составим неравенство и воспользуемся
  предельным переходом.

  \begin{equation*}
    \begin{aligned}
      \under{\sum_{k = 2}^{n} u_k}{\text{Нижние столбцы}}
      \le \int_1^n f(x) \dd x
      \le \under{\sum_{k = 1}^{n - 1} u_k}{\text{Верхние столбцы}}
    \\
      S_n - u_1 \le \int_1^n f(x) \dd x \le S_{n - 1}
    \\
      \lim_{n \to \infty} \prh{S_n - u_1}
      \le \int_1^{\infty} f(x) \dd x
      \le \lim_{n \to \infty} \prh{S_{n - 1}}
    \end{aligned}
  \end{equation*}

  \subsubheader{1.}{\(\sum u_n = S \in \RR\)}

  Интеграл сходится, т.к. ограничен снизу и сверху числами \(S - u_1\) и \(S\)
  соответственно.

  \subsubheader{2.}{\(\sum u_n \diverge\)}

  Интеграл расходится, т.к. он не менее бесконечности (левая часть неравенства).

  \subsubheader{3.}{\(\int \converge, \int = I\)}

  \(
    \lim_{n \to \infty} \prh{S_n - u_1} < I \in \RR
    \implies \sum u_n \converge
  \)

  \subsubheader{4.}{\(\int \diverge\)}

  Ряд расходится, т.к. предел его частичных сумм не менее бесконечности (правая
  часть неравенства).
\end{proof}

\begin{example}
  Рассмотрим обобщенный гармонический ряд \(\display{\sum \frac{1}{n^p}}\) и
  исследуем его сходимость с помощью интегрального признака Коши. Введем функцию
  \(\display{f(x) = \frac{1}{x^p}}\), \(f(x) \colon [1; + \infty] \to R^+\).
  Если \(p = 1\), то получаем

  \begin{equation*}
    \int_1^{\infty} \frac{\dd x}{x}
    = \ln x \Big \vert_1^{\infty}
    = \infty
  \end{equation*}

  Если \(p \neq 1\), то

  \begin{equation*}
    \int_1^{\infty} \frac{\dd x}{x^p}
    = \frac{x^{-p + 1}}{-p + 1} \bigg \vert_1^{\infty}
    = \frac{1}{1 - p} \cdot \lim_{x \to \infty} \prh{x^{-p + 1} - 1}
  \end{equation*}

  Значит, если \(-p + 1 < 0\), т.е. \(p > 1\), то интеграл (а значит и ряд)
  сходится. Если же \(p < 1\), то ряд расходится. В итоге

  \begin{equation*}
    \sum \frac{1}{n^p}
    \qquad
    \begin{cases}
      p > 1 \implies \converge \\
      p \le 1 \implies \diverge \\
    \end{cases}
  \end{equation*}
\end{example}

\subheader{Знако\textit{чередующиеся} ряды}

\begin{definition}
  Ряд вида

  \begin{equation*}
    \sum_{n = 1}^{\infty} (-1)^{n - 1} u_n
    = u_1 - u_2 + u_3 - \dotsc \pm u_n \mp
  \end{equation*}

  где \(u_n > 0\) называют знакочередующимся рядом.
\end{definition}

\begin{remark}
  Следует различать знакочередующиеся и знакопеременные ряды. В
  знакочередующихся рядах знак чередуется с каждым следующим членом ряда. В
  знакопеременных знак члена ряда не обязательно чередуется~--- он может
  меняться и по более сложным правилам. Примером знакопеременного ряда может
  быть

  \begin{equation*}
    \sum u_n = 1 + \frac{1}{2} + \frac{1}{3} - \frac{1}{4} + \frac{1}{5}
      + \frac{1}{6} - \frac{1}{7} + \dotsc
  \end{equation*}
\end{remark}

\begin{theorem}[Признак Лейбница о сходимости ряда] \label{thr:L-cond-conv}
  Пусть дан знакочередующийся ряд \(\sum (-1)^{n - 1} u_n\). Тогда если \(u_i\)
  монотонно убывают и \(\lim_{n \to \infty} u_n = 0\), то данный ряд сходится.
\end{theorem}

\begin{proof}
  Рассмотрим частичную сумму ряда \(S_{2 n}\)

  \begin{equation*}
    S_{2 n}
    = u_1 - u_2 + u_3 - u_4 + \dotsc + u_{2 n - 1} - u_{2 n}
    = (u_1 - u_2) + (u_3 - u_4) + \dotsc + (u_{2 n - 1} - u_{2 n})
  \end{equation*}

  Т.к. \(\seq{u_i}\) монотонно убывает, то \(u_i > u_{i + 1}\), значит все
  полученные скобки положительные, причем при увеличении \(n\) сумма \(S_{2 n}\)
  накапливается. Таким образом последовательность \(\seq{S_{2 n}}\) возрастает.
  Сгруппируем члены ряда в частичной суммой по другому.

  \begin{equation*}
    S_{2 n}
    = u_1 - u_2 + u_3 - u_4 + \dotsc + u_{2 n - 1} - u_{2 n}
    = u_1 - (u_2 - u_3) - (u_4 - u_5) - \dotsc - (u_{2 n - 2} - u_{2 n - 1})
      - u_{2 n}
  \end{equation*}

  Опять же, в силу монотонности \(\seq{u_i}\) все полученные скобки
  положительные, а \(u_{2 n}\) положителен по условию. Значит \(S_{2 n} < u_1\).
  Итого последовательность \(\seq{S_{2 n}}\) ограничена сверху и монотонно
  возрастает, значит

  \begin{equation*}
    \exists \lim_{n \to \infty} S_{2 n} = S \in \RR
  \end{equation*}

  \(S_{2 n + 1}\) отличается от \(S_{2 n}\) одним слагаемым \(u_{2 n + 1}\),
  которое не влияет на сходимость.

  \begin{equation*}
    \lim_{n \to \infty} S_{2 n + 1}
    = \under{\lim_{n \to \infty} S_{2 n}}{S}
      + \under{\lim_{n \to \infty} u_{2 n + 1}}{0}
    = S
  \end{equation*}
\end{proof}

\begin{remark}[Об оценке остатка ряда] \label{rem:rem-est}
  В доказательстве теоремы \ref{thr:L-cond-conv} установили, что \(S_{2 n} <
  u_1\) (для \(S_{2 n + 1}\) это также верно). Рассмотрим следующий ряд

  \begin{equation*}
    \begin{aligned}
      \sum_{n = 1}^{\infty} (-1)^{n - 1} u_n = u_1 - u_2 + u_3 - u_4 + \dotsc
        \under{\pm u_{k + 1} \mp u_{k + 2} \pm \dotsc}{R_{k + 1}}
    \\
      R_{k + 1}
      = \sum_{n = k + 1}^{\infty} (-1)^{n - 1} u_n
      = \pm \sum_{m = 1}^{\infty} (-1)^{m - 1} u_n
    \end{aligned}
  \end{equation*}

  Если исходный ряд сходится, то и его остаток \(R_{k + 1}\) также сходится. При
  этом остаток можно оценить (по модулю) старшим членом, т.е. \(\abs{R_{k + 1}}
  < \abs{u_{k + 1}}\). Это позволяет определить, какая погрешность получится,
  если в приближенных вычислениях использовать частичную сумму ряда.
\end{remark}

\begin{example}
  Пусть дан ряд

  \begin{equation*}
    1 - \frac{1}{2} + \frac{1}{4} - \frac{1}{8} + \dotsc
    = \sum_{n = 0}^{\infty} (-1)^n \frac{1}{2^n}
  \end{equation*}

  Вычислим его остаток \(R_4\).

  \begin{equation*}
    R_4
    = \prh{\frac{1}{16} - \frac{1}{32}}
      + \prh{\frac{1}{64} - \frac{1}{128}}
      + \dotsc
    = \frac{1}{32} + \frac{1}{128} + \frac{1}{512} + \dotsc
    = \sum_{k = 2}^{\infty} \frac{1}{2^{2 k + 1}}
  \end{equation*}

  Преобразуем полученное выражение перенумеровав ряд.

  \begin{equation*}
    R_5
    = \frac{1}{2} \sum_{k = 2}^{\infty} \frac{1}{2^{2 k}}
    = \frac{1}{2} \sum_{k = 2}^{\infty} \frac{1}{4^k}
    = \frac{1}{2} \sum_{k = 0}^{\infty} \frac{1}{4^{k + 2}}
    = \frac{1}{32} \sum_{k = 0}^{\infty} \frac{1}{4^k}
    = \frac{1}{32} \cdot \frac{1}{1 - \frac{1}{4}}
    = \frac{1}{24} < \frac{1}{16} = u_4
  \end{equation*}
\end{example}

\begin{remark}
  Заметим, что оценка \ref{rem:rem-est} не работает для знакоположительных
  рядов. Приведем пример.

  \begin{equation*}
    \begin{aligned}
      1 + \frac{1}{2} + \frac{1}{4} + \frac{1}{8}
        + \under{\frac{1}{16} + \dotsc}{R_4}
    \\
      R_4
        = \frac{1}{16} + \frac{1}{32} + \dotsc
        = \frac{1}{16} \prh{1 + \frac{1}{2} + \dotsc}
        = \frac{1}{8} > \frac{1}{16} = u_4
    \end{aligned}
  \end{equation*}
\end{remark}

\begin{theorem}[Абсолютная сходимость]
  \begin{equation*}
    \sum \abs{u_n} \converge \implies \sum u_n \converge
  \end{equation*}
\end{theorem}

\begin{proof}
  Применим критерий Коши для ряда \(\display{\sum \abs{u_n}}\).

  \begin{equation*} \label{thr:abs-conv-1} \tag{1}
    \sum \abs{u_n} \converge \iff
    \forall \epsilon > 0 \given
    \exists n_0 \in \NN \given
    \forall n > p > n_0 \colon
    \abs{S_n - S_p} < \epsilon
  \end{equation*}

  Раскроем и преобразуем последнее неравенство. Воспользуемся тем, что модуль
  суммы не превышает суммы модулей.

  \begin{equation*} \label{thr:abs-conv-2} \tag{2}
    \begin{aligned}
      \abs{S_n - S_p} = \abs[\Big]{\abs{u_p} + \dotsc + \abs{u_n}} < \epsilon
    \\
      \abs{u_p + \dotsc + u_n} \le \abs{u_p} + \dotsc + \abs{u_n} < \epsilon
    \\
      \abs{u_p + \dotsc + u_n} < \epsilon
    \end{aligned}
  \end{equation*}

  Если вернуться к \ref{thr:abs-conv-1} и подставить полученное неравенство, то
  получим критерий Коши для ряда \(\sum u_n\).
\end{proof}

\begin{definition}
  Ряд называется абсолютно сходящимся, если ряд из модулей сходится. Ряд
  называется условно сходящимся, если он сходится, но ряд из модулей расходится.
\end{definition}

\begin{example}
  Примером условно сходящегося ряда может быть ряд \(\display{\sum (-1)^{n - 1}
  \frac{1}{n}}\).
\end{example}

\begin{remark}
  Перестановка членов абсолютно сходящегося ряда не меняет его суммы.
\end{remark}

\begin{remark}
  Ряд из модулей знакоположителен, значит для него можно применять все признаки
  для знакоположительных рядов рассмотренные ранее.
\end{remark}