\subsection{%
  Лекция \texttt{23.09.08}.%
}

\begin{remark}
  Можно доказать, что определенной перестановкой членов ряда в качестве суммы
  можно получить любое заданное число.
\end{remark}

\begin{remark}
  Также возможно перемножение рядов. Произведение сходящихся рядов~---
  сходящийся ряд. Формулы для произведения можно найти в литературе.
\end{remark}

\begin{important}
  Далее для краткости ряды будут записываться в виде \(\display{\sum u_n}\).
  Нижней границей по умолчанию будем считать единицу. В рядах с другой нижней
  границей и в местах, где необходимо сделать акцент на границе, будет
  использоваться запись вида \(\display{\sum_{n = 0} v_n}\).
\end{important}

Далее рассмотрим некоторые условия сходимости рядов.

\begin{theorem}[Необходимое условие сходимости ряда] \label{thr:ness-ser-conv}
  \begin{equation*}
    \sum u_n \converge \implies \lim_{n \to \infty} u_n = 0
  \end{equation*}
\end{theorem}

\begin{proof}
  \begin{equation*}
    \begin{aligned}
      \sum u_n \converge \iff \exists \lim_{n \to \infty} S_n = S \in \RR
    \\
      u_{n + 1} = S_{n + 1} - S_n
    \\
      \lim_{n \to \infty} u_{n + 1}
      = \lim_{n \to \infty} \prh{S_{n + 1} - S_n}
      = \lim_{n \to \infty} S_{n + 1} - \lim_{n \to \infty} S_n
      = S - S
      = 0
    \end{aligned}
  \end{equation*}
\end{proof}

\begin{remark}
  Обратное в общем случае неверно. Например

  \begin{equation*}
    \sum \frac{1}{n} \notconverge, \text{ но }
    \lim_{n \to \infty} \frac{1}{n} = 0
  \end{equation*}
\end{remark}

\begin{remark}
  Необходимым условием сходимости удобно пользоваться в обратную сторону, т.е. с
  его помощью проще показать, что ряд расходится.
\end{remark}

\begin{example}
  \begin{equation*}
    \begin{aligned}
      \sum \under{\prh{2n + 3} \cdot \sin \frac{1}{n}}{u_n}
    \\
      \lim_{n \to \infty} u_n
      = \lim_{n \to \infty} \frac{\sin \frac{1}{n}}{\frac{1}{2n + 3}}
      = \lim_{n \to \infty} \frac{\frac{1}{n}}{\frac{1}{2n + 3}}
      = \lim_{n \to \infty} \frac{2n + 3}{n}
      = 2
    \\
      \lim_{n \to \infty} u_n \neq 1 \implies
      \sum \prh{2n + 3} \cdot \sin \frac{1}{n} \notconverge
    \end{aligned}
  \end{equation*}
\end{example}

\begin{example}
  \begin{equation*}
    \sum \frac{1}{2n + 3}
    \qquad
    \lim_{n \to \infty} \frac{1}{2n + 3} = 0
  \end{equation*}

  Рассмотрим вспомогательный ряд \(\display{\sum \frac{1}{3n}}\). Можно
  убедиться, что начиная с \(n = 4\) члены вспомогательного ряда меньше
  соответствующих членов исследуемого ряда. Заметим, что

  \begin{equation*}
    \sum \frac{1}{3n} = \frac{1}{3} \sum \frac{1}{n} \implies \notconverge
  \end{equation*}

  Значит, исходный ряд также расходится.
\end{example}

\begin{theorem}[Критерий Коши для сходимости рядов] \label{thr:crt-C}
  \begin{equation*}
    \sum u_n \converge \iff \exists \lim_{n \to \infty} S_n = S \in \RR
    \; \boxed{\iff} \;
    \forall \epsilon > 0 \given
      \exists n_0 \in \NN \given
      n \ge p \ge n_0 \colon
      \abs{S_n - S_p} < \epsilon
  \end{equation*}

  Стоит отметить, что \(\display{\abs{S_n - S_p} = \abs{u_p + u_{p + 1} +
  \dotsc + u_n}}\). Такая форма записи иногда будет полезна в дальнейшем.
\end{theorem}

\begin{remark}
  Смысл критерия Коши в том, что у сходящегося ряда при заданном \(\epsilon\)
  начиная с \(n_0\) весь хвост попадает в \(\epsilon\)-трубу.
\end{remark}

\begin{remark}
  Критерий не удобен для исследования на сходимость, поэтому обычно используют
  признаки сходимости.
\end{remark}

\subheader{Достаточные условия (признаки) сходимости
\underline{знакоположительных} рядов}

\begin{remark}
  Будем рассматривать только ряды, в которых \(u_n > 0\), но описанные далее
  признаки можно применять для любых рядов, предварительно навесив модуль.
\end{remark}

\begin{theorem}[Признак сравнения в неравенствах]
  Пусть \(\display{\sum u_n}\)~--- исследуемый ряд, а \(\display{\sum v_n}\)~---
  вспомогательный ряд и \(u_n, v_n \ge 0\). Тогда

  \begin{align*}
    \begin{rcases}
      \forall n \in \NN \given u_n < v_n \\
      \sum v_n \converge
    \end{rcases}
    \implies
    \sum u_n \converge
    \label{thr:cmp-ineq-1} \tag{1}
  \\
    \begin{rcases}
      \forall n \in \NN \given u_n > v_n \\
      \sum v_n \notconverge
    \end{rcases}
    \implies
    \sum u_n \notconverge
    \label{thr:cmp-ineq-2} \tag{2}
  \end{align*}
\end{theorem}

\begin{proof}
  Сначала докажем \eqref{thr:cmp-ineq-1}. Пусть \(S_n = u_1 + u_2 + \dotsc\) и
  \(\sigma_n = v_1 + v_2 + \dotsc\), т.к. \(\forall n \in \NN \given u_n <
  v_n\), то \(S_n \le \sigma_n\). Причем эти последовательности возрастают, т.к.
  ряды знакоположительные. Далее

  \begin{equation*}
    \sum v_n \converge \implies
    \exists \lim_{n \to \infty} \sigma_n = \sigma \in \RR
  \end{equation*}

  Таким образом последовательность \(\seq{\sigma_n}\) ограничена числом
  \(\sigma\). Последовательность \(\seq{S_n}\) возрастает и также ограничена
  числом \(\sigma\). Значит по т. Вейерштрасса \(\display{\exists \lim_{n \to
  \infty} S_n = S}\), причем \(S \le \sigma\).

  Теперь от противного докажем \eqref{thr:cmp-ineq-2}. Пусть \(\display{\sum
  u_n}\) сходится, тогда согласно \eqref{thr:cmp-ineq-1} \(\display{\sum v_n}\)
  тоже должен сходится. Противоречие.
\end{proof}

\begin{remark}
  Для установления расходимости ряда в качестве вспомогательного не следует
  брать ряды с несуществующей как предел суммой.
\end{remark}

\begin{example}
  \begin{equation*}
    \begin{aligned}
      \sum \frac{1}{n^2}
      \qquad
      u_n = \frac{1}{n^2}
    \\
      \sum \frac{1}{n (n + 1)} = 1 \in \RR
      \qquad
      v_n = \frac{1}{n (n + 1)}
    \end{aligned}
  \end{equation*}

  Неравенство \(\display{\frac{1}{n^2} < \frac{1}{n (n + 1)}}\) неверно, однако
  заметим, что

  \begin{equation*}
    \frac{1}{n (n + 1)}
    = \frac{1}{n^2 + n}
    > \frac{1}{n^2 + 2 n + 1}
    = \frac{1}{(n + 1)^2}
  \end{equation*}

  Таким образом по признаку сравнения ряд \(\display{\sum_{n = 0} \frac{1}{(n +
  1)^2}}\) сходится. Если перенумеровать, то получим, что и ряд
  \(\display{\sum_{n = 1} \frac{1}{n^2}}\) сходится.
\end{example}

\begin{theorem}[Предельный признак]
  Пусть \(\display{\sum u_n}\)~--- исследуемый ряд, а \(\display{\sum v_n}\)~---
  вспомогательный ряд и \(u_n, v_n \ge 0\). Тогда, если

  \begin{equation*}
    \lim_{n \to \infty} \frac{u_n}{v_n} = q \in \RR \setminus \set{0}
  \end{equation*}

  то ряды имеют одинаковую сходимость.
\end{theorem}

\begin{proof}
  Распишем предел по определению, после чего раскроем получившийся модуль
  (учитывая то, что ряды знакоположительные).

  \begin{equation*} \label{thr:lim-attr-1} \tag{1}
    \begin{aligned}
      \lim_{n \to \infty} \frac{u_n}{v_n} = q \iff
      \forall \epsilon > 0 \given
        \exists n_0 \in \NN \given
        \forall n > n_0 \colon
        \abs{\frac{u_n}{v_n} - q} < \epsilon
    \\
      q - \epsilon < \frac{u_n}{v_n} < q + \epsilon
    \\
      (q - \epsilon) v_n < u_n < (q + \epsilon) v_n
    \end{aligned}
  \end{equation*}

  При достаточно малом \(\epsilon\) ряды \(\display{\sum (q + \epsilon) v_n}\),
  \(\display{\sum (q - \epsilon) v_n}\) и \(\display{\sum v_n}\) имеют
  одинаковую сходимость, т.к. домножение на ненулевую константу не влияет на
  сходимость. Применим признак сравнения.

  \begin{equation*} \label{thr:lim-attr-2} \tag{2}
    \sum v_n \notconverge \implies \sum u_n \notconverge
    \qquad
    \sum v_n \converge \implies \sum u_n \converge
  \end{equation*}

  В первом случае \(u_n\) расходится, т.к. он больше расходящегося ряда (левая
  часть неравенства \eqref{thr:lim-attr-1}), во втором случае \(u_n\) сходится,
  т.к. он меньше сходящего ряда (правая часть неравенства
  \eqref{thr:lim-attr-1}).
\end{proof}

\begin{remark}
  Т.к. \(u_n\) и \(v_n\) являются бесконечно малыми (иначе вопрос о расходимости
  ряда \(u_n\) решен, т.к. не выполнено необходимое условие сходимости
  \ref{thr:ness-ser-conv}), то в предельном признаке устанавливается порядок
  \(u_n\) по отношению к \(v_n\). Ряды имеют одинаковый характер сходимости при
  одином порядке малости.
\end{remark}

\begin{lemma}
  Пусть \(\display{\lim_{n \to \infty} \frac{u_n}{v_n} = 0 \iff u_n =
  \smallo(v_k)}\). Тогда \(\display{\sum v_n \converge \implies \sum u_n
  \converge}\).
\end{lemma}

\begin{proof}
  Распишем предел по определению.

  \begin{equation*} \label{eq:lim-attr-add-1} \tag{1}
    \lim_{n \to \infty} \frac{u_n}{v_n} = 0 \iff
      \forall \epsilon > 0 \given
      \exists n_0 \in \NN \given
      \forall n > n_0 \colon
      \abs{\frac{u_n}{v_n}} < \epsilon
    \implies
    u_n < \epsilon v_n
  \end{equation*}

  Т.к. ряд \(\display{\sum v_n}\) сходится, то выполнен критерий Коши
  (\ref{thr:crt-C}), имеем

  \begin{equation*} \label{eq:lim-attr-add-2} \tag{2}
    \forall \epsilon' > 0 \given
    \exists m_0 \in \NN \given
    \forall n \ge p \ge m_0 \colon
    \abs{v_p + \dotsc + v_n} < \epsilon'
  \end{equation*}

  Домножим последнее неравенство на \(\epsilon\) и объединив его с неравенством
  в \eqref{eq:lim-attr-add-1} получим, что

  \begin{equation*} \label{eq:lim-attr-add-3} \tag{3}
    \abs{u_p + \dotsc + u_n}
    < \epsilon \abs{v_p + \dotsc + v_n}
    < \under{\epsilon \epsilon'}{\tilde{\epsilon}}
  \end{equation*}

  Подставим это в \eqref{eq:lim-attr-add-2}, получим

  \begin{equation*}
    \forall \tilde{\epsilon} > 0 \given
    \exists m_0 \in \NN \given
    \forall n \ge p \ge m_0 \colon
    \abs{u_p + \dotsc + u_n} < \tilde{\epsilon}
  \end{equation*}

  Значит ряд \(\display{\sum u_n}\) сходится по критерию Коши.
\end{proof}

\begin{remark}
  Если в отношении общих членов ряда получилась бесконечность, то лучше
  использовать другие признаки.
\end{remark}

\begin{theorem}[Признак Даламбера]
  \begin{equation*}
    \lim_{n \to \infty} \frac{u_{n + 1}}{u_n}
    = D \in \RR
    = \begin{cases}
      0 < D < 1 & \implies \converge \\
      D = 1 & \implies \text{ необходимо дополнительное исследование} \\
      D > 1 & \implies \notconverge
    \end{cases}
  \end{equation*}
\end{theorem}

\begin{proof}
  Распишем предел по определению.

  \begin{equation*} \label{eq:dAlembert-attr-1} \tag{1}
    \lim_{n \to \infty} \frac{u_{n + 1}}{u_n} = D \iff
    \forall \epsilon > 0 \given
    \exists n_0 \in \NN \given
    \forall n > n_0 \colon
    \abs{\frac{u_{n + 1}}{u_n} - D} < \epsilon
    \implies
    (D + \epsilon) u_n < u_{n + 1} < (D + \epsilon) u_n
  \end{equation*}

  Рассмотрим правую часть полученного неравенства. Положим \(D + \epsilon = r <
  1\). Тогда \(u_{n + 1} < r \cdot u_n\) начиная с \(n_0\). Имеем

  \begin{equation*} \label{eq:dAlembert-attr-2} \tag{2}
    \begin{rcases}
      u_{n_0 + 1} < r u_{n_0} \\
      u_{n_0 + 2} < r u_{n_0 + 1} < r^2 u_{n_0} \\
      \dotsc \\
      u_{n_0 + k} < r^k u_{n_0}
    \end{rcases}
    \implies
    \begin{cases}
      \sum u_n = u_1 + u_2 + \dotsc + u_{n_0}
        + u_{n_0 + 1} + \dotsc + u_{n_0 + k} + \dotsc
      \\
      \sum v_n = \under{u_1 + u_2 + \dotsc + u_{n_0}}{\text{отбросим}}
        + r u_{n_0} + \dotsc + r^k u_{n_0} + \dotsc
      \\
      u_n \le v_n
    \end{cases}
  \end{equation*}

  Отбросим \quote{голову} полученных рядов до члена \(u_{n_0}\) включительно.
  Тогда из ряда \(\display{\sum v_n}\) получим ряд

  \begin{equation*} \label{eq:dAlembert-attr-3} \tag{3}
    \sum v'_n
    = \sum_{k = 1}^{\infty} r^k \under{u_{n_0}}{const}
    = u_{n_0} \sum_{k = 1}^{\infty} r^k
  \end{equation*}

  Получаем эталонный геометрический ряд, т.к. \(r < 1\) то он сходится. Значит
  по признаку сравнения сходится и ряд \(\sum u'_n\), полученный отбрасыванием
  \quote{головы} ряда \(\sum u_n\). Отбрасывание конечного числа членов ряда не
  влияет на его сходимость, а значит ряд \(\sum u_n\) также сходится.

  Аналогично рассмотрим левую часть неравенства \eqref{eq:dAlembert-attr-1} и
  положим \(D - \epsilon = r > 1\). Оценим члены ряда снизу вспомогательным
  рядом \(\display{\sum v'_{n_0 + k} = r^k u_{n_0}}\). При \(r > 1\) исходный
  ряд почленно больше расходящегося, значит тоже расходится.
\end{proof}
