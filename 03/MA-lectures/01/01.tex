\subsection{%
  Лекция \texttt{23.09.01}.%
}

\begin{definition}
  Числовым рядом называется выражение \(u_1 + u_2 + \dotsc + u_n\), где
  \(\seq{u_n}\) это некоторая числовая последовательность. Обозначается
  \(\sum_{n = 1}^{\infty} u_n\).
\end{definition}

\begin{remark}
  Нумерация может вестись с любого целого числа.
\end{remark}

\begin{definition}
  \(u_n\) называется общим членом ряда.
\end{definition}

\begin{definition}
  \(S_n = u_1 + \dotsc + u_k\) называется частичной суммой ряда.
\end{definition}

\begin{remark}
  \(S_n\) также образуют последовательность.
\end{remark}

\begin{definition}
  Если последовательность частичных сумм сходится, т.е. \(\display{\lim_{n \to
  \infty} S_n = S \in \RR}\), то говорят, что ряд сходится к сумме \(S\) (\(S\)
  называется суммой ряда). Если предел равен бесконечности или не существует, то
  ряд расходится.
\end{definition}

Иногда сумму ряда можно найти простой арифметикой.

\begin{example}[Непосредственное вычисление суммы ряда]
  \begin{equation*}
    \begin{aligned}
      u_n = \frac{1}{n (n + 1)}
      \implies
      \sum_{n = 1}^{\infty} \frac{1}{n (n + 1)}
      = \sum_{n = 1}^{\infty} \prh{\frac{1}{n} - \frac{1}{n + 1}}
    \\
      S_n
      = \sum_{k = 1}^{n} \prh{\frac{1}{k} - \frac{1}{k + 1}}
      = \under{\prh{1 - \frac{1}{2}}}{k = 1}
        + \under{\prh{\frac{1}{2} - \frac{1}{3}}}{k = 2}
        + \under{\prh{\frac{1}{3} - \frac{1}{4}}}{k = 3}
        + \dotsc +
        + \under{\prh{\frac{1}{n} - \frac{1}{n}}}{k = n - 1}
        + \under{\prh{\frac{1}{n} - \frac{1}{n + 1}}}{k = n}
      = 1 - \frac{1}{n + 1}
    \\
      \lim_{n \to \infty} S_n = 1 = S
    \end{aligned}
  \end{equation*}
\end{example}

\begin{example}[Геометрический ряд (эталонный)]
  Пусть \(b \neq 0, b \in \RR\).

  \begin{equation*}
    \begin{aligned}
      \sum_{n = 0}^{\infty} b q^n
      = b + b q + b q^2 + b q^3 + \dotsc + b q^n
      = b (1 + q + q^2 + q^3 + \dotsc + q^n)
      = b \frac{1 - q^{n + 1}}{1 - q}
      = S_n
    \\
      \lim_{n \to \infty} S_n
      = \frac{b}{1 - q} \lim_{n \to \infty} (1 - q^{n + 1})
    \end{aligned}
  \end{equation*}

  Далее значение предела зависит от \(q\).

  \begin{enumerate}
  \item
    \(\display{
      \abs{q} < 1
    \implies
      q^n \to 0
    \implies
      \lim_{n \to \infty} S_n = \frac{b}{1 - q} = S
    }\)

  \item
    \(\display{
      \abs{q} > 1
      \implies
      q^n \to \infty
      \implies
    }\) ряд расходится.

  \item
    \(\display{
      q = 1
      \implies
      S_n = b (n + 1) \to \infty
      \implies
    }\) ряд расходится.
  
  \item
    \(\display{
      q = -1
      \implies
      S_n
      = \frac{b}{2} (1 + 1 - 1 + \dotsc + 1 - 1)
      = \begin{cases}
        b \\
        0
      \end{cases}
      \implies
    }\) две подпоследовательность сходятся к разным числам, значит предела нет и
    ряд расходится.
  
  \end{enumerate}
\end{example}

\begin{remark}
  Чаще требуется только определить сходимость ряда не вычисляя его сумму.
\end{remark}

\subheader{Свойства числовых рядов}

\begin{theorem} \label{thr:srs-discard}
  \begin{equation*}
    \begin{aligned}
      \sum_{n = 1}^{\infty} u_n \converge
      \iff
      \sum_{n = k > 1}^{\infty} u_n \converge
    \\
      \sum_{n = 1}^{\infty} u_n \diverge
      \iff
      \sum_{n = k > 1}^{\infty} u_n \diverge
    \end{aligned}
  \end{equation*}
\end{theorem}

\begin{proof}
  \begin{equation*}
    \begin{aligned}
      \sum_{n = 1}^{\infty} u_n \converge
      \iff
      \exists \lim_{n \to \infty} S_n \in \RR
    \\
      \lim_{n \to \infty} S_n
      = \lim_{n \to \infty} \prh{
        \under{u_1 + u_2 + \dotsc + u_k}{v} + u_{k + 1} + \dotsc + u_n
      }
      = \under{\lim_{n \to \infty} v}{v \in \RR} + \lim_{n \to \infty} \prh{
        u_{k + 1} + \dotsc + u_n
      }
    \end{aligned}
  \end{equation*}

  Для расходящихся доказательство аналогично.
\end{proof}

\begin{remark}
  Теорему \ref{thr:srs-discard} можно сформулировать по-другому (не формально):
  ряд и его \quote{хвост} одновременно сходятся и расходятся.
\end{remark}

\begin{theorem}
  \begin{equation*}
    \begin{rcases}
      \sum_{n = 1}^{\infty} u_n = S \in \RR \\
      \alpha \in \RR
    \end{rcases}
    \implies
    \sum_{n = 1}^{\infty} \alpha u_n = \alpha S
  \end{equation*}
\end{theorem}

\begin{proof}
  \begin{equation*}
    \begin{aligned}
      \sum_{n = 1}^{\infty} u_n \converge
      \iff
      \exists \lim_{n \to \infty} S_n \in \RR
    \\
      \lim_{n \to \infty} \prh{\alpha u_1 + \dotsc + \alpha u_n}
      = \alpha \lim_{n \to \infty} \prh{u_1 + \dotsc + u_n}
      = \alpha S
    \end{aligned}
  \end{equation*}
\end{proof}

\begin{remark}
  Если ряд расходится, то умножение на \(\alpha \neq 0\) не меняет его
  расходимости.
\end{remark}

\begin{theorem}
  \begin{equation*}
    \begin{rcases}
      \sum_{n = 1}^{\infty} u_n = S \in \RR \\
      \sum_{n = 1}^{\infty} v_n = \sigma \in \RR \\
    \end{rcases}
    \implies
    \sum_{n = 1}^{\infty} (u_n \pm v_n) = S \pm \sigma
  \end{equation*}
\end{theorem}

\begin{proof}
  \begin{equation*}
    \under{\lim_{n \to \infty} S_n}{S} \pm
      \under{\lim_{n \to \infty} \sigma_n}{\sigma}
    = \lim_{n \to \infty} \prh{S_n \pm \sigma_n}
    = \sum_{n = 1}^{\infty} \prh{u_n \pm v_n}
  \end{equation*}
\end{proof}

\begin{remark}
  Ряды складываются и вычитаются почленно.
\end{remark}

\begin{remark}
  Из сходимости разности рядов \textbf{не следует} сходимость самих рядов.
  Например,

  \begin{equation*}
    \sum_{n = 1}^{\infty} \frac{1}{n (n + 1)}
    = \sum_{n = 1}^{\infty} \prh{\frac{1}{n} - \frac{1}{n + 1}}
      \neq \under{\sum_{n = 1}^{\infty} \frac{1}{n}
      - \sum_{n = 1}^{\infty} \frac{1}{n + 1}}{\text{расходятся}}
  \end{equation*}
\end{remark}


\subheader{Гармонический ряд (эталонный)}

\begin{equation*}
  \sum_{n = 1}^{\infty} \frac{1}{n}
  = \frac{1}{1} + \frac{1}{2} + \frac{1}{3} + \frac{1}{4} + \frac{1}{5}
    + \frac{1}{6} + \frac{1}{7} + \frac{1}{8} + \frac{1}{9} + \frac{1}{10}
    + \frac{1}{11} + \frac{1}{12} + \frac{1}{13} + \frac{1}{14} + \frac{1}{15}
    + \frac{1}{16} + \dotsc
\end{equation*}

Рассмотрим вспомогательный ряд и вычислим его частичные суммы

\begin{equation*}
  \begin{aligned}
    \frac{1}{1} + \frac{1}{2} + \under{\frac{1}{4} + \frac{1}{4}}{\frac{1}{2}}
    + \under{\frac{1}{8} + \frac{1}{8} + \frac{1}{8} + \frac{1}{8}}{\frac{1}{2}}
    + \under{\frac{1}{16} + \frac{1}{16} + \frac{1}{16} + \frac{1}{16}
    + \frac{1}{16} + \frac{1}{16} + \frac{1}{16} + \frac{1}{16}}{\frac{1}{2}}
    + \dotsc
  \\
    \sigma_1 = 1 + 0 \cdot \frac{1}{2} \qquad
    \sigma_2 = 1 + 1 \cdot \frac{1}{2} \qquad  
    \sigma_n = 1 + (n - 1) \cdot \frac{1}{2}
  \end{aligned}
\end{equation*}

Последовательность частичных сумм \(\sigma_n\) расходится при \(n \to \infty\).
Последовательность частичных сумм исходного ряда почленно не меньше
\(\sigma_n\), значит \(\display{\lim_{n \to \infty} S_n = \infty}\).

\begin{theorem}
  Члены сходящегося ряда можно группировать произвольным образом \textbf{не
  переставляя}.
\end{theorem}

\begin{proof}
  Группируя члены ряда получаем подпоследовательность последовательности
  частичных сумм. Если существует предел исходной последовательности, то
  существует и предел любой ее подпоследовательности.
\end{proof}

\begin{remark}
  Перестановка членов ряда может изменить сумму. Например, рассмотрим ряд
  \(\display{\sum_{n = 1}^{\infty} \frac{(-1)^{n + 1}}{n}}\). Он сходится
  (без доказательства). Далее имеем

  \begin{equation*}
    \begin{aligned}
      \sum_{n = 1}^{\infty} \frac{(-1)^{n + 1}}{n}
      & = 1 - \frac{1}{2} + \frac{1}{3} - \frac{1}{4} + \frac{1}{5}
        - \frac{1}{6} + \frac{1}{7} - \frac{1}{8} + \frac{1}{9} - \frac{1}{10}
        + \frac{1}{11} - \frac{1}{12} + \frac{1}{13} - \frac{1}{14} 
        + \frac{1}{15} - \frac{1}{16} + \dotsc
    \\
      & = \prh{1 - \frac{1}{2}} - \frac{1}{4}
        + \prh{\frac{1}{3} - \frac{1}{6}} - \frac{1}{8}
        + \prh{\frac{1}{5} - \frac{1}{10}} - \frac{1}{16}
        + \prh{\frac{1}{7} - \frac{1}{14}} - \frac{1}{32}
        + \dotsc
    \\
      & = \prh{1 - \frac{1}{2}} - \frac{1}{4}
        + \frac{1}{3} \cdot \prh{1 - \frac{1}{2}} - \frac{1}{8}
        + \frac{1}{5} \cdot \prh{1 - \frac{1}{2}} - \frac{1}{16}
        + \frac{1}{7} \cdot \prh{1 - \frac{1}{2}} - \frac{1}{32}
        + \dotsc
    \\
      & = \frac{1}{2} \prh{1 + \frac{1}{3} + \frac{1}{5} + \frac{1}{7} + \dotsc}
        - \frac{1}{4} - \frac{1}{8} - \frac{1}{16} - \frac{1}{32} - \dotsc
    \\
      & = \frac{1}{2} \prh{1 - \frac{1}{2} + \frac{1}{3} - \frac{1}{4} +
        \frac{1}{5} - \frac{1}{6} + \frac{1}{7} + \dotsc}
    \\
      & = \frac{1}{2} \sum_{n = 1}^{\infty} \frac{(-1)^{n + 1}}{n}
    \end{aligned}
  \end{equation*}
  
  Получили, что сумма ряда равна своей половине.
\end{remark}
