\subsection{%
  Лекция \texttt{23.10.06}.%
}

\begin{theorem}[Формула Тейлора]
  \begin{equation*}
    \begin{rcases}
      f(x) \isdiff[\infty]{\near{x_0}{\delta}} \\
      \forall x \in \interval{-R}{R}
        \sum_{n = 0}^{\infty} \frac{f^{(n)}(x_0)}{n!} (x - x_0)^n = T \\
      \lim_{n \to \infty} R_{n + 1}(x) = 0
    \end{rcases}
    \implies
    f(x) = T
  \end{equation*}
\end{theorem}

\begin{proof}
  \begin{equation*} \label{eq:T-ser-1} \tag{1}
    f(x)
    = T_n (x) + R_{n + 1} (x)
    = \sum_{k = 1}^{n}  \frac{f^{(k)} (x_0)}{k!} (x - x_0)^k
      + \frac{f^{(n + 1)} (\xi)}{(n + 1)!} (x - x_0)^{n + 1}
  \end{equation*}

  Заметим, что \(T_n (x)\) это частичная сумма ряда \eqref{eq:T-ser-2}. 
  
  \begin{equation*} \label{eq:T-ser-2} \tag{2}
    \sum_{n = 0}^{\infty} \frac{f^{(n)} (x_0)}{n!} (x - x_0)^n
  \end{equation*}

  Т.к. ряд степенной, то можно найти радиус сходимости. Пусть в круге радиуса
  \(R\) ряд Тейлора \eqref{eq:T-ser-2} сходится к сумме \(T\), тогда перейдем к
  пределу

  \begin{equation*}
    \lim_{n \to \infty} f(x)
    = \under{\lim_{n \to \infty} T_n (x)}{T}
      + \under{\lim_{n \to \infty} R_{n + 1} (x)}{= 0}
    \implies f(x) = T
  \end{equation*}
\end{proof}

\begin{definition}[Ряд Маклорена]
  \begin{equation*}
    \sum_{n = 0}^{\infty} \frac{f^{(n)} (0)}{n!} x^n
  \end{equation*}

  Такое разложение \(f(x)\) называется стандартным.
\end{definition}

\subheader{Стандартные разложения элементарных фукнций}

\subsubheader{I}{\(f(x) = e^x\)}

\begin{equation*}
  \begin{aligned}
    e^x
    = \sum_{n = 0}^{\infty} \frac{x^n}{n!}
    = 1 + \frac{x}{1!} + \frac{x^2}{2!} + \frac{x^3}{3!} + \dotsc
  \\
    R_{n + 1} (x) = \frac{e^{\xi}}{(n + 1)!} x^{n + 1}
    \Rarr{n \to \infty} 0
  \\
    \lim_{n \to \infty} \abs{\frac{u_{n + 1}}{u_n}}
    = \frac{\abs{x}}{n + 1} < 1
    \implies
    \text{область сходимости } \RR
  \end{aligned}
\end{equation*}

\subsubheader{II}{\(f(x) = \sin x\)}

\begin{equation*}
  \begin{aligned}
    f^{(n)} (x) = \sin \prh{x + \frac{\pi n}{2}}
  \\
    \sin x
    = \sum_{n = 0}^{\infty} \frac{\sin \prh{\frac{\pi n}{2}}}{n!}
    = \sum_{k = 0}^{\infty} \frac{(-1)^k}{(2 k + 1)!} x^{2 k + 1}
    = \frac{x}{1!} - \frac{x^3}{3!} + \frac{x^5}{5!}
  \\
    R_{n + 1} (x)
    = \frac{\sin \prh{\frac{\pi}{2} (n + 1) + \xi}}{(n + 1)!} x^{n + 1}
    \le \frac{x^{n + 1}}{(n + 1)!}
    \Rarr{n \to \infty} 0
  \\
    \lim_{k \to \infty} \abs{\frac{u_{k + 1}}{u_k}}
    = \lim_{k \to \infty} \frac{x^2}{(2 k + 2) (2 k + 3)}
    = 0
    \implies
    \text{область сходимости } \RR
  \end{aligned}
\end{equation*}

\subsubheader{III}{\(f(x) = \cos x\)}

\begin{equation*}
  \begin{aligned}
    \cos x
    = \sum_{k = 0}^{\infty} \frac{(-1)^k}{(2 k)!} x^{2 k}
    = 1 - \frac{x^2}{2!} + \frac{x^4}{4!} - \frac{x^6}{6!} + \dotsc
  \\
    R_{n + 1} (x) \Rarr{n \to \infty} 0
  \\
    \text{область сходимости } \RR
  \end{aligned}
\end{equation*}

\subsubheader{IV.1}{\(f(x) = \sinh x\)}

\begin{equation*}
  \sinh x
  = \frac{e^x - e^{-x}}{2}
  = \frac{1}{2} \prh{\sum_{n = 0}^{\infty} \frac{x^n}{n!}
    - \sum_{n = 0}^{\infty} \frac{(-x)^n}{n!}}
  = x + \frac{x^3}{3} + \frac{x^5}{5} + \dotsc
  = \sum_{n = 0}^{\infty} \frac{x^{2 n + 1}}{(2 n + 1)!}
\end{equation*}

\subsubheader{IV.2}{\(f(x) = \cosh x\)}

\begin{equation*}
  \cosh x
  = \frac{e^x + e^{-x}}{2}
  = 1 + \frac{x^2}{2} + \frac{x^4}{4} + \dotsc
  = \sum_{n = 0}^{\infty} \frac{x^{2 n}}{(2 n)!}
\end{equation*}

\begin{remark}
  \begin{equation*}
    e^{i \pi}
    = \sum_{(i \pi)^n}^{n!} 
    = 1 + \frac{i \pi}{1!} - \frac{\pi^2}{2!} - \frac{i \pi^3}{3!}
      + \frac{\pi^4}{4!}
    = \under{\prh{1 - \frac{\pi^2}{2!} + \frac{\pi^4}{4!} - \dotsc}}{\cos \pi}
      + i \under{\prh{\frac{\pi}{1!} - \frac{\pi^3}{3!} + \frac{\pi^5}{5!}
        - \dots}}{\sin \pi}
    = -1 + 0
    = -1
  \end{equation*}

  Или, если обобщить, \(e^{i x} = \cos x + i \sin x\).
\end{remark}

\subsubheader{V}{Биномиальный ряд \(f(x) = (1 + x)^m\), \(m \in \RR\)}

\begin{remark}
  Представление остаточного члена в форме Лагранжа и доказательство его
  сходимости к нулю это сложная задача, поэтому получим представление другим
  способом.
\end{remark}

\begin{equation*}
  \begin{aligned}
    f'(x) = m (1 + x)^{m - 1}
  \\
    (1 + x) f'(x) = m (1 + x)^m = m f(x)
  \\
    \begin{cases}
      (1 + x) f'(x) = m f(x) \\
      f(0) = 1
    \end{cases}
  \end{aligned}
\end{equation*}

Получили задачу Коши. Запишем ее для суммы ряда.

\begin{equation*}
  \begin{aligned}
    \begin{cases}
      (1 + x) S'(x) = m S(x) \\
      S(0) = 1
    \end{cases}
    \qquad
    \begin{cases}
      S(x) = 1 + a_1 x + a_2 x^2 + \dotsc + a_n x^n + \dotsc \\
      S'(x) = a_1 + 2 a_2 x + \dotsc + n a_n x^{n - 1} + \dotsc
    \end{cases}
  \\
    (1 + x) S'(x)
    = a_1 + \under{a_1 x + 2 a_2 x}{}
      + \under{2 a_2 x^2 + 3 a_3 x^2}{} + 3 a_3 x^3 + \dotsc
    = m S(x)
    = m + m a_1 x + m a_2 x^2 + m a_3 x^3 + \dotsc
  \end{aligned}
\end{equation*}

Приравняем коэффициенты.

\begin{equation*}
  \begin{aligned}
    a_1 = m
    & \qquad
    a_1 = \frac{m}{1}
  \\
    a_1 + 2 a_2 = m a_1
    & \qquad
    2 a_2 = a_1 (m - 1) \implies a_2 = \frac{m (m - 1)}{2}
  \\
    2 a_2 + 3 a_3 = m a_2
    & \qquad
    3 a_3 = m a_2 - 2 a_2 \implies a_3 = \frac{m (m - 1) (m - 2)}{2 \cdot 3}
  \\
    \vdots
    & \qquad
    \vdots
  \\
    & \qquad
    a_n
    = \frac{m (m - 1) \dotsc (m - n + 1)}{n!}
    = \frac{m!}{n! (m - n)!}
    = \binom{m}{n}
  \end{aligned}
\end{equation*}

Итого \((1 + x)^m = \sum_{n = 0}^{\infty} \comb{m}{n} x^n\). Выясним радиус
сходимости.

\begin{equation*}
  \begin{aligned}
    u_{n - 1} = \frac{m (m - 1) \dotsc (m - n + 2)}{(n - 1)!} x^{n - 1}
    \qquad
    u_n = \frac{m (m - 1) \dotsc (m - n + 1)}{n!} x^n
  \\
    \lim_{n \to \infty} \abs{\frac{u_n}{u_{n - 1}}}
    = \lim_{n \to \infty} \abs{\frac{(m - n + 1) x}{n}}
    = \abs{x} \lim_{n \to \infty} \abs{\frac{m}{n} - 1 + \frac{1}{n}}
    = \abs{x} < 1
  \end{aligned}
\end{equation*}

Таким образом, область сходимости \(\interval{-1}{1}\).

\begin{remark}
  В некоторых случаях (например \(m \in \ZZ^-\)) \(x = 1\) входит в область
  сходимости.
\end{remark}

\subsubheader{VI}{\(f(x) = \ln (1 + x)\)}

Заметим, что \(f'(x) = \frac{1}{1 + x} = (1 + x)^{-1}\) это бином. Получаем

\begin{equation*}
  \begin{aligned}
    f'(x)
    = \sum_{n = 0}^{\infty} \comb{m}{n} x^n
    = 1 - x + x^2 - x^3 + \dotsc
  \\
    f(x)
    = \ln (1 + x)
    = \int_0^x (1 - x + x^2 - x^3 + \dotsc) \dd x
    = x - \frac{x^2}{2} + \frac{x^3}{3} - \frac{x^4}{4} + \dotsc
    = \sum_{n = 1}^{\infty} \frac{(-1)^{n + 1} x^n}{n!}
  \end{aligned}
\end{equation*}

Область сходимости, как и у бинома, будет равна \(\interval{-1}{1}\).

\begin{remark}
  Если взять \(x = \frac{1}{k}\), то тогда

  \begin{equation*}
    \ln \prh{1 + x}
    = \ln \prh{1 + \frac{1}{k}}
    = \ln \prh{\frac{k + 1}{k}}
    = \ln \prh{k + 1} - \ln k
  \end{equation*}

  Т.е. можно рекурсивно получать значения натуральных логарифмов с помощью
  рядов.
\end{remark}

\begin{remark}[О применении к приближенным вычислениям]
  Рассмотрим \quote{неберущийся} интеграл.

  \begin{equation*}
    \begin{aligned}
      \int_0^a \frac{\sin x}{x} \dd x
      & = \int_0^a \frac{1}{x}
        \prh{\sum_{n = 0}^{\infty} \frac{(-1)^n x^{2 n + 1}}{(2 n + 1)!}} \dd x
    \\
      & = \int_0^a \sum_{n = 0}^{\infty} \frac{(-1)^n x^{2 n}}{(2 n + 1)!} \dd x
    \\
      & = \sum_{n = 0}^{\infty} \frac{(-1)^n}{(2 n + 1)!} \int_0^a x^{2 n} \dd x
    \\
      & = \sum_{n = 0}^{\infty} \frac{(-1)^n}{(2 n + 1)!}
        \cdot \frac{a^{2 n + 1}}{(2 n + 1)}
    \end{aligned}
  \end{equation*}

  Рассмотрим другой \quote{неберущийся} интеграл.

  \begin{equation*}
    \int_0^a e^{-x^2} \dd x
    = \int_0^a \sum_{n = 0}^{\infty} \frac{(-x^2)^n}{n!} \dd x
    = \sum_{n = 0}^{\infty} \frac{(-1)^n}{n!} \cdot \frac{a^{2 n + 1}}{2 n + 1}
  \end{equation*}
\end{remark}