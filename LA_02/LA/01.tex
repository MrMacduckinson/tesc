\question{Евклидово пространство: определение, неравенство Коши-Буняковского. Нормированное евклидово пространство.}

\begin{definition}
  Скалярным произведением называется функция двух элементов линейного
  пространства \(x, y \in L^{n}\) обозначаемая \((x, y) \to \RR\),
  для которой выполнены аксиомы: \(\forall x, y \in L^{n}, \lambda \in \CC\)

  \begin{enumerate}
    \item \((x, y) = (y, x)\)
    \item \((\lambda x, y) = \lambda (x, y)\)
    \item \((x_{1} + x_{2}, y) = (x_{1}, y) + (x_{2}, y)\)
    \item \((x, x) \ge 0\), \((x, x) = 0 \implies x = 0\)
  \end{enumerate}
\end{definition}

\begin{definition}
  Линейное пространство с введенным скалярным произведением называется
  евклидовым пространством \(E^{n}\).
\end{definition}

\begin{remark}
  Если \(L = C_{[a; b]}\), то скалярное произведение обычно определяется как
  \((f, g) = \int_{a}^{b} f(x) g(x) \dd x\)
\end{remark}

\begin{theorem}
  Неравенство Коши-Буняковского
  \begin{lequation}{CB}
    (x, y)^{2} \le (x, x) (y, y)
  \end{lequation}
\end{theorem}
\begin{proof}
  Рассмотрим скалярное произведение:

  \begin{lequation}{CB-proof-1}
    (\lambda x - y, \lambda x - y) \ge 0 \\
    (\lambda x - y, \lambda x - y)
    = \lambda^2 (x, x) - 2 \lambda (x, y) + (y, y) \ge 0
  \end{lequation}
  
  Полученное выражение можно рассмотреть как квадратное уравнение относительно
  \(\lambda\). Т.к. оно неотрицательно \(\forall \lambda\), то его дискриминант
  будет \(\le 0\). Таким образом

  \begin{lequation}{CB-proof-2}
    4 \lambda^2 (x, y)^2 - 4 \lambda^2 (x, x) (y, y) \le 0 \\
    (x, y)^2 - (x, x) (y, y) \le 0 \\
    (x, y)^2 \le (x, x) (y, y)
  \end{lequation}
\end{proof}

\begin{definition}
  Нормой называется функция одного элемента линейного пространства
  \(x \in L^{n}\), обозначаемая \(\norm{x}\) и определяемая аксиомами:
  \(\forall x, y \in L^{n}, \lambda \in \CC\):

  \begin{enumerate}
    \item \(\norm{\lambda x} = \lambda \norm{x}\)
    \item \(\norm{x + y} \le \norm{x} + \norm{y}\)
    \item \(\norm{x} \ge 0\), \(\norm{x} = 0 \implies x = 0\)
  \end{enumerate}
\end{definition}

\begin{definition}
  Евклидово пространство называется \textit{нормированным}, если в нем
  определена норма.
\end{definition}

\begin{remark}
  Чаще всего норма определяется как \(\norm{x} = \sqrt{(x, x)}\).
\end{remark}

