\question{Ортогональная матрица и ортогональный оператор. Поворот плоскости и пространства как ортогональное преобразование.}

\begin{definition}
  Оператор \(\opA\) называется ортогональным, если

  \begin{lequation}{ort-lo-1}
    (\opA x, \opA y) = (x, y)
  \end{lequation}
\end{definition}

\begin{definition}
  Альтернативное определение: \(\opA\) называется ортогональным, если его
  матрица ортогональна:

  \begin{lequation}{ort-lo-2}
    A^{-1} = A^{T}
  \end{lequation}
\end{definition}

\begin{remark}
  Ортогональный оператор сохраняет скалярное произведение, т.е. не меняет норму
  элементов. Таким образом к ортогональным преобразованиям можно отнести
  параллельный перенос, поворот и осевую симметрию.

  Также ортогональное преобразование переводит ортонормированный базис в
  ортонормированный.
\end{remark}

Некоторые примеры ортогональных преобразований:
\begin{enumerate}
\item Поворот на угол \(\alpha\): \(A = \begin{pmatrix}
  \cos \alpha & \sin \alpha \\
  -\sin \alpha & \cos \alpha
\end{pmatrix}\)

\item Осевая симметрия относительно \(Ox\): \(A = \begin{pmatrix}
  1 & 0 \\
  0 & -1
\end{pmatrix}\)
\end{enumerate}

Матрицы выше составлены из базисных векторов
\(\basis_{i} \bot \basis_{j}\),
\(\norm{\basis_{i}} = 1\).

В пространстве $\RR^3$ существуют движения: перенос, поворот,
зеркальная симметрия относительно осевой плоскости

При этом ортонормированный базис переходит в ортонормированный

\begin{remark}
  Заметим, что для $\opA = \opA^*$ и $T$ (оператор с ортогональной матрицей) верно:
  $\opA \opA^* = \opA^* \opA$ и $TT^* = TT^{-1} = I = T^{-1}T = T^*T$
\end{remark}

\begin{definition}
  Оператор $\opA$ такой, что $\opA \opA^* = \opA^* \opA$ называется нормальным
\end{definition}

\begin{remark}
  Можно доказать, что для нормальных операторов собственные числа и векторы $\opA$ и $\opA^*$ совпадают
\end{remark}