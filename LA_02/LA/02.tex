\question{Ортонормированный базис, ортогонализация базиса. Матрица Грама.}

\begin{definition}
  \textit{Углом} между двумя элементами Евклидова пространства называется
  \begin{lequation}{angle-def}
    \cos \angle (x, y) = \frac{(x, y)}{\norm{x} \cdot \norm{y}}
  \end{lequation}
\end{definition}

\begin{definition}
  Для элемента Евклидова пространства ортогональны, если их скалярное
  произведение равно нулю.
  \begin{lequation}{ort-def}
    x \bot y \iff (x, y) = 0
  \end{lequation}
\end{definition}

\begin{theorem}
  Во всяком \(E^{n}\) можно выделить ортонормированный базис размера \(n\).
\end{theorem}
\begin{proof}
  Пусть у нас есть базис \(B = \{ \beta_{1}, \dotsc, \beta_{n} \}\).
  Ортогонализируем его, полученный базис обозначим
  \(\Basis' = \{ \basis'_{1}, \dotsc, \basis'_{n} \}\).
  Этот базис можно нормировать и получить искомый ортонормированный базис
  \(\Basis = \{ \basis_{1}, \dotsc, \basis_{n} \}\).

  \textbf{Процесс ортогонализации Грама-Шмидта:}

  Будем добавлять векторы в базис \(\Basis'\) из базиса \(B\) по-одному:

  \textbf{База:} начнем с одного произвольного вектора \(\beta_1\).
  Тогда \(\basis'_{1} = \beta_{1}\).

  \textbf{Переход:} пусть у нас уже выделен набор из \(k - 1\) ортогональных
  векторов \(\{ \basis'_{1}, \dotsc, \basis'_{k - 1} \}\) и в него требуется
  добавить вектор \(\beta_{k}\).

  Будем искать \(\basis'_{k}\) в виде

  \begin{lequation}{ort-basic-proof-1}
    \basis'_{k}
      = \beta_{k}
      + \lambda_{1} \basis'_{k - 1}
      + \lambda_{2} \basis'_{k - 2}
      + \dotsc
      + \lambda_{k - 1} \basis'_{1}
  \end{lequation}

  Чтобы \(\basis'_{k}\) был ортогонален остальным векторам уже построенной
  системы, необходимо, чтобы скалярные произведение \(\basis'_{k}\) с остальными
  векторами системы равнялись нулю. Рассмотрим на примере \(\basis'_{1}\):

  \begin{lequation}{ort-basic-proof-2}
    (\basis'_{k}, \basis'_{1})
      = (\beta_{k}, \basis'_{1})
      + \lambda_{1} (\basis'_{k - 1}, e'_{1})
      + \dotsc
      + \lambda_{k - 1} (\basis'_{1}, \basis'_{1})
      = 0
  \end{lequation}

  Учитывая то, что построенная система ортогональна, то
  \((\basis'_{i}, \basis'_{j}) = 0 \; (i, j < k)\).
  Значит выражение выше упрощается и остается:

  \begin{lequation}{ort-basic-proof-3}
     (\beta_{k}, \basis'_{1})
     + \lambda_{k - 1} (\basis'_{1}, \basis'_{1}) = 0 \\
     \lambda_{k - 1}
     = -\frac{(\beta_{k}, \basis'_{1})}{(\basis'_{1}, \basis'_{1})}
  \end{lequation}

  Аналогично можно получить оставшиеся коэффициенты \(\lambda_{i}\). Тогда
  добавляемый в систему вектор \(\basis'_{k}\) будет иметь вид:

  \begin{lequation}{ort-basic-proof-4}
    \basis'_{k}
    = \beta_{k}
    - \frac{(\beta_{k}, \basis'_{k - 1})}
      {(\basis'_{k - 1}, \basis'_{k - 1})} \cdot \basis'_{k - 1}
    - \dotsc
    - \frac{(\beta_{k}, \basis'_{1})}
      {(\basis'_{1}, \basis'_{1})} \cdot \basis'_{1}
  \end{lequation}

  Далее каждый вектор из получившегося ортогонального базиса поделим на его норму и получим ортонормированный базис \(\Basis\):

  \begin{lequation}{ort-basic-proof-5}
    \Basis = \Bigg\{ \frac{\basis'_{1}}{\norm{\basis'_{1}}}, \dotsc, \frac{\basis'_{n}}{\norm{\basis'_{n}}} \Bigg\} = \{ \basis_{1}, \dotsc, \basis_{n} \}
  \end{lequation}

\end{proof}

\begin{definition}
  Матрицей Грама называется матрица составленная из скалярных произведений

  \begin{lequation}{G-mtx}
    \begin{pmatrix}
      (\basis_{1}, \basis_{1}) & \dots  & (\basis_{k}, \basis_{1}) \\
      \vdots                   & \ddots & \vdots                   \\
      (\basis_{1}, \basis_{k}) & \dots  & (\basis_{k}, \basis_{k}) \\
    \end{pmatrix}
  \end{lequation}
\end{definition}

\begin{remark}
  В ортогональном базисе матрица Грама диагональная, а в ортонормированном~---
  единичная.
\end{remark}
