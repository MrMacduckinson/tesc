\question{Линейный оператор: определение, основные свойства.}

\begin{definition}
  Пусть \(V^{n}, W^{m}\) линейные пространства. Отображение
  \(\opA \colon V^{n} \to W^{m}\), которое \(\forall x \in V^{n}\) сопоставляет
  \(y \in W^{m}\) называется линейным оператором при выполнении следующих
  условий: \(\forall x_{1}, x_{2} \in V^{n}, \lambda \in \CC\):

  \begin{enumerate}
    \item \(\opA(x_{1} + x_{2}) = \opA x_{1} + \opA x_{2}\)
    \item \(\opA(\lambda x_{1}) = \lambda (\opA x_{1})\)
  \end{enumerate}
\end{definition}

\begin{remark}
  \(y = \opA x\) означает, что \(y\) порождается применением оператора \(\opA\)
\end{remark}

Обозначим некоторые базовые свойства линейных операторов. Пусть
\(\opA, \opB \colon V^{n} \to W^{m}\) это линейные операторы, тогда определены:
\begin{enumerate}
  \item Сумма \((\opA + \opB) x = \opA x + \opB x\)
  \item Умножение на число \((\lambda \opA) x = \lambda (\opA x)\)
  \item Нулевой оператор \(\opZero x = 0, \forall x \in V^{n}\)
  \item Противоположный оператор \(-\opA = (-1) \cdot \opA\)
\end{enumerate}

Далее рассмотрим операторы \(\opA, \opB, \opC \colon V^{n} \to V^{n}\)
действующие в одном линейном пространстве.
Для таких операторов определена композиция (произведение)
\((\opA \cdot \opB) x = \opA (\opB x)\).
В общем случае она не коммутативна \(\opA \cdot \opB \neq \opB \cdot \opA\).

Свойства композиции операторов:
\begin{enumerate}
  \item \(\lambda (\opA \cdot \opB) = (\lambda \opA) \cdot \opB\)
  \item \((\opA + \opB) \cdot \opC = \opA \cdot \opC + \opB \cdot \opC\)
  \item \(\opA \cdot (\opB + \opC) = \opA \cdot \opB + \opA \cdot \opC\)
  \item \(\opA \cdot (\opB \cdot C) = (\opA \cdot \opB) \cdot \opC\)
\end{enumerate}

\begin{definition}
  Композиция оператора самим с собой \(n\) раз называется \(n\)-ой степенью
  оператора: \(\opA^{n} = \underbrace{A \cdot \dotsc \cdot A}_{n}\)
\end{definition}

Для степени оператора справедливо равенство
\(\opA^{n + m} = \opA^{n} \cdot \opA^{m}\)

