\question{Структура образа самосопряженного оператора. Проектор. Спектральное разложение оператора.}

\begin{theorem}\label{sconj-lo-img}
  Образ самосопряженного оператора \(\opA \colon V^{n} \to V^{n}\) имеет вид

  \begin{align*}
    \Img \opA = \left\{
      \sum_{i = 1}^{n}
      \lambda_{i} (x, \basis_{i}) \basis_{i}
      \mid x \in V^{n}
    \right\}
  \end{align*}

  где \(\Basis = \{ \basis_{i} \}_{i = 1}^{n}\) это ортонормированный базис,
  \(\lambda_{i}\)~--- собственные числа оператора \(\opA\).
\end{theorem}
\begin{proof}
  \begin{align*}
    \opA x = y =
    \\
      y_{1} \basis_{1}
        + \dotsc + y_{n} \basis_{n} =
    \\
      (y, \basis_{1}) \basis_{1}
        + \dotsc + (y, \basis_{n}) \basis_{n} =
    \\
      (\opA x, \basis_{1}) \basis_{1}
        + \dotsc + (\opA x, \basis_{n}) \basis_{n} =
    \\
      (x, \opA \basis_{1}) \basis_{1}
        + \dotsc + (x, \opA \basis_{n}) \basis_{n} =
    \\
      (x, \lambda_{1} \basis_{1}) \basis_{1}
        + \dotsc + (x, \lambda_{n} \basis_{n}) \basis_{n} =
    \\
      \lambda_{1} (x, \basis_{1}) \basis_{1}
        + \dotsc + \lambda_{n} (x,\basis_{n}) \basis_{n} =
    \\
    \sum_{i = 1}^{n} \lambda_{i} (x, \basis_{i}) \basis_{i}
  \end{align*}
\end{proof}

\begin{remark}
  \((x, \basis_{i})\) это проекция вектора \(x\) на собственный вектор.

  Таким образом \(y\) получается следующим образом: \(x\) проецируется на 
  \(\basis_{1}, \dotsc, \basis_{n}\), после чего проекции 'растягиваются' в
  \(\lambda_{1}, \dotsc, \lambda_{n}\) раз и суммируются.
\end{remark}

\begin{definition}\label{lo-proj}
  Оператор вида

  \begin{align*}
    P_{i}(x) = (x, \basis_{i}) \basis_{i}
  \end{align*}

  называется проектором на одномерное пространство, порожденное собственным
  вектором.
\end{definition}

\begin{remark}
  Проектор является самосопряженным оператором.
\end{remark}

\begin{definition}
  Спектральным разложением оператора называется представление его в виде
  линейной комбинации проекторов

  \begin{align*}
    \opA = \sum_{i = 1}^{n} \lambda_{i} P_{i}
  \end{align*}

  Корректность такого представления следует из \ref{sconj-lo-img} и
  \ref{lo-proj}.
\end{definition}