\question{Матрица линейного оператора. Преобразование матрицы при переходе к новому базису.}

Пусть дан оператор \(\opA \colon V^{n} \to V^{n}\) и \(x, y \in V^{n}\),
\(\opA x = y\).

Выделим в \(V^{n}\) базис, разложим \(x\) по этому базису. После чего применим
к нему оператор \(\opA\):

\begin{lequation}{lo-mtx-1}
  \Basis = \{ \basis_{1}, \dots, \basis_{n} \} \\
  x = x_{1} \basis_{1} + \dots + x_{n} \basis_{n} \\
  y = x_{1} \opA \basis_{1} + \dots + x_{n} \opA \basis_{n} \\
\end{lequation}

Далее применим оператор к каждому из базисных векторов:

\begin{lequation}{lo-mtx-2}
  \opA \basis_{i} = a_{1,i} \basis_{1} + \dotsc + a_{n, i} \basis_{n} \\
  y
  = x_{1} \Big( a_{1,1} \basis_{1} + \dotsc + a_{n,1} \basis_{n} \Big)
  + \dots
  + x_{n} \Big( a_{1,n} \basis_{1} + \dotsc + a_{n,n} \basis_{n} \Big) \\
  y
  = \basis_{1} \Big( x_{1} a_{1,1} + \dotsc + x_{n} a_{1,n} \Big)
  + \dots
  + \basis_{n} \Big( x_{1} a_{n,1} + \dotsc + x_{n} a_{n,n} \Big)
\end{lequation}

Заметим, что \(y\) также можно разложить по базису. Составим СЛАУ и запишем её
в матричном виде:

\begin{lequation}{lo-mtx-3}
  \begin{Bmatrix}
    x_{1} a_{1,1} & + \dotsc + & x_{n} a_{1,n} = & y_{1} \\
    \vdots        & \ddots    & \vdots          & \vdots \\
    x_{n} a_{n,1} & + \dotsc + & x_{n} a_{n,n} = & y_{n} 
  \end{Bmatrix} \iff AX = Y
\end{lequation}

\begin{definition}
  Матрицей оператора \(\opA\) \textbf{в данном базисе} называется матрица
  составленная из столбцов-коэффициентов разложения образов базисных векторов
  по этому же базису.
\end{definition}

\begin{remark}
  Если \(A^{-1} = A^{T}\), то матрица оператора называется ортогональной.
\end{remark}

\begin{definition}
  Матрица 

  \begin{lequation}{trs-mtx}
    \begin{pmatrix}
      \tau_{1,1} & \dots & \tau_{1,n} \\
      \vdots & \ddots & \vdots \\
      \tau_{n,1} & \dots & \tau_{n,n} \\
    \end{pmatrix}
  \end{lequation}

  такая, что при переходе из базиса \(\{ \basis_{i} \}_{i = 1}^{n}\) в базис
  \(\{ \basis_{i}' \}_{i = 1}^{n}\) выполняется равенство
  \((e_1', e_2', \dots, e_n') = (e_1, e_2, \dots, e_n)T\), называется матрицей преобразования координат.
\end{definition}

\textit{Примечание от автора:}
  Матрицей перехода от старого базиса $(e_1, \dots, e_n)$ к новому $(e_1', \dots, e_n')$,
  называется матрица $T = (t_{ij})$, в столбцах которой стоят координаты новых базисных векторов
  в старом базисе. Таким образом,
  $$(e_1', e_2', \dots, e_n') = (e_1, e_2, \dots, e_n)T,$$

  а координаты вектора x в старом и новом базисах связаны равенствами
  $x_i = \sum^{n}_{j=1}t_{ij}x_{j}'$, или, в матричной записи,
  $$\begin{pmatrix}
    x_1 \\ x_2 \\ \vdots \\ x_n
  \end{pmatrix} = T \begin{pmatrix}  
    x_1' \\ x_2' \\ \vdots \\ x_n'  
  \end{pmatrix}$$

\begin{theorem}
  Пусть дан линейный оператор \(\opA \colon V^{n} \to V^{n}\), который в базисе
  \(\Basis = \{ \basis_{i} \}_{i = 1}^{n}\) имеет матрицу \(A\), в базисе
  \(\Basis' = \{ \basis_{i}' \}_{i = 1}^{n}\)~---\(A'\).
  Тогда

  \begin{lequation}{trs-lo-mtx}
    A' = T^{-1} A T
  \end{lequation}
\end{theorem}
\begin{proof}
  Рассмотрим столбцы \(x, y\) в базисе \(\Basis\) такие, что \(A x = y\). В
  базисе \(\Basis'\) будет выполняться равенство \(A' x' = y'\), получаем:

  \begin{lequation}{trs-lo-mtx-proof}
    \begin{cases}
      x = T x' \\
      y = T y'
    \end{cases}
    \implies A (T x') = (T y') \mid \cdot T^{-1} \\
    T^{-1} A T x' = T^{-1} T y' \\
    T^{-1} A T x' = y' \\
    A' = T^{-1} A T \\
  \end{lequation}
\end{proof}

Некоторые свойства замены базиса:
\begin{enumerate}
  \item \(\opA, \opB \colon V^{n} \to V^{n}, \lambda \in \CC\), тогда
  
  \begin{lequation}{trs-lo-prop-1}
    \begin{matrix}
      \Basis \colon && A + \lambda B \\
      \Basis' \colon & T^{-1} (A + \lambda B) T = & A' + \lambda B'\\
    \end{matrix}
  \end{lequation}

  \item Матрица тождественного оператора в любом базисе единичная.
  
  \begin{lequation}{trs-lo-prop-2}
    \begin{matrix}
      \Basis \colon && E \\
      \Basis' \colon & T^{-1} E T = & E\\
    \end{matrix}
  \end{lequation}

\end{enumerate}

\begin{lemma}
  Определитель матрицы оператора не зависит от базиса, в котором эта матрица
  рассматривается.
\end{lemma}
\begin{proof}
  \begin{lequation}{trs-lo-prop-3}
    \det A'
    = \det (T^{-1} A T)
    = \det T^{-1} \cdot \det A \cdot \det T
    = \det A \cdot \det T^{-1} \cdot \det T
    = \det A
  \end{lequation}
\end{proof}