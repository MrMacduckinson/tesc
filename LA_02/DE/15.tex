\question{Структура решения ЛОДУ\(_n\): линейная независимость решений, нахождение фундаментальной системы решений по корням характеристического уравнения.}

\begin{remark}
  О существовании и единственности решения задачи Коши.

  Пусть 

  \begin{lequation}{loden-ex-un-sol-1}
    y^{(n)} = g(x, y, y', \dotsc, y^{(n - 1)})
  \end{lequation}

  при этом \(y, y', \dots, y^{(n - 1)}, g\) непрерывны и ограничены в области

  \begin{lequation}{loden-ex-un-sol-2}
    \abs{x - x_{0}} < h_{0}, \\
    \abs{y - y_{0}} < h_{1}, \\
    \dots \\
    \abs{y^{(n - 1)} - y^{(n - 1)}_{0}} < h_{n},
  \end{lequation}

  где \((x_{0}, y_{0}, \dots, y^{(n - 1)}_{0})\) это начальные условия.
  Тогда существует единственное решение задачи Коши.

\end{remark}

По аналогии с ЛОДУ\(_2\) для ЛОДУ\(_n\) можно составить характеристическое
уравнение:

\begin{lequation}{lode-char-eq}
  \Linear{y}
  = y^{(n)}(x) + p_{1} y^{(n - 1)} (x) + \dotsc + p_{n - 1} y' + p_{n} y = 0
  \hspace{10pt} p_{i} \in \RR
  \\
  \Linear{e^{k x}}
  = k^{n} e^{k x} + p_{1} \cdot k^{n - 1} e^{k x} + \dotsc + p_{n} e^{k x} = 0
  \mid \colon e^{k x} \neq 0 
  \\
  k^{n} + p_{1} k^{n - 1}+ \dotsc + p_{n} = 0
\end{lequation}

Далее аналогично рассмотрим некоторые случаи:
\begin{enumerate}
  \item Набору \(k_{1}, \dotsc, k_{m} \in \RR\) различных вещественных корней
  соответствует набор частных линейно-независимых решений однородного уравнения
  \(y_{1} = e^{k_{1} x}, \dotsc, y_{m} = e^{k_{m} x}\).

  \item Набору \(k_{1} = k_{2} = \dotsc = k_{m} = k \in \RR\) повторяющихся
  вещественных корней соответствует набор частных линейно-независимых решений
  однородного уравнения \(
    y_{1} = e^{k x}, y_{2} = x e^{k x},
    \dotsc,
    y_{m} = x^{m - 1} e^{k x}
  \).

  \item Каждой уникальной паре вида \(k = \alpha + i \beta\) соответствует пара
  частных линейно-независимых решений однородного уравнения
  \(y_{1} = e^{\alpha x} \cos \beta x\) и
  \(y_{2} = e^{\alpha x} \sin \beta x\).
  
  \item Каждой паре кратности \(m\) вида \(k = \alpha + i \beta\) соответствует
  \(m\) пар частных линейно-независимых решений однородного уравнения вида
  
  \begin{lequation}{loden-cases-4}
    \begin{matrix}
      y_{1} & = & e^{\alpha x} \cos \beta x, &&
        y_{2} & = & e^{\alpha x} \sin \beta x \\
      y_{3} & = & x e^{\alpha x} \cos \beta x, &&
        y_{4} & = & x e^{\alpha x} \sin \beta x \\
      \vdots && \vdots && \vdots & \vdots \\
      y_{2m - 1} & = & x^{m - 1} e^{\alpha x} \cos \beta x, &&
      y_{2m} & = & x^{m - 1} e^{\alpha x} \sin \beta x  
    \end{matrix}
  \end{lequation}
\end{enumerate}

\begin{remark}
  Общим решением ЛОДУ\(_n\) будет линейная оболочка набора частных решений,
  соответствующих корням характеристического уравнения.
\end{remark}

\begin{definition}
  Вронскианом ДУ называется вронскиан его ФСР.
\end{definition}

\begin{remark}
  Все доказанные свойства вронскиана распространяются и на бОльшую размерность.
\end{remark}
