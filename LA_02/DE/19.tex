\question{Системы дифференциальных уравнений: определения, решение матричным методом в случае различных вещественных собственных чисел.}

Обозначим
\((y_{1}, \dotsc, y_{n}) = Y\)~--- вектор неизвестных,
\(\{a_{i,j}\} = A\)~--- коэффициенты,
\((y'_{1}, \dotsc, y'_{n}) = Y'\)~--- вектор производных.

Тогда СДУ можно записать в матричном виде как \(Y' = AY\). Пусть \(A\) это
матрица линейного оператора \(\opA\). Для этого оператора можно найти
собственные числа и соответствующие им собственные векторы.

Обозначим собственные числа \(\lambda_{1}, \dotsc, \lambda_{n}\), а
\(\Gamma_{1}, \dotsc, \Gamma_{n}\)~--- соответствующие им собственные векторы.

Можно убедиться, что \(Y_{i} = \Gamma_{i} e^{\lambda_{i} x}\) будет являться
решением СДУ:

\begin{align*}
  \begin{cases}
    Y'_{i}
      = \Gamma_{i} \lambda_{i} e^{\lambda_{i} x}
      = \lambda_{i} (\Gamma_{i} e^{\lambda_{i} x}) \\  
    \opA Y_{i}
      = \opA (\Gamma_{i} e^{\lambda_{i} x})
      = \lambda_{i} (\Gamma_{i} e^{\lambda_{i} x})
  \end{cases} \implies Y'_{i} = \opA Y_{i}
\end{align*}

Пусть все собственные числа различные и вещественные, тогда
\(\forall \lambda_{i} \colon Y_{i} = \Gamma_{i} e^{\lambda_{i} x}\) это решение,
причем \(\forall i \neq j \colon Y_{i}\) и \(Y_{j}\) линейно-независимы.
Общее решение СДУ можно записать в виде:

\begin{align*}
  \overline{Y} = c_{1} Y_{1} + \dotsc + c_{n} Y_{n}
\end{align*}

\underline{Пример}:

\begin{align*}
  \begin{cases}
    x' = x + y \\
    y' = 8x + 3y
  \end{cases} \iff
  Y' = \begin{pmatrix}
    1 & 1 \\
    8 & 3
  \end{pmatrix} Y \iff
  \begin{cases}
    \lambda_{1} = -1
      \implies \Gamma_{1} = \begin{pmatrix} -1 \\ 2 \end{pmatrix}
    \\
    \lambda_{2} = 5
      \implies \Gamma_{2} = \begin{pmatrix} 1 \\ 4 \end{pmatrix}
  \end{cases} \\
  \begin{cases}
    Y_{1} = \Gamma_{1} e^{\lambda_{1} t}  \\
    Y_{2} = \Gamma_{2} e^{\lambda_{2} t}  \\
  \end{cases} \iff
  \begin{cases}
    \overline{x}(t)
      = c_{1} \cdot (-1) \cdot e^{-t}
      + c_{2} \cdot 1 \cdot e^{5t}
    \\
    \overline{y}(t)
      = c_{1} \cdot 2 \cdot e^{-t}
      + c_{2} \cdot 4 \cdot e^{5t}
    \\
  \end{cases}
\end{align*}

