\question{Теория устойчивости: определение устойчивости по Ляпунову, фазовая плоскость, траектории ДУ. Примеры устойчивого и неустойчивого решения.}

\begin{definition}
  Пусть дана СДУ

  \begin{lequation}{st-def-1}
    \begin{cases}
      x' = f_{1}(t, x, y) \\
      y' = f_{2}(t, x, y)
    \end{cases}
  \end{lequation}

  и \(x(t), y(t)\) это её решение, удовлетворяющее начальным условиям
  \(x(t = 0) = x_{0}, y(t = 0) = y_{0}\). Рассмотрим
  \(\widetilde{x}(t), \widetilde{y}(t)\)~--- другое решение, удовлетворяющее
  начальным условиям с отклонением \(
    \widetilde{x}(t = 0) = \widetilde{x}_{0},
    \widetilde{y}(t = 0) = \widetilde{y}_{0}
  \).

  Решение \(x(t), y(t)\) называется \textit{устойчивым по Ляпунову}, если

  \begin{lequation}{st-def-2}
    \forall \epsilon > 0 \;\;
    \exists \delta > 0 \mid
    \forall t > 0 \;\;
    \forall x_{0}, y_{0}, \widetilde{x}_{0}, \widetilde{y}_{0} \colon
    \begin{cases}
      \abs{x_{0} - \widetilde{x}_{0}} < \delta \\
      \abs{y_{0} - \widetilde{y}_{0}} < \delta
    \end{cases}
    \implies
    \begin{cases}
      \abs{\widetilde{x}(t) - x(t)} < \epsilon \\
      \abs{\widetilde{y}(t) - y(t)} < \epsilon
    \end{cases}
  \end{lequation}
\end{definition}

\underline{Пример}:

Рассмотрим ДУ:
\(y' + y = 1, \;\; y(0) = 1, \widetilde{y}(0) = \widetilde{y}_{0}\)

Найдем опорное решение \(y(t)\):

\begin{lequation}{st-ex-1}
  y' + y = 1 \\
  y(t) = c_{1} e^{-t} + 1 \\
  \begin{cases}
    y(t) = c_{1} e^{-t} + 1 \\
    y(0) = 1
  \end{cases}
  \implies c_{1} = 0 \\
  y(t) = 1
\end{lequation}

Теперь найдем решение при измененных начальных условиях

\begin{lequation}{st-ex-2}
  \begin{cases}
    \widetilde{y}(t) = c_{1} e^{-t} + 1 \\
    \widetilde{y}(0) = \widetilde{y}_{0}
  \end{cases}
  \implies c_{1} = \widetilde{y}_{0} - 1 \\
  \widetilde{y}(t) = (\widetilde{y}_{0} - 1) e^{-t} + 1 \\
\end{lequation}

Далее посмотрим на разницу полученных решений при \(t \to \infty\):

\begin{lequation}{st-ex-3}
  \widetilde{y}(t) - y(t)
  = (\widetilde{y}_{0} - 1) e^{-t} + 1 - 1
  = (\widetilde{y}_{0} - 1) e^{-t}
  \xrightarrow{t \to \infty}
  0
\end{lequation}

Таким образом при \(\abs{\widetilde{y}_{0} - 1} < \delta\) получаем
\(\abs{\widetilde{y}(t) - y(t)} < \epsilon \implies\) решение устойчиво.

\todo Исследование на устойчивость линейной автономной системы ДУ.
Разбор случаев.

