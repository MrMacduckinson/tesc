\question{Свойства решений ЛОДУ\(_2\) : линейная комбинация решений, линейная зависимость решений. Теорема о структуре общего решения ЛОДУ\(_2\). Фундаментальная система решений (определение).}

\begin{theorem}\label{lode-gen}
  О структуре общего решения ЛОДУ\(_2\).

  Если \(\Linear{y_{1}} = 0\), \(\Linear{y_{2}} = 0\) и \(y_{1}, y_{2}\)
  линейно независимы, то \(\overline{y} = c_{1} y_{1} + c_{2} y_{2}\) -- общее
  решение ЛОДУ\(_2\).
\end{theorem}
\begin{proof}
  Начнем с того, что \(\overline{y} = c_{1} y_{1} + c_{2} y_{2}\) это решение
  как линейная комбинация решений (см. \ref{lode-sol-lc}).

  Рассмотрим точку \((x_{0}, y_{0})\) в рамках задачи Коши:

  \begin{lequation}{lode-sln-struct-proof-1}
    \begin{cases}
      \Linear{y} = 0 \\
      y_{0} = y(x_{0}) \\
      y'_{0} = y'(x_{0})
    \end{cases} \iff
    \begin{cases}
      c_{1} y_{1}(x_{0}) + c_{2} y_{2}(x_{0}) = y_{0} \\
      c_{1} y'_{1}(x_{0}) + c_{2} y'_{2}(x_{0}) = y'_{0}
    \end{cases} \iff
    \begin{pmatrix}
      y_{1} & y_{2} \\
      y'_{1} & y'_{2}
    \end{pmatrix}
    \begin{pmatrix}
      c_{1} \\
      c_{2}
    \end{pmatrix}
    =
    \begin{pmatrix}
      y_{0} \\
      y'_{0}
    \end{pmatrix}
  \end{lequation}

  По т. Крамера решение полученной СЛАУ будет единственным только в том случае,
  если определитель главной матрицы не равен нулю. Это выполняется, т.к.
  этот определитель это вронскиан, который не равен нулю, т.к. решения
  линейно-независимы.
\end{proof}

\begin{definition}
  Фундаментальная система решений (ФСР) ЛОДУ\(_n\) это максимальный
  (по включению) набор линейно независимых решений ДУ.
\end{definition}
