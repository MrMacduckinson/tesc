\question{Решение ЛОДУ\(_2\) с постоянными коэффициентами для случая комплексных корней характеристического уравнения.}

\textbf{\eqref{eq:lode2-cases} случай III}:
\(k_{1,2} = \alpha + \beta i, k_{1,2} \in \CC\)

В заданных ограничениях получаем

\begin{lequation}{lode2-sln-3-3-1}
  C_{2}'(x) = c_{1} e^{(k_{1} - k_{2}) x} \\
  C_{2}(x) = \frac{c_1}{k_{1} - k_{2}} e^{(k_{1} - k_{2}) x} + \tilde{c_{2}} \\
  y(x)
  = C_{2}(x) y_{1}(x)
  = \underbrace{\frac{c_1}{k_{1} - k_{2}}}_{\tilde{c_1}} e^{k_{1} x}
  + \tilde{c_2} e^{k_{2} x} \\
  y(x) = \tilde{c_{1}} e^{k_{1}x} + \tilde{c_{2}} e^{k_{2}x} \\
  y(x)
  = \tilde{c_{1}} e^{\alpha x} e^{\beta i x}
  + \tilde{c_{2}} e^{\alpha x} e^{-\beta i x}
\end{lequation}

Далее используем формулу \(e^{i \phi} = \cos \phi + i \sin \phi\):

\begin{lequation}{lode2-sln-3-3-2}
  y(x) = e^{\alpha x} \bigg(
    \tilde{c_{1}} \Big(\cos (\beta x) + i \sin (\beta x)\Big) +
    \tilde{c_{2}} \Big(\cos (\beta x) - i \sin (\beta x)\Big)
  \bigg) \\
  y(x) = e^{\alpha x} \bigg(
    \cos (\beta x)
      \underbrace{\Big(\tilde{c_{1}} + \tilde{c_{2}}\Big)}_{\widehat{c_{1}}}
    +
    i \sin (\beta x)
    \underbrace{\Big(\tilde{c_{1}} - \tilde{c_{2}}\Big)}_{\widehat{c_{2}}}
  \bigg) \\
  y(x) = e^{\alpha x} \Big(
    \widehat{c_{1}} \cos (\beta x) +
    \widehat{c_{2}} i \sin (\beta x)
  \Big)
\end{lequation}

\begin{lemma}\label{lode2-sln-3-3-lm}
  Если \(y(x) = u(x) + i v(x)\) это решение ЛОДУ\(_2\), то
  \(y(x) = u(x) + v(x)\) также являются решением ЛОДУ\(_2\).
\end{lemma}
\begin{proof}
  Рассмотрим функцию \(y(x) = u(x) + v(x)\):
  \begin{lequation}{lode2-sln-3-3-lm-1}
    \begin{cases}
      y(x) = u(x) + v(x) \\
      y'(x) = u'(x) + v'(x) \\
      y''(x) = u''(x) + v''(x)
    \end{cases} \\
    y''(x) + p y'(x) + q y(x)
    = u''(x) + v''(x) + p u'(x) + p v'(x) + u(x) + q u(x) + q v(x) = 0 \\
    \Big( u''(x) + p u'(x) + q u(x) \Big) +
    \Big( v''(x) + p v'(x) + q v(x) \Big) = 0
  \end{lequation}
  Это равенство верно, т.к. \(u(x)\) и \(v(x)\) решения ЛОДУ\(_2\).
\end{proof}

Значит, по \ref{lode2-sln-3-3-lm} общее решение \eqref{eq:lode2-cases} в третьем
случае будет иметь вид

\begin{lequation}{}
  y(x) = e^{\alpha x} \Big(
    \widehat{c_{1}} \cos (\beta x) +
    \widehat{c_{2}} \sin (\beta x)
  \Big)
\end{lequation}



