\question{Уравнение в полных дифференциалах.}

\begin{definition}
  Дифференциальное уравнение \(P(x, y) \dd x + Q(x, y) \dd y = 0\) называется
  уравнением в полных дифференциалах, если
  
  \begin{align*}
    \exists z(x, y) \colon \dd z = P(x, y) \dd x + Q(x, y) \dd y
  \end{align*}
\end{definition}

Критерием того, что данное уравнение является уравнением в полных
дифференциалах может служить равенство

\begin{align*}
  \frac{\partial P}{\partial x} = \frac{\partial Q}{\partial y} 
\end{align*}

Решение уравнений в полных дифференциалах сводится к поиску функции \(z(x, y)\),
удовлетворяющей описанным условиям. После того, как такая функция будет найдена,
решение ДУ очевидно:

\begin{align*}
  P(x, y) \dd x + Q(x, y) \dd y = 0
  \implies \dd z = 0
  \implies z = C
\end{align*}

\begin{remark}
  О поиске \(z(x, y)\)

  \begin{align*}\label{eq:find-potential-1}\tag{\(1\)}
    z'_{x} = P(x, y) \implies z(x, y) = \int P(x, y) \dd x + \phi(y) \\
    z'_{y} = Q(x, y) = \left(\int P(x, y) \dd x\right)'_{y} + \phi'(y)
    \implies \phi'(y) = Q(x, y) - \left(\int P(x, y) \dd x\right)'_{y}
  \end{align*}

  Покажем, что \((\phi')'_{x} = 0\):

  \begin{align*}
    (\phi')'_{x}
    = Q(x, y)'_{x} - \left(\int P(x, y) \dd x\right)''_{yx}
    = Q(x, y)'_{x} - P(x, y)'_{y}
    = 0
  \end{align*}

  Таким образом из \eqref{eq:find-potential-1} получаем, что

  \begin{align*}
    \phi(y) = \int \phi'(y) \dd y
  \end{align*}

  Итого алгоритм поиска \(z(x, y)\) выглядит следующим образом:
  \begin{enumerate}
    \item Находим \(\int P(x, y) \dd x\) и прибавляем к нему некоторую
      неизвестную функцию \(\phi(y)\).

    \item Считаем производную по \(y\) от полученного выражения и приравниваем
      её к \(Q(x, y)\).

    \item Находим \(\phi'(y)\) из полученного выражения.
    
    \item Интегрируем её и подставляем в выражение, полученное в первом пункте.
  \end{enumerate}
\end{remark}


