\question{Уравнение в полных дифференциалах.}

\begin{definition}
  Дифференциальное уравнение \(P(x, y) \dd x + Q(x, y) \dd y = 0\) называется
  уравнением в полных дифференциалах, если
  \begin{lequation}{full-diff-def}
    \exists z(x, y) \colon \dd z = P(x, y) \dd x + Q(x, y) \dd y
  \end{lequation}
\end{definition}

Критерием того, что данное уравнение является уравнением в полных
дифференциалах может служить равенство

\begin{lequation}{full-diff-crt}
  \frac{\partial P}{\partial x} = \frac{\partial Q}{\partial y} 
\end{lequation}

Решение уравнений в полных дифференциалах сводится к поиску функции \(z(x, y)\),
удовлетворяющей условиям. Про то, как найти такую функцию можно прочитать в
конспекте по матанализу в разделе про интегралы, независящие от пути. После
того, как такая функция будет найдена, решить ДУ не составит проблем:

\begin{lequation}{full-diff-sln}
  P(x, y) \dd x + Q(x, y) \dd y = 0 \\
  \dd z = 0 \\
  z = C
\end{lequation}

\todo Интегрирующий множитель
