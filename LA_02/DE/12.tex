\question{Свойства решений ЛОДУ\(_2\): линейная комбинация решений, линейная зависимость решений. Определитель Вронского. Теоремы о вронскиане.}

\begin{lemma}\label{lode-sol-lc}
  Линейная комбинация решений ЛОДУ\(_2\) также является решением.

  \begin{align*}
    \begin{rcases}
      \Linear{y_{1}} = 0 \\
      \Linear{y_{2}} = 0
    \end{rcases} \implies
    \begin{cases}
      \Linear{y_{1} + y_{2}} = 0 \\
      \Linear{\lambda y_{1}} = 0 \; (\forall \lambda \in \RR) \\
    \end{cases}
  \end{align*}
\end{lemma}
\begin{proof}
  Рассмотрим на примере \(y = y_{1} + y_{2}\). Подставим в исходное ДУ,
  раскроем и сгруппируем

  \begin{align*}
    y'' + p y' + q y = 0 \\
    (y_{1} + y_{2})'' + p (y_{1} + y_{2})' + q (y_{1} + y_{2}) = 0 \\
    (y_{1}'' + p y_{1}' + q y_{1}) + (y_{2}'' + p y_{2}' + q y_{2}) = 0 \\
  \end{align*}

  Это верно, т.к. \(y_{1}\) и \(y_{2}\) это решения исходного ДУ. Случай
  \(y = \lambda y_{1}\) рассматривает аналогично.
\end{proof}

\begin{corollary}
  Множество решений ЛОДУ образует линейное пространство.
\end{corollary}

\begin{theorem}
  О существовании и единственности решения ЛОДУ\(_2\).

  \begin{align*}
    y'' = g(x, y, y') = f(x) - p y' - qy
  \end{align*}

  Если \(g, g'_{y}, g'_{y'}\) непрерывны в области \(D \owns (x_{0}, y_{0})\),
  то задача Коши

  \begin{align*}
    \begin{cases}
      \Linear{y} = f(x) \\
      y_{0} = y(x_{0}) \\
      y'_{0} = y'(x_{0})
    \end{cases}
  \end{align*}

  имеет единственное решение.
\end{theorem}
