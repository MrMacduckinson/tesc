\subsection{%
  Скалярное произведение и норма векторов. Ортонормированный базис.%
} \label{sec:01-20}

\begin{remark}
  Для геометрических векторов скалярное произведение определено как
  \(\dotpdtv{a}{b} = \vec{\abs{a}} \cdot \vec{\abs{b}} \cdot \cos \phi\), где
  \(\phi\) это угол между векторами \(\vec{a}\) и \(\vec{b}\).
\end{remark}

\begin{definition}
  В общем случае скалярным произведением \(x \in L^n\) и \(y \in L^n\)
  называется число \(G \in \RR\), такое что
  
  \begin{enumerate}
  \item
    \(\dotpdt{x}{y} = \dotpdt{y}{x}\)
    
  \item
    \(\dotpdt{x_1 + x_2}{y} = \dotpdt{x_1}{y} + \dotpdt{x_2}{y}\)
    
  \item
    \(\dotpdt{\lambda x}{y} = \lambda \dotpdt{x}{y} \; \forall \lambda \in \RR\)
    
  \item
    \(\dotpdt{x}{x} = x^2 \ge 0\), причем \(x^2 = 0 \implies x = 0\)
  \end{enumerate}
  
  Скалярное произведение обозначается как \(\dotpdt{x}{y}\) или \(x \cdot y\).
\end{definition}

\begin{remark}
  Совокупность второго и третьего свойств называется линейностью.
\end{remark}

\begin{remark} \label{rem:dot-pdf}
  \textbf{В ортонормированном базисе} для пространства числовых наборов \(L^n\)
  скалярное произведение определено как \(\dotpdt{x}{y} = x_1 y_1 + \dotsc + x_n
  y_n\). Если оно равно нулю, то векторы перпендикулярны (либо один из векторов
  нулевой).
\end{remark}

\begin{remark}
  Скалярное произведение коммутативно (по первому пункту определения) и
  дистрибутивно
  
  \begin{equation*}
    (\vec{a} + \vec{b}) \cdot \vec{c}
    = \vec{a} \cdot \vec{c} + \vec{b} \cdot \vec{c}    
  \end{equation*}
\end{remark}

\begin{definition}
  В общем случае число \(l \in \RR\) называется нормой \(x \in L^n\), если
  
  \begin{enumerate}
  \item
    \(\norm{\lambda x} = \lambda \norm{x} \given \forall \lambda \in \RR\)
    
  \item
    \(\norm{x + y} \le \norm{x} + \norm{y}\) (неравенство Минковского)
    
  \item
    \(\norm{x} \ge 0\), причем \(\norm{x} = 0 \implies x = 0\)
  \end{enumerate}
  
  Норма обозначается как \(\norm{x}\).
\end{definition}

\begin{remark}
  Для пространства числовых наборов \(L^n\) норма определена как \(\norm{x} =
  \sqrt{x_1^2 + \dotsc + x_n^2}\)
\end{remark}

\begin{remark}
  Для геометрических векторов нормой является длина вектора.
\end{remark}

\begin{remark}
  \textbf{В ортонормированном базисе} нормой (зачастую) является корень
  квадратный из скалярного произведения элемента самого на себя \(\norm{x}=
  \sqrt{\dotpdt{x}{x}} = \sqrt{x^2}\)
\end{remark}

\begin{definition}
  Базис называется ортогональным, если все попарные скалярные произведения
  векторов этого базиса равны нулю. Т.е. базис ортогонален, если все векторы в
  этом базисе перпендикулярны друг другу.
\end{definition}

\begin{remark}
  Нулевой вектор \textbf{не ортогонален} любому вектору, т.к. он коллинеарен
  любому вектору, а ортогональность любому вектору привела бы к противоречию.
\end{remark}

\begin{definition}
  Базис называется нормированным, если норма (в данном случае норма это тоже
  самое, что и длина) всех векторов в этом базисе равна единице.
\end{definition}

\begin{definition}
  Базис \(\set{\basis_i}_{i = 1}^n\) называется ортонормированным, если он
  ортогональный и нормированный
  
  \begin{equation*}
    \dotpdt{e_i}{e_j}
    = \begin{cases}
      1 & i = j \\
      0 & i \ne j
    \end{cases}    
  \end{equation*}

  Первая строчка дает нормированность, а вторая ортогональность.
\end{definition}
