\subsection{%
  Линейное пространство направленных отрезков с общим началом. Определение,
  проверка аксиом.%
}

\begin{remark}
  Множество направленных отрезков образует линейное пространство.
\end{remark}

Сложение определим геометрически, по правилу параллелограмма
(\figref{01_03_01a}). Умножение на число определим как растяжение/сжатие
направленного отрезка, причем при отрицательном множителе происходит смена
направления на противоположное (\figref{01_03_01b}).

\gallerytwo{01_03_01}
  {Введение операций для направленных отрезков}
  {Сложение}
  {Умножение}

\begin{remark}
  Проверим аксиомы линейного пространства.

  \begin{enumerate}
  \item
    Замкнутость. По определению сложения получаем, что сумма двух направленных
    отрезков это направленный отрезок.
  
  \item
    Ассоциативность. Проверим геометрически (\figref{01_03_02}).

    \gallerytwo{01_03_02}
      {Проверка ассоциативности сложения направленных отрезков}
      {Левая ассоциативность}
      {Правая ассоциативность}
  
  \item
    Наличие нейтрального элемента по сложению. Нейтральный элемент по сложению
    это нулевой вектор (длина равна нулю, направление не определено). Нулевой
    вектор можно получить, умножив любой вектор на ноль.
    
  \item
    Наличие обратного элемента по сложению. Обратным для вектора \(\vec{a}\)
    будет вектор \(-1 \cdot \vec{a}\) (вектор той же длины, но с противоположным
    направлением).
  
  \item
    Коммутативность. Коммутативность по сложению выполнена исходя из определения
    суммы двух направленных отрезков. Коммутативность по умножению на число
    выполнена из определения умножения направленного отрезка на число.

  \item
    Дистрибутивность и ассоциативность для умножения на число. \textit{Очевидно}
    доказываются геометрически.
    
  \item
    Наличие нейтрального элемента относительно умножения на число. Нейтральным
    элементом относительно умножения на число будет единица (вектор не
    сжимается, не растягивается, не меняет направление).
  \end{enumerate}
\end{remark}