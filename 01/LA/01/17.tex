\subsection{%
  Изоморфизм линейных пространств.%
}

\begin{definition}
  Биекция \(F \colon L \to L'\) называется изоморфизмом, если
  
  \begin{enumerate}
  \item
    \(\forall x \in L (x' \in L') \colon x' = F(x)\)
    
  \item
    \(F(x + y) = x' + y' = F(x) + F(y)\)
    
  \item
    \(F(\lambda x) = \lambda x' = \lambda F(x)\)
  \end{enumerate}
\end{definition}

\begin{remark}
  Последние два свойства определяют линейность отображения (сохранение линейной
  структуры пространства). Иными словами изоморфизм сохраняет \quote{расстояние}
  между элементами.
\end{remark}

\begin{remark}
  Размерности изоморфных пространств совпадают: \(L\) изоморфно \(L’ \implies
  \dim L = \dim L'\).
\end{remark}

\begin{remark}
  При изоморфизме линейная комбинация переходит в линейную комбинацию (сохраняя
  тривиальности/нетривиальность), а нулевой вектор в нулевой вектор.
  
  \begin{equation*}
    F(\varnothing)
    = F(0 \cdot X)
    = 0 \cdot F(X)
    = 0 \cdot X'
    = \emptyset'
  \end{equation*}
\end{remark}

\begin{theorem}
  Если \(\dim L = \dim L'\), то \(L\) изоморфно \(L'\).
\end{theorem}

\begin{proof}
  Пусть \(\Basis = \set{\basis_i}_{i = 1}^n\) базис \(L\), а \(\Basis' =
  \set{\basis_i'}_{i = 1}^n\) базис \(L'\). Разложим произвольный \(x \in L\) по
  этому базису \(\Basis\).

  \begin{equation*}
    x = \sum_{i = 1}^n x_i \basis_i
  \end{equation*}
  
  Поставим этому \(x\) в соответствие \(x'\) такой, что \(F(x) = x' = \sum_{i =
  1}^n x_i \basis'_i\). Причем это соответствие взаимно-однозначно, т.к.
  \(\Basis\) и \(\Basis'\) базисы. Тогда
  
  \begin{equation*}
    \begin{aligned}
      \begin{rcases}
        F(x + y) = \sum_{i = 1}^n (x_i + y_i) \basis'_i
      \\
        \sum_{i = 1}^n (x_i + y_i) \basis'_i
        = \sum_{i = 1}^n x_i \basis'_i + \sum_{i = 1}^n y_i \basis'_i
        = x' + y'
        = F(x) + F(y)
      \end{rcases}
      & \implies
      F(x + y) = F(x) + F(y)
    \\
      \begin{rcases}
        F(\lambda x) = \sum_{i = 1}^n \lambda x_i \basis'_i
      \\
        \sum_{i = 1}^n \lambda x_i \basis'_i
        = \lambda \sum_{i = 1}^n x_i \basis'_i
        = \lambda x' = \lambda F(x)
      \end{rcases}
      & \implies
      F(\lambda x) = \lambda F(x)    
    \end{aligned}
  \end{equation*}

  Значит \(L\) изоморфно \(L'\) по определению.
\end{proof}
