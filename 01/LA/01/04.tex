\subsection{%
  Матрицы. Определение. Арифметика матриц.%
}

\begin{definition}
  Матрица это упорядоченный набор арифметических векторов одинаковой размерности
  
  \begin{equation*}
    A_{k, n} = \mtxp{
      . & . & . \\
      . & . & . \\
      . & . & .
    }
  \end{equation*}

  где \(k\) это количество строк, \(n\) это количество столбцов, \(A_{i, j}\)
  это элемент матрицы в \(i\)-ой строке и \(j\)-ом столбце.
\end{definition}

\begin{remark}
  Матрицы одной размерности образуют линейное пространство.
\end{remark}

\begin{remark}
  Равенство матриц определено как \(A_{n, k} = B_{n, k} \iff \forall i, j \given
  A_{i, j} = B_{i, j}\).
\end{remark}

\begin{remark}
  Сложение матриц определено как \(A_{n, k} + B_{n, k} = C_{n, k} \iff \forall
  i, j \given c_{i, j} = a_{i, j} + b_{i, j}\).
  
  Сложение матриц коммутативно \(A + B = B + A\).
\end{remark}

\begin{remark}
  Умножение на число определено как \(\lambda A = B \iff \forall i, j \given
  b_{i,j} = \lambda a_{i, j}\).
  
  Умножение матрицы на число дистрибутивно и ассоциативно
  
  \begin{enumerate}
  \item
    \(\lambda (A + B) = \lambda A + \lambda B\)
    
  \item
    \(\lambda (\mu A) = (\lambda \mu) A\)
    
  \item
    \((\lambda + \mu) A = \lambda A + \mu A\)
  \end{enumerate}
\end{remark}

\begin{remark}
  Умножение матриц определено как \(A_{n, k} \cdot B_{k, m} = C_{n, m} \iff
  \forall c_{i, j} = \sum_{p = 1}^k a_{i, p} \cdot b_{p, j}\).
  
  Умножение определено не для всех матриц: для того, чтобы можно было
  перемножить две матрицы, необходимо чтобы число столбцов первой матрицы было
  равно числу строк второй матрицы.
\end{remark}

\begin{remark}
  Умножение матриц в общем случае некоммутативно \(A \cdot B \ne B \cdot A\),
  причем умножение в обратную сторону может быть даже не определено. Матрицы,
  для которых выполняем равенство \(A \cdot B = B \cdot A\) называют
  коммутирующими.
\end{remark}

\begin{remark}
  Умножение ассоциативно \(A \cdot (B \cdot C) = (A \cdot B) \cdot C\), если не
  менять порядок множителей.
\end{remark}

\begin{definition}
  Если поменять местами строки и столбцы матрицы \(A_{n, k}\), то получится
  транспонированная матрица \(A_{k, n}^T\).
\end{definition}

\begin{remark}
  Некоторые виды матриц.

  \begin{twocolumns}
    \begin{enumerate}
    \item
      Квадратная (\(n = m\)) \(\mtxp{. & . & . \\ . & . & . \\ . & . & .}\)

    \item 
      Треугольная

      \begin{enumerate}
      \item 
        Верхняя \(\mtxp{. & . & . \\ 0 & . & . \\ 0  & 0 & .}\)

      \item 
        Нижняя \(\mtxp{. & 0 & 0 \\ . & . & 0 \\ .  & . & .}\)
      \end{enumerate}

    \item 
      Диагональная \(
        \diag(a_{1, 1} \dots a_{n, n})
        = \mtxp{. & 0 & 0 \\ 0 & . & 0 \\ 0  & 0 & .}
      \)
    
    \item
      Ступенчатая \(\mtxp{. & . & . & . \\ 0 & . & . & . \\ 0  & 0 & . & .}\)

    \item
      Нулевая \(\mtxp{0 & 0 & 0 \\ 0 & 0 & 0 \\ 0  & 0 & 0}\)
    
    \item
      Единичная \(E = \mtxp{1 & 0 & 0 \\ 0 & 1 & 0 \\ 0 & 0 & 1}\)

    \item
      Симметричная (\(A^T = A\)) \(\mtxp{1 & 2 & 3 \\ 2 & 7 & 4 \\ 3 & 4 & 0}\)
    
    \item
      Антисимметричная (\(A^T = -A\))
        \(\mtxp{0 & -2 & 3 \\ 2 & 0 & -4 \\ -3 & 4 & 0}\)
    \end{enumerate}
  \end{twocolumns}
\end{remark}
