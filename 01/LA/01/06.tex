\subsection{%
  Обратная матрица. Существование и единственность.%
}

\begin{definition}
  Матрица \(A^{-1}\) называется обратной матрицей для матрицы \(A\), если
  
  \begin{equation*}
    A \cdot A^{-1} = A^{-1} \cdot A = E
  \end{equation*}
\end{definition}

\begin{definition}
   Матрица называется невырожденной, если её определитель не равен нулю. В
  противном случае она называется вырожденной.
\end{definition}

\begin{theorem}
  У каждой невырожденной матрицы существует ровно одна обратная матрица.
\end{theorem}

\begin{proof}
  От противного: пусть \(A'\) и \(A''\) обратные матрицы к матрице \(A\), тогда
  по определению получаем
  
  \begin{equation*}
    \begin{aligned}
      A \cdot A' = E = A \cdot A'' & \qquad \Big\vert \cdot A' \text{ (слева)}
    \\
      A' \cdot A \cdot A' = A' \cdot A \cdot A'' &
    \\
      E \cdot A' = E \cdot A'' &
    \\
      A' = A'' &
    \end{aligned}
  \end{equation*}
\end{proof}

\begin{theorem}
  У любой квадратной невырожденной матрицы есть обратная.
\end{theorem}

\begin{proof}
  Рассмотрим матрицу \(D = \mtxp{A_{11} & A_{12} \\ A_{21} & A_{22}}\), где
  \(A_{ij} = (-1)^{i + j} \cdot M_{ij}\), \(M_{ij}\)~--- минор элемента
  \(a_{ij}\) матрицы \(A\). Далее рассмотрим произведение
  
  \begin{equation*}
    \frac{1}{\Delta} D^T \cdot A
    = \mtxp{
      \frac{A_{11}}{\Delta} & \frac{A_{21}}{\Delta} \\
      \frac{A_{12}}{\Delta} & \frac{A_{22}}{\Delta}
    }
    \cdot
    \mtxp{
      a_{11} & a_{12} \\
      a_{21} & a_{22}
    }
    = \mtxp{
      \frac{a_{11} (-1)^{1 + 1} M_{11} + a_{21} (-1)^{2 + 1} M_{21}}{\Delta} &
        \frac{a_{12} (-1)^{1 + 1} M_{11} + a_{22} (-1)^{2 + 1} M_{21}}{\Delta}
      \\
      \frac{a_{11} (-1)^{1 + 2} M_{12} + a_{21} (-1)^{2 + 2} M_{22}}{\Delta} &
        \frac{a_{12} (-1)^{1 + 2} M_{12} + a_{22} (-1)^{2 + 2} M_{22}}{\Delta}
    }
  \end{equation*}
      
  Заметим, что
  
  \begin{equation*}
    a_{11} (-1)^{1 + 1} M_{11} + a_{21} (-1)^{2 + 1} M_{21} = \Delta
  \end{equation*}
  
  как разложение по первому столбцу. Проверив все элементы, получим
  
  \begin{equation*}
    \frac{1}{\Delta} D^T \cdot A
    = \mtxp{
      \frac{\Delta}{\Delta} & \frac{0}{\Delta} \\
      \frac{0}{\Delta}   & \frac{\Delta}{\Delta}
    }
    = \mtxp{
      1 & 0 \\
      0 & 1
    }
    = E
  \end{equation*}
\end{proof}

\begin{remark}
  Обратная матрица существует только для невырожденных матриц.
\end{remark}

\subheader{Вычисление обратной матрицы с помощью определителя}

\begin{enumerate}
\item
  Вычислим определитель исходной матрицы.
  
\item
  Составим матрицу алгебраических дополнений (каждый элемент исходной матрицы
  нужно заменить на его алгебраическое дополнение).
  
\item
  Транспонируем получившуюся матрицу. Обозначим полученную матрицу \(D\).
  
\item
  Обратная матрица будет равна \(A^{-1} = \frac{1}{\det A} \cdot D\).
\end{enumerate}

\subheader{Вычисление обратной матрицы методом элементарных преобразований}

\begin{enumerate}
\item
  Припишем к исходной матрице единичную матрицу такого же размера.
  
\item
  С помощью элементарных преобразований сделаем из левой половины получившейся
  матрицы единичную матрицу (при этом элементарные преобразования затрагивают
  всю матрицу, а не только её половину).
  
\item
  Матрица, которая окажется в правой половине получившейся матрицы, будет
  обратной матрицей к исходной матрице.
\end{enumerate}

\begin{example}
  \begin{equation*}
    \begin{aligned}
      & A
      = \mtxp{
        1 & 2 \\
        3 & 4
      }
        \rarr{\text{Приписываем } E}
      \mtxp{
        1 & 2 & 1 & 0 \\
        3 & 4 & 0 & 1
      }
        \rarr{II - 3I}
      \mtxp{
        1 & 2 & 1 & 0 \\
        0 & -2 & -3 & 1
      }
        \rarr{I + II}
      \mtxp{
        1 & 0 & -2 & 1 \\
        0 & -2 & -3 & 1
      }
        \rarr{II \cdot (-0.5)}
      \mtxp{
        1 & 0 & -2 & 1 \\
        0 & 1 & 1.5 & -0.5
      }
    \\
      & \implies A^{-1}
      = \mtxp{
        -2 & 1 \\
        1.5 & -0.5
      }
    \end{aligned}
  \end{equation*}
\end{example}
