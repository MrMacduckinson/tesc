\subsection{%
  Цилиндрические и сферические координаты в пространстве. Преобразование
  уравнений поверхности.%
}

В цилиндрических координатах точка имеет три координаты \(M (\rho, \phi, z)\),
где

\begin{enumerate}
\item
  \(\rho\) это расстояние от начала координат до проекции точки \(M\) на
  плоскость \(Oxy\).
  
\item
  \(\phi\) это угол, которая данная проекция составляет с положительным
  направлением оси \(Ox\).
  
\item
  \(z\) это \quote{высота} точки над плоскостью \(Oxy\).
\end{enumerate}

\subheader{Ограничения}

\begin{enumerate}
\item
  \(\rho > 0\)
  
\item
  \(\phi \in [0, 2 \pi)\)
  
\item
  \(z \in \RR\)
\end{enumerate}

Преобразование цилиндрических координат в декартовы:

\begin{equation*}
  \begin{cases}
    x = \rho \cos \phi \\
    y = \rho \sin \phi \\
    z = z
  \end{cases}  
\end{equation*}

В сферических координатах точка имеет три координаты \(M (\rho, \theta, \phi)\),
где

\begin{enumerate}
\item
  \(\rho\) это расстояние от начала координат до точки \(M\).
  
\item
  \(\theta\) это угол между \(OM\) и положительным направлением оси \(Oz\)
  (зенитный угол).
  
\item
  \(\phi\) это угол между проекцией \(OM\) на плоскость \(Oxy\) и положительным
  направлением оси \(Ox\) (азимутальный угол).
\end{enumerate}

\subheader{Ограничения}

\begin{enumerate}
\item
  \(\rho > 0\)
  
\item
  \(\phi \in [0, 2 \pi)\)
  
\item
  \(\theta \in [0, \pi)\)
\end{enumerate}

Преобразование сферических координат в декартовы

\begin{equation*}
  \begin{cases}
    x = \rho \sin \theta \cos \phi \\
    y = \rho \sin \theta \sin \phi \\
    z = \rho \cos \theta
  \end{cases}
\end{equation*}
