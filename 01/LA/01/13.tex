\subsection{%
  Теория СЛАУ. Теорема Кронекера-Капелли. Два случая совместности (определенные
  и неопределенные СЛАУ).%
}

\begin{remark}
  Матрица \(A \mid B\) называется расширенной матрицей системы \(AX = B\) Она
  получается путем приписывания столбца \(B\) к матрице \(A\) справа.
\end{remark}

\begin{theorem}[Кронекера-Капелли] \label{thr:K-C}
  СЛАУ совместна тогда и только тогда, когда ранг расширенной матрицы равен
  рангу обычной матрицы.
  
  \begin{equation*}
    \rank A = \rank (A \mid B) \iff AX = B \text{ совместна}
  \end{equation*}
\end{theorem}

\begin{proof}
  \(\impliedby\) Если СЛАУ \(AX = B\) совместна, то столбец \(B\) разложим по
  матрице \(A\) (т.к СЛАУ имеет решение), а значит и по базису этой матрицы.
  
  Ранг исходной матрицы это количество линейно независимых столбцов, т.к. новый
  столбец \(B\) раскладывается по базису, то количество базисных столбцов не
  изменится, а значит и ранг не изменится.
  
  \(\implies\) Если \(\rank A = \rank (A|B)\), то базисы \(A\) и \(A|B\)
  совпадают, а значит столбец \(B\) и матрица \(A\) разложимы по этому базису.
  
  Таким образом столбец \(B\) разложим по матрице \(A\) (разложим \(B\) по
  общему базису, а потом добавим оставшиеся столбцы из \(A\) с нулевыми
  коэффициентами). Если столбец \(B\) разложим по матрице \(A\), значит
  \(\exists X \given AX = B\), т.е. система совместна.
\end{proof}

\subheader{Теория решения СЛАУ (анализ случаев по теореме Кронекера-Капелли)}

Рассмотрим \quote{вертикальную матрицу} \(A_{k, n}\; (n < k)\). \(\rank A \le
n\), значит как минимум \(k - n\) строк можно обнулить с помощью метода Гаусса,
получаем

\begin{equation*}
  A_{k, n} = \left( \begin{array}{cc|c}
    . & . & . \\
    . & . & . \\
    . & . & . \\
    . & . & .
  \end{array} \right)
    \Rarr{}
  A_{k, n} = \left( \begin{array}{cc|c}
    . & . & . \\
    . & . & . \\
    0 & 0 & * \\
    0 & 0 & *
  \end{array} \right)
\end{equation*}

где на месте \(*\) может стоять любое число.

\begin{enumerate}
\item
  Если хотя бы одна из \(*\) не равна нулю, то система несовместна.
  
\item
  Если все \(*\) равны нулю, то отбрасываем все нулевые строки и работаем с
  квадратной матрицей (используя теорему \ref{thr:K-C} и метод Гаусса).
\end{enumerate}

Рассмотрим \quote{горизонтальную матрицу} \(A_{k, n}\; (n > k)\).  Методом
Гаусса её можно свести к ступенчатому видy.

\begin{equation*}
  A_{k, n} = \left( \begin{array}{cccc|c}
    . & . & . & . & . \\
    . & . & . & . & . \\
    . & . & . & . & .
  \end{array} \right)
    \Rarr{}
  A_{k, n} = \left( \begin{array}{cccc|c}
    . & . & . & . & . \\
    0 & . & . & . & . \\
    0 & 0 & . & . & .
  \end{array} \right)
\end{equation*}

Этой матрице соответствует совместная, но неопределенная СЛАУ. Переменные,
которым \quote{не хватило} строк в матрице (в данном случае это \(x_4\))
называются свободными и могут принимать любые значения, остальные переменные
называются связными (зависимыми). Их количество равно \(n - \rank A\)

\begin{remark}
  Не смотря на то, что такая СЛАУ имеет бесконечно много решений, не любой набор
  будет её решением, т.е. решения это СЛАУ имеют определенную структуру.
\end{remark}
