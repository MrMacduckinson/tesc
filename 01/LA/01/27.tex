\subsection{%
  Уравнения прямой в пространстве.%
}

Прямая в пространстве может быть задана с помощью

\begin{enumerate}
\item
  2ух точек.
  
\item
  Точки и направляющего вектора.
  
\item
  Двух пересекающихся плоскостей.
\end{enumerate}

\subheader{Параметрическое уравнение прямой в пространстве}

Аналогично прямой на плоскости имеем вектор по направлению \(\vec{s}(m, n, k)\),
который коллинеарен вектору \(\vec{M_0M}\), где \(M_0 (x_0, y_0, z_0)\)
фиксированная точка, а \(M\) плавающая точка на искомой прямой.
  
\begin{equation*}
  \begin{cases}
    x = x_0 + t m \\
    y = y_0 + t n \\
    z = z_0 + t k
  \end{cases}
\end{equation*}

\begin{remark}
  По данному уравнению можно получить координаты вектора по направлению
  \(\vec{s}(m, n, k)\) и точку на прямой \(M_0 (x_0, y_0, z_0)\).
\end{remark}

\begin{remark}
  Физический смысл заключается в том, что \((x_0, y_0, z_0)\) это начальное
  положение, а \((m, n, k)\) это вектор скорости.
\end{remark}

\subheader{Каноническое уравнение прямой в пространстве}

Если в параметрически заданном уравнении прямой в пространстве выразить параметр
\(t\) во всех уравнениях и приравнять результаты, то получится каноническое
уравнение прямой
  
\begin{equation*}
  \frac{x - x_0}{m} = \frac{y - y_0}{n} = \frac{z - z_0}{k}
\end{equation*}

\begin{remark}
  По данному уравнению можно получить координаты вектора по направлению
  \(\vec{s}(m, n, k)\) и точку на прямой \(M_0(x_0, y_0, z_0)\).
\end{remark}

\subheader{Уравнение прямой в пространстве через две точки}

Две точки в пространстве можно рассмотреть как точку и вектор по направлению
искомой прямой. Таким образом задача построения прямой через две точки сводится
к задаче построения прямой через точку коллинеарно вектору. В итоге получится
следующая формула
  
\begin{equation*}
  \frac{x - x_1}{x_2 - x_1}
  = \frac{y - y_1}{y_2 - y_1}
  = \frac{z - z_1}{z_2 - z_1}
\end{equation*}

\subheader{Уравнение прямой в пространстве через пересечение двух плоскостей}

Рассмотрим систему из двух уравнений плоскостей
  
\begin{equation*}
  \begin{cases}
    A_1 x + B_1 y + C_1 z + D_1 = 0 \\
    A_2 x + B_2 y + C_2 z + D_2 = 0
  \end{cases}  
\end{equation*}

Возможны несколько вариантов.
  
\begin{enumerate}
\item
  Система несовместна (плоскости параллельны).
  
\item
  Система совместна, но не определена.
  
  \begin{enumerate}
  \item
    \(\rank = 1\) (плоскости совпадают).
    
  \item
    \(\rank = 2\) (плоскости пересекаются).
  \end{enumerate}
\end{enumerate}

В случае 2.b ФСР содержит 1 вектор \(X\), а общее решение СЛАУ можно записать
как \(\mathbb{X} = t X + X^0\). Эта запись эквивалентна уравнению \(\vec{r} = t
\vec{s} + \vec{r_0}\), которое является векторным уравнением прямой.


\begin{remark}
  Чтобы найти уравнение прямой, по которой пересекаются две плоскости необходимо
  найти одну (любую) точку пересечения этих плоскостей, а также векторное
  произведение их нормалей, которое будет являться вектором по направлению для
  искомой прямой.
\end{remark}
