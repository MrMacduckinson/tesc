\subsection{%
  Определение СЛАУ. Совместность, определенность. Теорема Крамера.%
}

\begin{definition}
  Система линейных алгебраических уравнений (\textit{СЛАУ}) это
  
  \begin{equation*}
    \left\{
      \begin{array}{clclclclc}
        a_{1, 1} x_1 & + & \dotsc & + & a_{1, m} x_m & = & b_1
      \\
        \vdots       &   & \ddots &   & \vdots       &   & \vdots
      \\
        a_{n, 1} x_1 & + & \dotsc & + & a_{n, m} x_m & = & b_n
      \end{array}
    \right.
  \end{equation*}
\end{definition}

\begin{remark}
  СЛАУ может быть записана в векторном виде как
  
  \begin{equation*}
    \vec{a_1} x_1 + \vec{a_2} x_2 \dotsc \vec{a_m} x_m = \vec{b}
  \end{equation*}
  
  где \(\vec{a}_i = (a_{1, i} \dotsc a_{n, i})^T\) и \(\vec{b} = (b_1 \dotsc
  b_n)^T\). Также  СЛАУ может быть записана в матричном виде: \(AX = B\), где
  \(A = (\vec{a}_1 \dotsc \vec{a}_m)\), а \(B\) это матрица-столбец (размера \(n
  \times 1\)).
\end{remark}

\begin{definition}
  Решение СЛАУ \((x_1 \dotsc x_n)\) это упорядоченный набор, который
  удовлетворяет всем уравнениям СЛАУ.
\end{definition}

\begin{definition}
  СЛАУ называется совместной, если она имеет хотя бы одно решение, в противном
  случае СЛАУ называется несовместной.
\end{definition}

\begin{definition}
  Совместная СЛАУ называется определенной, если она имеет ровно одно решение, в
  противном случае СЛАУ называется неопределенной.
\end{definition}

\begin{theorem}[Крамера]
  Если в СЛАУ \(AX = B\) выполнено условие \(\det A \ne 0\), то эта СЛАУ имеет
  единственное решение.
\end{theorem}

\begin{proof}
  Докажем существование решения. Пусть \(X^* = A^{-1} B\) решение данной
  системы, тогда
  
  \begin{enumerate}
  \item
    Обратная матрица \(A^{-1}\) определена, т.к. \(\det A \ne 0\) по условию
    теоремы.
    
  \item
    Умножение определено, т.к. размерности матриц корректны \(A^{-1}_{n, n}
    \cdot B_{n, 1}\).
  \end{enumerate}
  
  Докажем единственность решения. Пусть \(X_1 \ne X_2\) это различные решения,
  тогда \(X_1 = A^{-1} B = X_2 = A^{-1} B\), т.е. эти решения равны.
\end{proof}

\begin{remark}
  СЛАУ в которых \(\det A \ne 0\) называют СЛАУ Крамеровского типа.
\end{remark}

\subheader{Решение СЛАУ методом Крамера}

Если данная СЛАУ Крамеровского типа, то её решением является вектор \((x_1
\dotsc x_n)\), где

\begin{enumerate}
\item
  \(x_i = \dfrac{\Delta_i}{\Delta}\)
  
\item
  \(\Delta\) это определитель исходной матрицы.
  
\item
  \(\Delta_j\) это определитель матрицы, полученной заменой \(j\)-го столбца на
  столбец \(B\).
\end{enumerate}
