\subsection{%
  Ранг матрицы. Элементарные преобразования.%
}

\begin{definition}
  Рангом системы векторов называется количество базисных векторов \(\rank A\)
\end{definition}

\begin{remark}
  Пусть \(\set{\basis}_{j = 1}^k\) это базис системы \(\set{A}_{i = 1}^n\),
  тогда \(k \le n\) и при этом
  
  \begin{enumerate}
  \item
    Если \(k = n\), то \(A\) линейно независимая система.
    
  \item
    Если \(k < n\), то \(A\) линейно зависимая система.
  \end{enumerate}
\end{remark}

\begin{definition}
  Ранг матрицы это ранг системы её строк или ранг системы её столбцов (они
  равны).
\end{definition}

\begin{lemma} \label{lem:n-plus-one}
   Система из \(n + 1\) столбцов высоты \(n\) всегда линейно зависима.
\end{lemma}

\begin{proof}
  Если первые \(n\) столбцов линейно зависимы, то и вся система зависима. Если
  первые \(n\) столбцов независимы, то они образуют квадратную невырожденную
  матрицу \(A\) порядка \(n\).
  
  Составим матричное уравнение \(AX = b\), где \(b\) это последний столбце
  исходной системы. Т.к. матрица \(A\) не вырождена, то решением данной системы
  будет \(X = A^{-1} b\). Мы нашли такой набор коэффициентов \(X\), что \(b\)
  является линейной комбинацией системы \(A\), а значит что исходная система
  линейно зависима.
\end{proof}

\begin{theorem}
  Ранг системы строк равен рангу системы столбцов.
\end{theorem}

\begin{proof}
  Несложно доказать, что при элементарных операциях ранг матрицы не изменяется.
  Вычтем из всех небазисных строк такую линейную комбинацию базисных строк,
  чтобы они обнулились, при этом базисные строки останутся неизменными.
  
  Рассмотрим матрицу \(D\), которая расположена на пересечении базис строк и
  базисных столбцов. Она также останется неизменной, т.к. базисные строки не
  изменились.
  
  Покажем, что столбцы матрицы \(D\) линейно независимы. Если бы они были
  линейно зависимыми, то мы могли бы дополнить их нулями и при этом линейная
  зависимость сохранилась. Однако при дополнении этих столбцов нулями мы
  получаем базисные столбцы исходной матрицы (после обнуления линейно зависимых
  строк), а они линейно независимы. Получаем противоречие.
  
  По лемме \ref{lem:n-plus-one} получаем, что столбцов в матрице \(D\) (а значит
  и базисных столбцов в исходной матрице) не больше, чем строк в матрице \(D\)
  (а значит и базисных строк в исходной матрице), в противном случае столбцы
  были бы линейно зависимы, что противоречит доказанному выше.
  
  Таким образом, получаем, что столбцовый ранг не больше строчного ранга.
  Проделав аналогичные операции для транспонированной исходной матрицы и
  учитывая, что ранг столбцов исходной матрицы это ранг строк транспонированной
  матрицы и наоборот, получим, что строчный ранг не больше столбцового. Это
  значит, что строчный и столбцовый ранги равны.
\end{proof}

\begin{remark}
  К элементарным преобразованиям относятся
  
  \begin{enumerate}
  \item
    Умножение строки на ненулевое число.
    
  \item
    Перестановка строк.
    
  \item
    Прибавление одной строки к другой.
  \end{enumerate}
\end{remark}

\begin{definition}
  Матрица \(A\) эквивалентна матрице \(B\) (\(A \sim B\)), если она получена из
  матрицы \(B\) с помощью элементарных преобразований.
\end{definition}
