\subsection{%
  Подпространство. Линейная оболочка.%
}

\begin{definition}
  \(L'\) называется подпространством линейного пространства \(L\), если
  
  \begin{enumerate}
  \item
    \(L' \subset L\)
    
  \item
    Любая линейная комбинация элементов из \(L'\) лежит в \(L'\).
  \end{enumerate}
\end{definition}

\begin{remark}
  Линейное подпространство является линейным пространством
  
  \begin{enumerate}
  \item
    Нейтральный элемент получается по \(2^{\circ}\) для \(\lambda = 0\).
    
  \item
    Обратный элемент получается по \(2^{\circ}\) для \(\lambda = -1\).
    
  \item
    Остальные аксиомы линейного пространства очевидны.
  \end{enumerate}
\end{remark}

\begin{definition}
  \(L'\) называется собственным подпространством \(L\), если \(L' \ne
  \varnothing\) и \(L' \ne L\). В противном случае \(L'\) называется
  несобственным подпространством.
\end{definition}

\begin{theorem}
  Размерность собственного подпространства меньше размерности пространства.
  
  \begin{equation*}
    L' \subset L \implies \dim L' < \dim L
  \end{equation*}
\end{theorem}

\begin{proof}
  От противного, пусть \(\dim L' = \dim L\). \(L' \subset L \implies \exists x
  \in L \given x \notin L'\).
  
  Мы предположили, что \(\dim L' = \dim L\), значит размерности базисов также
  совпадают. Выберем такую систему \(\{ \basis_i \}\), которая будет базисом и
  для \(L'\), и для \(L\). Разложим \(x\) по этой системе \(x = \sum \lambda_i
  \basis_i\).
  
  По определению подпространства, любая линейная комбинация элементов этого
  подпространства должна лежать в этом подпространстве, т.е. \(x \in L'\).
  Противоречие.
\end{proof}

\begin{definition}
  Линейной оболочкой системы векторов \(\set{\alpha_i}_{i = 1}^n\) называется
  совокупность всевозможных линейных комбинаций  векторов этой системы.
\end{definition}

\begin{remark}
  Система векторов \(\set{\alpha_i}_{i = 1}^n\) может содержать линейно
  зависимые векторы.
\end{remark}

\begin{remark}
  Линейная оболочка нулевого вектора это несобственное подпространство.
\end{remark}

\begin{remark}
  Допустим, мы взяли несколько векторов из линейного пространства. Тогда их
  линейная оболочка будет являться подпространством исходного линейного
  пространства.
\end{remark}

\begin{definition}
  Суммой подпространств \(L_1\) и \(L_2\) называется
  
  \begin{equation*}
    L_1 + L_2
    = \set{z = x + y \given \forall x \in L_1 \land \forall y \in L_2}
  \end{equation*}
\end{definition}

\begin{definition}
  Пересечением подпространств \(L_1\) и \(L_2\) называется
  
  \begin{equation*}
    L_1 \cdot L_2
    = \set{x \given \forall x \in L \given x \in L_1 \land x \in L_2}
  \end{equation*}
\end{definition}

\begin{definition}
  Прямой суммой подпространств \(L_1\) и \(L_2\) называется
  
  \begin{equation*}
    L_1 \oplus L_2
    = \set{
      x = x_1 + x_2
      \given \existsone x_1 \in L_1 \land \existsone x_2 \in L_2
    }
  \end{equation*}
\end{definition}

\begin{remark}
  Другими словами, (обычная) сумма подпространств \(L_1\) и \(L_2\) называется
  прямой суммой, если каждой вектор из этой суммы единственным образом разложим
  по подпространствам \(L_1\) и \(L_2\).
\end{remark}

\begin{theorem}[Формула Грассмана]
  Пусть \(L_1\) и \(L_2\) подпространства линейного пространства \(L\), тогда
  
  \begin{equation*}
    \dim L_1 + \dim L_2 = \dim(L_1 \cdot L_2) + \dim(L_1 + L_2)
  \end{equation*}
\end{theorem}
