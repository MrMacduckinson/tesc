\subsection{%
  Общее уравнение поверхности 2-го порядка. Канонические уравнения. Метод
  сечений.%
}
 
Общее уравнение поверхности 2ого порядка имеет вид

\begin{equation*}
  a_{11} x^2 + a_{22} y^2 + a_{33} z^2
  + a_{12} xy + a_{13} xz + a_{23} yz
  + a_{14} x + a_{24} y + a_{34} z + a_{44}
  = 0
\end{equation*}

\gallerydouble
  {01_33_01}{Эллиптический цилиндр \(\display{
    \frac{x^2}{a^2} + \frac{y^2}{b^2} = 1
  }\)}
  {01_33_02}{Гиперболический цилиндр \(\display{
    \frac{x^2}{a^2} - \frac{y^2}{b^2} = 1
  }\)}

\gallerydouble
  {01_33_03}{Параболический цилиндр \(\display{
    y^2 = 2px
  }\)}
  {01_33_04}{Эллипсоид \(\display{
    \frac{x^2}{a^2} + \frac{y^2}{b^2} + \frac{z^2}{c^2} = 1
  }\)}

\gallerydouble
  {01_33_05}{Эллиптический параболоид \(\display{
    \frac{x^2}{a^2} + \frac{y^2}{b^2} = \frac{z}{c}
  }\)}
  {01_33_06}{Гиперболический параболоид \(\display{
    \frac{x^2}{a^2} - \frac{y^2}{b^2} = \frac{z}{c}
  }\)}

\gallerydouble
  {01_33_07}{Однополостный гиперболоид \(\display{
    \frac{x^2}{a^2} + \frac{y^2}{b^2} - \frac{z^2}{c^2} = 1
  }\)}
  {01_33_08}{Двуполостный гиперболоид \(\display{
    \frac{x^2}{a^2} + \frac{y^2}{b^2} - \frac{z^2}{c^2} = -1
  }\)}

\galleryone{01_33_09}{Конус \(\display{
  \frac{x^2}{a^2} + \frac{y^2}{b^2} = \frac{z^2}{c^2}
}\)}

\begin{table}
  \setlength{\tabcolsep}{10pt}
  \renewcommand{\arraystretch}{1.5}

  \begin{tabular}{l|l|l|l}
    Кривая & Направляющая & Образующая & Примечание
  \\ \hline
    Эллиптический цилиндр & Эллипс & Прямая &
  \\
    Гиперболический цилиндр & Гипербола & Прямая &
  \\
    Параболический цилиндр & Парабола & Прямая &
  \\
    Эллипсоид & Эллипс & Эллипс &
  \\
    Эллиптический параболоид & Эллипс & Парабола &
  \\
    Гиперболический параболоид & Парабола & Парабола &
  \\
    Однополостный гиперболоид & Эллипс & Гипербола &
      Вращаем гиперболу вокруг оси \(Oz\)
  \\
    Двуполостный гиперболоид & Эллипс & Гипербола &
      Вращаем гиперболу вокруг оси \(Oy\)
  \\
    Конус & Эллипс & Прямая &
      Вращаем прямую вокруг оси \(Oz\)
  \end{tabular}  
\end{table}

\begin{remark}
  Метод сечений заключается в выяснении типа кривой в сечении плоскостью,
  параллельной какой-либо оси координат. Т.е. исходное уравнение исследуется при
  \(x = const\) или \(y = const\) или \(z = const\). Найдя уравнения кривых в
  3-4 разных сечениях можно сделать набросок поверхности.
\end{remark}
