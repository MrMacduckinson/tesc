\subsection{%
  Определители. Свойства.%
}

\begin{definition}
  Минор элемента \(a_{i, j}\) это определитель матрицы, полученной вычеркиванием
  \(i\)-ой строки и \(j\)-ого столбца (при сохранении порядка элементов).
\end{definition}

\begin{definition}
  Алгебраическим дополнением элемента \(a_{i, j}\) называется \(A_{i, j} =
  (-1)^{i + j} \cdot M_{i, j}\), где \(M_{i, j}\) это минор элемента \(a_{i,
  j}\).
\end{definition}

\begin{definition}
  Определитель матрицы это сумма произведений элементов в строке/столбце и их
  алгебраических дополнений.
  
  \begin{equation*}
    \abs{A}
    = \det A
    = \sum_{j/i = 1}^{n} (-1)^{i + j} \cdot M_{i, j} \cdot a_{i, j}
  \end{equation*}
  
  Причем в данном выражении либо \(i = const\), тогда говорят о разложении
  матрицы по строке, либо \(j = const\), тогда говорят о разложении по столбцу.
  Запись \(j/i\) означает, что мы идем либо по строке, либо по столбцу, в то
  время как другая переменная остается постоянной.
\end{definition}

\begin{remark}
  Можно вычислить определитель только квадратной матрицы.
\end{remark}

Определитель можно ввести по-другому.

\begin{definition}
  Для вычисления определителя необходимо
  
  \begin{enumerate}
  \item
    Взять из каждой строки и каждого столбца один элемент без повтора номера
    строки/столбца.
    
  \item
    Составить произведение \(n\) множителей \(a_{1, \alpha} \cdot a_{2,\beta}
    \cdot a_{3, \gamma} \cdot \dotsc \cdot a_{n, \omega}\) (множители
    сортируются по номеру строки, из которой они были взяты).
    
  \item
    Если полученная перестановка \textbf{номеров столбцов} четная, то это
    произведение берется с плюсом. Иначе~--- с минусом.
    
  \item
    Определитель будет равен сумме всех полученных произведений.
  \end{enumerate}
\end{definition}

\begin{remark}
  Перестановка чисел \(a_1, \dotsc, a_p\) называется четной, если её можно
  отсортировать за четное число обменов соседних элементов. В противном случае
  она называется нечетной.
\end{remark}

\begin{example}
  Перестановка \((2, 3, 1)\)~--- четная, потому что \((2, 3, 1) \to (2, 1, 3)
  \to (1, 2, 3)\). Потребовалось 2 обмена соседних элементов, чтобы
  отсортировать перестановку, значит она четная.
  
  Перестановка \((2, 1, 3)\)~--- нечетная, потому что \((2, 1, 3) \to (1, 2,
  3)\). Потребовался один обмен, значит перестановка нечетная.
\end{example}

\begin{theorem}
  Значение определителя не зависит от выбора строки/столбца, по которому он
  будет разложен.
\end{theorem}

\begin{theorem}
  Определитель матрицы с нулевой строкой/столбцом равен нулю.
\end{theorem}

\begin{proof}
  Разложим по этой строке/столбцу.
\end{proof}

\begin{theorem}
  Определитель транспонированной матрицы равен определителю исходной матрицы.
\end{theorem}

\begin{proof}
  Разложим определитель исходной матрицы по произвольной строке. Далее разложим
  определитель транспонированной матрицы по соответствующему столбцу. Будем
  раскладывать определители, пока не дойдем до произведений чисел. В итоге
  получим два одинаковых выражения.
\end{proof}

\begin{theorem} \label{thr:det-swap}
  Перестановка строк/столбцов местами меняет знак определителя на
  противоположный.
\end{theorem}

\begin{proof}
  Мат. индукция. В качестве \textit{базы} рассмотрим матрицу \(2 \times 2\).
  
  \begin{equation*}
    \mtxv{
      a_{1, 1} & a_{1, 2} \\
      a_{2,1} & a_{2,2}
    }
    = a_{1, 1} a_{2,2} - a_{2,1} a_{1, 2}
  \end{equation*}
    
  Заметим, что
  
  \begin{equation*}
    a_{1, 1} a_{2, 2} - a_{2, 1} a_{1, 2}
    = -(a_{2, 1} a_{1, 2} - a_{1, 1} a_{2, 2})
    = \mtxv{
      a_{2, 1} & a_{2, 2} \\
      a_{1, 1} & a_{1, 2}
    }
  \end{equation*}
  
  Получается, если поменять местами первую и вторую строку, то определитель
  поменяет свой знак.
  
  Далее \textit{переход}. Пусть для матрицы \(k \times k\) выполняется это
  свойство. Рассмотрим матрицу \((k + 1) \times (k + 1)\).
  
  \begin{equation*}
    \mtxv{
      a_{1, 1}     & a_{1, 2}     & \dotsc & a_{1, k + 1}     \\
      a_{2, 1}     & a_{2, 2}     & \dotsc & a_{1, k + 1}     \\
      \vdots       & \vdots       & \ddots & \vdots           \\
      a_{k + 1, 1} & a_{k + 1, 2} & \dotsc & a_{k + 1, k + 1}
    }
    = \sum_{i = 1}^{k + 1} (-1)^{1 + i} a_{1, i} M_{1, i}
  \end{equation*}
  
  В данной сумме все миноры имеют размер \(k \times k\). Поменяем местами
  \(i\)-тую и \(j\)-тую строку. Обозначим \(M'_{1, i} = -M_{1, i}\). Тогда
  получим
  
  \begin{equation*}
    \det A'_{k + 1}
    = \sum_{i = 1}^{k + 1} (-1)^{1 + i} a_{1, i} M'_{1, i}
    = -\sum_{i = 1}^{k + 1} (-1)^{1 + i} a_{1, i} M_{1, i}
    = -\det A_{k + 1}
  \end{equation*}
  
  Доказательство для столбцов аналогично.
\end{proof}

\begin{theorem}
  Если в матрице есть равные строки/столбцы, то определитель равен нулю.
\end{theorem}

\begin{proof}
  Поменяем местами равные строки/столбцы. По \ref{thr:det-swap}, определитель
  должен изменить свой знак, однако матрица осталась прежней. Такое возможно
  только в случае, если определитель равен нулю.
\end{proof}

\begin{theorem}
  Если умножить строку/столбец на число, то определитель также умножится на это
  число.
\end{theorem}

\begin{proof}
  Разложим определитель по строке \(i\), которую мы умножили на некоторое число
  \(\lambda\).
  
  \begin{equation*}
    \begin{aligned}
      \abs{A'}
      & = \sum_{j = 1}^{n} (-1)^{i + j} \cdot \lambda a_{i, j} \cdot M_{i, j}
    \\
      & = \lambda \sum_{j = 1}^{n} (-1)^{i + j} \cdot a_{i, j} \cdot M_{i, j}
    \\
      & = \lambda \abs{A}
    \end{aligned}
  \end{equation*}
\end{proof}

\begin{lemma} \label{lem:det-compose}
  Если все элементы \(k\)-той строки/столбца определителя представлены в виде
  сумм \(a_{k, j} + b_{k, j}\), то определитель можно представить в виде суммы
  соответствующих определителей.
  
  \begin{equation*}
    \mtxv{
      a_{1, 1}            & a_{1, 2}            & \dotsc & a_{1, n}
    \\
      \vdots              & \vdots              & \ddots & \vdots
    \\
      a_{k, 1} + b_{k, 1} & a_{k, 2} + b_{k, 2} & \dotsc & a_{k, n} + b_{k, n}
    \\
      \vdots              & \vdots              & \ddots & \vdots
    \\
      a_{n, 1}            & a_{n,2}             & \dotsc & a_{n, n}
    }
    = \mtxv{
        a_{1, 1} & a_{1, 2} & \dotsc & a_{1, n}
      \\
        \vdots   & \vdots   & \ddots & \vdots
      \\
        a_{k, 1} & a_{k, 2} & \dotsc & a_{k, n}
      \\
        \vdots   & \vdots   & \ddots & \vdots
      \\
        a_{n, 1} & a_{n,2}  & \dotsc & a_{n, n}
    } + \mtxv{
        a_{1, 1} & a_{1, 2} & \dotsc & a_{1, n}
      \\
        \vdots   & \vdots   & \ddots & \vdots
      \\
        b_{k, 1} & b_{k, 2} & \dotsc & b_{k, n}
      \\
        \vdots   & \vdots   & \ddots & \vdots
      \\
        a_{n, 1} & a_{n,2}  & \dotsc & a_{n, n}
    }
  \end{equation*}
  
  Преобразуем исходный определитель.
  
  \begin{equation*}
    \begin{split}
      \sum_{i_1, i_2, \dotsc, i_n}
      a_{1, i_1} \dotsc (a_{k, i_k} + b_{k, i_k}) \dotsc a_{n, i_n}
      (-1)^{p \setminus \set{ i_1, i_2, \dotsc, i_n }}
      =
    \\
      \sum_{i_1, i_2, \dotsc, i_n}
      \dotsc a_{k, i_k} \dotsc
      (-1)^{p \setminus \set{i_1, i_2, \dotsc, i_n}}
      +
      \sum_{i_1, i_2, \dotsc, i_n}
      \dotsc b_{k, i_k} \dotsc
      (-1)^{p \setminus \set{i_1, i_2, \dotsc, i_n}}
      =
    \\
      \mtxv{
        a_{1, 1} & a_{1, 2} & \dotsc & a_{1, n}
      \\
        \vdots   & \vdots   & \ddots & \vdots
      \\
        a_{k, 1} & a_{k, 2} & \dotsc & a_{k, n}
      \\
        \vdots   & \vdots   & \ddots & \vdots
      \\
        a_{n, 1} & a_{n,2}  & \dotsc & a_{n, n}
      }
      +
      \mtxv{
        a_{1, 1} & a_{1, 2} & \dotsc & a_{1, n}
      \\
        \vdots   & \vdots   & \ddots & \vdots
      \\
        b_{k, 1} & b_{k, 2} & \dotsc & b_{k, n}
      \\
        \vdots   & \vdots   & \ddots & \vdots
      \\
        a_{n, 1} & a_{n,2}  & \dotsc & a_{n, n}
      }
    \end{split}
  \end{equation*}
\end{lemma}

\begin{theorem}
  Если к любой строке/столбцу можно добавить другую строку/столбец умноженную на
  число, то определитель матрицы в этом случае не изменится.
\end{theorem}

\begin{proof}
  Требуется доказать, что
  
  \begin{equation*}
    \mtxv{
      \dotsc   & \dotsc   & \dotsc & \dotsc   \\
      a_{i, 1} & a_{i, 2} & \dotsc & a_{i, n} \\
      \vdots   & \vdots   & \ddots & \vdots   \\
      a_{k, 1} & a_{k, 2} & \dotsc & a_{k, n} \\
      \dotsc   & \dotsc   & \dotsc & \dotsc
    }
    = \mtxv{
      \dotsc   & \dotsc   & \dotsc & \dotsc   \\
      a_{i, 1} & a_{i, 2} & \dotsc & a_{i, n} \\
      \vdots   & \vdots   & \ddots & \vdots    \\
        a_{k, 1} + \lambda a_{i, 1} &
        a_{k, 2} + \lambda a_{i, 2} &
        \dotsc &
        a_{k, n} + \lambda a_{i, n} \\
      \dotsc   & \dotsc   & \dotsc & \dotsc
    }
  \end{equation*}
\end{proof}

Определитель, стоящий в правой части этого равенства, можно представить в виде
суммы двух определителей, один из которых является исходным, а второй имеет две
пропорциональные друг другу строки и, следовательно, равен нулю.

\begin{theorem}
  Если одна из строк/столбцов является линейной комбинацией других
  строк/столбцов, то определитель матрицы равен нулю.
\end{theorem}

\begin{proof}
   По \ref{lem:det-compose}, определитель такой матрицы можно разложить на
   несколько других, которые будут иметь две пропорциональные друг другу строки
  \(\implies\) будут равны нулю \(\implies\) начальный определитель равен нулю.
\end{proof}

\begin{theorem}
  Определитель обратной матрицы равен \(\det (A^{-1}) = \frac{1}{\det A}\)
\end{theorem}

\begin{proof}
  Подставим в \ref{thr:det-mul} вместо \(B\)~--- \(A^{-1}\), а вместо \(A \cdot
  B\)~--- единицу.
\end{proof}

\begin{definition}
  Главной диагональю матрицы называются элементы, для которых номер их строки
  равен номеру их столбца.
\end{definition}

\begin{theorem}
  Определитель диагональной или треугольной матрицы равен произведению элементов
  на главной диагонали.
\end{theorem}

\begin{proof}
  \textbf{Диагональная матрица:} разложим по первому столбцу
  
  \begin{equation*}
    \mtxv{
      a_{1, 1} & 0 & 0 \\
      0 & a_{2, 2} & 0 \\
      0 & 0 & a_{3, 3}
    }
    = a_{1, 1} \cdot \mtxv{
      a_{2, 2} & 0 \\
      0 & a_{3,3}
    }
  \end{equation*}
  
  Остальные слагаемые уйдут, т.к. имеют коэффициент ноль. Продолжая раскладывать
  по первому столбцу, мы получим искомое равенство.
  
  \begin{equation*}
    \abs{A} = a_{1, 1} \cdot \dotsc \cdot a_{n, n}
  \end{equation*}

  \textbf{Треугольная матрица:} все аналогично, но
  
  \begin{enumerate}
  \item
    Если он верхняя, то нужно раскладывать по первому столбцу.
    
  \item
    Если она нижняя, то по первой строке.
  \end{enumerate}
\end{proof}

\begin{theorem} \label{thr:det-mul}
  Определитель произведения матриц равен произведению определителей матриц.
  
  \begin{equation*}
    \det (A \cdot B) = \det A \cdot \det B
  \end{equation*}
\end{theorem}
