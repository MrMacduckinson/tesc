\subsection{%
  Кривые второго порядка. Специальные определения. Канонические уравнения.
  Характеристики.%
} \label{sec:01-29}

Кривые второго порядка также называются коническими сечениями, т.к. они
получаются при сечении конуса под различными углами (\figref{01_29_01}). Также
можно встретить название \quote{квадрики}.

\galleryone{01_29_01}{Конические сечения}

\begin{definition}
  Параболой называется множество точек плоскости равноудаленных от данной точки
  и данной прямой (точка не лежит на этой прямой). Данная точка называется
  фокусом \(F\), а данная прямая~--- директрисой \(d\).
\end{definition}

\begin{theorem}
  Каноническое уравнение параболы \(y^2 = 2px\).
\end{theorem}

\begin{proof}
  Возьмем произвольную точку \(M (x, y)\), принадлежащую параболе, введём
  \quote{удобную} СК (\figref{01_29_02}). Имеем

  \begin{equation*}
    \rho = \sqrt{\prh{x - \frac{p}{2}}^2 + y^2}
    \qquad
    h = x + \frac{p}{2}
  \end{equation*}

  Согласно данному ранее определению приравняем их. Упростим и получим

  \begin{equation*}
    \begin{aligned}
      \sqrt{\prh{x - \frac{p}{2}}^2 + y^2} = x + \frac{p}{2}
    \\
      x^2 - x p + \frac{p^2}{4} + y^2 = x^2 + x p + \frac{p^2}{4}
    \\
      y^2 = 2 p x
    \end{aligned}
  \end{equation*}
\end{proof}

\begin{remark}
  Проверить, что данное уравнение описывает \textbf{только} точки параболы можно
  подставив \(y^2 = 2 p x\) (при \(x \ge 0\)) в уравнения для \(\rho\) и \(h\) и
  сравнив результаты.

  \begin{equation*}
    \begin{aligned}
      \rho
      = \sqrt{\prh{x - \frac{p}{2}}^2 + y^2}
      = \sqrt{x^2 - px + \frac{p^2}{4} + 2px}
      = \sqrt{x^2 + px + \frac{p^2}{4}}
      = \sqrt{\prh{x + \frac{p}{2}}^2}
      = \abs{x + \frac{p}{2}}
    \\
      x \ge 0 \implies \rho = x + \frac{p}{2}
    \end{aligned}
  \end{equation*}
\end{remark}

\begin{remark}
  \(p\) называется параболическим параметром.
\end{remark}

\gallerydouble
  {01_29_02}{Каноническое уравнение параболы}
  {01_29_03}{Каноническое уравнение эллипса}

\begin{definition}
  Эллипсом называется множество точек плоскости сумма расстояний от которых до
  двух данных точек постоянная. Данные точки называются фокусами \(F_1\) и
  \(F_2\).
\end{definition}

\begin{theorem}
  Каноническое уравнение эллипса \(\display{\frac{x^2}{a^2} + \frac{y^2}{b^2} =
  1}\).
\end{theorem}

\begin{proof}
  Возьмем произвольную точку \(M (x, y)\), принадлежащую эллипсу, введём
  \quote{удобную} СК (\figref{01_29_03}). Имеем

  \begin{equation*}
    r_1 = \sqrt{(x + c)^2 + y^2}
    \qquad
    r_2 = \sqrt{(x - c)^2 + y^2}
  \end{equation*}

  Согласно данному ранее определению их сумма должна быть постоянная, обозначим
  её \(2 a\). Упростим и получим

  \begin{equation*}
    \begin{aligned}
      \sqrt{(x + c)^2 + y^2} + \sqrt{(x - c)^2 + y^2} = 2a    
    \\
      (x + c)^2 + y^2 = 4 a^2 - 4a \sqrt{(x - c)^2 + y^2} + (x - c)^2 + y^2
    \\
      2 c x = 4 a^2 - 4 a \sqrt{(x - c)^2 + y^2} - 2 c x
    \\
      a \sqrt{(x - c)^2 + y^2} = a^2 - cx
    \\
      a^2 x^2 - 2 a^2 c x + a^2 c^2 + a^2 y^2 = a^4 - 2 a^2 c x + c^2 x^2
    \\
      (a^2 x^2 - c^2 x^2) + a^2 y^2 = (a^4 - a^2 c^2)
    \\
      x^2 (a^2 - c^2) + a^2 y^2 = a^2 (a^2 - c^2)
    \end{aligned}
  \end{equation*}

  Исходя из определения \(2 a > 2 c\), значит \(a > c\), тогда обозначим
  \(a^2 - c^2 = b^2\) и получим искомое уравнение.

  \begin{equation*}
    \begin{aligned}
      x^2 b^2 + y^2 a^2 = a^2 b^2
    \\
      \frac{x^2}{a^2} + \frac{y^2}{b^2} = 1
    \end{aligned}
  \end{equation*}
\end{proof}

\begin{remark}
  Параметр эллипса \(a\) называется большой полуосью, а \(b\)~--- малой
  полуосью.
\end{remark}

Эллипс обладает следующими свойствами

\begin{enumerate}
\item
  Симметрия относительно \(Ox\), \(Oy\) и \(O\).

\item
  Ограничен прямоугольником размера \(2 a \times 2 b\).

\item
  Степень \quote{сжатости} эллипса характеризуется отношением
  \(\display{\frac{b}{a}}\).
\end{enumerate}

\begin{definition}
  Гиперболой называется множество точек плоскости модуль разности расстояний от
  которых до двух данных точек постоянен. Данные точки называются фокусами
  \(F_1\) и \(F_2\).
\end{definition}

\begin{theorem}
  Каноническое уравнение гиперболы \(\display{\frac{x^2}{a^2} - \frac{y^2}{b^2}
  = 1}\).
\end{theorem}

\begin{proof}
  Вывод уравнения гиперболы полностью аналогичен выводу уравнения эллипса,
  только в случае гиперболы мы за \(b^2\) обозначаем \(a^2 + c^2\).
\end{proof}

\begin{remark}
  Параметр гиперболы \(a\) называется действительной полуосью, а \(b\)~---
   мнимой полуосью.
\end{remark}

Гипербола обладает следующими свойствами

\begin{enumerate}
\item
  Симметрия относительно \(Ox\), \(Oy\) и \(O\).

\item
  Не ограничена, но на бесконечности ведёт себя как пара прямых.
\end{enumerate}

\begin{remark}
  Асимптоты гиперболы на бесконечности задаются уравнениями

  \begin{equation*}
    y_{ac} = \pm \frac{b}{a} \cdot x
  \end{equation*}
\end{remark}

\galleryone{01_29_04}{Сопряженные гиперболы}

\begin{remark}
  Сопряженная к данной гипербола задается уравнением

  \begin{equation*}
    \frac{x^2}{a^2} - \frac{y^2}{b^2} = -1
  \end{equation*}

  Например, на \figref{01_29_04} изображены две сопряженные гиперболы.
\end{remark}

\begin{definition}
  На данный момент мы определяем эксцентриситет как величину равную единице для
  параболы и \(\display{\frac{c}{a}}\) для гиперболы и эллипса. Далее будет
  доказано (\ref{thr:conic-ecc}), что эксцентриситет это характеристика кривой
  второго порядка, показывающая отношение расстояния от фокуса до точки на
  кривой к расстоянию от директрисы до той же точки кривой, т.е. \(\ecc =
  \frac{\rho}{h}\).
\end{definition}

Рассмотрим эксцентриситет эллипса.

\begin{equation*}
  \ecc_{\text{элл}}
  = \frac{c}{a}
  = \frac{\sqrt{a^2 - b^2}}{a}
  = \sqrt{1 - \frac{b^2}{a^2}}
\end{equation*}

Т.к. \(b < a\), то \(0 \le \ecc_{\text{элл}} < 1\). Если у эллипса большая и
малая полуоси совпадают (\(a = b\)), то это окружность и её эксцентриситет равен
нулю. Аналогично можно рассмотреть эксцентриситет гиперболы и показать, что
\(\ecc_{\text{гип}} > 1\).

\begin{definition}
  Директрисой \(d_i\) (соответствующей фокусу \(F_i\)) называется прямая,
  которая

  \begin{enumerate}
  \item
    Лежит в той же полуплоскости, что и фокус \(F_i\).

  \item
    Перпендикулярна оси, проходящей через фокусы (оси \(Ox\) в каноническом
    случае).

  \item
    Находится на расстоянии $\display{\frac{a}{\ecc}}$ от центра кривой (начала
    координат в каноническом случае).
  \end{enumerate}
\end{definition}

Уравнение директрисы для эллипса и гиперболы можно вычислить по формуле
\(\display{x = \pm \frac{a}{\ecc}}\), для параболы~--- \(\display{x =
-\frac{p}{2}}\).
