\subsection{%
  Сравнение бесконечно малых. Теоремы об эквивалентных функциях.%
}

Пусть \(\infsmall_1(x)\) и \(\infsmall_2(x)\)~--- б.м. в точке \(a\), тогда

\begin{enumerate}
\item
  \(
    \lim_{x \to a} \frac{\infsmall_1(x)}{\infsmall_2(x)} = 0
    \implies \infsmall_1(x)
  \) более высокого порядка, чем \(\infsmall_2(x)\).

\item
  \(
    \lim_{x \to a} \frac{\infsmall_1(x)}{\infsmall_2(x)} \in \RR \setminus
    \set{0} \implies \infsmall_1(x)
  \) и \(\infsmall_2(x)\) одного порядка.

\item
  \(
    \lim_{x \to a} \frac{\infsmall_1(x)}{\infsmall_2(x)} = \infty
    \implies \infsmall_1(x)
  \) более низкого порядка, чем \(\infsmall_2(x)\).

\item
  \(
    \lim_{x \to a} \frac{\infsmall_1(x)}{\infsmall_2(x)^k} \in \RR \setminus
    \set{0} \implies \infsmall_1(x)
  \) имеет порядок \(k\) по отношению к \(\infsmall_2(x)\).
\end{enumerate}

Пусть \(\infbig_1(x)\) и \(\infbig_2(x)\)~--- б.б. в точке \(a\), тогда

\begin{enumerate}
\item
  \(
    \lim_{x \to a} \frac{\infbig_1(x)}{\infbig_2(x)} = 0
    \implies \infbig_1(x)
  \) более низкого порядка, чем \(\infbig_2(x)\).

\item
  \(
    \lim_{x \to a} \frac{\infbig_1(x)}{\infbig_2(x)} \in \RR \setminus
    \set{0} \implies \infbig_1(x)
  \) и \(\infbig_2(x)\) одного порядка.

\item
  \(
    \lim_{x \to a} \frac{\infbig_1(x)}{\infbig_2(x)} = \infty
    \implies \infbig_1(x)
  \) более высокого порядка, чем \(\infbig_2(x)\).

\item
  \(
    \lim_{x \to a} \frac{\infbig_1(x)}{\infbig_2(x)^k} \in \RR \setminus
    \set{0} \implies \infbig_1(x)
  \) имеет порядок \(k\) по отношению к \(\infbig_2(x)\).
\end{enumerate}

\begin{remark}
  \(\smallo(f(x))\)~--- обозначение б.м. более высокого порядка для функции
  \(f(x)\). Обычно это обозначение используют для сравнения с некоторым эталоном
  \(\Delta x\), например

  \begin{equation*}
    \infsmall(x) = \smallo(\Delta x)
    \iff
    \lim_{x \to a} \frac{\infsmall(x)}{\Delta x} = 0
  \end{equation*}
\end{remark}

\begin{remark}
  Таким образом, разрешение неопределенностей вида
  \(\display{\left[\frac{0}{0}\right]}\) и
  \(\display{\left[\frac{\infty}{\infty}\right]}\) это на самом деле сравнение
  бесконечно малых и бесконечно больших функций.
\end{remark}

\begin{definition}
  Если \(\lim_{x \to a} \frac{\infsmall_1(x)}{\infsmall_2(x)} = 1\), то
  \(\infsmall_1(x)\) и \(\infsmall_2(x)\) называются эквивалентными б.м.
  функциями.

  \begin{equation*}
    \lim_{x \to a} \frac{\infsmall_1(x)}{\infsmall_2(x)} = 1
    \iff
    \infsmall_1(x) \sim \infsmall_2(x)
  \end{equation*}
\end{definition}

\begin{remark}
  Геометрический смысл эквивалентных функций заключается в том, что в малой
  окрестности точки \(a\) графики функций сливаются. В качестве примера можно
  рассмотреть функции \(\sin x\) и \(x\) (\figref{01_08_01}).
\end{remark}

\galleryone{01_08_01}{Эквивалентные функции}

\begin{theorem}
  В произведении можно заменять эквивалентные функции друг на друга при
  условии, что они эквивалентны \textbf{в одной точке}.

  \begin{equation*}
    \alpha(x) \sim \gamma(x)
    \implies
    \lim_{x \to a} \frac{\infsmall(x)}{\beta(x)}
    = \lim_{x \to a} \frac{\gamma(x)}{\beta(x)}
  \end{equation*}
\end{theorem}

\begin{proof}
  Домножим на \(\gamma(x)\) и воспользуемся свойством произведения пределов.

  \begin{equation*}
    \lim_{x \to a} \frac{\alpha(x)}{\beta(x)} 
    = \lim_{x \to a} \frac{\alpha(x) \cdot \gamma(x)}{\beta(x) \cdot \gamma(x)}
    = \lim_{x \to a} \frac{\alpha(x)}{\gamma(x)} \cdot
      \lim_{x \to a} \frac{\gamma(x)}{\beta(x)}
  \end{equation*}

  По определению эквивалентных функций левый предел равен единице, значит

  \begin{equation*}
    \lim_{x \to a} \frac{\alpha(x)}{\beta(x)}
    = \lim_{x \to a} \frac{\gamma(x)}{\beta(x)}
  \end{equation*}
\end{proof}

\begin{theorem}
  Разность эквивалентных \textbf{в одной точке} б.м. функций это б.м. более
  высокого порядка, чем эти функции.

  \begin{equation*}
    \alpha(x) \sim \beta(x)
    \implies
    \alpha(x) - \beta(x) = \smallo(\alpha(x)) = \smallo(\beta(x))
  \end{equation*}
\end{theorem}

\begin{proof}
  Сравним \(\alpha(x) - \beta(x)\) по определению сравнения б.м. функций, для
  этого вычислим следующий предел

  \begin{equation*}
    \lim_{x \to a} \frac{\alpha(x) - \beta(x)}{\alpha(x)}
    = \lim_{x \to a} 1 - \frac{\beta(x)}{\alpha(x)}
    = 1 - \lim_{x \to a} \frac{\beta(x)}{\alpha(x)}
  \end{equation*}

  По определению эквивалентных функций правый предел равен единице, значит

  \begin{equation*}
    \lim_{x \to a} \frac{\alpha(x) - \beta(x)}{\alpha(x)}
    = 1 - 1
    = 0
  \end{equation*}

  Таким образом \(\alpha(x) - \beta(x)\) это б.м. более высокого порядка, чем
  \(\alpha(x)\). Для \(\beta(x)\) доказательство аналогично.
\end{proof}

\begin{theorem}
  Произведение двух б.м. \textbf{в одной точке} функций это б.м. более высокого
  порядка, чем эти функции.

  \begin{equation*}
    \alpha(x) \cdot \beta(x) = \smallo(\alpha(x)) = \smallo(\beta(x))
  \end{equation*}
\end{theorem}

\begin{proof}  
  Сравним \(\alpha(x) \cdot \beta(x)\) по определению сравнения б.м. функций,
  для этого вычислим следующий предел

  \begin{equation*}
    \lim_{x \to a} \frac{\alpha(x) \cdot \beta(x)}{\alpha(x)}
    = \lim_{x \to a} \beta(x)
  \end{equation*}

  По определению б.м. функции этот предел равен нулю. Таким образом \(\alpha(x)
  \cdot \beta(x)\) это б.м. более высокого порядка, чем \(\alpha(x)\). Для
  \(\beta(x)\) доказательство аналогично.
\end{proof}
