\subsection{%
  Правила дифференцирования: производная и дифференциал суммы и произведения
  функций.%
} \label{sec:01-15}

\begin{theorem}
  Производная суммы равна сумме производных.  

  \begin{equation*}
    \prh{f(x) + g(x)}' = f'(x) + g'(x)
  \end{equation*}
\end{theorem}

\begin{proof}
  Воспользуемся определением производной.  

  \begin{equation*}
    \begin{aligned}
      (f(x) + g(x))'
      = \lim_{\Delta x \to 0} \frac{\Delta(f(x) + g(x))}{\Delta x}
      = \lim_{\Delta x \to 0} \frac{f(x + \Delta x) - f(x) + g(x + \Delta x) -
        g(x)}{\Delta x}
    \\
    (f(x) \cdot g(x))'
      = \lim_{\Delta x \to 0} \frac{f(x + \Delta x) - f(x)}{\Delta x}
        + \lim_{\Delta x \to 0} \frac{g(x + \Delta x) - g(x)}{\Delta x}
    \end{aligned}
  \end{equation*}

  Два полученных предела равны соответствующим производным по определению.
\end{proof}

\begin{theorem}
  Производная произведения функций равна

  \begin{equation*}
    (f(x) \cdot g(x))' = f'(x) \cdot g(x) + f(x) \cdot g'(x)
  \end{equation*}
\end{theorem}

\begin{proof}
  Воспользуемся определением производной.

  \begin{equation*}
    (f(x) \cdot g(x))'
    = \lim_{\Delta x \to 0} \frac{\Delta(f(x) \cdot g(x))}{\Delta x}
    = \lim_{\Delta x \to 0} \frac{f(x + \Delta x) \cdot g(x + \Delta x) - f(x)
      \cdot g(x)}{\Delta x}
  \end{equation*}

  В числителе добавим и вычтем \(f(x + \Delta x) \cdot g(x)\), чтобы разложить
  на множители.

  \begin{equation*}
    \begin{aligned}
      (f(x) \cdot g(x))'
      = \lim_{\Delta x \to 0} \frac{f(x + \Delta x) \cdot (g(x + \Delta x) - g(x))
        + g(x) (f(x + \Delta x) - f(x))}{\Delta x}
    \\
      (f(x) \cdot g(x))'
      = \lim_{\Delta x \to 0} (f(x + \Delta x)) \cdot
        \lim_{\Delta x \to 0} \frac{g(x + \Delta x) - g(x)}{\Delta x}
        + \lim_{\Delta x \to 0} (g(x)) \cdot
        \lim_{\Delta x \to 0} \frac{f(x + \Delta x) - f(x)}{\Delta x}
    \end{aligned}
  \end{equation*}

  По определению производной заменим два предела производными, а оставшиеся
  пределы просто вычислим и получим искомую формулу.

  \begin{equation*}
    (f(x) \cdot g(x))' = f(x) \cdot g'(x) + g(x) \cdot f'(x)
  \end{equation*}
\end{proof}

\begin{remark}
  Дифференциалы вычисляются аналогично производным.

  \begin{equation*}
    \begin{aligned}
      \dd (f(x) + g(x)) = \dd (f(x)) + \dd (g(x))
    \\
      \dd (f(x) \cdot g(x)) = \dd (f(x)) \cdot g(x) + f(x) \cdot \dd (g(x))
    \end{aligned}
  \end{equation*}
\end{remark}