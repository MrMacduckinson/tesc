\subsection{%
  Определение и классификация разрывов.%
}

\begin{definition}
  Разрывом называется отсутствие непрерывности в точке.
\end{definition}

\subheader{Классификация разрывов}

Классификация разрывов заключается в вычислении односторонних пределов.

К разрывам I-ого рода относятся

\begin{enumerate}
\item
  Устранимый разрыв.

\item
  Конечный неустранимый разрыв.

\item
  Разрыв со скачком.
\end{enumerate}

К разрывам II-ого рода относятся

\begin{enumerate}
\item
  Бесконечный разрыв.

\item
  Несобственный бесконечный разрыв.
\end{enumerate}

\gallerydouble
  {01_13_01}{Устранимый разрыв}
  {01_13_02}{Конечный неустранимый разрыв}

\begin{definition}
  Разрыв называется устранимым (\figref{01_13_01}), если

  \begin{enumerate}
  \item
    Функция не существует в этой точке.

  \item
    Предел в этой точке существует и конечен.
  \end{enumerate}
\end{definition}

\begin{definition}
  Разрыв называется конечным неустранимым (\figref{01_13_02}), если
  
  \begin{enumerate}
  \item
    Функция существует в этой точке.

  \item 
    Предел существует в этой точке и конечен.

  \item
    Значение функции не равно значению предела в этой точке.
  \end{enumerate}
\end{definition}

\gallerydouble
  {01_13_03}{Разрыв со скачком}
  {01_13_04}{Бесконечный разрыв}

\begin{definition}
  Разрыв называется разрывом со скачком (\figref{01_13_03}), если односторонние
  пределы в этой точке существуют и конечны, но не равны.
\end{definition}

\begin{remark}
  Функция может быть не определена в этой точке. Скачок определяется как модуль
  разности односторонних пределов.
\end{remark}

\begin{definition}
  Разрыв называется бесконечным (\figref{01_13_04}), если хотя бы один из
  односторонних пределов бесконечен.
\end{definition}

\galleryone{01_13_05}{Несобственный бесконечный разрыв}

\begin{definition}
  Разрыв называется несобственным бесконечным (\figref{01_13_05}), если не
  существует односторонних пределов.
\end{definition}
