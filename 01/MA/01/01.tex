\subsection{%
  Вещественная ось. Бесконечность. Окрестность точки.%
} \label{sec:01-01}

\begin{definition}
  Вещественная ось это геометрическое изображение множества всех вещественных
  чисел.
\end{definition}

\begin{remark}
  Вещественная ось не содержит \quote{дыр}. Это значит, что переменная проходя
  все точки интервала \(\interval{a}{b}\) принимает все вещественные значения из
  интервала \(\interval{a}{b}\). Другими словами, между геометрической осью
  (множеством точек) и \(\RR\) действует биекция.
\end{remark}

Отрезок \(A\) лежит \textit{левее} отрезка \(B\), если любая точка из отрезка
\(A\) не больше любой точки из отрезка \(B\): \(\forall a \in A, b \in B \given
a \le b\). Если один отрезок лежит левее другого, то между ними найдется как
минимум одна точка.

\begin{equation*}
  \forall A \text{ левее } B \given
  \exists c \in \RR \given
  \forall a \in A, b \in B \given
  a \le c \le b
\end{equation*}

Расширенная вещественная ось определяется следующими аксиомами

\begin{enumerate}
\item
  \(\bar{\RR} = \RR \cup \set{\pm\infty}\)

\item
  \(\forall x \in \RR \given -\infty < x < +\infty\)

\item
  \(x \cdot \pm \infty = \pm \infty\) (знаки как обычно)

\item
  \(x + \pm \infty = \pm \infty\)

\item
  \(x / \pm \Zero = \pm \infty\) (не ноль, а бесконечное приближение к нему)

\item
  \(x / \pm \infty = \pm \Zero\) (у нуля есть знак)
\end{enumerate}

\begin{definition}
  Дельта--окрестностью точки \(a\) называют \(\near{a}{\delta} =
  \interval{-\delta + a}{a + \delta}\).

  Если \(a = +\infty\), то \(\near{a}{\delta} = \interval{\delta}{+\infty}\).
  Если \(a = -\infty\), то \(\near{a}{\delta} = \interval{-\infty}{-\delta}\).
\end{definition}

\begin{definition}
  Проколотой дельта-окрестностью точки \(a\) называют \(\nearo{a}{\delta} =
  \near{a}{\delta} \setminus \set{a}\).
\end{definition}

\begin{remark}
  Если в некоторых окрестностях \(\delta_1 \dotsc \delta_n\) точки \(a\)
  выполняются условия \(P_1 \dotsc P_n\), то все эти условия будут выполняться в
  пересечении этих окрестностей.
\end{remark}

\begin{remark}
  Пересечение двух окрестностей одной точки это окрестность меньшего радиуса
  \(\near{a}{\delta} \cap \near{a}{\epsilon} = \near{a}{\min(\delta,
  \epsilon)}\).
\end{remark}

\begin{remark}
  Для двух различных точек всегда найдутся непересекающиеся окрестности \(a \ne
  b \implies \exists \delta > 0, \epsilon > 0 \given \near{a}{\delta} \cap
  \near{b}{\epsilon} = \varnothing\).
\end{remark}
