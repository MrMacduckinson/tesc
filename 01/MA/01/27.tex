\subsection{%
  Теоремы о дифференцируемых функциях. Правило Лопиталя.%
}

\begin{theorem}
  При неопределенностях вида \(\display{\left[\frac{0}{0}\right]}\) и
  \(\left[\frac{\infty}{\infty}\right]\) справедливо равенство

  \begin{equation*}
    \lim_{x \to a} \frac{f(x)}{g(x)} = \lim_{x \to a} \frac{f'(x)}{g'(x)}
  \end{equation*}

  при условии, что

  \begin{enumerate}
  \item
    \(f(x)\) и \(g(x)\) дифференцируемы в окрестности точки \(a\).

  \item
    \(g'(x) \neq 0\) в окрестности точки \(a\).
  
  \item
    \(\display{\exists \lim_{x \to a}{\frac{f'(x)}{g'(x)}}}\)
  \end{enumerate}
\end{theorem}

\begin{proof}
  \subsubheader{Шаг I}{\(\display{\left[\frac{0}{0}\right]}\)}

  По теореме Коши

  \begin{equation*}
    \begin{aligned}
      \exists \xi \in \near{a}{\abs{x - a}} \colon
      \frac{f(x) - f(a)}{g(x) - g(a)} = \frac{f'(\xi)}{g'(\xi)}
    \\
      \begin{cases}
        f(x) \isdiffd{a}
        \implies
        f(x) \iscontd{a}
        \implies
        \lim_{x \to a} f(x) = f(a)
      \\
        g(x) \isdiffd{a}
        \implies
        g(x) \iscontd{a}
        \implies
        \lim_{x \to a} g(x) = g(a)
      \end{cases}
      \qquad
      \left[\frac{0}{0}\right] \implies f(a) = g(a) = 0
    \\
      \frac{f(x)}{g(x)} = \frac{f'(\xi)}{g'(\xi)}
    \\
      \lim_{x \to a} \frac{f(x)}{g(x)} = \lim_{x \to a} \frac{f'(\xi)}{g'(\xi)}
    \end{aligned}
  \end{equation*}

  Рассмотрим предел \(\display{\lim_{x \to a}\frac{f'(x)}{g'(x)} = L}\),
  распишем его по определению

  \begin{equation*}
    \forall \epsilon > 0 \exists \delta > 0 \given
    \forall x \colon 0 < \abs{x - a} < \delta \implies
    \abs{\frac{f'(x)}{g'(x)} - L} < \epsilon
  \end{equation*}

  Однако \(\display{\forall x \colon 0 < |x - a| < \delta \given \exists \xi}\),
  такая что

  \begin{equation*}
    \begin{rcases}
      \lim_{x \to a} \frac{f(x)}{g(x)}
      = \lim_{x \to a} \frac{f'(\xi)}{g'(\xi)}
    \\
      \xi \in \near{a}{\abs{x - a}}
    \end{rcases}
    \implies
    \forall \epsilon > 0 \exists \delta_{\xi} > 0 \given
    \forall x \colon 0 < \abs{x - a} < \delta \implies
    \abs{\frac{f(x)}{g(x)} - L} < \epsilon
  \end{equation*}

  Это определение предела, значит мы получаем, что

  \begin{equation*}
    \lim_{x \to a} \frac{f(x)}{g(x)} = L = \lim_{x \to a} \frac{f'(x)}{g'(x)}
  \end{equation*}

  \subsubheader{Шаг II}{\(\display{\left[\frac{\infty}{\infty}\right]}\)}
  
  Так как \(f\), \(g\) б.б функции, то \(\display{\frac{1}{f}}\),
  \(\display{\frac{1}{g}}\) б.м.

  \begin{equation*}
    \lim_{x \to a} \frac{f(x)}{g(x)}
    = \lim_{x \to a} \frac{1}{\frac{1}{f(x)}}
        \cdot \lim_{x \to a} \frac{1}{g(x)}
    = \lim_{x \to a} {\frac{\frac{1}{g(x)}}{\frac{1}{f(x)}}}
  \end{equation*}

  Данный предел имеет неопределённость вида
  \(\display{\left[\frac{0}{0}\right]}\), как отношение б.м. Применим уже ранее
  доказанное правило Лопиталя.

  \begin{equation*}
    \lim_{x \to a} \frac{\frac{1}{g(x)}}{\frac{1}{f(x)}}
    = \lim_{x \to a} \frac{\prh{\frac{1}{g(x)}}'}{\prh{\frac{1}{f(x)}}'}
    = \lim_{x \to a} \frac{\frac{g'(x)}{g(x)^2}}{\frac{f'(x)}{f(x)^2}}
    = \lim_{x \to a} \frac{g'(x)}{f'(x)}
      \cdot \lim_{x \to a} \frac{f(x)^2}{g(x)^2}
    = \lim_{x \to a} \frac{g'(x)}{f'(x)}
      \cdot \lim_{x \to a} \prh{\frac{f(x)}{g(x)}}^2 
  \end{equation*}

  Итого получаем, что

  \begin{equation*}
    \begin{aligned}
      \lim_{x \to a} \frac{f(x)}{g(x)}
      = \lim_{x \to a} \frac{g'(x)}{f'(x)}
        \cdot \lim_{x \to a} \prh{\frac{f(x)}{g(x)}}^2
    \\
      \frac{1}{\lim_{x \to a} \frac{f(x)}{g(x)}}
      = \lim_{x \to a} \frac{g'(x)}{f'(x)}
    \\
      \lim_{x \to a} \frac{f(x)}{g(x)} = \lim_{x \to a} \frac{f'(x)}{g'(x)}
    \end{aligned}
  \end{equation*}
\end{proof}

\begin{remark}
  Правило Лопиталя можно применять несколько раз, например

  \begin{equation*}
    \lim_{x \to \infty} \frac{e^x}{x^3}
    = \lim_{x \to \infty} \frac{e^x}{3 x^2}
    = \lim_{x \to \infty} \frac{e^x}{6 x}
    = \lim_{x \to \infty} \frac{e^x}{6}
    = \infty
  \end{equation*}
\end{remark}

\begin{remark}
  Если после применения правила Лопиталя, получился несуществующий предел, это
  не значит, что исходный предел не существует, например

  \begin{equation*}
    \begin{aligned}
      \lim_{x \to \infty} \frac{x - \sin x}{x} = 1 - 0 = 1
    \\
      \lim_{x \to \infty} \frac{(x - \sin x)'}{x'}
      = \lim_{x \to \infty} \frac{1 - \cos x}{1} \to \nexists
    \end{aligned}
  \end{equation*}

  В данном случае правило Лопиталя неприменимо.
\end{remark}
