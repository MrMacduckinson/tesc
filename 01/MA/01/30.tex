\subsection{%
  Исследование функции: Выпуклость функции. Точки перегиба. Необходимое и
  достаточное условия перегиба.%
}

\begin{definition}
  Функция \(f(x)\) называется выпуклой вверх в точке \(x_0\), если она
  непрерывна в окрестности этой точки и касательные к графику функции \(f(x)\)
  для любого \(x\) из этой окрестности лежат выше графика функции
  (\figref{01_30_01}).
\end{definition}

\begin{definition}
  Функция \(f(x)\) называется выпуклой вниз в точке \(x_0\), если она непрерывна
  в окрестности этой точки и касательные к графику функции \(f(x)\) для любого
  \(x\) из этой окрестности лежат ниже графика функции.
\end{definition}

\begin{definition}
  Касательной к графику функции \(y = f(x)\) в точке \(x_0\) называется прямая,
  задающаяся уравнением \(y_{\text{кас}} = f'(x_0)(x - x_0) + f(x_0)\).
\end{definition}

\gallerydouble
  {01_30_01}{Выпуклая вверх функция}
  {01_30_02}{Точка перегиба}

\begin{definition}
  Если функция \(f(x)\) дифференцируема в точке \(x_0\) и при этом меняет
  характер выпуклости в этой точке, то точка \(x_0\) называется точкой перегиба
  (\figref{01_30_02}).
\end{definition}

\begin{remark}
  В малой окрестности точки перегиба график функции становится прямой, т.е.
  имеет нулевую выпуклость.
\end{remark}

\begin{theorem}[Необходимое условие \textbf{(гладкого?)} перегиба]
  Если \(f(x)\) дважды дифференцируема в точке \(x_0\) и при этом \(x_0\) точка
  перегиба, то \(f''(x_0) = 0\).
\end{theorem}

\begin{proof}
  Переобозначим \(g(x) = f'(x)\), тогда \(f''(x) = g'(x)\). Т.к. по определению
  точки перегиба в \(x_0\) меняется характер выпуклости, то \(f''(\xi)\) (где
  \(\xi \in \near{x_0}{x}\)) меняет свой знак, значит (в новых обозначениях)
  \(g'(x)\) меняет свой знак.
  
  Если \(g'(x)\) меняет свой знак, то согласно достаточному условию экстремума
  \(x_0\) это точка (гладкого) экстремума для функции \(g(x)\). Таким образом
  \(g'(x_0) = 0 = f''(x_0)\).
\end{proof}

\galleryone{01_30_03}{Достаточное условие перегиба}

\begin{theorem}[Достаточное условие перегиба]
  Если функция \(f(x)\) дважды дифференцируема в точке \(x_0\) и её вторая
  производная меняет знак при проходе через эту точку, то \(x_0\) это точка
  перегиба.

  \textbf{Отдельно стоит рассмотреть случай}
  
  Если функция \(f(x)\) дважды дифференцируема в точке \(x_0\) и её первая
  производная в этой точке бесконечна, то \(x_0\) это точка перегиба.
\end{theorem}

\begin{proof}
  Сделаем рисунок (\figref{01_30_03}), проведем касательную \(k(x) = f(x_0) +
  f'(x_0) (x - x_0)\). Выпуклость зависит от знака разности \(f(x) - k(x)\) в
  окрестности точки \(x_0\) (по определению выпуклой вверх/вниз функции
  необходимо проверить, лежит ли касательная над или под графиком функции).
  Распишем \(f(x)\) по формуле Тейлора в точке \(x_0\) до \(n = 1\), получим

  \begin{equation*}
    f(x) = f(x_0) + f'(x_0)(x - x_0) + r_1(f, x)
  \end{equation*}

  Подставим это и уравнение касательной в разность \(f(x) - k(x)\), получим

  \begin{equation*}
    \begin{aligned}
      f(x) - k(x)
      = f(x_0) + f'(x_0)(x - x_0) + r_1(f, x) - f(x_0) - f'(x_0)(x - x_0)
    \\
      f(x) - k(x) = r_1(f, x)
    \end{aligned}
  \end{equation*}
  
  Таким образом, необходимо узнать знак остаточного члена в формуле Тейлора для
  исследуемой функции. Чтобы это сделать, запишем этот остаток в форме Лагранжа,
  получим

  \begin{equation*}
    r_1(f, x) = \frac{f''(\xi)}{2!} \cdot (x - x_0)^2
    \qquad
    (\xi \in \interval{x_0}{x})
  \end{equation*}
  
  Т.к. \((x - x_0)^2 > 0\), то знак остатка (а следовательно и выпуклость)
  зависит от знака второй производной в окрестности точки \(x_0\).
\end{proof}
