\subsection{%
  Определение производной функции. Дифференцируемая функция. Дифференциал
  \(1\)-го порядка.%
}

\begin{important}
  Запись \(f(x) \isdiff[n]{\segment{a}{b}}\) означает, что функция \(f(x)\)
  дифференцируема на отрезке \(\segment{a}{b}\) (\(n\) раз).
  
  Запись \(f(x) \isdiffd[n]{x_0}\) означает, что функция \(f(x)\)
  дифференцируема в точке \(x_0\) (\(n\) раз).
\end{important}

\begin{definition}
  Производной функции в точке \(x_0\) называют предел отношения приращения
  функции к приращению аргумента.

  \begin{equation*}
    f'(x_0)
    = \lim_{\Delta x \to 0} \frac{\Delta y}{\Delta x}
    = \lim_{\Delta x \to 0} \frac{f(x_0 + \Delta x) - f(x_0)}{\Delta x}
  \end{equation*}
\end{definition}

\begin{definition}
  Функция дифференцируема в точке \(x_0 \iff \exists A \given \Delta y = A \cdot
  \Delta x + \smallo(\Delta x)\) при \(\Delta x \to 0\).
\end{definition}

\begin{theorem}[Критерий дифференцируемости]
  Функция называется дифференцируемой в точке \(x_0\), если существует конечная
  производная в этой точке.
\end{theorem}

\begin{proof}
  \ness Распишем производную по определению, после чего воспользуемся
  представлением функции пределом.

  \begin{equation*}
    \begin{aligned}
      f'(x) = \lim_{\Delta x \to 0} \frac{\Delta y}{\Delta x}
    \\
      \frac{\Delta y}{\Delta x} = f'(x) + \alpha(\Delta x)
        & \qquad \Big\vert \cdot \Delta x
    \\
      \Delta y = f'(x) \Delta x
        + \under{\alpha(\Delta x) \Delta x}{\smallo(\Delta x)}
    \end{aligned}
  \end{equation*}

  Обозначив \(f'(x) = A\), получаем требуемое равенство.

  \suff Имеем \(\Delta y = A \cdot \Delta x + \smallo(\Delta x)\), разделим это
  на \(\Delta x \neq 0\) и воспользуемся представлением функции пределом в
  обратную сторону.
  
  \begin{equation*}
    \frac{\Delta y}{\Delta x} = A + \alpha(\Delta x)
    \implies
    \lim_{\Delta x \to 0} \frac{\Delta y}{\Delta x} = A = f'(x)  
  \end{equation*}
\end{proof}

\begin{definition}
  Пусть \(\display{\lim_{x \to a} \frac{\beta(x)}{k \cdot \alpha^n (x)} = 1}\),
  тогда \(k \cdot \alpha^n(x)\) называют главной частью функции \(\beta(x)\).

  \begin{equation*}
    \beta(x) \sim k \cdot \alpha^n (x)
    \qquad
    \beta(x) = k \cdot \alpha^n (x) + \smallo(k \cdot \alpha^n (x))
  \end{equation*}
\end{definition}

\begin{remark}
  Главная часть функции позволяет с некоторой погрешностью заменять сложную
  (показательную, логарифмическую) функцию на простую (степенную), при этом
  порядок \(n\) показывает точность вычисления \(\beta(x)\).  
\end{remark}

\begin{definition}
  Дифференциалом (первого порядка) функции в точке называется главная (линейная)
  часть её приращения в этой точке \(\dd y = y '\Delta x\).
\end{definition}

\begin{remark}
  Дифференциал аргумента равен приращению \(\dd x = x' \Delta x = \Delta x\),
  поэтому производную можно обозначить в виде

  \begin{equation*}
    \dd y = y' \dd x \implies y' = \fullder{y}{x}
  \end{equation*}
\end{remark}

\begin{remark}
  Дифференциал служит простейшим приближением приращения, т.к. обычно его проще
  вычислить. При этом \(\dd y \neq \Delta y\), но при \(\Delta x \to 0 \given
  \dd y \to \Delta \to 0\) или \(dd y \approx \Delta y\).
\end{remark}

\begin{remark}
  С помощью дифференциала можно вычислять приближенное значение функции.

  \begin{equation*}
    f(x_0 + \Delta x)
    = f(x_0) + \dd f(x_0)
    = f(x_0) + f'(x_0) \Delta x
  \end{equation*}

  Чем меньше будет \(\Delta x\), тем точнее будут вычисления.
\end{remark}

\begin{example}
  Пусть \(f(x) = \sqrt[3]{x}\) и требуется найти \(f(67)\), тогда

  \begin{equation*}
    f(67)
    = f(64 + 3)
    \approx 4 + \frac{1}{\sqrt[3]{64^2}} \cdot 3
    \approx 4 + \frac{1}{16}
    \approx 4.0625
  \end{equation*}
\end{example}

\begin{theorem}
  Если функция дифференцируема в точке, то она непрерывна в этой точке.  
\end{theorem}

\begin{proof}
  По определению дифференцируемости \(\Delta y = A \Delta x + \smallo(\Delta
  x)\). Воспользуемся предельным переходом.

  \begin{equation*}
    \lim_{\Delta x \to 0} \Delta y
    = \lim_{\Delta x \to 0} (A \Delta x + \smallo(\Delta x))
    = 0    
  \end{equation*}

  Получаем, что \(\Delta y \to 0\) при \(\Delta x \to 0\), что является
  определением непрерывности.
\end{proof}

\begin{remark}
  Обратное в общем случае \textbf{неверно}, например: \(f(x) = \abs{x}\) в точке
  \(x_0 = 0\) непрерывна, но не дифференцируема.
\end{remark}
