\subsection{%
  Теоремы о дифференцируемых функциях. Теорема Коши.%
}

\begin{theorem}[Коши]
  \begin{equation*}
    \begin{rcases}
      f(x), g(x) \isdiff{\segment{a}{b}} \\
      \forall x \in \segment{a}{b} \given g'(x) \neq 0
    \end{rcases}
    \implies
    \exists \xi \in \interval{a}{b} \given
    \frac{f(b) - f(a)}{g(b) - g(a)} = \frac{f'(\xi)}{g'(\xi)}
  \end{equation*}
\end{theorem}

\begin{proof}
  Рассмотрим вспомогательную функцию

  \begin{equation*}
    \phi(x) = \prh[\Big]{f(x) - f(a)} \prh[\Big]{g(b) - g(a)}
      - \prh[\Big]{f(b) - f(a)} \prh[\Big]{g(x) - g(b)}
  \end{equation*}

  которая удовлетворяет условиям теоремы Ролля.

  \begin{enumerate}
  \item
    \(\phi(x) \isdiff{\segment{a}{b}}\) как линейная комбинация функций
    дифференцируемых на \(\segment{a}{b}\).
  
  \item
    Она принимает равные значения на концах отрезка \(\segment{a}{b}\).

    \begin{enumerate}
    \item
      \(\phi(a) = \prh[\Big]{f(a) - f(b)} \prh[\Big]{g(a) - g(b)}\)
    
    \item
      \(
        \phi(b)
        = \prh[\Big]{f(b) - f(a)} \prh[\Big]{g(b) - g(a)}
        = \prh[\Big]{f(a) - f(b)} \prh[\Big]{g(a) - g(b)}
      \)
    \end{enumerate}
  \end{enumerate}

  Таким образом по теореме Ролля

  \begin{equation*}
    \begin{aligned}
      \exists \xi \in \interval{a}{b} \given \phi'(\xi) = 0
    \\
      \phi'(\xi) = f'(\xi) \prh[\Big]{g(b) - g(a)}
        - \prh[\Big]{f(b) - f(a)} g'(\xi) = 0
    \\
      f'(\xi) \prh[\Big]{g(b) - g(a)} = \prh[\Big]{f(b) - f(a)} g'(\xi)
    \\
      \frac{f'(\xi)}{g'(\xi)} = \frac{f(b) - f(a)}{g(b) - g(a)}
    \end{aligned}
  \end{equation*}
\end{proof}

\begin{remark}
  Если \(g'(x) = 0\), то \(g = const \implies g(b) - g(a) = 0\). При такой
  ситуации полученная формула не работает.
\end{remark}

\begin{remark}
  Данная формула применима и к параметрически заданной функции.
\end{remark}
