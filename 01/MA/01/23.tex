\subsection{%
  Теоремы о дифференцируемых функциях. Теорема Ферма.%
}

\begin{theorem}[Ферма] \label{thr:lem-F}
  \begin{equation*}
    \begin{rcases}
      f(x) \isdiffd{x_0} \\
      x_0 \text{ точка гладкого экстремума}
    \end{rcases}
    \implies
    f'(x) = 0
  \end{equation*}

  Теорема Ферма (чаще ее называют леммой Ферма) является необходимым (но
  недостаточным!) условием \textbf{гладкого} экстремума.
\end{theorem}

\begin{proof}
  Пусть для определённости \(f(x)\) достигает максимума в точке \(x_0\) (для
  минимума доказательство аналогично). Тогда \(f(x_0) \ge f(x)\) в некоторой
  окрестности точки \(x_0\). Получаем, что \(f(x_0) \ge f(x) \implies f(x) -
  f(x_0) \le 0\). Заметим, что \(f(x) - f(x_0)\) это приращение функции \(\Delta
  y \le 0\), а \(x - x_0\)~--- приращение аргумента \(\Delta x\). Рассмотрим два случая.

  \subsubheader{Случай I}{\(x < x_0 \implies \Delta x < 0\)}

  \begin{equation*}
    \begin{rcases}
      f'(x_0 - 0) = \lim_{x \to x_0 - 0} \frac{\Delta y}{\Delta x} \\
      \frac{\Delta y}{\Delta x} \ge 0
    \end{rcases}
    \implies[предельный переход]
    f'(x_0 - 0) \ge 0
  \end{equation*}

  \subsubheader{Случай II}{\(x > x_0 \implies \Delta x > 0\)}
  
  Аналогично случаю I получаем, что \(f'(x_0 + 0) \le 0\).
  
  Т.к. функция дифференцируема в точке \(x_0\), то предел приращений в этой
  точке существует, а раз он не меньше нуля, но в то же время он не больше нуля,
  значит он равен нулю. Таким образом производная функции в точке \(x_0\) равна
  нулю.
\end{proof}

\begin{remark}
  Обратное утверждение в общем случае неверно. Например, \(f(x) = x^3, x_0 =
  0\), \(f'(x) = 3x^2\) и \(f'(0) = 0\), но при этом в точке \(x_0 = 0\) нет
  экстремума.
\end{remark}

\begin{remark}
  Теорема Ферма описывает только гладкие экстремумы.
\end{remark}

Геометрический смысл теоремы Ферма заключается в том, что в точке экстремума
касательная параллельна оси \(Ox\) (\figref{01_23_01}).

\galleryone{01_23_01}{Геометрический смысл теоремы Ферма}
