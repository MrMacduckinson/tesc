\subsection{%
  Приложения: градиент, производная по направлению.%
}

\begin{definition}
  Пусть дана функция двух переменных \(z = f(x, y)\), тогда если \(z = const\),
  то на поверхности будет отсечена линия уровня.
\end{definition}

\begin{definition}
  Производной функции \(z = f(x, y)\) по направлению вектора \(\vec{s}\)
  называют \textbf{число}, равное

  \begin{equation*}
    \frac{\partial z}{\partial \vec{s}} =
    \prh{\partder{z}{x}, \partder{z}{y}} \cdot \vec{s}_0
  \end{equation*}

  где \(\vec{s}_0\) это нормированный вектор \(\vec{s}\).
\end{definition}

Пусть есть функция \(z = f(x, y)\). Т.к. функция дифференцируема, то её
приращение представимо в виде

\begin{equation*} \label{eq:dir-der-1} \tag{1}
  \Delta z = \partder{z}{x} \Delta x + \partder{z}{y} \Delta y + \smallo(\rho)
  \qquad
  (\rho = \sqrt{(\Delta x)^2 + (\Delta y)^2})
\end{equation*}

Рассмотрим вектор \(\vec{s} = (\Delta x, \Delta y)\), обозначим \(\Delta s =
\rho = \sqrt{(\Delta x)^2 + (\Delta y)^2}\). Вернемся к \eqref{eq:dir-der-1}
и разделим все на \(\Delta s\).

\begin{equation*} \label{def:dir-der-2} \tag{2}
  \frac{\Delta z}{\Delta s}
  = \partder{z}{x} \cdot \frac{\Delta x}{\Delta s}
    + \partder{z}{y} \cdot \frac{\Delta y}{\Delta s} + \frac{\smallo(s)}{s}
\end{equation*}

Заметим, что (\figref{01_39_01})

\begin{equation*} \label{eq:dir-der-3} \tag{3}
  \frac{\Delta x}{\Delta s} = \cos \alpha
  \qquad
  \frac{\Delta x}{\Delta s} = \cos \beta
\end{equation*}

как направляющие вектора \(\vec{s}\). Перейдем к пределу при \(\Delta s \to 0\),
тогда получим

\begin{equation*} \label{eq:dir-der-4} \tag{4}
  \begin{aligned}
    \lim_{\Delta s \to 0}\frac{\Delta z}{\Delta s} 
    = \lim_{\Delta s \to 0} \prh{\partder{z}{x} \cdot \cos \alpha}
      + \lim_{\Delta s \to 0} \prh{\partder{z}{y} \cdot \cos \beta}
      + \lim_{\Delta s \to 0} \frac{\smallo(s)}{s}
  \\
    \partder{z}{s}
    = \partder{z}{x} \cdot \cos \alpha + \partder{z}{y} \cdot \cos \beta
  \\
    \frac{\partial z}{\partial \vec{s}}
    = \prh{\partder{z}{x}, \partder{z}{y}} \cdot \prh{\cos \alpha, \cos \beta}
  \end{aligned}  
\end{equation*}

Учитывая, что вектор из направляющих это нормированный вектор, в итоге получаем
искомую формулу

\begin{equation*} \label{eq:dir-der-5} \tag{5}
  \frac{\partial z}{\partial \vec{s}}
  = \prh{\partder{z}{x}, \partder{z}{y}} \cdot \vec{s}_0
\end{equation*}

\galleryone{01_39_01}{Производная по направлению}

\begin{remark}
  Производная по направлению определена и для функций с б\'oльшим число
  аргументов. Например, производная по направлению \(\vec{s}\) для функции \(u =
  f(x, y, z)\) будет вычисляться по формуле

  \begin{equation*}
    \frac{\partial u}{\partial \vec{s}}
    = \prh{\partder{u}{x}, \partder{u}{y}, \partder{u}{z}} \cdot \vec{s}_0
  \end{equation*}
\end{remark}

\begin{remark}
  Производные по направлению осей будут являться соответствующими частными
  производными. Например, если \(\vec{s} = (1, 0)\), тогда

  \begin{equation*}
    \frac{\partial z}{\partial \vec{s}}
    = \prh{\partder{z}{x}, \partder{z}{y}} \cdot (1, 0)
    = \partder{z}{x}
  \end{equation*}
\end{remark}

\begin{definition}
  Градиентом функции \(z = f(x, y)\) в точке \(M_0\) называется \textbf{вектор},
  показывающий направление наискорейшего подъема функции.

  \begin{equation*}
    \vec{\grad_{M_0}} z(x, y)
    = \partder{z}{x} \cdot \orti + \partder{z}{y} \cdot \ortj
  \end{equation*}
\end{definition}

\begin{remark}
  Также градиент можно определить как вектор, составленный из частных
  производных.

  \begin{equation*}
    \gradient = \prh{\partder{z}{x}, \partder{z}{y}}
  \end{equation*}

  где \(\gradient\) (читается \quote{набла}) это другое обозначение градиента.
\end{remark}

\begin{remark}
  Градиент показывает направление наибольшего значения производной по
  направлению.
\end{remark}

\begin{theorem}
  Производная по направлению вектора \(\vec{s}\) это проекция градиента на этот
  вектор.
\end{theorem}

\begin{proof}
  По определению производной по направлению 

  \begin{equation*}
    \frac{\partial z}{\partial \vec{s}}
    = \prh{\partder{z}{x}, \partder{z}{y}} \cdot \vec{s}_0
    = \gradient \cdot \vec{s}_0
  \end{equation*}

  Раскроем скалярное произведение по геометрического определению. Получаем

  \begin{equation*}
    \frac{\partial z}{\partial \vec{s}} = \abs{\gradient} \cdot
      \under{\abs{\vec{s}_0}}{= 1} \cdot \cos \phi
  \end{equation*}

  где \(\phi\) это угол между \(\vec{s}\) и градиентом. Значит
  \(\display{\frac{\partial z}{\partial \vec{s}}}\) это проекция градиента на
  вектор \(\vec{s}\) по определению.
\end{proof}

\begin{remark}
  Если вектор перпендикулярен градиенту, то производная по направлению равна
  нулю и её проекция на градиент это точка.

  \begin{equation*}
    \vec{s} \perp \nabla \implies \frac{\partial z}{\partial \vec{s}} = 0
  \end{equation*}
\end{remark}

\galleryone{01_39_02}{Наклон градиента}

\begin{theorem}
  Градиент в точке \(M_0\) перпендикулярен линии уровня, проведенной через эту
  точку.
\end{theorem}

\begin{proof}
  Пусть есть функция \(z = f(x, y)\), точка \(M_0\) и линия уровня \(l \colon z
  = c\). Найдем наклон касательной \(l\) (обозначим его \(k\)). Рассмотрим \(z -
  c = 0\) как неявную функцию \(F(x, y) = 0\). По формуле, выведенной ранее,
  получаем, что

  \begin{equation*}
    k
    = \fullder{y}{x}
    = -\frac{\fullder{F}{x}}{\fullder{F}{y}}
    = -\frac{\partder{z}{x}}{\partder{z}{y}}
    = - \frac{z'_x}{z'_y}
  \end{equation*}

  Найдем наклон градиента (обозначим его \(k^*\)). Из определения градиента
  (\figref{01_39_02}) получаем

  \begin{equation*}
    k^*
    = \frac{\partder{z}{y}}{\partder{z}{x}}
    = \frac{z'_y}{z'_x}
  \end{equation*}

  Т.к. произведение наклона касательной \(l\) и наклона градиента равно минус
  единице, то они перпендикулярны.
\end{proof}
