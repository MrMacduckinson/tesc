\subsection{%
  Исследование функции: Монотонность. Экстремумы. Необходимое и достаточное
  условия экстремума.%
}

\begin{definition}
  Функция \(f(x)\) называется возрастающей, если

  \begin{equation*}
    \forall x, y \given x < y \implies f(x) < f(y)
    \bydef
    f(x) \increase \segment{a}{b}
  \end{equation*}
\end{definition}

\begin{definition}
  Функция \(f(x)\) называется неубывающей, если

  \begin{equation*}
    \forall x, y \given x < y \implies f(x) \le f(y)
    \bydef
    f(x) \nondescrease \segment{a}{b}
  \end{equation*}
\end{definition}

\begin{definition}
  Функция \(f(x)\) называется убывающей, если

  \begin{equation*}
    \forall x, y \given x < y \implies f(x) > f(y)
    \bydef
    f(x) \decrease \segment{a}{b}
  \end{equation*}
\end{definition}

\begin{definition}
  Функция \(f(x)\) называется невозрастающей, если

  \begin{equation*}
    \forall x, y \given x < y \implies f(x) \ge f(y)
    \bydef
    f(x) \nonincrease \segment{a}{b}
  \end{equation*}
\end{definition}

\begin{definition}
  Функция называется монотонной на отрезке, если она либо невозрастающая, либо
  неубывающая на этом отрезке.
\end{definition}

\begin{definition}
  Если функция возрастает (убывает) на отрезке, то она называется строго
  монотонной.
\end{definition}

\begin{theorem}[Достаточное условие монотонности на интервале]
  \begin{equation*}
    \begin{aligned}
      \begin{rcases}
        f(x) \isdiff{\interval{a}{b}} \\
        \forall x \in \interval{a}{b} \given f'(x) > 0
      \end{rcases}
      \implies
      f(x) \increase \interval{a}{b}
    \\
      \begin{rcases}
        f(x) \isdiff{\interval{a}{b}} \\
        \forall x \in \interval{a}{b} \given f'(x) < 0
      \end{rcases}
      \implies
      f(x) \decrease \interval{a}{b}
    \end{aligned}
  \end{equation*}
\end{theorem}

\begin{proof}
  Рассмотрим случай строго возрастающей функции (для строго убывающей функции
  доказательство аналогично). Возьмем две произвольные точки \(x_1, x_2 \in
  \interval{a}{b}\) так, чтобы \(x_1 < x_2\). По теореме Лагранжа получаем

  \begin{equation*}
    \begin{aligned}
      \frac{f(x_2) - f(x_1)}{x_2 - x_1} = f'(\xi)
      & \qquad
      (\xi \in \interval{a}{b})
    \\
      f(x_2) - f(x_1) = f'(\xi) \cdot (x_2 - x_1)
    \end{aligned}
  \end{equation*}

  Далее рассмотрим правую часть.

  \begin{equation*}
    \begin{aligned}
      \begin{rcases}
        \forall x \in \interval{a}{b} f'(x) > 0 \text{ (по условию)} \\
        \xi \in \interval{a}{b} \text{ (по т. Лагранжа)}
      \end{rcases}
      \implies
      f'(\xi) > 0
    \\
      \begin{rcases}
        f'(\xi) > 0 \\  
        x_2 > x_1 \text{ (по предположению)}
      \end{rcases}
      \implies
      f'(\xi) \cdot (x_2 - x_1) > 0
    \\
      \forall x_2 > x_1 \given f(x_2) - f(x_1) > 0
    \end{aligned}
  \end{equation*}

  Значит \(f(x)\) строго возрастает по определению.
\end{proof}

Лемма Ферма (\ref{thr:lem-F}) это необходимое условие только \textbf{гладкого}
экстремума.

\begin{theorem}[Необходимое условие экстремума]
  \begin{equation*}
    \begin{rcases}
      f(x) \isdiff{\near{x_0}{\delta}} \\
      f(x) \iscont{x_0} \\
      x_0 \text{ точка экстремума}
    \end{rcases}
    \implies
    \begin{cases}
      f'(x_0) \in \set{0, \pm \infty} \\
      \text{ либо } \\
      \nexists f'(x_0)
    \end{cases}
  \end{equation*}
\end{theorem}

\begin{theorem}[Достаточное условие экстремума]
  \begin{equation*}
    \begin{rcases}
      f(x) \isdiff{\near{x_0}{\delta}} \\
      f(x) \iscont{x_0} \\ 
      f'(x) \text{ меняет свой знак в } x_0  
    \end{rcases}
    \implies
    x_0 \text{ точка экстремума}
  \end{equation*}
\end{theorem}

\begin{proof}
  Рассмотрим случай смены знака производной с \quote{\(+\)} на \quote{\(-\)}
  (случай смены \quote{\(-\)} на \quote{\(+\)} доказывается аналогично).

  \subsubheader{Слева}{\(x < x_0, f'(x) > 0\)}

  Распишем производную по определению.

  \begin{equation*}
    f'(x) = \frac{f(x) - f(x_0)}{x - x_0} > 0
  \end{equation*}

  Т.к. \(x < x_0\), то \(x - x_0 < 0\), значит знаменатель меньше нуля, но т.к.
  вся дробь больше нуля, то числитель тоже должен быть меньше нуля, значит
  \(f(x) - f(x_0) < 0\), т.е. \(f(x) < f(x_0)\).

  \subsubheader{Справа}{\(x > x_0, f'(x) < 0\)}

  Распишем производную по определению.

  \begin{equation*}
    f'(x) = \frac{f(x) - f(x_0)}{x - x_0} <  0
  \end{equation*}

  Т.к. \(x > x_0\), то \(x - x_0 > 0\), значит знаменатель больше нуля, но т.к.
  вся дробь меньше нуля, то числитель тоже должен быть меньше нуля, значит
  \(f(x) - f(x_0) < 0\), т.е. \(f(x) < f(x_0)\).

  Таким образом \(f(x) < f(x_0)\), значит \(x_0\) это точка максимума, т.е. в
  ней достигается экстремум.
\end{proof}

\begin{remark}
  Для того, чтобы найти точки экстремума сначала с помощью необходимого условия
  ищем точки, подозрительные на экстремум (точки, в которых производная равна
  нулю/бесконечности или не существует), а потом проверяем их с помощью
  достаточного условия.
\end{remark}

\begin{theorem}
  В некоторых случая все производные в точке \(x_0\), кроме \((n + 1)\)--ой
  обнуляются, тогда удобно провести исследование функции на экстремум с помощью
  формулы Тейлора (при условии, что \((n + 1)\)--ая производная непрерывна).

  \begin{enumerate}
  \item
    \(n + 1\)~--- нечетное число \(\implies\) нет экстремума.

  \item
    \(n + 1\)~--- четное число \(\implies\) есть экстремум, причем

    \begin{enumerate}
    \item
      \(f^{(n + 1)}(x_0) > 0 \implies x_0\) это точка минимума.
      
    \item 
      \(f^{(n + 1)}(x_0) < 0 \implies x_0\) это точка максимума.
    \end{enumerate}
  \end{enumerate}
\end{theorem}

\begin{proof}
  Рассмотрим формулу Тейлора с остатком в форме Лагранжа. Т.к. первые \(n\)
  производных равны нулю, то формула упрощается

  \begin{equation*}
    \begin{aligned}
      f(x)
        = \sum_{k = 0}^n \frac{f^{(k)}(x_0)}{k!} \cdot (x - x_0)^k +
        \frac{f^{(n + 1)}(\xi)}{(n + 1)!} \cdot (x - x_0)^{n + 1}
        & \qquad
        (\xi \in \interval{x_0}{x})
      \\
        f(x) - f(x_0)
        = \frac{f^{(n + 1)}(\xi)}{(n + 1)!} \cdot (x - x_0)^{n + 1}
    \end{aligned}
  \end{equation*}

  \(f^{(n + 1)}\) непрерывна и \(\xi \to x_0\) при \(x \to x_0\), значит по
  теореме о стабилизации знака \(f^{(n + 1)}(\xi)\) имеет такой же знак, как и
  \(f^{(n + 1)}(x_0)\). Рассмотрим два случая.

  \subsubheader{Случай I}{\(n + 1\) нечетное}

  Тогда \((x - x_0)^{n + 1}\) меняет свой знак при переходе через \(x_0\). При
  этом \(f^{(n + 1)}(\xi)\) имеет постоянный знак (неважно какой). Значит \(f(x)
  - f(x_0)\) тоже меняет свой знак при переходе через \(x_0\). Таким образом в
  \(x_0\) нет экстремума.

  \subsubheader{Случай II}{\(n + 1\) четное}

  Тогда \((x - x_0)^{n + 1} > 0\), а это значит что знак \(f(x) - f(x_0)\)
  совпадает со знаком \(f^{(n + 1)}(\xi)\), который (как уже было показано)
  совпадает со знаком \(f^{(n + 1)}(x_0)\). Рассмотрим два варианта.

  \subsubheader{Случай II.a}{\(f^{(n + 1)}(x_0) > 0\)}

  Тогда \(f(x) - f(x_0) > 0\), т.е. \(f(x) > f(x_0)\). Таким образом \(x_0\)
  точка минимума.

  \subsubheader{Случай II.b}{\(f^{(n + 1)}(x_0) < 0\)}

  Тогда \(f(x) - f(x_0) < 0\), т.е. \(f(x) < f(x_0)\). Таким образом \(x_0\)
  точка максимума.
\end{proof}
