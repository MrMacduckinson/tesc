%%
%% Большая часть взята отсюда:
%% https://tex.stackexchange.com/questions/25928/how-to-draw-tangent-line-of-an-arbitrary-point-on-a-path-in-tikz
%%

\begin{tikzpicture}[
  tangent/.style = {
    decoration = {
      markings,
      mark = at position #1 with {
        \coordinate (tangent point-\pgfkeysvalueof{/pgf/decoration/mark info/sequence number}) at (0pt, 0pt);
        \coordinate (tangent unit vector-\pgfkeysvalueof{/pgf/decoration/mark info/sequence number}) at (1, 0pt);
        \coordinate (tangent orthogonal unit vector-\pgfkeysvalueof{/pgf/decoration/mark info/sequence number}) at (0pt, 1);
      }
    },
    postaction = decorate
  },
  use tangent/.style = {
    shift = (tangent point-#1),
    x = (tangent unit vector-#1),
    y = (tangent orthogonal unit vector-#1)
  }
]
  \draw[draw = olive!60, fill = olive!20, very thick]
    (-2.5, -1.5) rectangle (6, 6);

  \draw[->, thick] (0, -0.5) -- (0, 5)
    node[right] {\(y\)};
  \draw[->, thick] (-0.5, 0) -- (5, 0)
    node[below] {\(x\)};

  \draw[
    thick,
    tangent = 0.2,
    tangent = 0.4,
    tangent = 0.6,
    tangent = 0.8,
  ] (1, 1) .. controls (1.5, 3) and (3, 3) .. (4, 1);

  \foreach \i in {1, ..., 4} {
    \draw[fill = black, use tangent = \i] (0, 0) circle (1pt);
    \draw[color = blue!80, use tangent = \i] (-1, 0) -- (1, 0);
  }
  \draw[dashed, thick] (2.25, 0) -- (2.25, 2.5)
    node[pos = 0, below] {\(x_0\)};

\end{tikzpicture}