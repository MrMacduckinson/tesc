\question{Разбиения чисел.}

\begin{definition}
  Разбиение числа \(n\) на \(k\) частей это решение уравнения
  \(b_{1} + \dotsc + b_{k} = n\) при условии
  \(b_{1} \ge \dots \ge b_{k} \ge 0\).
\end{definition}

Для подсчета числа разбиений существует специальная функция \(p(n, k)\), которая
определена как

\begin{align*}
  p(n, k) = p(n - 1, k - 1) + p(n - k, k)
\end{align*}

Это соотношение построено исходя из следующих соображений: пусть необходимо
разбить число \(n\) на \(k\) слагаемых. Есть две стратегии действий:

\begin{itemize}
  \item Можно разбить число \(n - 1\) на \(k - 1\) слагаемое, а \(k\)-тым
  слагаемым будет единица.

  \item Можно разбить число \(n - k\) на \(k\) слагаемых, после чего добавить ко
  всем слагаемым по единице.
\end{itemize}

Две эти стратегии определяют два слагаемых в формуле выше.

\begin{remark}
  О крайних случаях

  \begin{itemize}
    \item \(p(0, 0) = 1\)~--- полагаем так, чтобы все было хорошо.
    
    \item \(p(n, k) = 0\), если \(n \le 0\), т.к. невозможно представить
    неположительное число в виде суммы положительных слагаемых.

    \item \(p(n, k) = 0\), если \(k \le 0\), т.к число слагаемых должно быть
    положительным.
  \end{itemize}
\end{remark}
