\question{Мультимножества.}

\begin{definition}
  Мультимножество \(\Sigma^{*}\) это упорядоченная пара вида
  \(\Pair{\Sigma, r}\), где \(\Sigma\) это множество, а \(r\) это функция,
  показывающая число повторений элемента \(r \colon \Sigma \to \NN\).
\end{definition}

Количество перестановок мультимножества можно найти по формуле
(мультиномиальная теорема):

\begin{align*}
  \abs{P(\Sigma^{*}, n)} = \frac{n!}{r_{1}! \cdot \dotsc \cdot r_{s}!}
\end{align*}

где \(n = \abs{\Sigma^{*}}\) и \(s = \abs{\Sigma}\).

\begin{remark}
  О сочетаниях бесконечных мультимножеств

  Пусть дано мультимножество \(\Sigma^{*} = \Pair{\Sigma, r}\), в котором
  \(\forall r_{i} \ge k\), \(\abs{\Sigma} = s\) и требуется найти количество
  сочетаний.

  Для решения этой задачи можно воспользоваться методом Stars\&Bars. Пусть у нас
  есть \(s - 1\) перегородка и \(k\) элементов из \(\Sigma^{*}\), тогда
  количество \(k\)-сочетаний будет равно

  \begin{align*}
    \binom{k + s - 1}{s - 1} = \binom{k + s - 1}{k}
  \end{align*}

  т.к. нам нужно из \(k + s - 1\) позиций выбрать \(s - 1\) позицию для
  перегородок, а остальные позиции займут элементы.
\end{remark}
