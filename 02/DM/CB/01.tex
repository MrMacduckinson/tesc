\question{Упорядоченные и неупорядоченные размещения.}

\begin{definition}
  Упорядоченным размещением \(n\) элементов из \(\Sigma\) называется функция
  \(s \colon [n] \to \Sigma\), где \(\Sigma\) это алфавит,
  \([n] = \{ 1, \dotsc, n \}\).
\end{definition}

\begin{itemize}
  \item \([n]\) это домен функции
  \item \(\Sigma\) это кодомен функции
  \item \(s(i)\) это изображение (\textit{image}) для \(i\)
  \item \(\ran s = \{ x \in \Sigma \mid \exists i \in [n] \colon s(i) = x \}\).
\end{itemize}

\begin{remark}
  Также упорядоченное размещение может быть представлено в виде
  кортежа \(s = (s_{1}, \dotsc, s_{n})\),
  строки \(s = s(1)s(2) \dotsc s_{n}\) или
  последовательности \(s(i) = 3i + 2\).
\end{remark}

\begin{definition}
  Перестановка это биективная функция \(\pi \colon [n] \to \Sigma\).
\end{definition}

\(S_{n}\) это множество всех перестановок для \(\Sigma = [n]\).
\(\abs{S_{n}} = n!\)

\begin{definition}
  \(k\)-перестановка это упорядоченное размещение \(k\) различных элементов из
  \(\Sigma\), т.е. это инъективная функция \(\pi_{k} \colon [k] \to \Sigma\).
\end{definition}

\(P(n, k)\) это множество всех \(k\)-перестановок для \(\Sigma = [n]\).
\(\displaystyle \abs{P(n, k) = \frac{n!}{(n - k)!}}\)

\begin{remark}
  Существуют также циклические \(k\)-перестановки. \(\pi_{1}\) и \(\pi_{2}\)
  называются циклическими эквивалентами, если существует сдвиг \(s \in [k]\)
  такой, что \(
    \forall i, j \in [k] \colon i + s = j (\mod k)
    \implies \pi_{1}(i) = \pi_{2}(j)
  \)

  Количество циклических \(k\)-перестановок в \(k\) раз меньше, чем количество
  обычных \(k\)-перестановок:
  \(\displaystyle P_{c}(n, k) = \frac{n!}{k \cdot (n - k)!}\).
\end{remark}

\begin{definition}
  Мультимножество \(\Sigma^{*}\) это упорядоченная пара вида
  \(\Pair{\Sigma, r}\), где \(\Sigma\) это множество, а \(r\) это функция,
  показывающая число повторений элемента \(r \colon \Sigma \to \NN\).
\end{definition}

Количество перестановок мультимножества можно найти по формуле
(мультиномиальная теорема):

\begin{align*}
  \abs{P(\Sigma^{*}, n)} = \frac{n!}{r_{1}! \cdot \dotsc \cdot r_{s}!}
\end{align*}

где \(n = \abs{\Sigma^{*}}\) и \(s = \abs{\Sigma}\).


\begin{definition}
  Неупорядоченное размещение \(k\) элементов множества \(\Sigma\) это
  мультимножество \(S\) размера \(k\).
\end{definition}

\begin{definition}
  \(k\)-сочетание (\(k\)-подмножество) это неупорядоченное размещение \(k\)
  различных элементов из \(\Sigma\).
\end{definition}

Множество всех \(k\)-подмножеств обозначается \(\binom{\Sigma}{k}\), если
\(\abs{\Sigma} = n\), то получаем \(\binom{n}{k} = C_{n}^{k}\).

Для подсчета количества \(k\)-подмножеств воспользуемся тем, что каждое
\(k\)-подмножество имеет \(k!\) перестановок:

\begin{align*}
  \abs{P(n, k)} = \binom{n}{k} \cdot k!
  \implies \binom{n}{k} = \frac{n!}{k! \cdot (n - k)!}
\end{align*}

\begin{remark}
  О сочетаниях бесконечных мультимножеств

  Пусть дано мультимножество \(\Sigma^{*} = \Pair{\Sigma, r}\), в котором
  \(\forall r_{i} \ge k\), \(\abs{\Sigma} = s\) и требуется найти количество
  сочетаний.

  Для решения этой задачи можно воспользоваться методом Stars&Bars. Пусть у нас
  есть \(s - 1\) перегородка и \(k\) из \(\Sigma^{*}\), тогда количество
  \(k\)-сочетаний будет равно

  \begin{align*}
    \binom{k + s - 1}{s - 1} = \binom{k + s - 1}{k}
  \end{align*}

  т.к. нам нужно из \(k + s - 1\) позиций выбрать \(s - 1\) позицию для
  перегородок, а остальные позиции займут элементы.
\end{remark}