\question{Упорядоченные размещения. Перестановки и \(k\)-перестановки.}

\begin{definition}
  Упорядоченным размещением \(n\) элементов из \(\Sigma\) называется функция
  \(s \colon [n] \to \Sigma\), где \(\Sigma\) это алфавит,
  \([n] = \{ 1, \dotsc, n \}\).
\end{definition}

\begin{itemize}
  \item \([n]\) это домен функции
  \item \(\Sigma\) это кодомен функции
  \item \(s(i)\) это изображение (\textit{image}) для \(i\)
  \item \(\ran s = \{ x \in \Sigma \mid \exists i \in [n] \colon s(i) = x \}\).
\end{itemize}

\begin{remark}
  Также упорядоченное размещение может быть представлено в виде
  кортежа \(s = (s_{1}, \dotsc, s_{n})\),
  строки \(s = s(1)s(2) \dotsc s_{n}\) или
  последовательности \(s(i) = 3i + 2\).
\end{remark}

\begin{definition}
  Перестановка это биективная функция \(\pi \colon [n] \to \Sigma\).
\end{definition}

\(S_{n}\) это множество всех перестановок для \(\Sigma = [n]\).
\(\abs{S_{n}} = n!\)

\begin{definition}
  \(k\)-перестановка это упорядоченное размещение \(k\) различных элементов из
  \(\Sigma\), т.е. это инъективная функция \(\pi_{k} \colon [k] \to \Sigma\).
\end{definition}

\(P(n, k)\) это множество всех \(k\)-перестановок для \(\Sigma = [n]\).
\(\displaystyle \abs{P(n, k)} = \frac{n!}{(n - k)!}\)

\begin{remark}
  Существуют также циклические \(k\)-перестановки. \(\pi_{1}\) и \(\pi_{2}\)
  называются циклическими эквивалентами, если существует сдвиг \(s \in [k]\)
  такой, что \(
    \forall i, j \in [k] \colon i + s = j (\mod k)
    \implies \pi_{1}(i) = \pi_{2}(j)
  \)

  Количество циклических \(k\)-перестановок в \(k\) раз меньше, чем количество
  обычных \(k\)-перестановок:

  \(\displaystyle P_{c}(n, k) = \frac{n!}{k \cdot (n - k)!}\).
\end{remark}
