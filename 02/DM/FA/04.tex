\question{Недетерминированный конечный автомат.}

Недетерминированный конечный автомат (НКА) это кортеж вида
\(\Tuple{\Sigma, Q, q_{0}, F, \delta}\), где
\begin{itemize}
  \item \(\Sigma\) это алфавит
  \item \(Q\) это множество состояний
  \item \(q_{0}\) это начальное состояние (\(q_{0} \in Q\))
  \item \(F\) это множество принимающих состояний (\(F \subseteq Q\))
  \item \(\delta\) это функция перехода
    \(\delta \colon Q \times \Sigma \to 2^{Q}\)
\end{itemize}

Таким образом отличие НКА от ДКА заключается в функции перехода: если в ДКА мы
переходили от одного состояния к другому по определенному символу, то в НКА по
одному символу можно перейти сразу в несколько новых состояний.

\underline{Вычисление}:

Как и для ДКА, для НКА можно определить снимки:

\begin{align*}
  \Pair{q_{1}, s_{1}} \vdash_{NFA} \Pair{q_{2}, s_{2}}
  \iff
  \begin{cases}
    s_{1} = \alpha s_{2} \\
    q_{2} \in \delta(q_{1}, \alpha)
  \end{cases}
\end{align*}

Тогда язык, принимаемый НКА будет иметь вид

\begin{align*}
  \Lang(\mathcal{A})
    = \{ s \mid \Pair{a_{0}, s} \vdash^{*}_{NFA} \Pair{f, \epsilon}, f \in F \}
\end{align*}
