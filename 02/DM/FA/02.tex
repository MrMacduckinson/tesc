\question{Регулярные языки. Регулярные выражения.}

Множество регулярных языков задается рекуррентно. Определим нулевое поколение
регулярных языков как

\begin{align*}
  Reg_{0} = \{ \varnothing, \epsilon, \{ c \} \mid c \in \Sigma \}
\end{align*}

Далее определим переход к следующему поколению:

\begin{align*}
  Reg_{i + 1} = Reg_{i} \cup \{
    \Lang_{1} \cup \Lang_{2},
    \Lang_{1} \cdot \Lang_{2},
    \Lang_{1}^{*}
    \mid \Lang_{1}, \Lang_{2} \in Reg_{i}
  \}
\end{align*}

Тогда множество регулярных языков задается как

\begin{align*}
  REG = \bigcup\limits_{k = 0}^{\infty} Reg_{k} = Reg_{\infty}
\end{align*}

\begin{lemma}
  Множество регулярных языков замкнуто относительно объединения,
  конкатенации и звезды Клини.
\end{lemma}
\begin{proof}
  Пусть \(\Lang_{1} \in Reg_{i}\), \(\Lang_{2} \in Reg_{j}\), тогда

  \begin{align*}
    \Lang_{1} \cup \Lang_{1},
    \Lang_{1} \cdot \Lang_{2},
    \Lang_{1}^{*}
    \in Reg_{max(i, j) + 1} \subseteq REG
  \end{align*}
\end{proof}

\begin{definition}
  Регулярные выражения это один из способов задания регулярного языка. Они
  определяются рекурсивно.  
\end{definition}

Пусть \(\alpha\) и \(\beta\) это регулярные выражения соответствующие
регулярным языкам \(\mathcal{A}\) и \(\mathcal{B}\).

\begin{table}[H]
  \centering
  
  \renewcommand{\arraystretch}{1.5}
  \begin{tabular}{c|c|c}
    Регулярный язык & Регулярное выражение & Примечание \\
    \hline
    \(\varnothing\) & \(\varnothing\) & Пустое множество \\
    \(\epsilon\) & \(\epsilon\) & Пустое слово \\
    \(c\) & \(c\) & Один символ из алфавита \\
    \(\mathcal{A} \cup \mathcal{B}\) & \(\alpha | \beta\) & Объединение \\
    \(\mathcal{A} \cdot \mathcal{B}\) & \(\alpha \beta\) & Конкатенация \\
    \(\mathcal{A}^{*}\) & \(\alpha^{*}\) & Звезда Клини
  \end{tabular}
\end{table}

Также для удобства в синтаксис регулярных выражений добавлены:

\begin{itemize}
  \item Точка соответствует одного любому символу из алфавита.  
  \item Скобки для определения порядка операций.
  \item Группировка \([abc]\), которая означает 'один любой символ из
  перечисленных в квадратных скобках'.
  \item Дополнение \([\verb|^|abc]\), которая означает 'один любой символ кроме
  перечисленных в квадратных скобках'.
\end{itemize}
