\question{Детерминированный конечный автомат.}

Детерминированный конечный автомат (ДКА) это кортеж вида
\(\Tuple{\Sigma, Q, q_{0}, F, \delta}\), где
\begin{itemize}
  \item \(\Sigma\) это алфавит
  \item \(Q\) это множество состояний
  \item \(q_{0}\) это начальное состояние (\(q_{0} \in Q\))
  \item \(F\) это множество принимающих состояний (\(F \subseteq Q\))
  \item \(\delta\) это функция перехода \(\delta \colon Q \times \Sigma \to Q\)
\end{itemize}

Мы рассматриваем только автоматы-детекторы, т.е. им на вход подается слово,
после обработки которого автомат оказывается в одном из состояний. Если это
принимающее состояние, то говорят, что слово принимается автоматом (автомат
допускает слово), в противном случае~--- слово отвергается автоматом.

\underline{Вычисление}:

\begin{definition}
  Снимок (\textit{snapshot}) это упорядоченная пара вида \(\Pair{q, s}\), где
  \(q \in Q\) это текущее слово, а \(s\) это еще непросмотренная часть входной
  строки.
\end{definition}

Множество всех снимков обозначается \(SNAP\). На этом множество можно ввести
бинарное отношение \(\vdash\), которое будет показывать, можно ли перейти от
одного снимка к другому:

\begin{align*}
  \Pair{q_{1}, s_{1}} \vdash \Pair{q_{2}, s_{2}}
  \iff
  \begin{cases}
    s_{1} = \alpha s_{2} \\
    \delta(q_{1}, \alpha) = q_{2}
  \end{cases}
\end{align*}

\begin{definition}
  Автоматный язык \(\Lang(\mathcal{A})\) это язык, состоящий из слов,
  принимаемых автоматом:

  \begin{align*}
    \Lang(\mathcal{A})
    = \{ s \mid \Pair{a_{0}, s} \vdash^{*} \Pair{f, \epsilon}, f \in F \}
  \end{align*}
\end{definition}

\begin{definition}
  Множество автоматных языков \(AUT\) это множество языков, для которых
  существует автомат их принимающий.

  \begin{align*}
    AUT = \{
      \mathcal{X} \mid \exists \mathcal{A} \colon
      \Lang(\mathcal{A}) = \mathcal{X}
    \}
  \end{align*}
\end{definition}
