\question{Гамильтоновы графы.}

\begin{definition}
  Гамильтонов путь это путь, который проходит все вершины в графе ровно по
  одному разу.
\end{definition}

\begin{definition}
  Гамильтонов цикл это гамильтонов путь, у которого начальная вершина совпадает
  с конечной.

  Несмотря на то, что каждая вершина должна быть пройдена ровно один раз, для
  стартовой вершины делается исключение.
\end{definition}

\begin{definition}
  Граф называется гамильтоновым, если в нем есть гамильтонов цикл, и
  полугамильтоновым, если в нем есть только гамильтонов путь.
\end{definition}

\begin{theorem}\label{Ore}
  Теорема Оре.

  Если \(n \ge 3\) и \(\deg u + \deg v \ge n\) для любых двух \textbf{несмежных}
  вершин неориентированного графа \(G\), то этот граф гамильтонов.
\end{theorem}
\begin{proof}
  От противного: пусть дан граф, удовлетворяющий условиям теоремы, но не
  являющийся гамильтоновым. Будем добавлять к нем ребра до тех пор, пока он
  будет будет оставаться не гамильтоновым. Т.к. мы только добавляем ребра, то
  условия теоремы не нарушатся.

  Пусть в полученном графе есть две несмежные вершины \(u\) и \(v\) такие, что
  добавление ребра \((u, v)\) приводит к появлению гамильтонова цикла. Это
  значит, что путь \(u \leadsto v\) гамильтонов. Обозначим вершины на этом пути
  \(t_{j}\). Рассмотрим два множества:

  \begin{itemize}
    \item \(S = \{ i \mid \exists (u, t_{i + 1} \in E) \}\)
    \item \(T = \{ i \mid \exists (t_{i}, v) \in E) \}\)
  \end{itemize}

  Тогда \(\abs{S} + \abs{T} = \deg u + \deg v \ge n\), при этом
  \(\abs{S \cup T} < n\). Из этого следует, что
  \(\abs{S \cap T} = \abs{S} + \abs{T} - \abs{S \cup T} > 0\), т.е. найдутся две
  смежные вершины \(t_{1}\) и \(t_{2}\) такие, что будут существовать ребра
  \((u, t_{2})\) и \((t_{1}, v)\).

  Построим гамильтонов цикл: \(u \leadsto t_{1} \to v \leadsto t_{2} \to v\).
  Получаем противоречие: предполагалось, что граф не гамильтонов, однако мы
  смогли построить гамильтонов цикл.
\end{proof}

\begin{theorem}
  Теорема Дирака.

  Если \(\abs{V} \ge 3\) и \(\minDegree{G} \ge \frac{n}{2}\), то
  неориентированный граф \(G\)~--- гамильтонов.
\end{theorem}
\begin{proof}
  Возьмем две произвольные несмежные вершины \(u\) и \(v\), тогда
  \(\deg u + \deg v \ge \frac{n}{2} + \frac{n}{2} = n\). Значит по т. Оре
  (\ref{Ore}) граф \(G\) гамильтонов.
\end{proof}

\begin{remark}
  Эти две теоремы являются достаточными, но не обходимыми условиями
  гамильтоновости графа, т.е. если они не выполняются, то это не значит, что
  граф не гамильтонов.
\end{remark}
