\question{Морфизмы графов.}~-

\begin{definition}
  Два графа называются изоморфными,если между из вершинами существует биекция,
  причем два вершины в первом графе смежны тогда и только тогда, когда смежны
  соответствующие им (по биекции) вершины второго графа.
\end{definition}

\begin{definition}
  Автоморфизм графа это отображение множества вершин графа в себя, сохраняющее
  смежность.
\end{definition}

\begin{remark}
  Граф, для которого единственный возможный автоморфизм это тождественное
  отображение называется асимметрическим.
\end{remark}

\begin{definition}
  Гомоморфизм графа это отображение одного графа в другой, которое сохраняет
  смежность вершин.
\end{definition}

\begin{definition}
  Подразделением графа \(G\) называется граф \(G'\), полученный делением ребер
  графа \(G\) новыми вершинами.
\end{definition}

\begin{definition}
  Два графа \(G_{1}\) и \(G_{2}\) называются гомеоморфными, если существует
  изоморфизм некоторого подразделения графа \(G_{1}\) и некоторого подразделения
  графа \(G_{2}\)
\end{definition}
