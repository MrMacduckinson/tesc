\question{Связность.}

\begin{definition}
  Две вершины неориентированного графа \(u\) и \(v\) называются связными, если
  существует путь из \(u\) в \(v\).
\end{definition}

\begin{remark}
  Связность это отношение эквивалентности, а классы эквивалентности этого
  отношения называются компонентами связности графа.
\end{remark}

\begin{definition}
  Неориентированный граф называется связным, если он состоит из одной компоненты
  связности.
\end{definition}

\begin{definition}
  Две вершины \(u\) и \(v\) ориентированного графа называются слабо связными,
  если существует путь из \(u\) в \(v\) без учета ориентации ребер.
\end{definition}

\begin{remark}
  Слабая связность это отношение эквивалентности, а классы эквивалентности этого
  отношения называются компонентами слабой связности.
\end{remark}

\begin{definition}
  Ориентированный граф называется слабо связным, если он состоит из одной
  компоненты слабой связности.
\end{definition}

\begin{definition}
  Две вершины \(u\) и \(v\) ориентированного графа называются сильно связными,
  если существует путь из \(u\) в \(v\) и путь из \(v\) в \(u\).
\end{definition}

\begin{remark}
  Сильная связность это отношение эквивалентности, а классы эквивалентности
  этого отношения называются компонентами сильной связности.
\end{remark}

\begin{definition}
  Ориентированный граф называется сильно связным, если он состоит из одной
  компоненты сильной связности.
\end{definition}

\begin{definition}
  Конденсацией ориентированного графа \(G\) называется граф \(G'\), вершины
  которого соответствуют компонентам сильной связности графа \(G\), а ребра
  между вершинами существуют лишь в том случае, если между соответствующими
  компонентами сильной связности есть ребро.
\end{definition}

\begin{definition}
  Вершинная связность \(\vertexConnectivity{G}\) это минимальное число вершин,
  которое необходимо удалить для того, чтобы граф стал несвязным или
  тривиальным.
\end{definition}

\begin{definition}
  Вершинная двусвязность это бинарное отношение на ребрах. Два ребра называются
  двусвязными, если существует два вершиннно-независимых пути между концами этих
  ребер.
\end{definition}

\begin{definition}
  Вершинная двусвязность это отношение эквивалентности, а классы эквивалентности
  этого отношения называются вершинно-двусвязными компонентами или
  \textit{блоками}.
\end{definition}

\begin{definition}
  Граф называется \(k\)-вершинно связным, если после удаления менее чем \(k\)
  вершин он остается связным, т.е. \(\vertexConnectivity{G} \ge k\).
\end{definition}


\begin{definition}
  Реберная связность \(\edgeConnectivity{G}\) это минимальное число ребер,
  которое необходимо удалить для того, чтобы граф стал несвязным или
  тривиальным.
\end{definition}

\begin{definition}
  Реберная двусвязность это бинарное отношение на вершинах. Две вершины
  называются реберно двусвязными, если между ними существует два 
  реберно-независимых пути.
\end{definition}

\begin{definition}
  Реберная двусвязность это отношение эквивалентности, а классы эквивалентности
  этого отношениями называются реберно-двусвязными компонентами.
\end{definition}

\begin{definition}
  Граф называется \(k\)-реберно связным, если после удаления менее чем \(k\)
  ребер он остается связным, т.е. \(\edgeConnectivity{G} \ge k\).
\end{definition}


\begin{remark}
  Крайние случаи:
  \begin{itemize}
    \item \(K_{1}\) не является \(1\)-вершинно связным, но считается связным.
    
    \item \(K_{1}\) не является \(2\)-реберно связным, но считается
      реберно-двусвязным. Т.е. одна вершина может считаться компонентой реберной
      двусвязности.

    \item \(K_{2}\) не является \(2\)-вершинно связным, но считается
      вершинно-двусвязным, поэтому \(K_{2}\) может быть блоком.
  \end{itemize}
\end{remark}

\begin{definition}
  Вершина называется точкой сочленения (шарниром), если её удаление увеличивает
  число компонент связности графа.
\end{definition}

\begin{definition}
  Ребро называется мостом, если его удаление увеличивает число компонент
  связности графа.
\end{definition}

\begin{remark}
  Таким образом точки сочленения являются 'границами' между блоками (но при
  этом в блоки они входят), а мосты~--- между компонентами
  реберной-двусвязности (но в сами компоненты они не входят).
\end{remark}

\begin{definition}
  Дерево блоков-точек сочленения это двудольный граф, в котором в одной доле
  находятся вершины соответствующие блокам, а в другой~--- точки сочленения.

  Если точка сочленения принадлежит блоку, то между ними будет ребро, в
  противном случае~--- нет.
\end{definition}

