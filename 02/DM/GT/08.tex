\question{Теорема Татта.}

\begin{definition}
$1$-фактор графа $G$ это $1$-регулярный остовный подграф графа $G$ (по сути Perfect Matching)
\end{definition}

\begin{theorem} 
    Теорема Татта
    
    Граф $G$ содержит $1$-фактор тогда и только тогда, когда $k_0(G - S) \leq |S|$ 
    для любого $S \subseteq V$, где $k_0(G)$ это число нечетных (т.е. содержащих нечетное количество вершин) компонент графа $G$
\end{theorem}

\begin{proof}
    Для начала предположим, что граф $G$ содержит $1$-фактор $F$, пусть $S \subseteq V$. Если граф $G - S$ не содержит нечетных компонент, то тогда $k_0(G - S) = 0$ и очевидно, что $k_0(G - S) \leq |S|$

Предположим, что $k_0(G - S) = k \geq 1$, обозначим $G_1, \dots, G_k$ нечетные компоненты графа $G - S$. Т.к. граф $G$ содержит $1$-фактор, то некоторе ребро из $F$ должно быть инцидентно вершине из $G_i$ и вершине из $S$. Таким образом $k_0(G - S) \leq |S|$

> Достаточность:

Пусть $k_0(G - S) \leq |S|$ для любого $S \subseteq V$. В частности для $S = \varnothing$ получаем, что $k_0(G - S) = k_0(G) = 0$. Это значит, что каждая компонента графа $G$ четная, а значит и сам граф содержит четное количество вершин. Далее по индукции покажем, что любой граф $G$ четного размера, обладающий этим свойством, содержит $1$-фактор

База: граф $K_2$ очевидно имеет $1$-фактор

Переход: пусть все четные графы $H$ размера меньше, чем $n$, и удовлетворяющие свойству $k_0(H - S) \leq |S|$ для любого $S \subseteq V_H$, имеют
$1$-фактор. Рассмотрим граф $G$, который имеет четный размер $n$ и также удовлетворяет этому свойству

Заметим, что любая нетривиальная компонента графа $G$ содержит вершину, которая не является точкой сочленения, 
поэтому существуют такие $R_j \subset V$ для которых $k_0(G - R_j) = |R_j|$. Пусть $S$ это максимальное по мощности множество $R_j$. 
Обозначим $G_1, \dots, G_k$ нечетные компоненты $G - S$, тогда $|S| = k \geq 1$

Покажем, что $G - S$ содержит только нечетные компоненты, которые мы обозначили $G_1, \dots, G_k$. 
Пусть есть четная компонента $G_p$, содержащая вершину $u$, которая не является точкой сочленения. 
Тогда рассмотрим множество $S_p = S \cup \{u\}$, мощность которого будет равна $k + 1$. Получаем, что $k_0(G - S_p) = |S_p| = k + 1$, а это невозможно
\end{proof}
