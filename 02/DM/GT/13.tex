\question{Алгоритм Дейкстры.}

\underline{Цель}: ищем кратчайшие пути от одной вершины до всех остальных
в графе с неотрицательными весами.

\underline{Оценка по времени} зависит от реализации приоритетной очереди:
\begin{itemize}
  \item Наивная \(O(V^2 + E)\)
  \item На бинарной куче \(O(V + E \log V)\)
  \item На Фибоначчиевой куче \(O(V \log V + E)\)
\end{itemize}

\underline{Алгоритм}:
\begin{enumerate}
  \item Будем поддерживать множество вершин \(A\), расстояние до которых
  минимально.

  \item Изначально
  \(A = \varnothing\),
  \(\forall v \neq s \in V \colon d_{v} = +\infty\),
  \(d_{s} = 0\),
  где \(d_{x}\) это расстояние от стартовой вершины до вершины \(x\).

  \item Из непосещенных вершин выберем вершину \(u\), у для которой \(d_{u}\)
  минимально.
  
  \item Добавим \(u\) в \(A\), причем \(d_{u}\) будет являться кратчайшим
  расстоянием до вершины \(u\).

  \item Далее будем обновлять расстояние до непосещенных вершин, смежных с
  выбранной:
  
  \(
    \forall (u, p) \in E \colon
    d_{p} = \min(d_{p}, d_{u} + w_{up})
  \), где \(w_{up}\) это вес ребра между вершинами \(u\) и \(p\).

  \item Будем повторять шаги \(3-5\) пока в \(A\) не попадут все достижимые
  вершины (либо в \(A\) не попадет финишная вершина).
\end{enumerate}

\underline{О корректности}:

Корректность покажем по индукции.

\textbf{База}: согласно шагу \(2\) расстояние до стартовой вершины \(s\) равно
нулю.

\textbf{Переход}: пусть \(\forall a \in A \colon d_{a} = \dist{a}\) это
действительно кратчайшие расстояния до вершин в \(A\) (кратчайшие расстояния
будем обозначать \(\dist{a}\), а расстояния, найденные в ходе работы
алгоритма~--- \(d_{a}\)).
Покажем, что \(d_{u} = \minL_{a \in A} (d_{a} + w_{au})\) это кратчайшее
расстояние до \(u\).

Рассмотрим кратчайший путь \(s \leadsto u\). Пусть \(a \in A\) это последняя
вершина на этом пути, которая принадлежит \(A\), а \(b\)~--- это следующая за
ней вершина на пути в \(u\). Тогда \(\dist{u} \ge d_{a} + w_{ab}\).

Но т.к. \(d_{u} = \minL_{a \in A} (d_{a} + w_{au})\), то
\(d_{a} + w_{ab} \ge d_{u}\), значит \(\dist{u} \ge d_{u}\). Итого \(d_{u}\)
это кратчайшее расстояние до \(u\).
