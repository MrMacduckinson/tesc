\question{Эйлеровы графы.}

\begin{definition}
  Эйлеров путь это путь, который проходит все ребра графа ровно по одному разу.
\end{definition}

\begin{definition}
  Эйлеров цикл это эйлеров путь, у которого начальная вершина совпадает с
  конечной.
\end{definition}

\begin{definition}
  Граф называется эйлеровым, если в нем есть эйлеров цикл, и полуэйлеровым, если
  в нем есть только эйлеров путь.
\end{definition}

\begin{theorem}
  Условие существования Эйлерова цикла

  Эйлеров цикл существует тогда и только тогда, когда степени всех вершин четны.  
\end{theorem}
\begin{proof}
  Индукция по количеству ребер

  \textbf{База}: \(m = 0\), т.к. ребер нет, то считаем, что Эйлеров цикл есть

  \textbf{Переход}: пусть теорема верна для графов с менее чем \(m\) ребрами.
  Т.к. все степени четны, то войдя в вершины мы всегда сможет выйти из неё (за
  исключением стартовой вершины) \(\implies\) будем идти по ребрам в случайном
  порядке пока не вернемся в стартовую вершину. Далее возможны две ситуации:

  \begin{itemize}
    \item Если мы обошли все ребра, то теорема верна
    \item Иначе удалим из графа все пройденные ребра. Он распадется на
    компоненты связности, в которых будет меньше ребер, чем в исходном графе
    \(\implies\) к ним применимо предположение индукции. Найдем в каждой
    компоненте Эйлеров цикл, после чего 'прикрепим' его к первоначально
    найденному циклу \(\implies\) получим Эйлеров цикл.
  \end{itemize}
\end{proof}

\begin{remark}
  Для существования Эйлерова пути необходимо и достаточно, чтобы количество
  вершин с нечетной степенью не превышало двух. В этом случае можно соединить
  эти вершины дополнительным ребром и построить Эйлеров цикл (т.к. в графе все
  степени вершин станут четными). После чего достаточно удалить добавленное
  ребро, чтобы получить Эйлеров путь.
\end{remark}
