\question{Реккурентные соотношения. Характеристические уравнения.}

\begin{definition}
  Рекуррентное соотношение это последовательность, в которой текущий член
  задается через предыдущие.
\end{definition}

\begin{definition}
  Рекуррентное соотношение называется однородным, если оно не содержит свободных
  членов.

  Рекуррентное соотношение со свободными членами называется неоднородными.
\end{definition}

\begin{definition}
  Характеристическое уравнение это уравнение, построенное по данному
  рекуррентному соотношению заменой \(a_{n}\) на \(r^{n}\).
\end{definition}

Оно может быть использовано для решения однородных рекуррентных соотношений. Для
этого нужно найти его корни, после чего записать общий вид решения. Далее нужно
подставить начальные условия, чтобы определить константы и получить замкнутую
формулу для рекуррентного соотношения.

\underline{Пример}:

Пусть требуется решить (т.е. найти замкнутую формулу) следующее рекуррентное
соотношение: \(a_{n} = 6 a_{n - 1} - 9 a_{n - 2}, a_{0} = 1, a_{2} = 45\).

Составим характеристическое уравнение и найдем его корни

\begin{align*}
  r^{n} = 6 r^{n - 1} - 9 r^{n - 2} \\
  r^{2} - 6 r + 9 = 0 \\
  r_{1} = 3, r_{2} = 3
\end{align*}

Значит общее решение будет иметь вид
\(a_{n} = c_{1} \cdot 3^{n} + c_{2} \cdot n \cdot 3^{n}\).
Найдем константы:

\begin{align*}
  a_{0} = c_{1} 3^{0} + c_{2} \cdot 0 \cdot 3^{0} = c_{1} = 1 \\
  a_{2} = c_{1} 3^{2} + c_{2} \cdot 2 \cdot 3^{2} = 9 c_{1} + 18 c_{2} = 45 \\
  c_{1} + 2 c_{2} = 5 \\
  c_{2} = 2
\end{align*}

Ответ: \(a_{n} = 3^{n} + 2n \cdot 3^{n}\).

\begin{remark}
  Решение однородных рекуррентных соотношений очень похоже на решение линейных
  однородных дифференциальных уравнений.
\end{remark}
