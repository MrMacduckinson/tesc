\question{Асимптотический анализ. Алгоритмы разделяй-и-властвуй.}

Асимптотический анализ это анализ поведения функции на бесконечности.

Пусть \(T(n)\) это функция 'количества работы' (время, память и т.п.) в
зависимости от объема входных данных, а \(g(n)\) это некоторая функция, тогда:

\begin{definition}
  \(g(n)\) называется оценкой сверху для \(T(n)\), если

  \begin{align*}
    T(n) \in O(g(n)) \iff
      \exists n_{0}, c > 0 \mid \forall n \ge n_{0} \colon T(n) \le c \cdot g(n)
  \end{align*}
\end{definition}

\begin{definition}
  \(g(n)\) называется оценкой cнизу для \(T(n)\), если
  
  \begin{align*}
    T(n) \in \Omega(g(n)) \iff
      \exists n_{0}, c > 0 \mid \forall n \ge n_{0} \colon c \cdot g(n) \le T(n)
  \end{align*}
\end{definition}

\begin{definition}
  Если \(g(n)\) одновременно является и оценкой сверху, и оценкой снизу, то
  \(g(n)\) называется точной оценкой.

  \begin{align*}
    T(n) \in \Theta(g(n)) \iff \begin{cases}
      T(n) \in O(g(n)) \\
      T(n) \in \Omega(g(n))
    \end{cases}
  \end{align*}
\end{definition}

Также можно определять оценки через пределы:

\begin{table}[H]
  \centering

  \renewcommand{\arraystretch}{1.5}
  \begin{tabular}{c|c|c}
    \(\lim_{n \to \infty} \frac{f(n)}{f(n)}\) & Оценка & Комментарий
    \\ \hline 
    \(c \neq 0, c \neq \infty\)
      & \(f \in \Theta(g)\)
      & \(f\) растет так же быстро, как и \(g\)
    \\
    \(c \neq \infty\)
      & \(f \in O(g)\)
      & \(f\) растет не быстрее \(g\)
    \\
      \(c \neq 0\)
        & \(f \in \Omega(g)\)
        & \(f\) растет не медленнее \(g\)
    \\
      \(c = 1\)
        & \(f \sim g\)
        & \(f\) и \(g\) эквивалентны
    \\
      \(c = 0\)
        & \(f \in o(g)\)
        & \(f\) растет медленнее \(g\)
    \\
      \(c = \infty\)
        & \(f \in \omega(g)\)
        & \(f\) растет быстрее \(g\)
  \end{tabular}
\end{table}

Некоторые свойства асимптотических оценок:

\begin{align*}
  \begin{rcases}
    f(n) \in O(g(n)) \\
    f(n) \in \Omega(g(n))
  \end{rcases} 
  \iff f(n) \in \Theta(g(n))
  \qquad
  f(n) \sim g(n) \implies f(n) \in \Theta(g(n))
  \\
  f(n) \in O(g(n)) \iff g(n) \in \Omega(f(n))
  \qquad
  f(n) \in o(g(n)) \implies f(n) \in O(g(n))
  \\
  f(n) \in o(g(n)) \iff g(n) \in \omega(f(n))
  \qquad
  f(n) \in \omega(g(n)) \implies f(n) \in \Omega(g(n))
\end{align*}

\begin{definition}
  Алгоритмы разделяй-и-властвуй основываются на том, что на каждой итерации они
  дробят текущую задачу на несколько частей, а потом объединяют результаты
  работы алгоритма для каждой из этих частей. Деление останавливается, когда
  алгоритм доходит до некоторого базового случая.
\end{definition}

На примере сортировки слиянияем:

\begin{align*}
  T(n) = 
  \under{2 T(n / 2)}{\text{рекурсивная работа}}
  +
  \under{n}{\text{работа по разделению и слиянию}}
\end{align*}

\todo Скобки пофиксить, либо оформить иначе 

Для асимптотической оценки времени работы таких алгоритмов чаще всего
используется Мастер теорема или метод Акра-Бацци.
