\question{Приложения определенного интеграла: вычисление длины дуги кривой, заданной параметрически.}

Рассмотрим формулу \(L = \int_{a}^{b} \sqrt{1 + y'(x)^2} \dd x\) при условии,
что кривая задана параметрически. Получим:

\begin{align*}
  x = \phi(t), y = \psi(t) \\
  \dd x = \phi'(t) \dd t \\
  a = \phi(\alpha), b = \phi(\beta), t \in [\alpha; \beta] \\
  y'(x) = \frac{\dd y}{\dd x} = \frac{\psi'(t)}{\phi'(t)}
\end{align*}

Подставим это в исходную формулу:

\begin{align*}
  L
  = \int_{a}^{b} \sqrt{1 + y'(x)^2} \dd x
  = \int_{\alpha}^{\beta}
    \sqrt{1 + \left(\frac{\psi'(t)}{\phi'(t)}\right)^2} \phi'(t) \dd t
  = \int_{\alpha}^{\beta} \sqrt{\phi'(t)^2 + \psi'(t)^2} \dd t \\
  a = \phi(\alpha), b = \phi(\beta)
\end{align*}

\begin{remark}\label{arc-diff-param}
  Таким образом, дифференциал дуги в при параметрическом задании будет равен
  
  \begin{align*}
    \dd l = \sqrt{\phi'(t)^2 + \psi'(t)^2} \dd t \\
    x = \phi(t), y = \psi(t) \\
  \end{align*}
\end{remark}