\question{Признаки сходимости несобственных интегралов: второй признак сравнения (предельный).}

\begin{theorem}
  Пусть \(f(x), g(x) \colon [a; +\infty] \to \RR\) и \(f(x) > 0, g(x) > 0\).
  Тогда если предел

  \begin{align*}
    \lim_{x \to +\infty} \frac{f(x)}{g(x)} = r \in \RR \setminus \{ 0 \} 
  \end{align*}

  существует, конечен и не равен нулю, то функции оба интеграла
  \(\int_{a}^{+\infty} f(x) \dd x\), \(\int_{a}^{+\infty} g(x) \dd x\)
  ведут себя одинаково в плане сходимости (т.е. либо оба сходятся, либо оба
  расходятся).
\end{theorem}
\begin{proof}
  По определению предела получаем:

  \begin{align*}
    \lim_{x \to +\infty} \frac{f(x)}{g(x)} = r
    \iff
    \forall \epsilon > 0 \exists \delta > 0 \mid
    \forall x \in [a; +\infty], x > \delta \colon
    \abs{\frac{f(x)}{g(x)} - r} < \epsilon
    \\
    r - \epsilon < \frac{f(x)}{g(x)} < r + \epsilon \mid \cdot \; g(x) > 0
    \\
    (r - \epsilon) g(x) < f(x) < (r + \epsilon) g(x)
  \end{align*}

  Далее используем признак сравнения в неравенствах (\ref{cnv-cmp}). Рассмотрим
  два случая:

  \begin{align*}
    \lets \int_{a}^{+\infty} f(x) \dd x \converge
    \implies \int_{a}^{+\infty} (r - \epsilon) g(x) \dd x \converge
  \end{align*}

  Т.к. \(r \in \RR\), а \(\epsilon > 0\) произвольное положительное число, то
  интеграл \(\int_{a}^{+\infty} g(x) \dd x\) также будет сходится. Второй случай
  рассматривается аналогично:

  \begin{align*}
    \lets \int_{a}^{+\infty} f(x) \dd x \notconverge
    \implies \int_{a}^{+\infty} (r + \epsilon) g(x) \dd x \notconverge
    \implies \int_{a}^{+\infty} g(x) \dd x \notconverge
  \end{align*}
\end{proof}

