\question{Признаки сходимости несобственных интегралов: теорема об абсолютной сходимости. Понятие условной сходимости.}

\begin{theorem}
  Пусть \(f(x) \colon [a; +\infty] \to \RR\). Тогда

  \begin{align*}
    \int_{a}^{+\infty} \abs{f(x)} \dd x \converge
    \implies \int_{a}^{+\infty} f(x) \dd x \converge
  \end{align*}
\end{theorem}
\begin{proof}
  Раскроем интегралы
  \(\abs{\int_{a}^{+\infty} f(x) \dd x}\)
  и
  \(\int_{a}^{+\infty} \abs{f(x)} \dd x\)
  по определению:

  \begin{align*}
    \abs{\int_{a}^{+\infty} f(x) \dd x}
    = \abs{\lim_{\beta \to +\infty} \int_{a}^{\beta} f(x) \dd x}
    = \lim_{\beta \to +\infty} \abs{\int_{a}^{\beta} f(x) \dd x}
    \\
    \int_{a}^{+\infty} \abs{f(x)} \dd x
    = \lim_{\beta \to +\infty} \int_{a}^{\beta} \abs{f(x)} \dd x
  \end{align*}

  Далее воспользуемся свойством определенных интегралов (\ref{dint-abs-prop})
  \(\abs{\int_{a}^{b} f(x) \dd x} \le \int_{a}^{b} \abs{f(x)} \dd x\) и
  предельным переходом:

  \begin{align*}
    \lim_{\beta \to +\infty} \abs{\int_{a}^{\beta} f(x) \dd x}
    \le
    \lim_{\beta \to +\infty} \int_{a}^{\beta} \abs{f(x)} \dd x
    \\
    \abs{\int_{a}^{+\infty} f(x) \dd x} \le \int_{a}^{+\infty} \abs{f(x)} \dd x
  \end{align*}

  Т.к. интеграл в правой части сходится, то обозначим его значение
  \(r \in \RR\). Раскрывая модуль по определению получаем:

  \begin{align*}
    -r \le \int_{a}^{+\infty} f(x) \dd x \le r
  \end{align*}

  Другими словами значение интеграла ограничено, а значит интеграл сходится.
\end{proof}

\begin{definition}
  Если \(\int_{a}^{+\infty} \abs{f(x)} \dd x \converge\), то интеграл
  \(\int_{a}^{+\infty} f(x) \dd x\) называется абсолютного сходящимся.
\end{definition}

\begin{definition}
  Если \(\int_{a}^{+\infty} \abs{f(x)} \dd x \notconverge\), а
  \(I = \int_{a}^{+\infty} f(x) \dd x \converge\) интеграл \(I\) называется
  условно сходящимся.
\end{definition}


