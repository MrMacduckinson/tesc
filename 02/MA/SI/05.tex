\question{Интегрирование тригонометрических функций. Универсальная тригонометрическая подстановка.}

\begin{remark}
  Всякая рациональная дробь интегрируемая, поэтому можно попытаться с помощью
  замены свести функции другого вида к рациональным дробям.
\end{remark}

Если требуется вычислить интеграл вида \(\int R(\sin x, \cos x) \dd x\), где
\(R\) это некоторая \textit{рациональная} функция, то можно применить
универсальную тригонометрическую подстановку:

\begin{align*}
  x = 2 \arctg t \iff t = \tg \frac{x}{2}
\end{align*}

Тогда составляющие интеграла преобразуется следующим образом:

\begin{align*}
  \sin x
  =
  2 \sin \frac{x}{2} \cos \frac{x}{2}
  = 
  \frac{2 \sin \sfrac{x}{2} \cos \sfrac{x}{2}}
  {\sin^2 \sfrac{x}{2} + \cos^2 \sfrac{x}{2}}
  =
  \frac{2 \tg \sfrac{x}{2}}{\tg^2 \sfrac{x}{2} + 1}
  =
  \frac{2 t}{1 + t^2}
  \\
  \cos x
  =
  \cos^2 \frac{x}{2} - \sin^2 \frac{x}{2}
  = 
  \frac{\cos^2 \sfrac{x}{2} - \sin^2 \sfrac{x}{2}}
  {\sin^2 \sfrac{x}{2} + \cos^2 \sfrac{x}{2}}
  =
  \frac{1 - \tg^2 \sfrac{x}{2}}{\tg^2 \sfrac{x}{2} + 1}
  =
  \frac{1 - t^2}{1 + t^2}
  \\
  \dd x = \dd (2 \arctg t) = \frac{2}{1 + t^2} \dd t
\end{align*}

Подставляя полученные выражения в исходный интеграл, получаем:

\begin{align*}
  \int R(\sin x, \cos x) \dd x
  \eqby{УТП}
  \int R \left(\frac{2 t}{1 + t^2}, \frac{1 - t^2}{1 + t^2}\right)
    \cdot \frac{2}{1 + t^2} \dd t
\end{align*}