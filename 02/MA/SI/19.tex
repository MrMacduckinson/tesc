\question{Вычисление несобственного интеграла 1-го рода: формула Ньютона-Лейбница, интегрирование по частям, замена переменной.}

Т.к. несобственный интеграл первого рода это по сути предел, то его можно
вычислить с помощью формулы Ньютона-Лейбница:

\begin{align*}
  \int_{a}^{+\infty} f(x) \dd x
  = \left( \lim_{\beta \to +\infty} F(\beta) \right) - F(a)
  = \lim_{\beta \to +\infty} F(x) \bigg\vert_{a}^{\beta}
\end{align*}

Интегрирование по частями и замена переменной выполняются также, как и в
определенном интеграле (аккуратнее с пределами интегрирования при замене).

\begin{remark}
  Иногда после замены несобственный интеграл может превратиться в собственный.
\end{remark}