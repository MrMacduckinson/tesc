\question{Определение и свойства неопределенного интеграла.}

\begin{definition}
  Кусочная дифференцируемая функция \(F(x)\) называется первообразной для
  функции \(f(x)\), если \(F'(x) = f(x)\).
\end{definition}

\begin{theorem}
  Разность двух первообразных для одной и той же функции~--- константа.
\end{theorem}
\begin{proof}
  Пусть дана функция \(f(x)\) и две её первообразные \(F_{1}(x)\), \(F_{2}(x)\).
  Обозначим их разность как \(\phi(x) = F_{1}(x) - F_{2}(x)\). Производная этой
  функции будет равна \(
    \phi'(x)
    = (F_{1}(x) - F_{2}(x))'
    = F'_{1}(x) - F'_{2}(x)
    = f(x) - f(x)
    = 0
  \).

  Из множества дифференцируемости \(F_{1}(x)\) и \(F_{2}(x)\) выберем наименьшее
  и выделим в нем отрезок \([a; x]\). По т. Лагранжа:

  \begin{align*}
    \exists \xi \in (a; x) \colon \phi'(\xi) = \frac{\phi(x) - \phi(a)}{x - a}
  \end{align*}

  Т.к. \(\forall \xi \colon \phi'(\xi) = 0\), то \(\phi(x) - \phi(a) = 0\), т.е.
  \(\phi(x) = \phi(a)\). Т.к. отрезок произвольный, то значения функции
  \(\phi(x)\) равны во всех точках, т.е. она константа.
\end{proof}

\begin{corollary}
  Первообразные для \(f(x)\) составляют множество функций вида
  \(\{ F(x) + C \mid C \in \RR \}\), где \(F(x)\) это какая-либо первообразная.
\end{corollary}

\begin{definition}
  Семейство первообразных функции \(f(x)\) называется неопределенным интегралом
  функции \(f(x)\) по аргументу \(x\).
\end{definition}

\begin{remark}
  О существовании первообразной

  Не для каждой функции существует первообразная, но для каждой непрерывной на
  отрезке. Даже если первообразная существует, то она не всегда выражается в
  элементарных функциях, например, \(\int e^{-x^2} \dd x\).
\end{remark}

Далее рассмотрим некоторые свойства неопределенного интеграла.

\begin{lemma}\label{ad-prop-1}
  \begin{align*}
    \int \dd F(x) = F(x) + C
  \end{align*}
\end{lemma}
\begin{proof}
  \begin{align*}
    \dd F(x) = F'(x) \dd x = f(x) \dd x
    \implies
    \int \dd F(x) = \int f(x) \dd x = F(x) + C
  \end{align*}
\end{proof}

\begin{lemma}\label{ad-prop-2}
  \begin{align*}
    \left( \int f(x) \dd x \right)' = f(x)
  \end{align*}
\end{lemma}
\begin{proof}
  \begin{align*}
    \left( \int f(x) \dd x \right)' = (F(x) + C)' = F'(x) = f(x)
  \end{align*}
\end{proof}

\begin{lemma}\label{ad-prop-3}
  Линейность

  \begin{align*}
    \int \alpha f(x) \dd x = \alpha \int f(x) \dd x \\
    \int (f(x) + g(x)) \dd x = \int f(x) \dd x + \int g(x) \dd x
  \end{align*}
\end{lemma}
\begin{proof}
  \begin{align*}
    \int \alpha f(x) \dd x
    = \int \dd (\alpha F(x))
    = \alpha F(x) + C
  \end{align*}

  При первым переходе используется свойство линейности дифференциала,
  а при втором~--- \ref{ad-prop-1}. Доказательство для суммы функций аналогично.
\end{proof}
