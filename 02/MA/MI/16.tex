\question{Поверхностный интеграл 2-го рода: математическое определение, вычисление, свойства.}

\begin{remark}
  О математическом определении

  Чтобы математически определить поверхностный интеграл 2-ого рода, сначала
  отдельно определяются интегралы в проекциях на координатные плоскости, после
  чего вычисляется их сумма. При построении поверхностного интеграла 2-ого в
  проекциях была получена соответствующая формула связи
  (\ref{surf-int-coords} \(\to\) \ref{surf-int-proj}):

  \begin{align*}
    \under{\iint_{S} \Big(
      P \cos \alpha +
      Q \cos \beta +
      R \cos \gamma\Big
    ) \dd \sigma}{I \text{ род}} =
    \under{\iint_{S}
      P \dd y \dd z +
      Q \dd x \dd z +
      R \dd y \dd z
    }{II \text{ род}}
  \end{align*}
\end{remark}

\begin{remark}
  О вычислении

  Рассмотрим криволинейный интеграл 2-ого рода в проекциях
  (\ref{surf-int-proj}):

  \begin{align*}
    \iint_{S} P \dd y \dd z + Q \dd x \dd z + R \dd x \dd y  
  \end{align*}

  Спроецируем поверхность \(S\) на координатную плоскость \(Oxy\), получим
  некоторую область \(D_{xy}\):

  \begin{align*}\label{eq:surf-int-proj-calc}\tag{\(\bigstar\)}
    \iint_{S} R(x, y, z) \dd x \dd y
    = \pm \iint_{D_{xy}} R(x, y, z(x, y)) \dd x \dd y
  \end{align*}

  Знак \(\pm\) ставится потому, что в поверхностном интеграле
  \(\dd x \dd y \approx \cos \gamma \dd \sigma\) это проекция и косинус
  учитывает знак (т.е. направление обхода \(\dd \sigma\)). А в двойном интеграле
  \(\dd x \dd y\) это площадь элементарного участка
  \(\implies \dd x \dd y > 0\).

  Таким образом вычисление поверхностного интеграла 2-ого рода в проекциях
  сводится к вычислению трех двойных интегралов в проекции на каждую из
  координатных плоскостей (но нужно не забывать про знаки).
\end{remark}
