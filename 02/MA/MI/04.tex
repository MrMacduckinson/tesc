\question{Криволинейные координаты.}

Полярные координаты определяются как:

\begin{align*}
  \begin{cases}
    x = \rho \cos \phi \\
    y = \rho \sin \phi \\
  \end{cases}
  \rho \ge 0, \phi \in [0; 2 \pi) \\
  \dd x \dd y \longrightarrow \rho \dd \rho \dd \phi
\end{align*}

Цилиндрические координаты определяются как:

\begin{align*}
  \begin{cases}
    x = \rho \cos \phi \\
    y = \rho \sin \phi \\
    z = z
  \end{cases}
  \rho \ge 0, \phi \in [0; 2 \pi), z \in \RR \\
  \dd x \dd y \dd z \longrightarrow \rho \dd \rho \dd \phi \dd z
\end{align*}

Сферические координаты определяются как:

\begin{align*}
  \begin{cases}
    x = \rho \cos \phi \sin \theta \\
    y = \rho \sin \phi \sin \theta \\
    z = \rho \cos \theta
  \end{cases}
  \rho \ge 0, \phi \in [0; 2 \pi), \theta \in [0, \pi] \\
  \dd x \dd y \dd z \longrightarrow
    \rho^2 \sin \theta \dd \rho \dd \phi \dd \theta
\end{align*}

\begin{remark}
  О том, почему элементы объема имеют именно такое задание можно прочитать в
  следующем вопросе.
\end{remark}
