\question{Механический смысл потока и дивергенции.}

\begin{theorem}
  О механическом смысле потока

  Поток это количество жидкости, протекающей за единицу времени через площадку
  \(S\) в заданном направлении.
\end{theorem}
\begin{proof}
  Механический смыл потока был выяснен при построении поверхностного интеграла
  2ого рода.
\end{proof}

\begin{theorem}
  О механическом смысле дивергенции

  Дивергенция \(\div \vec{F}(M_{0})\) это мощность точечного источника поля
  \(\vec{F}\).
\end{theorem}
\begin{proof}
  Рассмотрим равенство в т. Гаусса-Остроградского (\ref{GO}):

  \begin{align*}
    \iiint_{T} \div \vec{F} \dd v
    = \oiint_{S_{T}} \vec{F} \dd \vec{\sigma}
    = \flow
  \end{align*}

  Выберем в пространстве, где действует \(\vec{F}\), точку \(M_{0}\) и окружим
  её объемом в границей \(S\). К тройному интегралу в левой части равенства
  применима т. Лагранжа о среднем:

  \begin{align*}
    \exists M \in V \colon
      \iiint_{T} \div \vec{F} \dd v = \div \vec{F}(M) \cdot V
  \end{align*}

  Будем стягивать выделенный ранее объем в точку \(M_{0}\), получим

  \begin{align*}
    \lim_{\substack{M \to M_{0} \\ V \to 0}}
      \div \vec{F}(M) \cdot V = 
    \lim_{\substack{M \to M_{0} \\ V \to 0}}
      \flow
    \hspace{20pt} \mid \colon V
    \\
    \div \vec{F}(M_{0}) = \frac{\flow}{V}
  \end{align*}

  Выражение в правой части это и есть мощность точечного источника.
\end{proof}

\begin{corollary}
  Таким образом т. Гаусса-Остроградского (\ref{GO}) утверждает, что поток равен
  сумме мощностей точечных источников.
\end{corollary}