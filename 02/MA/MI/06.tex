\question{Криволинейный интеграл 1-го рода: определение, свойства, вычисление, геометрический и физический смысл.}

Задача: найти массу \(m\), распределенную с плотностью \(f\) по
участку плоской кривой (простая дуга \(\breve{AB}\)).

\begin{remark}
  Постановка задачи определяет физический смысл криволинейного интеграла первого
  рода.
\end{remark}

Составим интеграл: разобьем дугу \(\breve{AB}\) на элементарные
дуги \(\dd l\). Масса таких дуг будет равна \(f(x, y) \dd l\), значит масса всей
дуги будет равна

\begin{align*}
  \int_{AB} f(x, y) \dd l
\end{align*}

Полученный интеграл называется криволинейным интегралом 1-ого рода.

\begin{remark}
  О математическом определении

  \begin{enumerate}
    \item Введем ДПСК \(\breve{AB} \to y = y(x), x \in [a; b]\).
    \item Разобьем дугу на элементарные дуги \(l_{i}\),
      тогда элементарная масса будет равна
      \(m_{i} = f(\xi_{i}, \eta_{i}) \Delta l_{i}\).
  
    \item Составим предел интегральных сумм
    
    \begin{align*}
      \lim_{\substack{n \to \infty \\ \tau \to 0}}
        \sum_{i = 1}^{n} f(\xi_{i}, \eta_{i}) \Delta l_{i}
    \end{align*}
    
    \item Перейдем к интегралу и получим такое же выражение, что и выше.
  \end{enumerate}
\end{remark}

\begin{remark}
  О вычислении
  
  \(\dd l\) это дифференциал дуги (см. \ref{arc-diff}), значит получаем, что

  \begin{align*}
    \int_{AB} f(x, y) \dd l
      = \int_{x_{1}}^{x_{2}} f(x, y(x)) \sqrt{1 + y'(x)^2} \abs{\dd x}
  \end{align*}

  или в параметрическом виде (см. \ref{arc-diff-param}):

  \begin{align*}
    \int_{AB} f(x, y) \dd l
    = \int_{t_{1}}^{t_{2}}
      f(\phi(t), \psi(t))
      \sqrt{\phi'(t)^2 + \psi'(t)^2} \abs{\dd t}
  \end{align*}

  Дифференциалы \(\dd x\) и \(\dd t\) находятся под модулем, т.к. если дуга
  проходится в обратном направлении (т.е. \(\dd x, \dd y, \dd t < 0\)), то
  получится отрицательное число.
  Однако \(\dd l\) здесь имеет смысл длины и не может быть отрицательным.
\end{remark}

\begin{remark}
  Криволинейный интеграл первого рода не зависит от направления прохода дуги:

  \begin{align*}
    \int_{AB} f(x, y) \dd l = \int_{BA} f(x, y) \dd l
  \end{align*}

  Остальные его свойства совпадают со свойствами определенного интеграла.
\end{remark}

\begin{remark}
  Геометрический смысл криволинейного интеграла заключается в том, что он равен
  части площади поверхности криволинейного цилиндра, основанием которого
  является дуга \(\breve{AB}\).
\end{remark}
