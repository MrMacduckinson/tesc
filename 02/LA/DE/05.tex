\question{Линейное уравнение первого порядка. Метод Лагранжа.}

\begin{definition}
  Линейным однородным уравнением первого порядка (ЛОДУ\(_1\)) называется
  уравнение вида

  \begin{align*}
    y' + p(x) y = 0
  \end{align*}
\end{definition}

ЛОДУ\(_1\) является уравнением с разделяющими переменными, поэтому оно решается
следующим образом:

\begin{align*}
  y' + p(x) y = 0 \\
  \frac{\dd y}{\dd x} = -p(x) y \\
  \frac{\dd y}{y} = -p(x) \dd x \\
  \overline{y} = C \cdot \under{e^{-\int p(x) \dd x}}{y_1}
\end{align*}

\begin{remark}
  При решении данного уравнения мы поделили на \(y \neq 0\). Заметим, что
  \(y = 0\) также является решением ЛОДУ\(_1\), однако оно получаемо из общего
  решения при \(C = 0\).
\end{remark}
  
\begin{definition}
  Линейным неоднородным уравнением первого порядка (ЛНДУ\(_1\)) называется
  уравнение вида

  \begin{align*}
    y' + p(x) y = q(x), \hspace{10pt} q(x) \neq 0
  \end{align*}
\end{definition}

\textbf{Метод Лагранжа} (метод вариации произвольной постоянной) для решения
ЛНДУ\(_1\):

\begin{enumerate}
  \item Найдем частное решение \(y_{1}\) соответствующего однородного уравнения.
  \item Будем искать решение ЛНДУ\(_1\) в виде \(y(x) = y_{1}(x) \cdot C(x)\).
    Преобразуем ДУ в соответствии с этой заменой:

    \begin{align*}
      y' + p(x) y = q(x) \\
      y_{1}'(x) C(x) + y_{1}(x) C'(x) + p(x) y_{1}(x)  C(x) = q(x) \\
      y_{1}(x) C'(x) + C(x) 
      \under{\Big( y_{1}'(x) + p(x) y_{1}(x) \Big)}{\text{= 0}}
      = q(x) \\
      y_{1}(x) C'(x) = q(x) \\
      C(x) = \int \frac{q(x)}{y_{1}(x)} \dd x + C
    \end{align*}

  \item Подставим найденную функцию \(C(x)\) в \(y(x) = y_{1}(x) \cdot C(x)\).
\end{enumerate}

\begin{remark}
  Помимо рассмотренных интегрируемыми являются уравнения Лагранжа, Клеро,
  Риккати, Бернулли. 
\end{remark}

\begin{definition}
  ДУ называется уравнением Бернулли, если его можно привести к виду   

  \begin{align*}
    y' + p(x) y = q(x) y^{n}
  \end{align*}

  Оно решается с помощью  подстановки Бернулли:

  \begin{align*}
    y(x) = u(x) \cdot v(x) \\
    y'(x) = u'(x) \cdot v(x) + u(x) \cdot v'(x)
  \end{align*}
\end{definition}

\underline{Пример}:

\begin{align*}
  y' + 2 y = y^2 e^x \\
  u' v + u v' + 2 u v = v^2 u^2 e^x \\
  u' v + u(v'+ 2v) = u^2v^2e^x \\
  \begin{cases}
    v' + 2v = 0 \\
    u'v = u^2 v^2 e^x
  \end{cases}
\end{align*}

Из первого полученного уравнения находим \(v(x)\), потом подставляем его во
второе уравнение и находим \(u(x)\). Ответом будет \(y(x) = u(x) \cdot v(x)\).
