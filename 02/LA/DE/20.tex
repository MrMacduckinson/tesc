\question{Теория устойчивости: определение устойчивости по Ляпунову, фазовая плоскость, траектории ДУ. Примеры устойчивого и неустойчивого решения.}

\begin{definition}
  Пусть дана СДУ

  \begin{align*}
    \begin{cases}
      x' = f_{1}(t, x, y) \\
      y' = f_{2}(t, x, y)
    \end{cases}
  \end{align*}

  и \(x(t), y(t)\) это её решение, удовлетворяющее начальным условиям
  \(x(t = 0) = x_{0}, y(t = 0) = y_{0}\). Рассмотрим
  \(\widetilde{x}(t), \widetilde{y}(t)\)~--- другое решение, удовлетворяющее
  начальным условиям с отклонением \(
    \widetilde{x}(t = 0) = \widetilde{x}_{0},
    \widetilde{y}(t = 0) = \widetilde{y}_{0}
  \).

  Решение \(x(t), y(t)\) называется \textit{устойчивым по Ляпунову}, если

  \begin{align*}
    \forall \epsilon > 0 \;\;
    \exists \delta > 0 \mid
    \forall t > 0 \;\;
    \forall x_{0}, y_{0}, \widetilde{x}_{0}, \widetilde{y}_{0} \colon
    \begin{cases}
      \abs{x_{0} - \widetilde{x}_{0}} < \delta \\
      \abs{y_{0} - \widetilde{y}_{0}} < \delta
    \end{cases}
    \implies
    \begin{cases}
      \abs{\widetilde{x}(t) - x(t)} < \epsilon \\
      \abs{\widetilde{y}(t) - y(t)} < \epsilon
    \end{cases}
  \end{align*}
\end{definition}

\underline{Исследование на устойчивость линейной автономной системы ДУ}:

Пусть дана ДУ:

\begin{align*}\label{eq:given-sde}\tag{\(\blacklozenge\)}
  \begin{cases}
    \frac{\dd x}{\dd t} = a x + b y \\
    \frac{\dd y}{\dd t} = c x + m y
  \end{cases}
  \implies \frac{\frac{\dd y}{\dd t} }{\frac{\dd x}{\dd t}}
  = \frac{\dd y}{\dd x} 
  = \frac{c x + m y}{a x + b y} 
\end{align*}

\begin{multicols}{2}
  \begin{figure}[H]
  \centering
  \begin{tikzpicture}[scale = 0.5]
    
    \draw[->] (-5, 0) -- (5, 0) node[right] {\(x\)};
    \draw[->] (0, -5) -- (0, 5) node[above] {\(y\)};

    \foreach \d in {1, ..., 4} {
      \draw[dashed] plot[smooth cycle] coordinates {
        (-\d, 0) (0, \d) (\d, \d  * 0.5) (\d, -\d) (0, -\d * 0.5)
      };
    }

  \end{tikzpicture}
\end{figure}
  \columnbreak

  \begin{definition}
    Общий интеграл для \ref{eq:given-sde} \(\phi(x, y) = C\) (семейство
    интегральных кривых) называется фазовым портретом системы или траекторией
    ДУ.
  \end{definition}

  \begin{remark}
    Решения \(x = 0\), \(y = 0\) являются особыми для \ref{eq:given-sde}.
  \end{remark}
\end{multicols}

Запишем СДУ в матричном виде:

\begin{align*}
  Y' = \begin{pmatrix}
    a & b \\
    c & m
  \end{pmatrix} Y
  \implies \begin{vmatrix}
    a - \lambda & b \\
    c & m - \lambda
  \end{vmatrix} = 0
\end{align*}

Далее все зависит от количества решений:

\begin{align*}
  \begin{matrix}
    \lambda_{1} \neq \lambda_{2} \in \RR
      & \implies & c_{1} e^{\lambda_{1} t} + c_{2} e^{\lambda_{2} t}
    \\
    \lambda_{1} = \lambda_{2} = \lambda \in \RR
      & \implies & c_{1} e^{\lambda t} + c_{2} t \cdot e^{\lambda t}
    \\
    \lambda_{1, 2} = \alpha + i \beta \in \CC
      & \implies & e^{\alpha t} (c_{1} \cos \beta t + c_{2} \sin \beta t)
  \end{matrix}
\end{align*}

Подробно рассмотрим случай \(\lambda_{1} \neq \lambda_{2} \in \RR\) причем
\(\lambda_{1}, \lambda_{2} < 0\).

Решаем ДУ методом исключения. Составим характеристическое уравнение:
\(\lambda^2 - (a + m) \lambda + (a m - b c) = 0\). Получаем:

\begin{align*}\label{sde-generic}
  \begin{cases}
    x(t) = c_{1} e^{\lambda_{1} t} + c_{2} e^{\lambda_{2} t} \\
    y(t) = \frac{1}{b} \left(
      c_{1} e^{\lambda_{1} t} (\lambda_{1} - a)
        + c_{2} e^{\lambda_{2} t} (\lambda_{2} - a)
    \right)
  \end{cases}
\end{align*}

Подберем константы, соответствующие начальным условиям \(x(t = 0) = x_{0}\),
\(y(t = 0) = y_{0}\) (отклонение от начальных условий \((0, 0)\)):

\begin{align*}
  x(t) =
    \frac{a x_{0} + b y_{0} - \lambda_{2} x_{0}}
    {\lambda_{1} - \lambda_{2}} e^{\lambda_{1} t}
    +
    \frac{-a x_{0} - b y_{0} + \lambda_{1} x_{0}}
    {\lambda_{1} - \lambda_{2}} e^{\lambda_{2} t}
  \\
  y(t) = \frac{1}{b} \left(
    \frac{a x_{0} + b y_{0} - \lambda_{2} x_{0}}
    {\lambda_{1} - \lambda_{2}} (\lambda_{1} - a) e^{\lambda_{1} t}
    +
    \frac{-a x_{0} - b y_{0} + \lambda_{1} x_{0}}
    {\lambda_{1} - \lambda_{2}} (\lambda_{2} - a) e^{\lambda_{2} t}
  \right)
\end{align*}

Таким образом

\begin{align*}
  \forall \epsilon > 0 \; \exists \delta > 0 \colon
    \abs{x_{0}} < \delta, \abs{y_{0}} < \delta
  \implies x(t) < \epsilon, y(t) < \epsilon
  \iff \begin{cases}
    t \to \infty \\
    x_{0} \to 0 \\
    y_{0} \to 0
  \end{cases} \implies \begin{cases}
    x(t) \to 0 \\
    y(t) \to 0
  \end{cases}
\end{align*}

Значит решение \(x = 0, y = 0\) устойчивое.

Для остальных случаев приведем пример и построим фазовый портрет:

\begin{figure}[H]
  \centering

  \begin{subfigure}[b]{0.3\textwidth}
    \begin{tikzpicture}[
  decoration = {
    markings,
    mark = between positions 0.1 and 0.4 step 0.1 with
      {\arrow[line width = 1pt]{stealth}},
    mark = between positions 0.6 and 0.9 step 0.1 with
      {\arrowreversed[line width = 1pt]{stealth}},
  }
]
    
  \draw[->] (-3, 0) -- (3, 0) node[right] {\(x\)};
  \draw[->] (0, -3) -- (0, 3) node[above] {\(y\)};

  \foreach \d in {-4, -3, -2, -1, 1, 2, 3, 4} {
    \draw[domain = -2.5 : 2.5, variable = \x, postaction = { decorate }]
      plot ({\x}, {0.1 * \d * \x * \x});
  }

  \node at (-1, -3) {\(\begin{cases}
    x' = -x \\
    y' = -2y
  \end{cases}\)};
  \node at (1.4, -3) {\(
    \begin{matrix}
      \lambda_{1} \neq \lambda_{2} \in \RR \\
      \lambda_{1, 2} < 0
    \end{matrix}
  \)};
\end{tikzpicture}

    \caption{Устойчивый узел}

  \end{subfigure}
  \qquad
  \begin{subfigure}[b]{0.3\textwidth}

    \begin{tikzpicture}[
  decoration = {
    markings,
    mark = between positions 0.1 and 0.4 step 0.1 with
      {\arrowreversed[line width = 1pt]{stealth}},
    mark = between positions 0.6 and 0.9 step 0.1 with
      {\arrow[line width = 1pt]{stealth}},
  }
]
    
  \draw[->] (-3, 0) -- (3, 0) node[right] {\(x\)};
  \draw[->] (0, -3) -- (0, 3) node[above] {\(y\)};

  \foreach \d in {-4, -3, -2, -1, 1, 2, 3, 4} {
    \draw[domain = -2.5 : 2.5, variable = \x, postaction = { decorate }]
      plot ({\x}, {0.1 * \d * \x * \x});
  }

  \node at (-1, -3) {\(\begin{cases}
    x' = x \\
    y' = 2y
  \end{cases}\)};
  \node at (1.4, -3) {\(
    \begin{matrix}
      \lambda_{1} \neq \lambda_{2} \in \RR \\
      \lambda_{1, 2} > 0
    \end{matrix}
  \)};
\end{tikzpicture}

    \caption{Неустойчивый узел}

  \end{subfigure}
  \qquad
  \begin{subfigure}[b]{0.3\textwidth}

    \begin{tikzpicture}
  \draw[->] (-3, 0) -- (3, 0) node[right] {\(x\)};
  \draw[->] (0, -3) -- (0, 3) node[above] {\(y\)};
   
  \begin{scope}[
    decoration = {
      markings,
      mark = between positions 0.2 and 0.8 step 0.1 with
        {\arrow[line width = 1pt]{stealth}},
    }
  ]
    \foreach \d in {1, 2, 3, 4} {
      \draw[domain = 0.5 : 2.5, variable = \x, postaction = { decorate }]
        plot ({\x}, {\d / (3 * \x)});
    }  
  \end{scope}

  \begin{scope}[
    decoration = {
      markings,
      mark = between positions 0.2 and 0.8 step 0.1 with
        {\arrow[line width = 1pt]{stealth}},
    }
  ]
    \foreach \d in {1, 2, 3, 4} {
      \draw[domain = 0.5 : 2.5, variable = \x, postaction = { decorate }]
        plot ({\x}, {-\d / (3 * \x)});
    }  
  \end{scope}

  \begin{scope}[
    decoration = {
      markings,
      mark = between positions 0.2 and 0.8 step 0.1 with
        {\arrowreversed[line width = 1pt]{stealth}},
    }
  ]
    \foreach \d in {1, 2, 3, 4} {
      \draw[domain = -2.5 : -0.5, variable = \x, postaction = { decorate }]
        plot ({\x}, {\d / (3 * \x)});
    }  
  \end{scope}

  \begin{scope}[
    decoration = {
      markings,
      mark = between positions 0.2 and 0.8 step 0.1 with
        {\arrowreversed[line width = 1pt]{stealth}},
    }
  ]
    \foreach \d in {1, 2, 3, 4} {
      \draw[domain = -2.5 : -0.5, variable = \x, postaction = { decorate }]
        plot ({\x}, {-\d / (3 * \x)});
    }  
  \end{scope}

  \node at (-1, -3.3) {\(\begin{cases}
    x' = x \\
    y' = -2y
  \end{cases}\)};
  \node at (1.4, -3.3) {\(
    \begin{matrix}
      \lambda_{1} \neq \lambda_{2} \in \RR \\
      \lambda_{1} \lambda_{2} <  0
    \end{matrix}
  \)};
\end{tikzpicture}

    \caption{Неустойчивое седло}

  \end{subfigure}
\end{figure}

\begin{figure}[H]
  \centering

  \begin{subfigure}[b]{0.3\textwidth}
    \begin{tikzpicture}[
  decoration = {
    markings,
    mark = between positions 0.4 and 0.9 step 0.1 with
      {\arrowreversed[line width = 1pt]{stealth}},
  }
]
    
  \draw[->] (-3, 0) -- (3, 0) node[right] {\(x\)};
  \draw[->] (0, -3) -- (0, 3) node[above] {\(y\)};

  \foreach \d in {3, 4} {
    \draw[
      domain = 1 : 25.1327,
      variable = \t,
      samples = 500,
      postaction = { decorate }
    ]
      plot ({\t r} : {0.001 * \d * \t * \t});
  }

  \node at (-1, -3) {\(\begin{cases}
    x' = -x + y \\
    y' = -x - y
  \end{cases}\)};
  \node at (1.4, -3) {\(
    \begin{matrix}
      \lambda_{1,2} = \alpha + i \beta \in \CC \\
      \alpha < 0
    \end{matrix}
  \)};
\end{tikzpicture}

    \caption{Устойчивый фокус}

  \end{subfigure}
  \qquad
  \begin{subfigure}[b]{0.3\textwidth}

    \begin{tikzpicture}[
  decoration = {
    markings,
    mark = between positions 0.4 and 0.9 step 0.1 with
      {\arrow[line width = 1pt]{stealth}},
  }
]
    
  \draw[->] (-3, 0) -- (3, 0) node[right] {\(x\)};
  \draw[->] (0, -3) -- (0, 3) node[above] {\(y\)};

  \foreach \d in {3, 4} {
    \draw[
      domain = 1 : 25.1327,
      variable = \t,
      samples = 500,
      postaction = { decorate }
    ]
      plot ({\t r} : {0.001 * \d * \t * \t});
  }

  \node at (-1, -3) {\(\begin{cases}
    x' = x + y \\
    y' = -x + y
  \end{cases}\)};
  \node at (1.6, -3) {\(
    \begin{matrix}
      \lambda_{1,2} = \alpha + i \beta \in \CC \\
      \alpha > 0
    \end{matrix}
  \)};
\end{tikzpicture}

    \caption{Неустойчивый фокус}

  \end{subfigure}
  \qquad
  \begin{subfigure}[b]{0.3\textwidth}

    \begin{tikzpicture}[
  decoration = {
    markings,
    mark = between positions 0.1 and 0.9 step 0.15 with
      {\arrow[line width = 1pt]{stealth}},
  }
]
    
  \draw[->] (-3, 0) -- (3, 0) node[right] {\(x\)};
  \draw[->] (0, -3) -- (0, 3) node[above] {\(y\)};

  \foreach \d in {0.5, 1, 1.5, 2} {
    \draw[postaction = { decorate }] (0, 0) ellipse (0.5 * \d cm and \d cm);
  }

  \node at (-1, -3) {\(\begin{cases}
    x' = y \\
    y' = -4x
  \end{cases}\)};
  \node at (1.4, -3) {\(
    \begin{matrix}
      \lambda_{1,2} = \alpha + i \beta \in \CC \\
      \alpha = 0
    \end{matrix}
  \)};
\end{tikzpicture}

    \caption{Центр (неусточивый)}

  \end{subfigure}
\end{figure}

\begin{figure}[H]
  \centering

  \begin{subfigure}[b]{0.3\textwidth}
    \begin{tikzpicture}[
  decoration = {
    markings,
    mark = between positions 0.1 and 0.4 step 0.1 with
      {\arrow[line width = 1pt]{stealth}},
    mark = between positions 0.6 and 0.9 step 0.1 with
      {\arrowreversed[line width = 1pt]{stealth}},
  }
]
    
  \draw[->] (-3, 0) -- (3, 0) node[right] {\(x\)};
  \draw[->] (0, -3) -- (0, 3) node[above] {\(y\)};

  \foreach \d in {-2.5, -1.5, -0.5, 0.5, 1.5, 2.5} {
    \draw[postaction = { decorate }] (\d, -2) -- (\d, 2.5);
  }

  \node at (-1, -3) {\(\begin{cases}
    x' = 0 \\
    y' = y
  \end{cases}\)};
  \node at (1.4, -3) {\(
    \begin{matrix}
      \lambda_{1} \neq \lambda_{2} \in \RR \\
      \lambda_{1} \lambda_{2} = 0 \\
      \lambda_{1} = 0, \lambda_{2} < 0
    \end{matrix}
  \)};
\end{tikzpicture}

    \caption{Неустойчивое}

  \end{subfigure}
  \qquad
  \begin{subfigure}[b]{0.3\textwidth}

    \begin{tikzpicture}[
  decoration = {
    markings,
    mark = between positions 0.1 and 0.4 step 0.1 with
      {\arrow[line width = 1pt]{stealth}},
    mark = between positions 0.6 and 0.9 step 0.1 with
      {\arrowreversed[line width = 1pt]{stealth}},
  }
]
    
  \draw[->] (-3, 0) -- (3, 0) node[right] {\(x\)};
  \draw[->] (0, -3) -- (0, 3) node[above] {\(y\)};

  \foreach \d in {-2.5, -1.5, -0.5, 0.5, 1.5, 2.5} {
    \draw[domain = -2 : 2, variable = \x, postaction = { decorate }]
      plot ({\x}, {0.4 * \d * \x});
  }

  \node at (-1, -3) {\(\begin{cases}
    x' = 0 \\
    y' = y
  \end{cases}\)};
  \node at (1.4, -3) {\(
    \begin{matrix}
      \lambda_{1, 2} \in \RR \\
      \lambda_{1} \lambda_{2} = 0 \\
      \lambda_{1} = 0, \lambda_{2} > 0
    \end{matrix}
  \)};
\end{tikzpicture}

    \caption{Вырожденный узел (устойчивое)}

  \end{subfigure}
  \qquad
  \begin{subfigure}[b]{0.3\textwidth}

    \begin{tikzpicture}[
  decoration = {
    markings,
    mark = between positions 0.1 and 0.9 step 0.1 with
      {\arrow[line width = 1pt]{stealth}}
  }
]
    
  \draw[->] (-3, 0) -- (3, 0) node[right] {\(x\)};
  \draw[->] (0, -3) -- (0, 3) node[above] {\(y\)};

  \foreach \d in {-1.5, -0.5, 0.5, 1.5, 2.5} {
    \draw[postaction = { decorate }] (-2.5, \d) -- (2.5, \d);
  }

  \node at (-1, -3) {\(\begin{cases}
    x' = y \\
    y' = 0
  \end{cases}\)};
  \node at (1.4, -3) {\(
    \begin{matrix}
      \lambda_{1} \neq \lambda_{2} \in \RR \\
      \lambda_{1} \lambda_{2} = 0 \\
      \lambda_{1} = 0, \lambda_{2} = 0
    \end{matrix}
  \)};
\end{tikzpicture}

    \caption{Вырожденное седло (неустойчивое)}

  \end{subfigure}
\end{figure}
