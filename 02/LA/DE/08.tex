\question{Линейные однородные дифференциальные уравнения (ЛОДУ) : определения, решение ЛОДУ\(_2\) с постоянными коэффициентами для случая различных вещественных корней характеристического уравнения.}

\begin{definition}
  Линейным дифференциальным уравнением \(n\)-ого порядка (ЛДУ\(_n\)) называется

  \begin{align*}
    a_{0}(x) y^{(n)}(x) + a_{1}(x) y^{(n - 1)}(x) + \dotsc + a_{n}(x) y(x)
    = f(x), \hspace{10pt} a_{0}(x) \neq 0
  \end{align*}
\end{definition}

\begin{definition}
  Разрешенным ЛДУ\(_n\) называется

  \begin{align*}
    y^{(n)}(x) + b_{1}(x) y^{(n - 1)}(x) + \dotsc + b_{n}(x) y(x) = f(x)
  \end{align*}
\end{definition}

\begin{definition}
  Если в ЛДУ\(_n\) \(\forall i \colon a_{i}(x) = p_{i} \in \RR\),
  то такое ЛДУ\(_n\) называется ЛДУ\(_n\) с постоянными коэффициентами.
  Оно имеет вид

  \begin{align*}
    y^{(n)}(x) + p_{1} y^{(n - 1)}(x) + \dotsc + p_{n} y(x) = f(x)
  \end{align*}
\end{definition}

\begin{definition}
  Линейным однородным дифференциальным уравнение \(n\)-ого порядка называется
  ЛДУ\(_n\) вида

  \begin{align*}
    a_{0}(x) y^{(n)}(x) + a_{1}(x) y^{(n - 1)}(x) + \dotsc + a_{n}(x) y(x) = 0,
  \end{align*}
\end{definition}

\begin{definition}
  Линейным неоднородным дифференциальным уравнение \(n\)-ого порядка называется
  ЛДУ\(_n\) вида

  \begin{align*}
    a_{0}(x) y^{(n)}(x) + a_{1}(x) y^{(n - 1)}(x) + \dotsc + a_{n}(x) y(x)
    = f(x), \hspace{10pt} f(x) \neq 0
  \end{align*}
\end{definition}

Рассмотрим ЛОДУ\(_2\) вида \(y'' + p y' + q y = 0\). Любой паре
\((p, q) \in \RR^{2}\) можно поставить в соответствие квадратное уравнение
\(k^2 + pk + q = 0\). По т. Виета \(p = -(k_{1} + k_{2}), q = k_{1} k_{2}\), где
\(k_{1}, k_{2}\) это корни уравнения. Подставим полученные выражения в исходное
ДУ:

\begin{align*}
  y'' - (k_{1} + k_{2}) y' + k_{1} k_{2} y = 0 \\
  y'' - k_{1} y' - k_{2} y' + k_{1} k_{2} y = 0 \\
  (y'' - k_{2} y') - k_{1} (y' - k_{2}y) = 0 \\
  \lets u(x) =  y' - k_{2} y \\
  u' - k_{1} u = 0
  \implies u(x) = c_1  e^{k_{1} x}
  \implies y' - k_{2} y = c_1  e^{k_{1} x}
\end{align*}

Сначала найдем частное решение соответствующего ЛОДУ\(_1\):
\(\overline{y} = c_{2} e^{k_{2} x}, y_{1} = e^{k_{2} x}\). Далее будем 
варьировать постоянную \(c_{2}\), тогда \(y(x) = C_{2}(x) e^{k_{2} x}\).
Подставим это в исходное ДУ:

\begin{align*}
  C_{2}'(x) e^{k_{2} x} + C_{2}(x) \cdot k_{2} \cdot e^{k_{2} x}
  - k_{2} \cdot C_{2}(x) e^{k_{2} x} = c_{1} e^{k_{1} x} \\
  C_{2}'(x) e^{k_{2} x} = c_{1} e^{k_{1} x} \\
\end{align*}

В итоге получаем уравнение

\begin{align*}\label{eq:lode2-cases}\tag{\(\bigstar\)}
  \boxed{C_{2}'(x) = c_{1} e^{(k_{1} - k_{2}) x}}
\end{align*}

Проанализируем это уравнение. Всего будет рассмотрено 3
случая: один в этом вопросе, остальные~--- в двух последующих.

\textbf{\eqref{eq:lode2-cases} случай I}:
\(k_{1} \neq k_{2}, k_{1}, k_{2} \in \RR\)

В заданных ограничениях имеем

\begin{align*}
  C_{2}'(x) = c_{1} e^{(k_{1} - k_{2}) x} \\
  C_{2}(x) = \frac{c_1}{k_{1} - k_{2}} e^{(k_{1} - k_{2}) x} + \tilde{c_{2}} \\
  y(x)
  = C_{2}(x) y_{1}(x)
  = \under{\frac{c_1}{k_{1} - k_{2}}}{\tilde{c_1}} e^{k_{1} x}
  + \tilde{c_2} e^{k_{2} x} \\
  y(x) = \tilde{c_{1}} e^{k_{1}x} + \tilde{c_{2}} e^{k_{2}x}
\end{align*}



