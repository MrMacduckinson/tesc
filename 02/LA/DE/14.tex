\question{Свойства решений ЛНДУ\(_2\) : теоремы о структуре общего решения и решении ДУ с суммой правых частей.}

\begin{theorem}
  Общее решение ЛНДУ\(_2\) представимо в виде суммы общего решения
  соответствующего ЛОДУ\(_2\) и некоторого частного решения ЛНДУ\(_2\).

  \begin{align*}
    y'' + p y' + q y = f(x) \\
    y = \overline{y} + y^{*} \\
    \Linear{\overline{y}} = 0, \Linear{y^{*}} = f(x)
  \end{align*}
\end{theorem}
\begin{proof}
  Сначала покажем, что \(y\) будет являться решением ДУ:

  \begin{align*}
    \Linear{y} = \Linear{\overline{y}} + \Linear{y^{*}} = 0 + f(x) = f(x)
  \end{align*}

  Показать, что это решение будет являться общим можно аналогично
  \ref{lode-gen}.
\end{proof}

\begin{lemma}
  Если правая часть ЛНДУ\(_2\) представлена суммой \(f_{1}(x) + f_{2}(x)\), то
  частное решение этого ЛНДУ\(_2\) будет суммой двух частных решений ЛНДУ\(_2\),
  в которых правая часть является каждым из слагаемых.

  \begin{align*}
    y'' + p y' + q y = f_{1}(x) + f_{2}(x) \\
    \Linear{y^{*}_{1}} = f_{1}(x), \Linear{y^{*}_{2}} = f_{2}(x) \\
    y^{*} = y^{*}_{1} + y^{*}_{2}
  \end{align*}
\end{lemma}
\begin{proof}
  \begin{align*}
    \Linear{y^{*}}
    = \Linear{y^{*}_{1}} + \Linear{y^{*}_{2}}
    = f_{1}(x) + f_{2}(x)
  \end{align*}
\end{proof}
