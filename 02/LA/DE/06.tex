\question{Теорема существования и единственности решения задачи Коши. Особые решения.}

\begin{theorem}\label{ex-un-Cp}
  О существовании и единственности решения задачи Коши.

  Пусть дана задача Коши

  \begin{align*}
    \begin{cases}
      y' = g(x, y) \\
      y_{0} = y(x_{0})
    \end{cases}
  \end{align*}

  и \(u(M_{0})\) -- окрестность точки \(M_{0}(x_{0}, y_{0})\). Тогда, если
  \(g\) непрерывна в \(u(M_{0})\), а \(g'_{y}\)~--- ограничена, то в окрестности
  \(u(M_{0})\) существует единственное решение задачи Коши.
\end{theorem}

\begin{definition}
  Особым решением ДУ называется решение, в каждой точке которого нарушается
  единственность.
\end{definition}

\begin{remark}
  Геометрически особое решение это интегральная кривая, через каждую точку
  которой проходит другая интегральная кривая.
\end{remark}

\begin{definition}
  Точка \(M(x, y) \in D\) (где \(D\) - область заполненная интегральными
  кривыми) называется обыкновенной, если через неё проходит ровно одна
  интегральная кривая.
\end{definition}

\begin{definition}
  Точка, не являющаяся обыкновенной, называется особой. Через неё может
  проходить несколько интегральных кривых, либо не проходить ни одной.
\end{definition}

\begin{lemma}
  Если ДУ задано в дифференциалах \(P \dd x + Q \dd y = 0\), то условие особой
  точки имеет вид \(P = 0\) или \(Q = 0\).
\end{lemma}
\begin{proof}
  ДУ в дифференциалах можно разрешить относительно каждой из переменных:

  \begin{align*}
    y' = - \frac{P}{Q} = g_{1}(x, y) \\
    x' = - \frac{Q}{P} = g_{2}(x, y)
  \end{align*}

  Далее можно применить теорему \ref{ex-un-Cp} о единственности к каждому
  из полученных уравнений. Непрерывность \(g_{1}, g_{2}\) нарушается при
  \(Q = 0\) и \(P = 0\) соответственно. Это и будет условием особой точки.
\end{proof}
