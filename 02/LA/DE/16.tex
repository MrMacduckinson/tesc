\question{Решение ЛНУ\(_2\) с постоянными коэффициентами: специальная правая часть, поиск частного решения методом неопределенных коэффициентов.}

\begin{definition}
  Специальной правой частью называется (СПЧ)

  \begin{align*}
    f(x) = e^{\alpha x} (P_{n}(x) \cos \beta x + Q_{m}(x) \sin \beta x)
  \end{align*}

  где \(\alpha, \beta, n, m\) некоторые коэффициенты.
\end{definition}

\textbf{Поиск частного решения методом неопределенных коэффициентов}

\underline{Идея}: пусть в ЛНДУ\(_n\) правая часть является специальной. Можно
предположить, что она была получена дифференцированием функции со схожей
структурой, поэтому будем искать частное решение ЛНДУ\(_n\) в виде

\begin{align*}
  y^{*} = x^{r} e^{\alpha x} (U_{l}(x) \cos \beta x + W_{l}(x) \sin \beta x),
  \; l = max(n, m)
\end{align*}

\underline{Алгоритм}:
\begin{enumerate}
  \item Составляем и решаем характеристическое уравнение.
  \item Извлекаем из СПЧ коэффициенты \(\alpha, \beta, n, m\).
  \item Считаем \(r\)~--- количество совпадений корней характеристического
  уравнения с \(\alpha \pm i \beta\). Совпадение комплексной пары считаем один
  раз.
  \item Составляем \(y^{*}\) с неопределенными коэффициентами: полиномы \(U, W\)
  степени \(l = max(n, m)\).
  \item Подставляем \(y^{*}\) в уравнение, находим неопределенные коэффициенты.
\end{enumerate}

\underline{Пример \#01}:
\begin{align*}
  y'' - 3 y' + 2y = 2 e^{3x} \\
  k^{2} - 3k + 2 = 0 \\
  k_{1} = 1, k_{2} = 2 \\
  2 e^{3x} \implies \alpha = 3, \beta = 0, n = 0, m = 0
\end{align*}

Число  \(\alpha + i \beta\) не совпадает с корнями характеристического
уравнения, поэтому будем искать частное решение в виде \(y^{*} = A e^{3x}\):

\begin{align*}
  (A e^{3x})'' - 3 (A e^{3x})' + 2 (A e^{3x}) = 2 e^{3x} \\
  9A e^{3x} - 9A e^{3x} + 2A e^{3x} = 2 e^{3x} \mid \colon e^{3x} \\
  9A - 9A + 2A = 2 \\
  A = 1
\end{align*}

Итого: частное решение будет равно \(y^{*} = e^{3x}\)

\underline{Пример \#02}:
\begin{align*}
  y'' - 3 y' + 2y = e^{x} \\
  k^{2} - 3k + 2 = 0 \\
  k_{1} = 1, k_{2} = 2 \\
  e^{x} \implies \alpha = 1, \beta = 0, n = 0, m = 0
\end{align*}

Корень характеристического уравнения \(k_{1}\) совпал с числом
\(\alpha + i \beta\), поэтому будем искать частное решение в виде
\(y^{*} = Ax e^{x}\):

\begin{align*}
  (Ax e^{x})'' - 3 (Ax e^{x})' + 2 (Ax e^{x}) = e^{x} \\
  e^{x} (2A + Ax) - 3 e^{x} (A + Ax) + 2Ax e^{3x} = 2 e^{3x}
    \mid \colon e^{x} \\
  2A + Ax - 3A - 3Ax + 2Ax = 2 \\
  A = -2
\end{align*}

Итого: частное решение будет равно \(y^{*} = -2 e^{x}\)

\begin{remark}
  Почему необходимо умножать на \(x^{r}\)? Если этого не делать, то полученное
  уравнение с неопределенными коэффициентами не будет иметь решений. Это
  происходит в тех случаях, когда выбранное частное решение совпадает с общим
  решением (именно поэтому мы смотрим на корни характеристического уравнения,
  потому что общее решение формируется на их основе). В этих случаях нарушается
  структура общего решения ЛНДУ\(_n\).
\end{remark}
