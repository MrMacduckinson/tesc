\question{Решение ЛНУ\(_2\) : метод вариации произвольных постоянных (Лагранжа).}

Метод универсален для правой части любого вида и даже для \(\Linear{y} = f(x)\)
с непостоянными коэффициентами.

Рассмотрим на примере:

\begin{align*}
  y'' - 3 y' + 2y = 2 e^{3x} \\
  k^{2} - 3k +  2 = 0 \\
  k_{1} = 1, k_{2} = 2 \\
  \overline{y}
  = c_{1} \under{e^{x}}{y_{1}}
  + c_{2} \under{e^{2x}}{y_{2}}
\end{align*}

Будем искать \(y(x)\) в виде \(y(x) = C_{1}(x) y_{1} + C_{2}(x) y_{2}(x)\).
Пусть \(C_{1}(x) = g(x) + c_{1}, C_{2}(x) = h(x) + c_{2}\), тогда:

\begin{align*}
  y(x) = (g(x) + c_{1}) y_{1} + (h(x) + c_{2}) y_{2} \\
  y(x)
    = \under{c_{1} y_{1} + c_{2} y_{2}}{\overline{y}}
    + \under{g(x) y_{1} + h(x) y_{2}}{y^{*}}
\end{align*}

В нашем примере получаем, что \(y^{*} = g(x) e^{x} + h(x) e^{2x}\). Заметим, что
функции \(g(x)\) и \(h(x)\) можно представить по-разному. Подберем их так, чтобы
выполнялось равенство

\begin{align*}\label{lhde-L-second}\tag{\(\blacktriangle\)}
  C'_{1}(x) y_{1} + C'_{2}(x) y_{2} = 0
\end{align*}

Вычислим производные \(y'(x)\) и \(y''(x)\):

\begin{align*}
  y(x) = C_{1}(x) y_{1} + C_{2}(x) y_{2}(x)
  \\
  y'(x)
    = \under{C'_{1}(x) y_{1} + C'_{2}(x) y_{2}}
      {\text{\ref{lhde-L-second}} = 0}
    + C_{1}(x) y'_{1} + C_{2}(x) y'_{2}
  \\
  y''(x)
    = C'_{1}(x) y'_{1} + C_{1}(x) y''_{1}
    + C'_{2}(x) y'_{2} + C_{2}(x) y''_{2}
\end{align*}

Вернемся к исходному ДУ и подставим в него все полученные равенства:

\begin{align*}
  \begin{matrix}
    y''(x) & = &
    C'_{1}(x) y'_{1} & + & C_{1}(x) y''_{1}
    & + & C'_{2}(x) y'_{2} & + & C_{2}(x) y''_{2}
    \\
    p y'(x) & = &&& p C_{1}(x) y'_{1} &&& + & p C_{2}(x) y'_{2}
    \\
    q y(x) & = &&& q C_{1}(x) y_{1} &&& + & q C_{2}(x) y_{2}
    \\
    f(x) & = & C'_{1}(x) y'_{1} & + & 0 & + & C'_{2}(x) y'_{2} & + & 0
  \end{matrix}
\end{align*}

Суммы во втором и четвертом столбцах обнуляются, т.к. если вынести из них
\(C_{1}(x)\) и \(C_{2}(x)\) соответственно, то в скобках останется ЛОДУ\(_2\),
а \(y_{1}, y_{2}\)~--- его корни.

Таким образом мы получили второе условие для системы (первое условие это
\ref{lhde-L-second}) для нахождения \(C_{1}(x)\) и \(C_{2}(x)\). Искомая
система будет иметь вид:

\begin{align*}
  \begin{cases}
    C'_{1}(x) y_{1} + C'_{2}(x) y_{2} = 0 \\
    C'_{1}(x) y'_{1} + C'_{2}(x) y'_{2} = f(x)
  \end{cases} \iff
  \begin{pmatrix}
    y_{1} & y_{2} \\
    y'_{1} & y'_{2} \\
  \end{pmatrix}
  \begin{pmatrix}
    C'_{1}(x) \\
    C'_{2}(x)
  \end{pmatrix}
  =
  \begin{pmatrix}
    0 \\
    f(x)
  \end{pmatrix}
\end{align*}

Подведем итог и обобщим алгоритм решения ЛНДУ\(_n\):
\begin{enumerate}
  \item Решаем соответствующее ЛОДУ\(_n\), получаем набор корней
  \(y_{1}, \dots, y_{n}\).

  \item Составляем СЛАУ следующего вида:
  
  \begin{align*}
    \begin{pmatrix}
      y_{1} & \dotsc & y_{n} \\
     \vdots & \ddots & \vdots \\
      y^{(n - 1)}_{1} & \dotsc & y^{(n - 1)}_{n} \\
    \end{pmatrix}
    \begin{pmatrix}
      C'_{1}(x) \\
      \vdots \\
      C'_{n}(x)
    \end{pmatrix}
    =
    \begin{pmatrix}
      0 \\
      \vdots \\
      f(x)
    \end{pmatrix}
  \end{align*}

  \item Решаем её и находим производные варьируемых функций. Интегрируем их
  (не забывая про константу).

  \item Общее решение ЛНДУ\(_n\) будет иметь вид
  \(y = C_{1}(x) y_{1} + \dotsc + C_{n}(x) y_{n}\)
\end{enumerate}
