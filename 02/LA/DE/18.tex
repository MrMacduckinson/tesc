\question{Системы дифференциальных уравнений: определения, решение методом исключения.}

\begin{definition}
  Пусть \(y_{1}, \dots, y_{n}\)~--- функции от \(x\), дифференцируемые \(m\)
  раз. Тогда система

  \begin{align*}
    \begin{cases}
      f_{1}(x,
        y_{1}, \dotsc, y_{n},
        y'_{1}, \dotsc, y'_{n},
        \dotsc,
        y_{1}^{(m - 1)}, \dotsc, y_{n}^{(m - 1)}
      ) = 0 \\
      \dotsc \\
      f_{n} (\cdots) = 0
    \end{cases}
  \end{align*}

  называется системой дифференциальных уравнений (СДУ).
\end{definition}

\begin{definition}
  СДУ называется \textit{нормальной}, если все её уравнения разрешены
  относительно старшей производной и при этом правые части не содержат
  производных.
\end{definition}

\begin{definition}
  Нормальная СДУ называется автономной, если функции в правой части каждого из
  её уравнений не зависят явно от независимой переменной.
\end{definition}

\begin{remark}\label{de-to-sde}
  С помощью введения новых переменных ДУ
  \(y^{(n)} = f(x, y, y', \dotsc, y^{(n - 1)})\) можно свести к системе ДУ
  следующего вида

  \begin{align*}
    \begin{cases}
      y = y_{1} \\
      y' = y_{2} \\
      \dots \\
      y^{(n - 1)} = y_{n} \\
      y^{(n)} = f(x, y_{1}, \dotsc, y_{n})
    \end{cases}
  \end{align*}

  где \(y_{1}, \dotsc, y_{n}\)~--- новые переменные.
\end{remark}

\begin{definition}
  Порядком системы называется сумма порядков старших производных каждой из
  переменных системы. Порядок системы равен порядку ДУ, соответствующему ей.
\end{definition}

\textbf{Решение СДУ методом исключения}:

Пусть дана следующая СДУ:

\begin{align*}
  \begin{cases}
    \frac{\dd y_{1}}{\dd x} = f_{1}(x, y_{1}, \dotsc, y_{n}) \\
    \frac{\dd y_{2}}{\dd x} = f_{2}(x, y_{1}, \dotsc, y_{n}) \\ 
    \dots, \\
    \frac{\dd y_{n}}{\dd x} = f_{n}(x, y_{1}, \dotsc, y_{n})
  \end{cases}  
\end{align*}

Обозначим \(f_{1}(x, y_{1}, \dotsc, y_{n}) = F_{1}(x, y_{1}, \dotsc, y_{n})\).
Дифференцируем первое уравнение по \(x\), получим:

\begin{align*}
  \frac{\dd^{2} y_{1}}{\dd x^{2}}
    = \frac{\partial f_{1}}{\partial x} 
    + \frac{\partial f_{1}}{\partial y_{1}} \cdot 
     \under{\frac{\dd y_{1}}{\dd x} }{f_{1}}
    + \dotsc
    + \frac{\partial f_{1}}{\partial y_{n}} \cdot 
     \under{\frac{\dd y_{n}}{\dd x} }{f_{n}}
    = F_{2}(x, y_{1}, \dotsc, y_{n})
\end{align*}

Полученное выражение можно продифференцировать еще раз. Подставляя производные
из изначального СДУ можно получить аналогичные функции вплоть до
\(F_{n}(x, y_{1}, \dotsc, y_{n})\). Итого получится следующая система:

\begin{align*}
  \begin{cases}
    \frac{\dd y_{1}}{\dd x} = F_{1}(x, y_{1}, \dotsc, y_{n}) \\
    \frac{\dd^{2} y_{1}}{\dd x^{2}} = F_{2}(x, y_{1}, \dotsc, y_{n}) \\ 
    \dots, \\
    \frac{\dd^{n} y_{1}}{\dd x^{n}} = F_{n}(x, y_{1}, \dotsc, y_{n})
  \end{cases}  
\end{align*}

Как видно из \ref{de-to-sde} полученный вид системы свидетельствует о том, что
её можно свести к равносильному ДУ \(\phi(x, y_{1}, \dotsc, y_{1}^{(n)})\).

\underline{Пример}:

\begin{align*}
  \begin{cases}
    y' = y + 5x \\
    x' = -y - 3x
  \end{cases} \iff
  \begin{cases}
    y'' = y' + 5x' \\
    x' = -y - 3x
  \end{cases} \iff
  \begin{cases}
    y'' = y' + 5 (-y - 3x) \\
    x' = -y - 3x
  \end{cases} \iff
  \begin{cases}
    y'' = y' - 5y + 15x \\
    x' = -y - 3x
  \end{cases}
\end{align*}

Выразим \(x\) из первого уравнения изначальной системы и подставим его в первое
уравнение полученной системы:

\begin{align*}
  \begin{cases}
    y' = y + 5x \implies x = \frac{1}{5} (y' - y) \\
    y'' = y' - 5y + 15x 
  \end{cases}
  \implies y'' = y' - 5y + 3y' - 3y 
  \implies y'' - 4y' + 8y = 0
\end{align*}

Из полученного ЛОДУ\(_2\) можно найти \(y\), после чего подставить его в СДУ и
найти \(x\).

\begin{remark}
  Линейная СДУ сводится к ЛОДУ, т.к. дифференцирование и исключение линейны.
  Аналогично СДУ с постоянными коэффициентами сводится к ДУ с постоянными
  коэффициентами.
\end{remark}
