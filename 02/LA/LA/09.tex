\question{Сопряженный и самосопряженный операторы в вещественном евклидовом пространстве: определения, основные свойства. Свойства собственных чисел и собственных векторов самосопряженного оператора.}

\begin{definition}\label{conj-lo-1}
  Рассмотрим оператор \(\opA \colon E^{n}_{\RR} \to E^{n}_{\RR}\). Оператор
  \(\opA^{*}\) называется сопряженным оператором для \(\opA\), если
  
  \begin{align*}
    (\opA x, y) = (x, \opA^{*} y)
  \end{align*}
\end{definition}

\begin{definition}\label{conj-lo-2}
  Альтернативное определение: оператор \(\opA^{*}\) называется сопряженным
  оператором для \(\opA\), если в любом ортонормированном базисе
  \(A^{*} = A^{T}\).
\end{definition}

\begin{theorem}
  Равносильность определений \ref{conj-lo-1} и \ref{conj-lo-2} сопряженного
  оператора.
\end{theorem}
\begin{proof}
  Выберем произвольный ортонормированный базис \(\Basis\). В нем векторам \(x\)
  и \(y\) соответствуют координатные столбцы \(X\) и \(Y\).

  Скалярное произведение \((\opA x, y)\) можно записать в виде
  \((A X)^{T} Y\), т.к. мы работаем в ортонормированном базисе. Преобразуем
  это выражение:

  \begin{align*}
    (A X)^{T} Y = X^{T} A^{T} Y = X^{T} A^{*} Y \implies (x, \opA^{*} y)
  \end{align*}
\end{proof}

Некоторые базовые свойства сопряженного оператора:
\begin{enumerate}
  \item \(I^{*} = I\): \((Ix, y) = (x, y) = (x, Iy)\)
  \item \((\opA + \opB)^{*} = \opA^{*} + \opB^{*}\)
  \item \((\lambda \opA)^{*} = \overline{\lambda} \opA^{*}\)
  \item \((\opA^{*})^{*} = \opA\)
  \item \((\opA \cdot \opB)^{*} = \opB^{*} \cdot \opA^{*}\)
  \item Для любого оператора существует единственный сопряженный оператор.
\end{enumerate}

\begin{definition}
  Самосопряженный оператор это оператор, который равен своему сопряженному.

  \begin{align*}
    \opA = \opA^{*}
  \end{align*}
\end{definition}

\begin{corollary}
  Матрица самосопряженного оператора симметрическая: \(A = A^{T} = A^{*}\)
\end{corollary}

Далее рассмотрим некоторые свойства самосопряженного оператора.

\begin{lemma}
  Собственные числа самосопряженного оператора  всегда вещественные.
\end{lemma}
\begin{proof}
  Пусть \(\opA\)~--- самосопряженный оператор. Рассмотрим собственное число
  \(\lambda\) и собственный вектор \(x\), соответствующий ему:

  \begin{align*}
    (\opA x, x) \in \RR \\
    (\opA x, x) = (\lambda x, x) = \lambda \norm{x}^{2} \\
    \begin{cases}
      \lambda \norm{x}^{2} \in \RR \\
      \norm{x}^{2} \in \RR
    \end{cases} \implies \lambda \in \RR
  \end{align*}
\end{proof}

\begin{lemma}
  Собственные векторы самосопряженного оператора, соответствующие различным
  собственным числам, ортогональны.
\end{lemma}
\begin{proof}
  Пусть \(\opA\)~--- самосопряженный оператор. Рассмотрим два собственных числа
  \(\lambda_{1} \neq \lambda_{2}\) и собственные векторы \(x_{1}, x_{2}\),
  соответствующие им:

  \begin{align*}
    \begin{cases}
      (\opA x_{1}, x_{2}) = (x_{1}, \opA x_{2}) \\
      (\opA x_{1}, x_{2}) = \lambda_{1} (x_{1}, x_{2}) \\
      (x_{1}, \opA x_{2}) = \lambda_{2} (x_{1}, x_{2})
    \end{cases}
    \implies \lambda_{1} (x_{1}, x_{2}) = \lambda_{2} (x_{1}, x_{2})
    \implies (\lambda_{1} - \lambda_{2}) (x_{1}, x_{2}) = 0
  \end{align*}

  Т.к. \(\lambda_{1} \neq \lambda_{2}\), то первая скобка не может
  равняться нулю, значит \((x_{1}, x_{2}) = 0 \implies x_{1} \bot x_{2}\).
\end{proof}

\begin{remark}
  Если \(\opA\) и \(\opB\) это самосопряженные операторы, то \(\opA \cdot \opB\)
  не обязательно самосопряженный оператор. Чтобы
  \(\opA \cdot \opB = (\opA \cdot \opB)^{*}\)
  необходимо, чтобы \(\opA\) и \(\opB\) коммутировали.
\end{remark}

\begin{lemma}\label{sconj-subspace-inv}
  Пусть дан самосопряженный оператор \(\opA \colon V^{n} \to V^{n}\), а
  \(\basis\)~--- его собственный вектор. Тогда подпространство
  \(V_{1} = \{ x \in V^{n} \mid x \bot \basis \}\) инвариантно относительно
  оператора \(\opA\) и его размерность равна \(n - 1\).
\end{lemma}
\begin{proof}
  Собственный вектор \(\basis\) это линейная оболочка некоторого \(\basis_{f}\),
  т.е. это подпространство \(V_{n}\). Если \(x \in V_{1}\), то
  \(x \bot \basis \implies x \bot \basis_{f}\), таким образом \(V_{1}\) это
  ортогональное дополнение \(\basis_{f}\).

  \(\dim \basis = 1\), т.к. это линейная оболочка одного вектора \(\basis_{f}\).
  Значит \(\dim V_{1} = n - 1\).

  Теперь докажем, что это пространство инвариантно:

  \begin{align*}
    x \bot \basis \implies (x, \basis) = 0 \\
    (\opA x, \basis) =
    (x, \opA \basis) =
    (x, \lambda \basis) =
    \lambda (x, \basis) = 0
    \implies \opA x \bot \basis
  \end{align*}

  Таким образом \(\opA x \in V_{1}\) по определению \(V_{1}\).
\end{proof}

\begin{theorem}
  У любого самосопряженного оператора есть ортонормированный базис, состоящий из
  собственных векторов.
\end{theorem}
\begin{proof}
  Возьмем одно произвольное собственное число \(\lambda\), ему будет
  соответствовать собственный вектор \(\basis_{1}\), который мы и возьмем в
  базис. Далее пользуясь \ref{sconj-subspace-inv} рассмотрим подпространство
  \(V_{1}\), причем \(\basis_{1}\) будет ему ортогонален. Проделаем
  с этим подпространством аналогичную операцию.
  
  Повторим это \(n\) раз, после чего нормируем все полученные векторы
  \(\implies\) получим ортонормированный базис.
\end{proof}
