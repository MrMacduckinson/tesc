\question{Обратный оператор. Взаимно-однозначный оператор. Ядро и образ оператора. Теорема о размерностях.}

\begin{definition}
  Оператор \(\opI \colon V^{n} \to V^{n}\) называется тождественным оператором,
  если \(\forall x \in V^{n} \colon \opI x = x\).
\end{definition}

\begin{definition}
  Пусть даны операторы \(\opA, \opB \colon V^{n} \to V^{n}\). Оператор \(\opB\)
  называется обратным для оператора \(\opA\), если их композиция равна
  тождественному оператору.

  \begin{align*}
    \opB = \opA^{-1} \bydef \opA \cdot \opB = \opB \cdot \opA = \opI
  \end{align*}
\end{definition}

\begin{definition}
  Оператор \(\opA \colon V^{n} \to V^{n}\) называется взаимно-однозначным, если
  разным \(x \in V^{n}\) сопоставляются разные \(y \in V^{n}\).
  
  \begin{align*}
    \forall x, y \in V^{n} \colon x \neq y \implies \opA x \neq \opA y
  \end{align*}
\end{definition}

\begin{lemma}\label{bij-lo-lm}
  Если оператор \(\opA \colon V^{n} \to V^{n}\) взаимно-однозначный, то
  \(\opA x = 0 \implies x = 0\).
\end{lemma}
\begin{proof}
  От противного:

  \begin{align*}
    \lets x = x_{1} - x_{2} \neq 0 \implies x_{1} \neq x_{2} \\
    \opA x = \opA (x_{1} - x_{2}) = \opA x_{1} - \opA x_{2} = 0
    \implies \opA x_{1} = \opA x_{2}
  \end{align*}

  Противоречие, т.к. \(\opA\) взаимно-однозначный.
\end{proof}

\begin{theorem}\label{bij-lo-ind-to-ind}
  Взаимно-однозначный оператор переводит линейно-независимый набор в
  линейно-независимый набор.
\end{theorem}
\begin{proof}
  Пусть дан взаимно-однозначный оператор \(\opA \colon V^{n} \to V^{n}\) и
  линейно-независимый набор \(\{ x_{1}, \dotsc, x_{n}\}\). Построим
  набор образов \(\{ \opA x_{1}, \dotsc, \opA x_{n}\}\). Составим его нулевую
  линейную комбинацию, после чего воспользуемся линейностью оператора:

  \begin{align*}
    \lambda_{1} \opA x_{1} + \dotsc + \lambda_{n} \opA x_{n} = 0 \\
    \opA \Big( \lambda_{1} x_{1} + \dotsc + \lambda_{n} x_{n} \Big) = 0 \\
  \end{align*}

  Т.к. \(\opA\) взаимно-однозначный, то по \ref{bij-lo-lm} получаем, что
  \(\lambda_{1} x_{1} + \dotsc + \lambda_{n} x_{n} = 0\).
  Набор \(\{ x_{1}, \dotsc, x_{n}\}\) линейно независим, значит
  \(\forall \lambda_{i} = 0\).
\end{proof}

\begin{corollary}
  Взаимно-однозначный оператор переводит базис в базис.
\end{corollary}

\begin{theorem}
  Оператор \(\opA \colon V^{n} \to V^{n}\) взаимно-однозначный
  \(\iff \exists \opA^{-1}\).
\end{theorem}
\begin{proof}
  \(\implies\) Пусть \(x \xrightarrow{\opA} y\).
  Рассмотрим оператор \(\opB\) такой, что \(y \xrightarrow{\opB} x\). Т.к.
  \(\opA\) взаимно-однозначный, то \(\opA \cdot \opB = I\).

  \(\impliedby\) От противного

  \begin{align*}
    \lets x_{1} \neq x_{2}, \opA x_{1} = \opA x_{2}\\
    \opA x_{1} = \opA x_{2}
    \implies \opA^{-1} \opA x_{1} = \opA^{-1} \opA x_{2}
    \implies x_{1} = x_{2}
  \end{align*}

  Получили противоречие.
\end{proof}

\begin{definition}
  Пусть дан линейный оператор \(\opA \colon V^{n} \to W^{m}\).
  Множество \(\Ker \opA = \{ x \in V^{n} \mid \opA x = 0\}\)
  называется ядром оператора \(\opA\).
\end{definition}

\begin{lemma}\label{lo-bij-ker}
  Оператор \(\opA \colon V^{n} \to V^{n}\) взаимно-однозначный
  \(\implies \Ker \opA = \{ 0 \}\).  
\end{lemma}
\begin{proof}
  От противного, пусть \(x \neq 0 \in \Ker \opA\).
  Тогда \(\opA x = 0\), но в то же время \(\opA \; 0 = 0\).
  Нарушается взаимно-однозначность.
\end{proof}

\begin{definition}
  Пусть дан линейный оператор \(\opA \colon V^{n} \to W^{m}\).
  Множество
  \(\Img \opA = \{ y \in W^{m} \mid \exists x \in V^{n} \colon y = \opA x\}\)
  называется образом оператора \(\opA\).
\end{definition}

\begin{theorem}\label{lo-sum-dim}
  Пусть дан оператор \(\opA \colon V^{n} \to V^{n}\). Тогда

  \begin{align*}
    \dim{\Ker \opA} + \dim{\Img \opA} = n
  \end{align*}
\end{theorem}
\begin{proof}
  Т.к. \(\Ker{\opA}\) и \(\Img{\opA}\) это подпространства \(V^{n}\), то
  \(\exists W \subset V^{n} \mid W \oplus \Ker{A} = V^{n}\). Тогда
  \(\dim{W} + \dim{\Ker A} = n\). Требуется доказать, что
  \(\dim{W} = \dim{\Img \opA}\).

  Сначала покажем, что \(\opA \colon W \to \Img \opA\) взаимно-однозначный.
  От противного: 

  \begin{align*}
    \lets x_{1} \neq x_{2} \in W \colon \opA x_{1} = \opA x_{2} \\
    \opA x_{1} = \opA x_{2}
      \implies \opA (x_{1} - x_{2}) = 0
      \implies (x_{1} - x_{2}) \in \Ker \opA
    \\  
    x_{1}, x_{2} \in W \implies (x_{1} - x_{2}) \in W
  \end{align*}

  Но это невозможно, т.к.
  \(W \oplus \Ker{A} = V^{n} \implies W \cap \Ker A = \varnothing\).

  В \(\Img \opA\) выделим базис \( \{ y_{1}, \dotsc, y_{k} \}\). Т.к. \(\opA\)
  взаимно-однозначный, то выделенный базис порождается линейно-независимым
  набором \(\{ x_{1}, \dotsc, x_{k} \}, x_{i} \in W\)
  (по \ref{bij-lo-ind-to-ind}).
  Значит \(\dim{W} \ge \dim{\Img \opA}\).

  Предположим, что \(\dim{W} > \dim{\Img \opA}\). Обозначим \(\dim{W} = p\),
  дополним систему \(\{ x_{1}, \dotsc, x_{k} \}\) до \(p\) линейно-независимых
  векторов. Т.к. оператор \(\opA \colon W \to \Img \opA\) взаимно-однозначный,
  то он должен перевести полученную линейно-независимую систему в
  линейно-независимую. Однако это невозможно, т.к. \(\Img \opA\) имеет базис
  меньшей размерности.
\end{proof}

\begin{remark}
  Можно доказать, что

  \begin{align*}
    \begin{cases}
      V_{1} \subset V^{n} \\
      V_{2} \subset V^{n} \\
      \dim{V_{1}} + \dim{V_{2}} = n
    \end{cases} \implies
    \exists \opA \colon V^{n} \to V^{n},
    \Ker \opA = V_{1},
    \Img \opA = V_{2}
  \end{align*}
\end{remark}

\begin{definition}
  Рангом оператора \(\opA \colon V^{n} \to V^{n}\) называется размерность его
  образа:

  \begin{align*}
    \Rang \opA = \dim{\Img \opA}
  \end{align*}
\end{definition}

Пусть \(\opA, \opB \colon V^{n} \to V^{n}\). Рассмотрим некоторые свойства
ранга линейного оператора:
\begin{enumerate}
  \item Если оператор \(\opA\) взаимно-однозначный, то
  \(\Rang \opA = n\) (это следствие из \ref{lo-bij-ker} и \ref{lo-sum-dim}).
  \item
    \(\Rang (\opA \cdot \opB) \le \Rang \opA\),
    \(\Rang (\opA \cdot \opB) \le \Rang \opB\)
  \item \(\Rang (\opA \cdot \opB) = \Rang \opA + \Rang \opB - \dim{V}\)
\end{enumerate}