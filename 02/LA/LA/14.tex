\question{Знакоопределенность квадратичной формы: необходимые и достаточные условия. Критерий Сильвестра.}

\begin{definition}
  Квадратичная форма \(\Quad(u)\) называется положительно определенной, если

  \(\forall u \neq 0 \colon \Quad(u) > 0\).
\end{definition}

\begin{definition}
  Квадратичная форма \(\Quad(u)\) называется отрицательно определенной, если

  \(\forall u \neq 0 \colon \Quad(u) < 0\).
\end{definition}

\begin{definition}
  Квадратичная форма \(\Quad(u)\) называется знакопеременной, если 

  \(\exists u \colon \Quad(u) < 0\) и \(\exists v \colon \Quad(v) > 0\).
\end{definition}

\begin{definition}
  Квадратичная форма \(\Quad(u)\) называется квазиположительноопределенной, если 

  \(\exists \widehat{u} \neq 0 \colon \Quad(\widehat{u}) = 0\) и
  \(\forall u \neq \widehat{u} \colon \Quad(u) > 0\).
\end{definition}

\begin{definition}
  Квадратичная форма \(\Quad(u)\) называется квазиотрицательноопределенной, если 

  \(\exists \widehat{u} \neq 0 \colon \Quad(\widehat{u}) = 0\) и
  \(\forall u \neq \widehat{u} \colon \Quad(u) < 0\).
\end{definition}

\begin{remark}
  Если квазиопределенная квадратичная форма порождена симметричной билинейной
  формой, то эта билинейная форма является скалярным произведением.
\end{remark}

Пусть дана квадратичная форма \(\Quad^{0}(x)\) в нормальной форме, тогда её
можно записать в виде

\begin{align*}
  \Quad^{0}(x)
  = (x_{1}^2 + \dotsc + x_{p}^2)
  - (x_{p + 1}^2 + \dotsc + x_{k}^2)
  + 0 \cdot (x_{k}^2 + \dotsc + x_{n}^2)
\end{align*}

где в первой скобке находятся квадраты, у которых в нормальной форме был
коэффициент \(1\), а во второй~--- квадраты с коэффициентом \(-1\). Квадраты с
нулевым коэффициентом записываем в третью скобку.

Обозначим \(p\)~--- количество квадратов в первой скобке, а \(q\)~--- во второй.

\begin{definition}
  \(p\) и \(q\) называются положительным и отрицательным индексами инерции
  квадратичной формы. Они являются инвариантами.
\end{definition}

\begin{theorem}\label{quad-sd}
  Необходимое и достаточное условие знакоопределенности квадратичной формы

  \(\Quad(u)\) знакоопределенная тогда и только тогда, когда один её индексов
  инерции равен \(n\).
\end{theorem}
\begin{proof}
  Рассмотрим случай, когда \(p = n\), в этом случае квадратичная форма будет
  положительно определенной. Случай с \(q = n\) и отрицательно определенной
  квадратичной формой рассматривается аналогично.

  \(\implies\) От противного

  \begin{align*}
    \lets p < n
    \implies
    \exists \widehat{u} = (
      \under{u_{1}, \dotsc, u_{p}}{0},
      \under{u_{p + 1}, \dotsc, u_{n}}{\neq 0},
    )
    \implies
    \begin{cases}
      \widehat{u} \neq 0 \\
      \Quad(\widehat{u}) \le 0
    \end{cases}
  \end{align*}

  Получаем противоречие.

  \(\impliedby\) Если \(p = n\), то квадратичная форма имеет следующий
  нормальный вид:
  
  \begin{align*}
    \Quad^{0}(u) = u_{1}^2 + \dotsc + u_{n}^2
  \end{align*}

  Очевидно, что \(\forall u \neq 0\) она будет больше нуля, а для
  \(u = 0\) она будет равна нулю. Таким образом она положительно определенная по
  определению.
\end{proof}

\begin{theorem}\label{quad-sc}
  \(\Quad(u)\) знакоопеременная тогда и только тогда, когда оба её индекса
  инерции больше нуля.
\end{theorem}

\begin{theorem}\label{quad-sdz}
  \(\Quad(u)\) квазизнакоопределенная тогда и только тогда, когда один из её
  индексов инерции равен нулю, а второй меньше \(n\).
\end{theorem}

\begin{remark}
  Теоремы \ref{quad-sc} и \ref{quad-sdz} доказываются аналогично теореме
  \ref{quad-sd}.
\end{remark}

\begin{definition}
  Угловой минор \(\Delta_{i}\) это минор квадратной подматрицы образованной
  первыми \(i\) столбцами и строками.
\end{definition}

\begin{theorem}
  Критерий Сильвестра

  Квадратичная форма положительно определенная тогда и только тогда, когда все
  угловые миноры её матрицы положительны.

  Квадратичная форма отрицательно определенная тогда и только тогда, когда
  первый угловой минор её матрицы отрицательный, а остальные угловые миноры
  последовательно чередуют свой знак.
\end{theorem}
\begin{proof}
  Докажем первое утверждение, второе доказывается аналогично.

  \(\implies\) Сначала покажем, что среди угловых миноров нет нулевого. От
  противного: пусть \(\Delta_{k} = 0\), тогда рассмотрим квадратичную форму
  \(\Quad_{k}\), которая образована угловой матрицей размера \(k\):

  \begin{align*}
    \begin{pmatrix}
      b_{1,1} & \dots  & b_{1,k} \\
      \vdots  & \ddots & \vdots \\
      b_{k,1} & \dots  & b_{k,k} \\
    \end{pmatrix}
    \widetilde{u}
    = \vec{0}
  \end{align*}

  Т.к. \(\det_{k} = 0\), то построенное уравнение имеет нетривиальное решение
  \(\widetilde{u}\). Далее составим вектор \(\widehat{u}\) такой, что первые его
  \(k\) компонент равны компонентам вектора \(\widetilde{u}\), а остальные~---
  нулю. Тогда \(\widehat{u} \neq 0\), но \(\Quad(\widehat{u}) = 0\) (первые
  \(k\) компонент дают ноль, т.к. они взяты из \(\widetilde{u}\), а остальные
  компоненты дают ноль, т.к. они нулевые). Получаем противоречие, ведь
  квадратичная форма должна быть положительно определенной.

  Таким образом ни один из угловых миноров не равен нулю, значит можно применить
  метод Якоби. Т.к. квадратичная форма положительно определенная, то все её
  собственные числа положительны, итого:

  \begin{align*}
    \forall \lambda_{i} = \frac{\Delta_{i}}{\Delta_{i - 1}} > 0
    \implies \forall \Delta_{i} > 0
  \end{align*}

  \(\impliedby\) Применим метод Якоби

  \begin{align*}
    \forall \Delta_{i} > 0
    \implies \forall \lambda_{i} = \frac{\Delta_{i}}{\Delta_{i - 1}} > 0
    \implies \begin{cases}
      \Quad^{d} = \lambda_{1} u_{1}^2 + \dotsc + \lambda_{n} u_{n}^2 \\
      \forall \lambda_{i} > 0
    \end{cases}
  \end{align*}

  Легко заметить, что полученная квадратичная форма положительно определенная.
\end{proof}