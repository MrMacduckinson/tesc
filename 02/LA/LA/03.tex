\question{Ортогональность вектора подпространству. Ортогональное дополнение. Теорема Пифагора.}

\begin{definition}\label{ort-subspace}
  Пусть дано Евклидово пространство \(E^{n}\).
  Элемент \(h \in E^{n}\) называется ортогональным (перпендикулярным)
  подпространству \(G \subset E^{n}\), если

  \begin{align*}
    \forall x \in G \colon h \perp x \bydef h \perp G
  \end{align*}
\end{definition}

\begin{corollary}\label{ort-subspace-by-basis}
  Выделим в подпространстве \(G\) базис
  \(\Basis = \{\basis_{1}, \dotsc, \basis_{k} \}\). Тогда

  \begin{align*}
    \forall \basis_{i} \in \Basis \colon h \perp e_{i}
    \implies h \bot G
  \end{align*}
\end{corollary}
\begin{proof}
  Выберем произвольный элемент \(x \in G\), разложим его по базису \(\Basis\),
  после чего воспользуемся свойствами скалярного произведения:

  \begin{align*}
    x = \sum_{i = 1}^{k} \lambda_{i} \basis_{i}
    \implies
    (h, x)
    = \sum_{i = 1}^{k} \lambda_{i} \under{(h, \basis_{i})}{\text{= 0}}
    = 0 
    \implies[\ref{item-ort}] \forall x \in G \colon h \perp x
    \idenby{\ref{ort-subspace}} h \perp G
  \end{align*}
\end{proof}

\begin{definition}\label{ort-compl}
  Пусть дано Евклидово пространство \(E^{n}\).
  Ортогональным дополнением \(F^{\perp}\) к подпространству
  \(F \subset E^{n}\) называется совокупность векторов \(h \perp F\).
\end{definition}

\begin{remark}
  Из определения \ref{ort-subspace} следует, что \(F\) также является
  подпространством \(E^{n}\).
\end{remark}

\begin{theorem}
  Евклидово пространство \(E^{n}\) является прямой суммой подпространства
  \(F \subset E^{n}\) и его ортогонального дополнения \(F^{\perp}\).
  
  \begin{align*}
    E^{n} = F \oplus F^{\bot}
  \end{align*}
\end{theorem}
\begin{proof}
  В \(E^{n}\) выделим ортогональный базис \(\Basis\) (по \ref{ort-basic}), после
  чего разложим произвольный \(x \in E^{n}\) по этому базису:

  \begin{align*}
    \Basis = \{
      \under{\basis_{1}, \dots, \basis_{k}}{\text{Базис } F},
      \basis_{k + 1}, \dots \basis_{n}
    \}
    \implies x
    = \under{x_{1} \basis_{1} + \dots + x_{k} \basis_{k}}{\hat{x}}
    + \under{x_{k + 1} \basis_{k + 1} + \dots + x_{n} \basis_{n}}{\bar{x}}
    = \hat{x} + \bar{x}
  \end{align*}

  Т.к. базис \(\Basis\) ортогональный, то вектор \(\basis_{k + 1}\) ортогонален
  каждому из векторов базиса \(F\)
  \(\implies[\ref{ort-subspace-by-basis}] \basis_{k + 1} \perp F\).
  Аналогично остальные векторы, не входящие в базис \(F\), также \(\perp F\).
  Значит \(\bar{x} \perp F\) как их линейная комбинация. Множество \(\bar{x}\)
  и составляет \(F^{\perp}\) по определению ортогонального дополнения
  (\ref{ort-compl}).
\end{proof}

\begin{theorem}\label{pythagoras}
  Теорема Пифагора

  \begin{align*}
    \forall x, y \in E^{n} \colon
      x \perp y \implies \norm{x + y}^2 = \norm{x}^2 + \norm{y}^2
  \end{align*}
\end{theorem}
\begin{proof}

  Раскроем норму как корень из скалярного произведения элемента на себя, после
  чего воспользуемся свойствами скалярного произведения:

  \begin{align*}
    \norm{x + y}^2
    = (x + y, x + y)
    = (x, x) + 2 (x, y) + (y, y)
    \eqby{\(x \perp y \implies (x, y) = 0\)}
    (x, x) + (y, y)
    = \norm{x}^2 + \norm{y}^2
  \end{align*}
\end{proof}
