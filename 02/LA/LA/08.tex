\question{Собственные числа и собственные векторы оператора. Теоремы о диагональной матрице оператора.}

\begin{definition}
  Пусть дан оператор \(\opA \colon V^{n} \to V^{n}\) с матрицей \(A\) в
  некотором базисе \(\Basis = \{ \basis_i \}_{i = 1}^{n}\). Тогда многочлен

  \begin{align*}
    \det (A - \lambda E) 
  \end{align*}

  относительно \(\lambda \in \RR\) называется характеристическим многочленом.
\end{definition}

\begin{definition}
  Пусть дан оператор \(\opA \colon V^{n} \to V^{n}\). Подпространство
  \(U \subseteq V^{n}\) называется \textit{инвариантным}, если

  \begin{align*}
    \forall x \in U \colon \opA x \in U
  \end{align*}
\end{definition}

\begin{definition}
  Пусть дан оператор \(\opA \colon V^{n} \to V^{n}\) с матрицей \(A\) в
  некотором базисе \(\Basis = \{ \basis_i \}_{i = 1}^{n}\).
  
  \(x \neq 0 \in V^{n}\) называется собственным вектором для оператора \(\opA\),
  если

  \begin{align*}
    \exists \lambda \in \CC \colon \opA x = \lambda x
  \end{align*}

  Тогда \(\lambda\) называется собственным числом (собственным значением)
  оператора \(\opA\).
\end{definition}

\begin{theorem}\label{lo-ei-roots}
  Собственные числа оператора являются корнями характеристического многочлена
  \(\det(A - \lambda E)\).
\end{theorem}
\begin{proof}
  \(\implies\) Пусть \(\lambda\) -- собственное число, тогда
  
  \begin{align*}
    \exists x \neq 0 \in V^{n} \mid \opA x = \lambda x \\
    \opA x = \lambda x
    \implies A x = (\lambda E) x
    \implies (A - \lambda E) x = 0
  \end{align*}

  Теперь рассмотрим оператор
  \(\opB \colon V^{n} \to V^{n}, \opB = \opA - \lambda I\):

  \begin{align*}
    (A - \lambda E) x = 0 \implies \opB x = 0 \\
    x \neq 0 \in \Ker \opB \implies \dim \Ker \opB > 0 \\
    \begin{cases}
      \dim \Ker \opB + \dim \Im \opB = n \; (\ref{lo-sum-dim}) \\
      \dim \Ker \opB > 0
    \end{cases}
    \implies \dim \Im \opB < n
    \\
    \dim \Im \opB < n
    \implies \Rang \opB < n
    \implies \Rang B < n
    \\
    \Rang B < n
    \implies \Rang (A - \lambda E) < n
    \implies \det (A - \lambda E) = 0
  \end{align*}

  \(\impliedby\) Аналогичные рассуждения, но в обратную сторону.
\end{proof}

\begin{definition}
  Полученное в процессе доказательства \ref{lo-ei-roots} уравнение

  \begin{align*}
    (A - \lambda E) x = 0
  \end{align*}

  называют характеристическим (вековым) уравнением.
\end{definition}

\begin{theorem}\label{ind-eivec}
  Пусть дан оператор \(\opA \colon V^{n} \to V^{n}\), у которого \(m\) различных
  собственных чисел \(\lambda_{1}, \dotsc, \lambda_{m}\). Тогда система из
  собственных векторов \(\basis_{1}, \dotsc, \basis_{m}\), соответствующих этим
  собственным числам, линейно-независима.
\end{theorem}
\begin{proof}
  По индукции.

  \textbf{База}: \(m = 1\), \(\{ \basis_{1} \}\) линейно-независима, т.к.
  \(\basis_{1}\) ненулевой по определению.

  \textbf{Переход}: Пусть
  \(\{ \basis_{1}, \dotsc, \basis_{k} \}\)
  линейно-независима, покажем, что система
  \(\{ \basis_{1}, \dotsc, \basis_{k}, \basis_{k + 1}\}\)
  также линейно-независима. 

  Составим её нулевую линейную комбинацию и применим к ней оператор \(\opA\):

  \begin{align*}\label{eq:ind-eivec-proof-1}\tag{\(1\)}
    c_{1} \basis_{1} + \dotsc + c_{k + 1} \basis_{k + 1} = 0
    \\
    \opA(c_{1} \basis_{1} + \dotsc + c_{k + 1} \basis_{k + 1}) = 0
    \\
    c_{1} \opA \basis_{1} + \dotsc + c_{k + 1} \opA \basis_{k + 1} = 0
    \\
    c_{1} \lambda_{1} \basis_{1}
      + \dotsc + c_{k + 1} \lambda_{k + 1} \basis_{k + 1} = 0
  \end{align*}
  
  Далее снова составим нулевую линейную комбинацию этой системы и домножим её
  на \(\lambda_{k + 1}\):
  
  \begin{align*}\label{eq:ind-eivec-proof-2}\tag{\(2\)}
    c_{1} \basis_{1} + \dotsc + c_{k + 1} \basis_{k + 1} = 0
      \mid \cdot \lambda_{k + 1}
    \\
    c_{1} \lambda_{k + 1} \basis_{1}
      + \dotsc + c_{k + 1} \lambda_{k + 1} \basis_{k + 1} = 0 
  \end{align*}

  Теперь из \eqref{eq:ind-eivec-proof-1} вычтем \eqref{eq:ind-eivec-proof-2}:

  \begin{align*}
    c_{1} \basis_{1} (\lambda_{1} - \lambda_{k + 1})
    + \dotsc
    + c_{k} \basis_{k} (\lambda_{k} - \lambda_{k + 1})
    + c_{k + 1} \basis_{k + 1} (\lambda_{k + 1} - \lambda_{k + 1}) = 0
    \\
    c_{1} \basis_{1} (\lambda_{1} - \lambda_{k + 1})
    + \dotsc
    + c_{k} \basis_{k} (\lambda_{k} - \lambda_{k + 1}) = 0
  \end{align*}
    
  Каждая из скобок \((\lambda_{i} - \lambda_{k + 1})\) не равна нулю, т.к. все
  собственные числа различны. Т.к. система
  \(\{ \basis_{1}, \dotsc, \basis_{k} \}\)
  является базисом по предположению индукции, то полученное равенство верно
  только при \(\forall c_{i} = 0 \; (1 \le i \le k)\).

  Подставляя это в исходную нулевую линейную комбинацию, получаем, что

  \begin{align*}
    c_{1} \basis_{1} + \dotsc + c_{k + 1} \basis_{k + 1} = 0 \\
    c_{k + 1} \basis_{k + 1} = 0
  \end{align*}

  Т.к. собственный вектор \(\basis_{k + 1}\) ненулевой по определению, то
  \(c_{k + 1} = 0\). Таким образом \(\forall c_{i} = 0\), значит нулевая
  линейная комбинация тривиальна, т.е. полученная система линейно независима.
\end{proof}

\begin{definition}
  Базис, составленный из собственных векторов, называют собственным базисом.
\end{definition}

\begin{theorem}\label{eibasis-diag}
  Матрица оператора в собственном базисе диагональна.
\end{theorem}
\begin{proof}
  Матрица оператора в некотором базисе это коэффициенты разложения образов
  базисных векторов по этому же базису. Рассмотрим первый базисный вектор:

  \begin{align*}
    \begin{cases}
      \opA \basis_1 = a_{1,1} \basis_{1} + \dots + a_{n,1} \basis_{n} \\
      \opA \basis_1 = \lambda_{1} \basis_1 \\
    \end{cases}
    \implies
    \begin{cases}
      a_{1,1} = \lambda_{1}, \\
      a_{i,1} = 0 \; \forall i \neq 1
    \end{cases}
  \end{align*}

  Аналогично можно рассмотреть все оставшиеся базисные векторы. Таким образом
  матрица оператора в базисе из собственных векторов будет иметь вид

  \begin{align*}
    A = \begin{pmatrix}
      \lambda_{1} & 0           & \dots  & 0           \\
      0           & \lambda_{2} & \dots  & 0           \\
      \vdots      & \vdots      & \ddots & \vdots      \\
      0           & 0           & \dots  & \lambda_{n} \\
    \end{pmatrix}
  \end{align*}
\end{proof}
\begin{corollary}
  Если у оператора \(\opA \colon V^{n} \to V^{n}\) есть \(n\) различных
  собственных чисел, то существует базис, в котором матрица этого оператора
  диагональна.
\end{corollary}
\begin{proof}
  Т.к. все собственные числа различны, то соответствующие им \(n\) собственных
  векторов будут линейно независимы (по \ref{ind-eivec}). Составим из них
  базис, по только что доказанной теореме матрица оператора \(\opA\) в этом
  базисе будет диагональной.
\end{proof}
