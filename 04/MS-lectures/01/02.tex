\subsection{%
  Лекция \texttt{24.??.??}.%
}

\subheader{Точечные оценки}

Пусть имеется выборка \(\sample{X} = (X_1, \dotsc, X_n)\) объема \(n\)~--- набор
независимых экземпляров случайной величины \(X\).

\begin{definition}
  Статистикой называется измеримая функция \(\Theta^* = \Theta^* (X_1, \dotsc,
  X_n)\).
\end{definition}

Пусть требуется найти приближенную оценку неизвестного параметра \(\Theta\) по
выборке \((X_1, \dotsc, X_n)\). Оценка считается при помощи некоторой статистики
\(\Theta^* = \Theta^* (X_1, \dotsc, X_n)\).

\subheader{Свойства статистических оценок}

\begin{definition}
  Статистика \(\Theta^* = \Theta^* (X_1, \dotsc, X_n)\) неизвестного параметра
  \(\Theta\) называется состоятельной, если \(\Theta^* \Rarr{\probP} \Theta\)
  при \(n \to \infty\).
\end{definition}

\begin{definition}
  Статистика \(\Theta^* = \Theta^* (X_1, \dotsc, X_n)\) неизвестного параметра
  \(\Theta\) называется несмещенной, если \(\expected{\Theta^*} = \Theta\).
\end{definition}

\begin{remark}
  Оценка называется асимптотически несмещенной, если \(\expected{\Theta^*} \to
  \Theta\) при \(n \to \infty\).
\end{remark}

\begin{definition}
  Оценка \(\Theta_1^*\) не хуже оценки \(\Theta_2^*\), если

  \begin{equation*}
    \expected{\prh{\Theta_1^* - \Theta}^2}
      \le \expected{\prh{\Theta_2^* - \Theta}^2}
  \end{equation*}

  Если \(\Theta_1^*\) и \(\Theta_2^*\)~--- несмещенные оценки, то это
  равносильно тому, что \(\variance{\Theta_1^*} \le \variance{\Theta_2^*}\).
\end{definition}

\begin{definition}
  Оценка \(\Theta^*\) называется эффективной, если она не хуже всех остальных
  оценок.
\end{definition}

\begin{remark}
  В классе всех возможных оценок не существует эффективной оценки.  
\end{remark}

\begin{theorem}
  В классе несмещенных оценок существует эффективная оценка, причем
  единственная.
\end{theorem}

\subheader{Точечные оценки моментов}

\begin{definition}
  Выборочным средним \(\avg{x}\) называется величина

  \begin{equation*}
    \avg{x} = \frac{1}{n} \sum_{i = 1}^n x_i
  \end{equation*}
\end{definition}

\begin{definition}
  Выборочной дисперсией \(\svarianceD\) называется величина

  \begin{equation*}
    \svarianceD = \frac{1}{n} \sum_{i = 1}^n \prh{x_i - \avg{x}}^2
  \end{equation*}
\end{definition}

\begin{definition}
  Исправленной выборочной дисперсией \(S^2\) называется величина

  \begin{equation*}
    S^2
    = \frac{1}{n - 1} \sum_{i = 1}^n \prh{x_i - \avg{x}}^2
    = \frac{n}{n - 1} \svarianceD
  \end{equation*}
\end{definition}

\begin{definition}
  Выборочным среднеквадратичным отклонением называется величина

  \begin{equation*}
    \sstderS = \sqrt{\svarianceD}
  \end{equation*}
\end{definition}

\begin{definition}
  Исправленным выборочным среднеквадратичным отклонением называется величина

  \begin{equation*}
    S = \sqrt{S^2}
  \end{equation*}
\end{definition}

\begin{definition}
  Выборочным \(k\)-ым моментов \(\avg{x^k}\) называется величина

  \begin{equation*}
    \avg{x^k} = \frac{1}{n} \sum_{i = 1}^n x_i^k
  \end{equation*}
\end{definition}

\begin{definition}
  Выборочной модой \(\sMo\) называется варианта с наибольшей частотой.

  \begin{equation*}
    \sMo = x_i \given n_i = \max(n_1, \dotsc, n_k)
  \end{equation*}
\end{definition}

\begin{definition}
  Выборочной медианой \(\sMe\) называется значение варианты \(x_i\) в середине
  ряда.

  \begin{equation*}
    \sMe = \begin{cases}
      x_{(k)}, & \text{ если } n = 2 k - 1 \\
      \frac{1}{2} \prh{x_{(k)} + x_{(k + 1)}}, & \text{ если } n = 2 k
    \end{cases}
  \end{equation*}
\end{definition}

\begin{theorem}
  Выборочное среднее \(\avg{x}\) является несмещенной состоятельной оценкой для
  математического ожидания.

  \begin{equation*}
    \begin{aligned}
      \expected{\avg{x}} = \expected{X} = a
      \qquad
      \avg{x} \Rarr{\probP} \expected{X} = a
    \end{aligned}
  \end{equation*}
\end{theorem}

\begin{proof}
  Покажем несмещенность.

  \begin{equation*}
    \expected{\avg{x}}
    = \expected{\frac{x_1 + \dotsc + x_n}{n}}
    = \frac{1}{n} \cdot n \cdot \expected{x_1}
    = \expected{x_1}
  \end{equation*}

  Покажем состоятельность.

  \begin{equation*}
    \avg{x} = \frac{x_1 + \dotsc + x_n}{n} \Rarr{\probP} \expected{X}
    \text{ при } n \to \infty
  \end{equation*}

  Это закон больших чисел.
\end{proof}

\begin{theorem}
  Выборочный \(k\)-ый момент \(\avg{x^k}\) является несмещенной состоятельной
  оценкой для теоретического \(k\)-ого момента.

  \begin{equation*}
    \begin{aligned}
      \expected{\avg{x^k}} = \expected{X^k} = m_k
      \qquad
      \expected{\avg{x^k}} \Rarr{\probP} \expected{X^k} = m_k
      \text{ при } n \to \infty
    \end{aligned}
  \end{equation*}
\end{theorem}

\begin{proof}
  Это следует из предыдущей теоремы, если в качестве случайной величины взять
  \(x^k\).
\end{proof}

\begin{theorem}
  Выборочные дисперсии \(\svarianceD\) и \(S^2\) являются состоятельным оценками
  для дисперсии. При этом \(\svarianceD\)~--- смещенная оценка (есть
  систематическая вниз), а \(S^2\)~--- несмещенная оценка.
\end{theorem}

\begin{proof}
  Заметим, что

  \begin{equation*}
    \begin{aligned}
      \svarianceD
      = \frac{1}{n} \sum \prh{x_i - \avg{x}}^2
      = \avg{x^2} - \prh{\avg{x}}^2
    \\
      \variance{\avg{x}}
      = \expected{\prh{\avg{x}}^2} - \prh{\expected{\avg{x}}}^2
      \implies
      \expected{\prh{\avg{x}}^2}
      = \prh{\expected{\avg{x}}}^2 + \variance{\avg{x}}
    \end{aligned}
  \end{equation*}

  Таким образом

  \begin{equation*}
    \expected{\svarianceD}
    = \expected{\avg{x^2}} - \expected{\prh{\avg{x}}^2}
    = \expected{x^2} - \prh{
      \prh{\expected{x}}^2 + \variance{\avg{x}}
    }
    = \prh{\expected{x^2} - \prh{\expected{x}}^2} - \variance{\avg{x}}
    = \variance{x} - \variance{\avg{x}}
  \end{equation*}

  Видно, что будет смещение на величину \(\variance{\avg{x}}\). Преобразуем
  полученное выражение в более удобную форму.

  \begin{equation*}
    \variance{x} - \variance{\avg{x}}
    = \variance{x} - \frac{1}{n^2} \sum_{i = 1}^n \variance{x_i}
    = \variance{x} - \frac{1}{n} \cdot \variance{x}
    = \frac{n - 1}{n} \cdot \variance{x}
   \end{equation*} 

   Значит

   \begin{equation*}
     \expected{S^2}
     = \expected{\frac{n}{n - 1} \svarianceD}
     = \frac{n}{n - 1} \cdot \frac{n - 1}{n} \cdot \variance{x}
     = \variance{x}
   \end{equation*}

   Т.е.  \(S^2\) это несмещенная оценка. Далее покажем состоятельность оценок.

   \begin{equation*}
     \begin{aligned}
        \svarianceD = \avg{x^2} - \prh{\avg{x}}^2
        \Rarr{\probP}
        \expected{x^2} - \prh{\expected{x}}^2 = \variance{x}
      \\
        S^2 = \under{\frac{n}{n - 1}}{\to 1} \svarianceD
        \Rarr{\probP}
        \variance{x}
     \end{aligned}
   \end{equation*}
\end{proof}

\begin{remark}
  Т.к. \(\frac{n}{n - 1} \to 1\) при \(n \to \infty\), то
  \(\expected{\svarianceD} \Rarr{\probP} \variance{x}\) при \(n \to \infty\).
  Значит выборочная дисперсия является асимптотически несмещенной оценкой,
  поэтому на практике при \(n \ge 100\) можно считать обычную выборочную
  дисперсию, а при \(n < 100\) следует ее заменить на исправленную выборочную
  дисперсию.
\end{remark}

\subheader{Метод моментов (Пирсона)}

Зная выборочные моменты можно дать оценки остальным параметрам распределения.
Пусть имеется выборка \(\sample{X} = \prh{X_1, \dotsc, X_n}\) неизвестного
распределения, но при этом знаем, что данное распределение определенного типа,
задаваемого \(k\) параметрами \(\Theta = \prh{\Theta_1, \dotsc, \Theta_k}\).
Зная параметры можем вычислить теоретические \(k\)-ые моменты. Например, если
распределение непрерывное, то

\begin{equation*}
  m_i
  = \int_{\RR} x^k f \prh{x, \Theta_1, \dotsc, \Theta_k} \dd x
  = h_i \prh{\Theta_1, \dotsc, \Theta_k}
\end{equation*}

Вычислим выборочные моменты и подставим в эти формулы, получим систему
уравнений.

\begin{equation*}
  \begin{cases}
    \avg{x}   = & h_1 \prh{\Theta_1, \dotsc, \Theta_k} \\
    \avg{x^2} = & h_2 \prh{\Theta_1, \dotsc, \Theta_k} \\
                & \vdots                               \\
    \avg{x^k} = & h_k \prh{\Theta_1, \dotsc, \Theta_k}
  \end{cases}
\end{equation*}

Решив эту систему, находим оценки \(\Theta_1^*, \dotsc, \Theta_k^*\) неизвестных
параметров.

\begin{remark}
  При этом обычно получаем состоятельные оценки, но смещенные.
\end{remark}

\begin{example}
  Пусть \(X \in \udist{a}{b}\). При обработке статданных получили оценки первого
  и второго момента \(\avg{x} = 2.25\) и \(\avg{x^2} = 6.75\). Необходимо дать
  оценки неизвестных параметров \(a\) и \(b\).

  \solution{} Плотность будет иметь вид

  \begin{equation*}
    f(x, a, b) = \begin{cases}
      0, & x < a \\
      \frac{1}{b - a}, & a \le x \le b \\
      0, & x > b
    \end{cases}
  \end{equation*}

  Тогда

  \begin{equation*}
    \expected{x} = \frac{a + b}{2}
    \qquad
    \expected{x^2} = \frac{a^2 + a b + b^2}{3}
  \end{equation*}

  Составим систему

  \begin{equation*}
    \begin{cases}
      2.25 = \frac{a^* + b^*}{2} \\
      6.75 = \frac{\prh{a^*}^2 + a^* b^* + \prh{b^*}^2}{3}
    \end{cases}
    \iff
    \begin{cases}
      a^* + b^* = 4.5 \\
      \prh{a^*}^2 + a^* b^* + \prh{b^*}^2 = 20.25
    \end{cases}
    \iff
    \begin{cases}
      a^* + b^* = 4.5 \\
      a^* b^* = 0
    \end{cases}
    \iff[a < b]
    \begin{cases}
      a^* = 0 \\
      b^* = 4.5 
    \end{cases}
  \end{equation*}
\end{example}