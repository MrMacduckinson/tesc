\subsection{%
  Лекция \texttt{24.??.??}.%
}

\subheader{Датчики случайных чисел}

\begin{definition}
  Члены последовательности \(y_1, \dotsc, y_n, \dotsc\), которые можно
  рассматривать как экспериментальные значения данной случайной величины,
  называются псевдослучайными числами, а устройства или алгоритмы для их
  получения~--- датчиками случайных чисел.
\end{definition}

\subsubheader{I.}{Физические датчики}

\begin{theorem}
  Случайная величина \(\eta \in \udist{0}{1}\) тогда и только тогда, когда
  разряды \(\xi_i\) ее двоичной записи \(\eta = \sum_{i = 1}^{\infty} 2^{-i}
  \xi_i\)~--- схема Бернулли с вероятностью \(p = 0.5\).
\end{theorem}

\begin{remark}
  Недостатки физических датчиков:

  \begin{enumerate}
  \item
    Необходимо оборудование.

  \item
    При повторном опыте нельзя получить ту же самую последовательность (а иногда
    это надо).
  \end{enumerate}
\end{remark}

\subsubheader{II.}{Таблицы случайных чисел}

Пусть имеется таблица чисел (обычно двузначных)~--- результат работы некоторого
датчика случайных чисел. Далее наугад выбирались строка и столбец и начиная с
этого места брались \quote{случайные} числа. При необходимости увеличения
точности дописывали числа из соседнего столбца.

\begin{remark}
  Недостатки:

  \begin{enumerate}
  \item
    Требует много оперативной памяти.

  \item
    Слишком большая предсказуемость.
  \end{enumerate}
\end{remark}

\subsubheader{III.}{Математические датчики}

Обычно это рекуррентные последовательности вида \(y_n = f(y_{n - 1})\). Основным
считается мультипликативный датчик, который задается так: \(k_0\)~--- начальное
число, с которого начинается генерация, \(a\)~--- это множитель и \(m\) это
модуль. Причем модуль выбирается так, что множитель \(a\) и \(k_0\) взаимно
просты с ним. В итоге последовательность псевдослучайных чисел строится по
формулам

\begin{equation*}
  \begin{cases}
    k_n = k_{n - 1} \cdot a \pmod m \\
    y_n = \frac{k_n}{m} \in \interval{0}{1}
  \end{cases}
\end{equation*}

\begin{remark}
  Рекомендации (старые, для \(32\)-ух битных компьютеров):

  \begin{enumerate}
  \item
    \(m = 2^{31} - 1\)

  \item
    \(a = 30360016\) или \(a = 764261123\)

  \item
    \(k_0\) практически неважно.
  \end{enumerate}
\end{remark}

\subsubheader{IV.}{Датчик Уичмана и Хилла (1982)}

Одновременно запускаются три мультипликативных датчика с параметрами \(a_1 =
171\), \(m_1 = 30269\), \(a_2 = 172\), \(m_2 = 30307\), \(a_3 = 170\), \(m_3 =
30323\). На \(i\)-ом шаге генерируется три псевдослучайных числа \(y_n', y_n'',
y_n'''\). Тогда \(y_n = \{ y_n' + y_n'' + y_n''' \}\) (дробная часть от суммы).
Преимущества:

\begin{enumerate}
\item
  Он быстрее предыдущего.

\item
  Период этого датчика примерно \(3 \cdot 10^{13}\), а предыдущего \(2 \cdot
  10^9\).
\end{enumerate}

\subheader{Моделирование случайных величин}

В начале рассмотрим непрерывное распределение. Будем использовать метод обратной
функции (квантильное преобразование).

\begin{theorem}
  Пусть \(F(x)\)~--- функция распределения абсолютно непрерывной случайной
  величины. Если \(\eta \in \udist{0}{1}\), то случайная величина \(\xi = F^{-1}
  (\eta)\) имеет функцию распределения \(F(x)\).
\end{theorem}

\begin{example}[Показательное распределение]
  Пусть \(\xi \in \Edist{\alpha}\), тогда

  \begin{equation*}
    F(x) = \begin{cases}
      0,                 & x < 0 \\
      1 - e^{-\alpha x}, & x \ge 0
    \end{cases}
  \end{equation*}

  Значит получаем

  \begin{equation*}
    y = 1 - e^{-\alpha x}
    \implies
    x = -\frac{1}{\alpha} \ln \prh{1 - y} = F^{-1} (y)
  \end{equation*}

  Тогда согласно теореме, если \(y_i\) это значение датчика, то \(x_i = F^{-1}
  (y) \in \Edist{\alpha}\).
\end{example}

\begin{example}[Нормальное распределение]
  Пусть \(\xi \in \ndist{a}{\sigma^2}\). Известно, что если \(\xi \in
  \ndist{0}{1}\), то \((\sigma \xi + a) \in \ndist{a}{\sigma}\), поэтому
  достаточно уметь моделировать стандартное нормальное распределение.

  \begin{equation*}
    F_0 (x)
    = \frac{1}{2 \sqrt{\pi}} \int_{-\infty}^x \exp \prh{-\frac{z^2}{2}} \dd z
  \end{equation*}
  
  Таким образом \(x_i = F_0^{-1} (y_i) \in \ndist{0}{1}\), если \(y_i \in
  \udist{0}{1}\).
\end{example}

\begin{remark}
  Преимущество этого метода заключается в простоте и универсальности, а
  недостаток заключается в не очень высокой эффективности (раньше было так,
  сейчас уже возможно иначе).
\end{remark}

\subheader{Нормальные случайные числа}

\subsubheader{I.}{На основе центральной предельной теоремы}

Пусть \(\eta_i \in \udist{0}{1}\), тогда \(\expected{\eta_i} = a = 0.5\) и
\(\variance{\eta_i} = \frac{1}{12}\). По центральной предельной теореме

\begin{equation*}
  \frac{S_n - n a}{\sqrt{n \variance{\xi}}}
  = \frac{S_n - \frac{n}{2}}{\sqrt{\frac{n}{12}}}
  \rightrightarrows
  \ndist{0}{1}
\end{equation*}

Уже при \(n = 12\) получаются довольно хорошие результаты.

\subsubheader{II.}{Точное моделирование пары случайных величин с распределением
\(\ndist{0}{1}\)}

\begin{theorem}
  Пусть \(\eta_1, \eta_2 \in \ndist{0}{1}\) и независимы. Тогда следующие
  величины \(X, Y \in \ndist{0}{1}\) и независимы.

  \begin{equation*}
    X = \sqrt{-2 \ln \eta_1} \cos \prh{2 \pi \eta_2}
    \qquad
    Y = \sqrt{-2 \ln \eta_1} \sin \prh{2 \pi \eta_2}
  \end{equation*}
\end{theorem}

\begin{proof}
  Пусть \(X, Y \in \ndist{0}{1}\) и независимы. Тогда плотность совместного
  распределения

  \begin{equation*}
    f_{X, Y} (x, y)
    = \frac{1}{2 \sqrt{\pi}} \exp \prh{-\frac{x^2}{2}}
      \frac{1}{2 \sqrt{\pi}} \exp \prh{-\frac{y^2}{2}}
    = \frac{1}{2 \pi} \exp \prh{-\frac{1}{2} \prh{x^2 + y^2}}
  \end{equation*}

  При переходе к полярным координатам плотность данного распределения примет вид

  \begin{equation*}
    f_{X, Y} (\rho, \phi)
    = \frac{1}{2 \pi} \exp \prh{-\frac{1}{2} \rho^2} \rho
    = \under{\frac{1}{2 \pi}}{f_{\phi}}
      \cdot \under{\exp \prh{-\frac{\rho^2}{2}} \rho}{f_{\rho}}
  \end{equation*}

  Найдем их функции распределения.

  \begin{equation*}
    \begin{aligned}
      F_{\phi} (x) = \frac{x}{2 \pi}
    \\
      F_{\rho} (x)
        = \int_0^{\rho} \rho \exp \prh{-\frac{\rho^2}{2}} \dd \rho
        = -\exp \prh{-\frac{\rho^2}{2}} \bigg\rvert_0^{\rho}
        = 1 - \exp \prh{-\frac{\rho^2}{2}}
    \end{aligned}
  \end{equation*}

  Отсюда методом обратной функции получаем формулы для моделирования случайных
  величин \(\phi\) и \(\rho\).

  \begin{equation*}
    \begin{aligned}
      y = \frac{x}{2 \pi}
      \implies
      x = 2 \pi y
      & \qquad &
      y \in \udist{0}{1}
    \\
      y = 1 - \exp \prh{-\frac{\rho^2}{2}}
      \implies
      \rho = \sqrt{-2 \ln (1 - y)}
      & \qquad &
      y \in \udist{0}{1}
    \end{aligned}
  \end{equation*}

  Т.к. \(y \in \udist{0}{1}\), то \(1 - y \in \udist{0}{1}\). В итоге получаем

  \begin{equation*}
    X = \rho \cos \phi = \sqrt{-2 \ln \eta_1} \cos \prh{2 \pi \eta_2}
    \qquad
    Y = \rho \sin \phi = \sqrt{-2 \ln \eta_1} \sin \prh{2 \pi \eta_2}
  \end{equation*}
\end{proof}

\subheader{Быстрый показательный датчик}

\begin{theorem}
  Пусть случайные величины \(\eta_1, \dotsc, \eta_{2 n - 1} \in \udist{0}{1}\).
  Обозначим \(\xi_1, \dotsc, \xi_{n - 1}\) это расставленные в порядке
  возрастания значения \(\eta_{n + 1}, \dotsc, \eta_{2 n - 1}\). Отдельно
  положим \(\xi_0 = 0\) и \(\xi_n = 1\). Тогда \(\mu_i\) независимы и имеют
  показательное распределение с параметром \(\alpha\).

  \begin{equation*}
    \mu_i = -\frac{1}{\alpha} \prh{\xi_i - \xi_{i - 1}}
      \ln \prh{\eta_1 \cdot \dotsc \cdot \eta_n}
    \qquad
    1 \le i \le n
  \end{equation*}
\end{theorem}

\begin{example}
  При \(n = 3\) получаем \(\eta_1, \eta_2, \eta_3, \eta_4, \eta_5\). Пусть
  \(\eta_4 < \eta_5\) (в противном случае переобозначим их), тогда

  \begin{equation*}
    \begin{aligned}
      \mu_1 = -\frac{1}{\alpha} \eta_4 \ln \prh{\eta_1 \eta_2 \eta_3}
    \\
      \mu_2 = -\frac{1}{\alpha} (\eta_5 - \eta_4) \ln \prh{\eta_1 \eta_2 \eta_3}
    \\
      \mu_3 = -\frac{1}{\alpha} (1 - \eta_5) \ln \prh{\eta_1 \eta_2 \eta_3}
    \end{aligned}
  \end{equation*}
\end{example}

\begin{remark}
  При этом методе экономим время при вычислении логарифма, но теряем на
  сортировке. Алгоритм оптимален при \(n = 3\). В этом случае он примерно в два
  раза быстрее метода обратной функции.
\end{remark}

\subheader{Моделирование дискретных распределений}

Пусть имеется дискретное распределение, которое задается парами \(x_k\) и \(p_k
= \prob{\xi = c_k}\), где \(k = 1, 2, \dotsc\). Разобьем отрезок
\(\segment{0}{1}\) на отрезки длины \(p_k\). Обозначим \(\eta_m = \sum_{k = 1}^m
p_k\) и \(\eta_0 = 0\)~--- границы отрезков. Пусть \(y_i \in \segment{0}{1}\)
это случайное число. Если \(y_i \in [\rho_{j - 1}; \rho_j)\), то полагаем, что
\(x_i = c_j\).

\subsubheader{I.}{Распределение Бернулли}

Пусть \(\xi \in \Bdist{p}\), тогда получаем два отрезка, т.е.

\begin{equation*}
  \begin{cases}
    y_i \in [0; 1 - p) \implies x_i = 0 \\
    y_i \in [1 - p; 1] \implies x_i = 1
  \end{cases}
\end{equation*}

\subsubheader{II.}{Биномиальное распределение}

Пусть \(\xi \in \BNdist{n}{p}\), т.е. \(\prob{\xi = k} = C_n^k p^k q^{1 - k}\),
где \(k = 0, \dotsc, n\). Исходя из смысла биномиального распределения \(\xi =
\xi_1 + \dotsc + \xi_n\), где \(\xi_i \in \Bdist{p}\). Далее берем \(n\)
значений датчика \(y_i \in \udist{0}{1}\) и

\begin{equation*}
  \begin{cases}
    y_i \in [0; 1 - p) \implies z_i = 0 \\
    y_i \in [1 - p; 1] \implies z_i = 1
  \end{cases}
  \implies
  x_k = \sum_{i = 1}^n z_i
\end{equation*}

\subsubheader{III.}{Геометрическое распределение}

Пусть \(\xi \in \Gdist{p}\), т.е. \(\prob{\xi = k} = q^{k - 1} p\), где \(k = 1,
2, \dotsc\). Исходя из смысла геометрического распределения \(\xi\) это номер
первого успешного испытания, поэтому берем значения датчика \(y_i \in
\udist{0}{1}\) до тех пор, пока оно не попадет в интервал
\(\segment{1 - p}{1}\). Далее полагаем \(x_k = i\).

\subsubheader{IV.}{Распределение Пуассона}

Пусть \(\xi \in \Pdist{\lambda}\), т.е. \(\prob{\xi = k} = \frac{\lambda^k}{k!}
e^{-\lambda}\), где \(k = 0, 1, \dotsc\).

\begin{theorem}
  Пусть \(\mu_1, \mu_2, \dotsc\)~--- независимые случайные величины, имеющие
  показательное распределение с параметром \(\lambda\). Положим \(S_n = \mu_1 +
  \dotsc + \mu_n\) и \(N = \max n \given[\big] S_n \in \segment{0}{1}\). Тогда
  \(N \in \Pdist{\lambda}\).
\end{theorem}

На основе данной теоремы и метода обратной функции получаем формулу для
моделирования распределение Пуассона.

\begin{equation*}
  n_i = \min \set{ k \given \prod_{j = 1}^k y_{i, j} < e^{-\lambda}}
\end{equation*}

где \(y_{i, j}\) это значение датчика в \(i\)-ой серии.
