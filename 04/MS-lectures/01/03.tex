\subsection{%
  Лекция \texttt{24.??.??}.%
}

Пусть имеется выборка распределения известного нам типа, но с неизвестными
параметрами. Требуется найти эти параметры. На прошлой лекции был рассмотрен
метод моментов. Его недостаток заключается в том, что он не гарантирует
эффективность оценок.

\subheader{Метод максимального правдоподобия (Фишера)}

Пусть имеется выборка \(\sample{X} = \prh{X_1, \dots, X_n}\) распределения
известного типа, задаваемого параметрами \(\Theta = \prh{\Theta_1, \dotsc,
\Theta_k}\).

Идея метода заключается в подборе параметров таким образом, чтобы вероятность
получения данной выборки при случайной эксперименте была наибольшей. Например,
если распределение дискретное, то эта вероятность будет равна

\begin{equation*}
  \probP_{\Theta} (x_1, \dotsc, x_n)
  = \probP_{\Theta} (x = x_1)
    \cdot \dotsc
    \cdot \probP_{\Theta} (x = x_n)
\end{equation*}

\begin{definition}
  Функцией правдоподобия \(L \prh{\sample{X}, \Theta}\) называется функция

  \begin{equation*}
    \begin{aligned}
      L \prh{\sample{X}, \Theta}
      = \prod_{i = 1}^n \prob{x = x_i}
      & \qquad & \text{при дискретном распределении}
    \\
      L \prh{\sample{X}, \Theta}
      = \prod_{i = 1}^n f_{\Theta} (x_i)
      & \qquad & \text{при непрерывном распределении}
    \end{aligned}
  \end{equation*}
\end{definition}

\begin{definition}
  Логарифмической Функцией правдоподобия называется функция \(\ln L
  \prh{\sample{X}, \Theta}\).
\end{definition}

\begin{remark}
  Т.к. \(y = \ln x\) возрастающая функция, то их точки экстремума совпадают, но
  искать их проще во втором случае.
\end{remark}

\begin{definition}
  Оценкой максимального правдоподобия \(\widehat{\Theta}\) называется значение
  \(\Theta\), при котором функция правдоподобия достигает наибольшего значения
  (при фиксированных значениях \(x_1, \dotsc, x_n\)).
\end{definition}

\begin{example}
  Пусть \(\sample{X} = \prh{X_1, \dotsc, X_n}\) из распределения Пуассона с
  неизвестным параметром \(\lambda > 0\). Требуется найти оценку максимального
  правдоподобия параметра \(\lambda\).
  
  \solution{} Составим функцию правдоподобия.

  \begin{equation*}
    L \prh{\sample{X}, \alpha}
    = \prod_{i = 1}^n \frac{\lambda^{x_i}}{x_i!} e^{-\lambda}
    = \frac{\lambda^{n \avg{x}}}{\prod_{i = 1}^n x_i!} e^{-n \lambda}
  \end{equation*}

  Перейдем к логарифмической функции правдоподобия.

  \begin{equation*}
    \ln L \prh{\sample{X}, \lambda}
    = n \avg{x} \ln \lambda - \ln \prod_{i = 1}^n x_i! - n \lambda
  \end{equation*}

  Вычислим производную по \(\lambda\) и приравняем ее к нулю, чтобы найти
  экстремум.

  \begin{equation*}
    \frac{\dd }{\dd \lambda} \prh{\ln L \prh{\sample{X}, \lambda}}
    = \frac{n \avg{x}}{\lambda} - n
    = 0
    \implies
    \lambda = \avg{x}
  \end{equation*}

  Покажем, что это действительно точка максимума.

  \begin{equation*}
    \frac{\dd^2 }{\dd \lambda^2} \prh{\ln L \prh{\sample{X}, \lambda}}
    = -\frac{n \avg{x}}{\lambda^2}
    < 0
  \end{equation*}

  Итого \(\lambda = \avg{x}\) это оценка максимального правдоподобия.
\end{example}

\begin{example}
  Пусть \(\sample{X} = \prh{X_1, \dotsc, X_n}\)~--- выборка из \(N(a;
  \sigma^2)\), где \(a \in \RR\) и \(\sigma > 0\). Требуется найти оценку
  максимального правдоподобия.

  \solution{} Составим функцию правдоподобия.

  \begin{equation*}
    L \prh{\sample{X}, a, \sigma}
    = \prod_{i = 1}^n \frac{1}{\sigma \sqrt{2 \pi}}
      \exp \prh{-\frac{(x_i - a)^2}{2 \sigma^2}}
    = \frac{1}{\sigma^n \prh{2 \pi}^{n / 2}} \exp \prh{
      -\frac{1}{2 \sigma^2} \sum_{i = 1}^n \prh{x_i - a}^2
    }
  \end{equation*}

  Перейдем к логарифмической функции правдоподобия.

  \begin{equation*}
    \ln L \prh{\sample{X}, a, \sigma}
    = -n \ln \sigma - \frac{n}{2} \ln (2 \pi)
      - \frac{1}{2 \sigma^2} \sum_{i = 1}^n \prh{x_i - a}^2
  \end{equation*}

  Найдем частные производные по \(a\) и \(\sigma\).

  \begin{equation*}
    \begin{aligned}
      \frac{\partial}{\partial a} \ln L \prh{\sample{X}, a, \sigma}
      = -\frac{1}{2 \sigma^2} \sum_{i = 1}^n 2 (x_i - a) \cdot (-1)
      = \frac{1}{\sigma^2} \sum_{i = 1}^n (x_i - a)
      = \frac{n}{\sigma^2} \prh{\avg{x} - a}
    \\
      \frac{\partial}{\partial \sigma} \ln L \prh{\sample{X}, a, \sigma}
      = -\frac{n}{\sigma} - \frac{1}{2} \cdot (-2) \sigma^{-3}
        \cdot \sum_{i = 1}^n \prh{x_i - a}^2
      = \frac{1}{\sigma^3} \sum_{i = 1}^n \prh{x_i - a}^2 - \frac{n}{\sigma}
    \end{aligned}
  \end{equation*}

  Приравняем их к нулю. Из производной по \(a\) получаем \(\widehat{a} =
  \avg{x}\). Поработаем с производной по \(\sigma\).

  \begin{equation*}
    \frac{1}{\sigma^3} \sum_{i = 1}^n \prh{x_i - a}^2 - \frac{n}{\sigma} = 0
    \implies
    \widehat{\sigma}^2 = \frac{1}{n} \sum_{i = 1}^n \prh{x_i - a}^2
    \implies
    \widehat{\sigma}^2 = \svarianceD
  \end{equation*}

  То, что это действительно точка максимума можно показать с помощью вторых
  дифференциалов (самостоятельно).
\end{example}

\begin{example}
  Пусть \(\sample{X} = \prh{X_1, \dotsc, X_n}\) из \(\evenly{0}{\Theta}\), где
  \(\Theta > 0\). Найти оценки \(\Theta\) методом моментов и методом
  максимального правдоподобия. Сравнить результаты.

  \solution{} По методу моментов

  \begin{equation*}
    \expected{X} = \frac{0 + \Theta}{2}
    \implies
    \avg{x} = \frac{\Theta^*}{2}
    \implies
    \Theta^* = 2 \avg{x}
  \end{equation*}

  Далее рассмотрим метод максимального правдоподобия. Обозначим \(X_{(n)}\)
  \(n\)-ую порядковую статистику (т.е. максимальный элемент выборки), тогда

  \begin{equation*}
    L \prh{\sample{X}, \Theta}
    = \prod_{i = 1}^n f_{\Theta} (x_i)
    = \begin{cases}
      0, & \Theta < X_{(n)} \\
      \frac{1}{\Theta^n}, & \Theta \ge X_{(n)}
    \end{cases}
  \end{equation*}

  Значит \(\widehat{\Theta} = X_{(n)}\) это оценка максимального правдоподобия.

  Видим, что полученные оценки отличаются. Сравним их. Сначала рассмотрим оценку
  по методу моментов. Отметим, что она будет несмещенной, т.к.

  \begin{equation*}
    \Theta^* = 2 \avg{x}
    \qquad
    \expected{\Theta^*}
    = 2 \expected{\avg{x}}
    = 2 \expected{x}
    = \Theta
  \end{equation*}

  Таким образом

  \begin{equation*}
    \expected{\prh{\Theta^* - \Theta}^2}
    = \variance{\Theta^*}
    = 4 \variance{\avg{x}}
    = 4 \frac{\variance{x}}{n}
    = \frac{4}{n} \cdot \frac{\Theta^2}{12}
    = \frac{\Theta^2}{3 n}
  \end{equation*}

  Далее рассмотрим оценку по методу максимального правдоподобия. Вначале
  рассмотрим случайную величину \(X_{(n)} = \max \prh{X_1, \dotsc, X_n}\).
  Найдем ее функцию распределения.

  \begin{equation*}
    F_{X_{(n)}}
    = \prob{X_{(n)} < x}
    = \prob{X_1 < x, \dotsc, X_n < x}
    = \prob{X_1 < x} \cdot \dotsc \cdot \prob{X_n < x}
    = \prh{F(x)}^n
    = \begin{cases}
      0, & x < 0 \\
      \frac{x^n}{\Theta^n}, & 0 \le x \le \Theta \\
      1, & x > \Theta
    \end{cases}
  \end{equation*}

  Далее найдем плотность.

  \begin{equation*}
    f_{X_{(n)}} (x)
    = F'_{X_{(n)}} (x)
    = \begin{cases}
      0, & x < 0 \\
      \frac{n x^{n - 1}}{\Theta^n}, & 0 \le x \le \Theta \\
      0, & x > \Theta
    \end{cases}
  \end{equation*}

  Вычислим первый момент.

  \begin{equation*}
    \expected{X_{(n)}}
    = \int_0^{\Theta} x \cdot \frac{n x^{n - 1}}{\Theta^n} \dd x
    = \frac{1}{\Theta^n} \int_0^{\Theta} n x^n \dd x
    = \frac{1}{\Theta^n}
      \left. \frac{n x^{n + 1}}{n + 1} \right\rvert_0^{\Theta}
    = \frac{1}{\Theta^n} \cdot \frac{n \Theta^{n + 1}}{n + 1}
    = \frac{n}{n + 1} \Theta
  \end{equation*}

  Значит эта оценка будет смещенной вниз. Подправим ее следующим образом

  \begin{equation*}
    \widetilde{\Theta}
    = \frac{n + 1}{n} \widehat{\Theta}
    = \frac{n + 1}{n} X_{(n)}
  \end{equation*}

  Такая оценка будет уже несмещенной. Вычислим ее второй момент.

  \begin{equation*}
    \expected{\widetilde{\Theta}^2}
    = \frac{(n + 1)^2}{n^2}
      \int_0^{\Theta} x^2 \frac{n x^{n - 1}}{\Theta^n} \dd x
    = \frac{(n + 1)^2}{n} \cdot \frac{1}{\Theta^n}
      \int_0^{\Theta} x^{n + 1} \dd x
    = \frac{(n + 1)^2}{n \Theta^n} \cdot 
      \left. \frac{x^{n + 2}}{n + 2} \right.\rvert_0^{\Theta}
    = \frac{(n + 1)^2 \Theta^2}{n (n + 2)}
  \end{equation*}

  Теперь вычислим дисперсию этой несмещенной оценки.

  \begin{equation*}
    \expected{\prh{\widetilde{\Theta} - \Theta}^2}
    = \variance{\widetilde{\Theta}}
    = \expected{\widetilde{\Theta}^2} - \prh{\expected{\widetilde{\Theta}}}^2
    = \frac{(n + 1)^2 \Theta^2}{n (n + 2)} - \Theta^2
    = \frac{n^2 + 2 n + 1 - n^2 - 2 n}{n (n + 2)} \cdot \Theta^2
    = \frac{\Theta^2}{n (n + 2)}
  \end{equation*}

  Итак, теперь можно сравнить эти две оценки.

  \begin{equation*}
    \variance{\Theta^*} = \frac{\Theta^2}{3 n}
    \qquad
    \variance{\widetilde{\Theta}} = \frac{\Theta^2}{n (n + 2)}
  \end{equation*}

  Хорошо видно, что вторая оценка сходится быстрее, т.е. эффективность оценки по
  методу максимального правдоподобия заметно выше. Оценка по методу моментов
  сходится со скоростью \(\sim \frac{1}{\sqrt{n}}\), а по методу максимального
  правдоподобия со скоростью \(\sim \frac{1}{n}\).
\end{example}

\begin{remark}
  Т.к. \(\expected{X} = \frac{\Theta}{2}\), то из примера выше следует, что
  оценка \(\avg{x}\) не является эффективной оценкой математического ожидания.
  В данном случае оценка \(\widetilde{\expected{X}} = \frac{n + 1}{2 n}
  X_{(n)}\) будет лучше. Можно показать, что эта оценка является эффективной.
\end{remark}

\begin{remark}
  В общем случае равномерного распределения несмещенной эффективной оценкой
  математического ожидания будет \(\frac{1}{2} \prh{X_{(1)} + X_{(n)}}\). А
  несмещенной эффективной оценкой длины интервала будет \(\frac{n + 1}{n - 1}
  \prh{X_{(n)} - X_{(1)}}\).
\end{remark}

\begin{remark}
  Таким образом мы видим, что при равномерном распределении нельзя откидывать
  крайние значения выборки.
\end{remark}

\begin{remark}
  При методе максимального правдоподобия как правило получем состоятельные и
  эффективные оценки.
\end{remark}

\subheader{Неравенство Рао---Крамера}

Пусть известно, что случайная величина \(X \in \mathcal{F}_{\Theta}\), где
\(\mathcal{F}_{\Theta}\) это семейство распределений с параметром \(\Theta\).

\begin{definition}
  Носителем семейства распределений \(\mathcal{F}_{\Theta}\) называется
  множество \(C \in \RR\) такое, что \(\forall \Theta \given \prob{X \in C} =
  1\).
\end{definition}

\begin{remark}
  Далее для того, чтобы не формулировать теорему дважды, под плотностью будем
  иметь в виду

  \begin{equation*}
    f_{\Theta} (x) = \begin{cases}
      \text{плотность } f_{\Theta} (x),
        & \text{если распределение абсолютно непрерывное}
    \\
      \probP_{\Theta} (X = x), & \text{если распределение дискретное}
    \end{cases}
  \end{equation*}
\end{remark}

\begin{definition}
  Информацией Фишера \(I(\Theta)\) семейства распределений
  \(\mathcal{F}_{\Theta}\) называется величина

  \begin{equation*}
    I(\Theta)
    = \expected{\prh{\frac{\partial}{\partial \Theta} \ln f_{\Theta} (x)}^2}
  \end{equation*}

  при условии, что она существует.
\end{definition}

\begin{definition} \label{def:reg-fam}
  Семейство распределений \(\mathcal{F}_{\Theta}\) называется регулярным, если

  \begin{enumerate}
  \item
    \(\exists C\)~--- носитель \(\mathcal{F}_{\Theta}\) такой, что
    \(\forall x \in C \given\) функция \(\ln f_{\Theta} (x)\) непрерывно
    дифференцируема по \(\Theta\).
  
  \item
    Информация Фишера \(I(\Theta)\) существует и непрерывна по \(\Theta\).
  \end{enumerate}
\end{definition}

\begin{theorem}[Неравенство Рао---Крамера] \label{thr:rao-cra-ineq}
  Пусть \(\prh{X_1, \dotsc, X_n}\) это выборка из регулярного семейства
  \(\mathcal{F}_{\Theta}\), \(\Theta^* = \Theta^* \prh{X_1, \dotsc, X_n}\)~---
  несмещенная оценка параметра \(\Theta\), дисперсия \(\variance{\Theta^*}\)
  ограничена на любом компакте (компакт \(\approx\) ограниченное замкнутое
  множество). Тогда

  \begin{equation*}
    \variance{\Theta^*} \ge \frac{1}{n I(\Theta)}
  \end{equation*}
\end{theorem}

\begin{remark}
  Если \(\variance{\Theta^*} = \frac{1}{n I(\Theta)}\), то оценка \(\Theta^*\)
  будет эффективной.
\end{remark}

\begin{example}
  Пусть \(\sample{X} = \prh{X_1, \dotsc, X_n}\)~--- выборка из \(N(a;
  \sigma^2)\), где \(a \in \RR\) и \(\sigma > 0\). Проверим эффективность оценки
  \(a^* = \avg{x}\).

  \solution{} В качестве носителя выберем \(C = \interval{-\infty}{\infty}\).
  Далее вычислим \(\ln f (x)\).

  \begin{equation*}
    f(x) = \frac{1}{\sigma \sqrt{2 \pi}} \exp \prh{
      -\frac{(x - a)^2}{2 \sigma^2}
    }
    \implies
    \ln f(x) = -\ln \sigma - \ln \sqrt{2 \pi} - \frac{(x - a)^2}{2 \sigma^2}
  \end{equation*}

  Теперь вычислим частную производную по \(a\).

  \begin{equation*}
    \frac{\partial}{\partial a} \ln f(x)
    = -\frac{1}{2 \sigma^2} \cdot 2 (x - a) \cdot (-1)
    = \frac{x - a}{\sigma^2}
  \end{equation*}

  Видим, что эта производная непрерывна по \(a\). Значит первое условие из
  определения \ref{def:reg-fam} выполнено. Теперь вычислим информацию Фишера.

  \begin{equation*}
    I(\Theta)
    = \expected{\prh{\frac{\partial}{\partial a} \ln f(x)}^2}
    = \expected{\prh{\frac{X - a}{\sigma^2}}^2}
    = \frac{\expected{\prh{X -\expected{X}}^2}}{\sigma^4}
    = \frac{\variance{X}}{\sigma^4}
    = \frac{1}{\sigma^2}
  \end{equation*}

  Видим, что информация Фишера существует и непрерывна по \(a\). Значит
  выполнено второе условие из определения \ref{def:reg-fam}. Таком образом
  семейство нормальных распределений \(N(a; \sigma^2)\) регулярно относительно
  параметра \(a\). Аналогично можно показать, что оно будет регулярно
  относительно параметра \(\sigma\). Вычислим дисперсию оценки.

  \begin{equation*}
    \variance{a^*}
    = \variance{\avg{x}}
    = \frac{\variance{X}}{n}
    = \frac{\sigma^2}{n}
  \end{equation*}

  В правой части неравенства \ref{thr:rao-cra-ineq} имеем

  \begin{equation*}
    \frac{1}{n I(\Theta)}
    = \frac{\sigma^2}{n}
  \end{equation*}

  Видим, что дисперсия оценки получилась минимально возможной. Значит
  \(\avg{x}\) это эффективная оценка параметра \(a\). Аналогично можно показать,
  что \(S^2\)~--- несмещенная эффективная оценка \(\sigma^2\).
\end{example}

\begin{remark}
  В практической статистике часто используют линейные оценки \(\Theta^* = \sum
  c_i x_i\). Лучшая оценка в таком классе называется наилучшей линейной
  несмещенной оценкой или BLUE-оценкой. В примере с нормальным распределение
  оценка \(a^* = \avg{x}\) будет BLUE-оценкой, а в примере с равномерным
  распределением оценка \(\Theta^* = 2 \avg{x}\) будет BLUE-оценкой.
\end{remark}
