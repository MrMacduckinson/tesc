\subsection{%
  Лекция \texttt{24.??.??}.%
}

\subheader{Метод Монте---Карло}

Цель метода состоит в том, чтобы находить неизвестные значения изучаемой
величины при помощи \quote{разыгрывания} некоторой случайной величины.

Общая постановка метода. Пусть требуется найти неизвестное число \(a\) и имеется
случайная величина \(\xi\) такая, что \(\expected{\xi} = a\). Тогда согласно
закону больших чисел

\begin{equation*}
  \frac{\xi_1 + \dotsc + \xi_n}{n} \Rarr{\text{п.н.}} a
\end{equation*}

Следовательно, при достаточно больших \(n\) среднее выборочное \(\avg{x} \approx
a\).

Оценка погрешности. Пусть \(\variance{\xi} < \infty\), тогда согласно
центральной предельной теореме

\begin{equation*}
  \frac{S_n - n a}{\sqrt{n \variance{\xi_1}}}
  = \frac{n \prh{\avg{x} - a}}{\sqrt{n \variance{\xi_1}}}
  \rightrightarrows
  z \in \ndist{0}{1}
\end{equation*}

По правилу трех сигм \(\prob{\abs{z} < 3} \approx 0.9973\). Следовательно, при
больших \(n\) можно считать, что

\begin{equation*}
  \abs{\frac{n \prh{\avg{x} - a}}{\sqrt{n \variance{\xi_1}}}} < 3
  \implies
  \abs{\prh{\avg{x} - a}} < 3 \frac{\sqrt{\variance{\xi_1}}}{\sqrt{n}}
\end{equation*}

Отсюда видим, что сходимость есть, но она достаточно медленная, порядка
\(\frac{1}{\sqrt{n}}\), поэтому на практике не удается получить очень точные 
оценки для \(a\).

\subheader{Вычисление определенных интегралов}

\begin{equation*}
  \int_a^b \phi(x) \dd x
  = \lim_{\Delta x_i \to 0} \sum_{i = 1}^n \phi (c_i) \Delta x_i
\end{equation*}

где \(\Delta x_i\) это длины интервалов разбиения, а \(c_i\)~--- точки внутри
интервалов. На этом определении основаны так называемые квадратурные формулы.

\subsubheader{I.}{Формула прямоугольников}

Разобьем отрезок \(\segment{a}{b}\) на \(n\) равных частей длины \(\Delta x_i =
\frac{b - a}{n}\). Обозначим \(x_i\)~--- середину \(i\)-ого интервала. Тогда

\begin{equation*}
  I
  = \int_a^b \phi(x) \dd x
  \approx \frac{b - a}{n} \sum_{i = 1}^n \phi (x_i)
  = I_n
\end{equation*}

Можно показать, что \(\abs{I - I_n} \le \frac{M_1}{n^2}\), где \(M_1 = const\).

\subsubheader{II.}{Формула трапеций}

Разобьем отрезок \(\segment{a}{b}\) на \(n\) равных частей \(\segment{x_i}{x_{i
+ 1}}\) длины \(\Delta x_i = \frac{b - a}{n}\). Обозначим \(y_i = \phi (x_i)\),
где \(0 \le i \le n\). Тогда

\begin{equation*}
  I
  = \int_a^b \phi(x) \dd x
  \approx \frac{b - a}{2 n} \prh{y_0 + y_n + 2 \prh{y_1 + \dotsc + y_{n - 1}}}
  = I_n
\end{equation*}

Можно показать, что \(\abs{I - I_n} \le \frac{M_2}{n^2}\), где \(M_2 = const\),
но \(M_2\) меньше, чем \(M_1\), примерно в два раза.

\subsubheader{III.}{Формула Симпсона (формула парабол)}

Разобьем отрезок \(\segment{a}{b}\) на \(n = 2 m\) равных отрезков
\(\segment{x_i}{x_{i + 1}}\) длины \(\Delta x_i = \frac{b - a}{n}\). Обозначим
\(y_i = \phi (x_i)\), где \(0 \le i \le n\). Тогда

\begin{equation*}
  I
  = \int_a^b \phi(x) \dd x
  \approx \frac{b - a}{3 n} \prh[\Big]{
    y_0 + y_n
    + 4 \prh{y_1 + y_3 + \dotsc + y_{n - 1}}
    + 2 \prh{y_2 + y_4 + \dotsc + y_{n - 2}}
  }
  = I_n
\end{equation*}

Можно показать, что \(\abs{I - I_n} \le \frac{M}{n^4}\), где \(M = const\).

\subsubheader{IV.}{Метод Монте---Карло}

В качестве узлов берутся псевдослучайные числа. Пусть \(I = \int_0^1 \phi(x) \dd
x\). Ясно, что \(\segment{0}{1} \to \segment{a}{b}\) при помощи линейной замены.
Обозначим \(\eta_i \in \udist{0}{1}\)~--- значение датчика случайных чисел,
\(f_{\eta_i} (x) \equiv 1, x \in \segment{0}{1}\). Пусть \(\xi_i =
\phi(\eta_i)\), тогда

\begin{equation*}
  \expected{\xi_i}
  = \int_{-\infty}^{\infty} \phi(x) \cdot f_{\eta_i} (x) \dd x
  = \int_0^1 \phi(x) \dd x
  = I
\end{equation*}

По методу Монте---Карло получаем, что

\begin{equation*}
  I \approx \widehat{I}_n = \frac{1}{n} \sum_{i = 1}^n \phi(\eta_i)
\end{equation*}

где \(\eta_i\) это значения датчика случайных чисел. Погрешность вычислений
не будет превосходить

\begin{equation*}
  \abs{I - I_n} \le \frac{3 \sqrt{\variance{\xi_1}}}{\sqrt{n}}
  \qquad
  \variance{\xi_1} = \int_0^1 \phi^2 (x) \dd x - I^2
\end{equation*}

\begin{remark}
  Скорость сходимости намного хуже, чем в квадратурных формулах. При этом для
  оценки погрешности потребуется вычислить (или оценить сверху) дополнительный
  интеграл, поэтому метод Монте---Карло не используется для вычисления
  определенных интегралов.
\end{remark}

\subheader{Кратные интегралы}

\begin{remark}
  При вычислении \(k\)-кратных интегралов число узлов сетки возрастает как
  \(n^k\) и аналог метода прямоугольников будет довольно трудоемким. Таким
  образом метод Монте---Карло становится уместным, т.к. он не зависит от
  размерности.
\end{remark}

Для вычисления интеграла

\begin{equation*}
  I = \int_0^1 \dotsi \int_0^1 \phi \prh{x_1, \dotsc x_k} \dd x_1 \dotsc \dd x_k
\end{equation*}

достаточно случайно бросить \(n\) точек в данный \(k\)-мерный единичный куб. Для
этого последовательные \(k\) значений датчика берем в качестве координат
соответствующей случайной точки. Вычисляем значение подынтегральной функции в
этих случайных точках и получаем

\begin{equation*}
  I \approx I_n = \frac{1}{n} \sum_{i = 1}^n \phi \prh{x_{i_1}, \dotsc, x_{i_k}}
\end{equation*}

При этом скорость сходимости остается равной \(\frac{1}{\sqrt{n}}\). В
частности, если функция \(\phi = I_D\) это индикатор области \(D\), то с помощью
метода Монте---Карло можно приблизительно найти объем области \(D\).

\begin{example}
  Пусть необходимо оценить площадь четверти единичной окружности. Генерируем
  последовательность псевдослучайных чисел и разбиваем их на пары
  \(\pair{x_{i_1}, x_{i_2}}\), которые будут координатами точек.

  \begin{equation*}
    I_D = \begin{cases}
      0, & x_{i_1}^2 + x_{i_2}^2 > 1 \\
      1, & x_{i_1}^2 + x_{i_2}^2 \le 1
    \end{cases}
  \end{equation*}

  Итого \(S = \frac{n_D}{n}\), где \(n_D\) это число точек, попавших в область
  \(D\).
\end{example}

\subheader{Метод расслоенной выборки}

Пусть имеется \(k\)-мерный интеграл

\begin{equation*}
  I = \int_0^1 \dotsi \int_0^1 \phi \prh{x_1, \dotsc x_k} \dd x_1 \dotsc \dd x_k
\end{equation*}

Каждую из сторон \(k\)-мерного куба разобьем \(N\) равных частей, тогда куб
разобьется на \(n = N^k\) маленьких кубиков \(\Delta_i\) со стороной
\(\frac{1}{N}\). В каждом из этих кубиков возьмем случайную точку, построенную с
помощью датчика случайных чисел, \(\eta_i = \prh{\eta_i^1, \dotsc, \eta_i^k} \in
\Delta_i\), где \(1 \le i \le n\). Интеграл оценивается при помощи суммы

\begin{equation*}
  I_n = \frac{1}{n} \sum_{i = 1}^n \phi \prh{\eta_i}
\end{equation*}

При этом методе погрешность будет лучше, а именно

\begin{equation*}
  \abs{I - I_n} \le C \cdot \frac{1}{n^{\frac{1}{2} + \frac{1}{k}}}
  \qquad
  C = const
\end{equation*}

\begin{remark}
  В самых лучших квадратурных формулах для \(k\)-кратных интегралов погрешность
  составляет

  \begin{equation*}
    \abs{I - \widehat{I}} \le C \cdot \frac{1}{n^{1 + \epsilon}}
    \qquad
    C = const
  \end{equation*}
\end{remark}

\subheader{Равномерность по Вейлю}

\begin{definition}
  Числовая последовательность \(x_1, \dotsc, x_n\), где \(x_i \in
  \segment{0}{1}\) называется равномерной по Вейлю, если частота попадания точек
  \(x_i\) на любой отрезок \(\segment{a}{b}\) стремится к его длине \(b - a\)
  при \(n \to \infty\).
\end{definition}

\begin{remark}
  Ясно, что значения равномерного стандартного распределения или датчика
  случайных чисел будут равномерными по Вейлю.
\end{remark}

\begin{theorem} \label{thr:weyl}
  Пусть \(\alpha\) это иррациональное число, тогда последовательность
  \(x_n = \{ n \alpha \}\) является равномерной по Вейлю.
\end{theorem}

\begin{remark}
  Если последовательность \(x_n\) равномерная по Вейлю, то

  \begin{equation*}
    \frac{1}{n} \sum_{i = 1}^n \phi (x_i)
    \to
    \int_0^1 \phi(x) \dd x
  \end{equation*}

  но возможно что

  \begin{equation*}
    \frac{1}{n} \sum_{i = 1}^n \psi (x_{i_1}, x_{i_2})
    \to C
    \neq
    \int_0^1 \int_0^1 \psi(x, y) \dd x \dd y
  \end{equation*}
\end{remark}

\begin{definition}
  Числовая последовательность \(x_1, \dotsc, x_n, \dotsc\) называется вполне
  равномерной по Вейлю, если для произвольного числа \(k\) частота попадания
  \(k\)-мерных точек \(\tuple{x_i^{(1)}, \dotsc, x_i^{(k)}}\) в любой
  параллелепипед внутри единичного куба стремится к его объему.
\end{definition}

\begin{remark}
  Хороший датчик случайных чисел должен выдавать вполне равномерную
  последовательность. Мультипликативный датчик не является вполне равномерным.
\end{remark}

\begin{example}[Парадокс первой цифры]
  Найти вероятность того, что число \(2^n\) начинается с цифры \(7\).

  \solution{} Пусть \(2^n\) начинается с цифры \(m\). Это означает, что

  \begin{equation*}
    \begin{aligned}
      m \cdot 10^l \le 2^n < (m + 1) \cdot 10^l
    \\
      \log_{10} m + l \le n \log_{10} 2 < \log_{10} (m + 1) + l
        \in [l; l + 1)
    \\
      \log_{10} m \le \{ n \log_{10} 2 \} < \log_{10} (m + 1)
    \end{aligned}
  \end{equation*}

  Согласно теореме \ref{thr:weyl} последовательность \(\seq{n \log_{10} 2}\)
  является равномерной по Вейлю, следовательно вероятность того, что число
  начнется с цифры \(m\) стремится к длине интервала, т.е. к \(\log_{10} (m + 1)
  - \log_{10} m = \log_{10} \prh{1 + \frac{1}{m}}\). Итого при \(m = 7\)
  получаем \(\log_{10} \prh{1 + \frac{1}{7}} \approx 0.058\).
\end{example}
