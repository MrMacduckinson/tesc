\question{Поверхностный интеграл 1-го рода: определение, свойства, вычисление, геометрический и физический смысл.}

\begin{twocolumns}
  %% based on https://tex.stackexchange.com/questions/488865/3d-volume-in-tikz/488869#488869

\tdplotsetmaincoords{60}{110}
\begin{tikzpicture}[
  tdplot_main_coords,
  >= stealth,
  declare function = {
    pfft(\x) = pi + 0.3 * sin(deg(\x));
  }
]
  \draw[->] (0, 0, 0) coordinate (O) -- (3, 0, 0) coordinate(X)
    node[pos = 1.1] {\(x\)};
  \draw[->] (O) -- (0, 5, 0)
    node[pos = 1.1] {\(y\)};
  \draw[->] (O) -- (0, 0, 4)
    node[pos = 1.1] {\(z\)};

  \draw[thick] plot[
    variable = \x,
    domain = 0.8 * pi : 1.2 * pi,
    smooth
  ] (2.2, \x, { pfft(2 * \x) }) coordinate (T1)
  -- plot[
    variable = \x,
    domain = 1.2 * pi : 0.8 * pi,
    smooth
  ] (0.8,\x,{ pfft(2 * \x) }) coordinate (T3)
  -- cycle;

  \draw[dashed] (2.2, 0.8 * pi, 0) coordinate (B4)
  -- (2.2, 1.2 * pi, 0) coordinate (B1)
  -- (0.8, 1.2 * pi, 0) coordinate (B2)
  -- (0.8, 0.8 * pi, 0) coordinate (B3)
  -- cycle;

  \path (2.2, 0.8 * pi, { pfft(2 * 0.8 * pi) }) coordinate (T4)
    (0.8, 1.2 * pi,{ pfft(2 * 1.2 * pi) }) coordinate (T2);

  \foreach \X in {1, ..., 4} {
    \draw[dashed] (B\X) -- (T\X);
  }

  \path[
    opacity = 0.3,
    left color = blue,
    right color = blue,
    middle color = blue!20,
    shading angle = 72
  ] plot[
    variable = \x,
    domain = 0 : 1.1 * pi,
    smooth
  ] (3, \x, { pfft(2 * \x) })
  -- plot[
    variable = \x,
    domain = 1.1 * pi : 0,
    smooth
  ] (0, \x, { pfft(2 * \x) })
  -- cycle;

  \path[
    opacity = 0.3,
    left color = blue,
    right color = blue,
    middle color = blue!20,
    shading angle = 72
  ] plot[
    variable = \x,
    domain = 1.1 * pi : 2.2 * pi,
    smooth
  ] (3, \x, { pfft(2 * \x) })
  -- plot[
    variable = \x,
    domain = 2.2 * pi : 1.1 * pi,
    smooth
  ] (0, \x, { pfft(2 * \x) })
  -- cycle;

  \draw plot[
    variable = \x,
    domain = 0 : 2.2 * pi,
    smooth
  ] (3, \x, { pfft(2 * \x) })
  -- plot[
    variable = \x,
    domain = 2.2 * pi : 0,
    smooth
  ] (0, \x, { pfft(2 * \x) })
  -- cycle;

  \draw node at (1.1, 3, 2.9) {\(\dd \sigma\)};
  \draw node at (0, 3, 4.3) {\(S \colon z = z(x, y)\)};
  \draw node at (1.55, 3.05, 0) {\(D_{xy}\)};
\end{tikzpicture}

  \columnbreak

  Пусть \(f(x, y, z)\) это плотность распределения некоторой скалярной величины.
  Введена ДПСК, поверхность простая \(z = z(x, y)\).

  Элемент поверхности \(\dd \sigma\) вырезается координатными плоскостями
  \(x = const, y = const\). Выделим элементарную массу \(\dd  m\). Умножая
  среднюю плотность на размер элементарного участка получаем
  \(\dd m = f(x, y, z) \dd \sigma\). Полную массу получим 'суммированием':

  \begin{align*}
    m = \iint_{S} \dd m = \iint_{S} f(x, y, z) \dd \sigma
  \end{align*}

  Получили поверхностный интеграл 1-ого рода (по участку поверхности).
\end{twocolumns}

\begin{remark}
  Физический смысл поверхностного интеграла 1-ого рода вытекает из его
  построения: он равен массе участка неоднородной поверхности.
\end{remark}

\begin{remark}
  О математическом определении

  Поверхностный интеграл 1-ого рода можно определить математически аналогично
  уже рассмотренным интегралам:

  \begin{enumerate}
    \item Дробим \(S\) плоскостями\(x = const, y = const\) на элементарные
    участки \(\Delta \sigma_{i}\).

    \item В каждом участке выбираем среднюю точку
    \(M_{i}(\xi_{i}, \eta_{i}, \zeta_{i})\) и вычисляем \(f(M_{i})\).

    \item Составляем предел интегральных сумм и переходим к интегралу:

    \begin{align*}
      \lim_{\substack{n \to \infty \\ \tau \to 0}}
        \sum_{i = 1}^{n} f(\xi_{i}, \eta_{i}, \zeta_{i}) \Delta \sigma_{i}
    \end{align*}
  \end{enumerate}
\end{remark}

\begin{remark}
  О вычислении

  Введем параметризацию \(z = z(x, y)\) и спроецируем поверхность на плоскость
  \(Oxy\), т.е. \(D_{xy} = S_{\text{пр. }xy}\). Получим

  \begin{align*}
    \iint_{S} f(x, y, z) \dd \sigma
    = \iint_{D_{xy}} f(x, y, z(x, y))
      \sqrt{1 + (z'_{x})^2 + (z'_{y})^2} \dd x \dd y
  \end{align*}

  При необходимости можно вводить другую параметризацию и проектировать
  поверхность на другую координатную плоскость.
\end{remark}

\begin{remark}
  С помощью поверхностного интеграла 1-ого рода можно найти площадь поверхности
  следующим образом:

  \begin{align*}
    S_{\text{пов.}} = \iint_{S} \dd \sigma
  \end{align*}
\end{remark}