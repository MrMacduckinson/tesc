\question{Теорема Стокса.}

\begin{multicols}{2}
  \todo Я устал рисовать картинки в TikZ'е
  \columnbreak

  Пусть поверхность \(S\) опирается на замкнутый контур \(L\).
  \(D_{xy} = S_{\text{пр. } Oxy}\), \(K = L_{\text{пр. } Oxy}\)
  
  В области, содержащей \(S\), определена тройка скалярных функций
  \(P(x, y, z)\), \(Q(x, y, z)\), \(R(x, y, z)\), каждая из которых
  дифференцируема и имеет непрерывные частные производные.
\end{multicols}
  
\begin{theorem}\label{ST}
  Теорема Стокса.

  При выполнении условий, описанных выше, справедливо равенство
  
  \begin{align*}
    \iint_{S}
      \left(
        \frac{\partial P}{\partial z} \cos \beta
        - \frac{\partial P}{\partial y} \cos \gamma
      \right)
      + \left(
        \frac{\partial Q}{\partial x} \cos \gamma
        - \frac{\partial Q}{\partial z} \cos \alpha
      \right)
      + \left(
        \frac{\partial R}{\partial y} \cos \alpha
        - \frac{\partial R}{\partial x} \cos \beta
      \right) \dd \sigma
    = \oint_{L^{+}} P \dd x + Q \dd y + R \dd z
  \end{align*}
\end{theorem}  
\begin{proof}
  Рассмотрим слагаемое \(\displaystyle \oint_{L^{+}} P \dd x\) в интеграле в
  правой части. Применяя формулу вычисления в обратную сторону, получаем, что

  \begin{align*}\label{eq:ST-proof-1}\tag{1}
    \oint_{L^{+}} P \dd x
    = \oint_{K^{+}} P(x, y, z(x, y)) \dd x + 0 \; \dd y
    \eqby{\ref{Green}}
    \iint_{D_{xy}} \left(0 - \frac{\partial P}{\partial y} \right) \dd x \dd y
    = -\iint_{D_{xy}} \frac{\partial P}{\partial y} \dd x \dd y
  \end{align*}

  Возьмем производную в подынтегральном выражении, учитывая то, что \(x\) и
  \(y\) это независимые переменные, а \(P\)~--- сложная функция:

  \begin{align*}
    \frac{\partial P}{\partial y}
    = \frac{\partial P}{\partial x} \cdot \frac{\partial x}{\partial y}
      + \frac{\partial P}{\partial y} \cdot \frac{\partial y}{\partial y}
      + \frac{\partial P}{\partial z} \cdot \frac{\partial z}{\partial y}
    = \frac{\partial P}{\partial y}
      + \frac{\partial P}{\partial z} \cdot \frac{\partial z}{\partial y}
  \end{align*}

  Первое слагаемое будет равно нулю, т.к \(x\) и \(y\) это независимые
  переменные. Подставим это в \eqref{eq:ST-proof-1}, а также заменим
  \(\dd x \dd y\) на \(\cos \gamma \dd \sigma\):

  \begin{align*}\label{eq:ST-proof-2}\tag{2}
    -\iint_{D_{xy}} \left(
      \frac{\partial P}{\partial y}
      + \frac{\partial P}{\partial z} \cdot \frac{\partial z}{\partial y}
    \right) \cos \gamma \dd \sigma
    =
    -\iint_{D_{xy}} \left(
      \frac{\partial P}{\partial y} \cos \gamma
      + \frac{\partial P}{\partial z}
        \cdot \frac{\partial z}{\partial y} \cos \gamma
    \right) \dd \sigma
  \end{align*}

  Упростим второе слагаемое, используя полученные ранее формулы для косинусов
  \eqref{eq:surf-angles}:

  \begin{align*}\label{eq:ST-proof-3}\tag{3}
    \frac{\partial z}{\partial y} \cos \gamma
    = \frac{z'_{y}}{\sqrt{1 + (z'_{x})^2 + (z'_{y})^2}}
    = -\cos \beta
  \end{align*}

  Подставим \eqref{eq:ST-proof-3} в \eqref{eq:ST-proof-2} и получим:

  \begin{align*}
    -\iint_{D_{xy}} \left(
      \frac{\partial P}{\partial y} \cos \gamma
      - \frac{\partial P}{\partial z} \cos \beta
    \right) \dd \sigma
    =
    \iint_{D_{xy}} \left(
      \frac{\partial P}{\partial z} \cos \beta
      - \frac{\partial P}{\partial y} \cos \gamma
    \right) \dd \sigma
  \end{align*}

  Таким образом мы получили первое слагаемое в поверхностном интеграле в левой
  части формулы. Аналогично можно рассмотреть
  \(\displaystyle \oint_{L^{+}} Q \dd y\),
  \(\displaystyle \oint_{L^{+}} R \dd z\)
  получить и оставшиеся два слагаемых.
\end{proof}

\begin{remark}
  Формулу Стокса можно записать в другом виде, если собрать коэффициенты при
  косинусах:

  \begin{align*}
    \iint_{S}
      \left(
        \frac{\partial R}{\partial y} 
        - \frac{\partial Q}{\partial z}
      \right) \cos \alpha \dd \sigma
      + \left(
        \frac{\partial P}{\partial z} 
        - \frac{\partial R}{\partial x}
      \right) \cos \beta \dd \sigma
      + \left(
        \frac{\partial Q}{\partial x} 
        - \frac{\partial P}{\partial y}
      \right) \cos \gamma \dd \sigma
    = \oint_{L^{+}} P \dd x + Q \dd y + R \dd z
  \end{align*}

  Также можно представить интеграл в левой части в виде поверхностного интеграла
  2-ого рода:

  \begin{align*}
    \iint_{S}
      \left(
        \frac{\partial R}{\partial y} 
        - \frac{\partial Q}{\partial z}
      \right) \dd y \dd z
      + \left(
        \frac{\partial P}{\partial z} 
        - \frac{\partial R}{\partial x}
      \right) \dd x \dd z
      + \left(
        \frac{\partial Q}{\partial x} 
        - \frac{\partial P}{\partial y}
      \right) \dd y \dd z
    = \oint_{L^{+}} P \dd x + Q \dd y + R \dd z
  \end{align*}
\end{remark}