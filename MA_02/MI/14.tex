\question{Поверхностный интеграл 2-го рода как поток жидкости через поверхность.}

\begin{remark}
  О поверхности, обходе участка и направлении нормали. Постановка задачи

  \begin{multicols}{2}
    %% based on https://tex.stackexchange.com/questions/488865/3d-volume-in-tikz/488869#488869

\tdplotsetmaincoords{60}{110}
\begin{tikzpicture}[
  tdplot_main_coords,
  >= stealth,
  declare function = {
    pfft(\x) = pi + 0.3 * sin(deg(\x));
  }
]
  \draw[->] (0, 0, 0) coordinate (O) -- (4, 0, 0) coordinate(X)
    node[pos = 1.1] {\(x\)};
  \draw[->] (O) -- (0, 3, 0)
    node[pos = 1.1] {\(y\)};
  \draw[->] (O) -- (0, 0, 4)
    node[pos = 1.1] {\(z\)};

  \begin{scope}[
    decoration = {
      markings,
      mark = between positions 0.1 and 1 step 30pt with
        {\arrow[line width = 2pt]{>}}
    }
  ]
    \draw[thick, postaction = { decorate }] plot[
      variable = \x,
      domain = 0.3 * pi : 0.9 * pi
    ] (3.0, \x, { pfft(2 * \x) }) coordinate (T1)
    -- plot[
      variable = \x,
      domain = 0.9 * pi : 0.3 * pi
    ] (0.8, \x, { pfft(2 * \x) }) coordinate (T3)
    -- cycle;
  \end{scope}

  \begin{scope}[
    decoration = {
      markings,
      mark = at position 0.7 with {\arrow[line width = 2pt]{>}}
    },
    every path/.style = {
      dashed,
    }
  ] 
    \draw[postaction = { decorate }]
      (3, 0.3 * pi, 0) coordinate (B4)
      -- (3, 0.9 * pi, 0) coordinate (B1);
    \draw[postaction = { decorate }]
      (0.8, 0.9 * pi, 0) coordinate (B2)
    -- (0.8, 0.3 * pi, 0) coordinate (B3);
    \draw[postaction = { decorate }] (B1) -- (B2);
    \draw[postaction = { decorate }] (B3) -- (B4);
  \end{scope}

  % \path (3, 0.3 * pi, { pfft(2 * 0.3 * pi) }) coordinate (T4)
  %   (0.8, 0.9 * pi,{ pfft(2 * 0.9 * pi) }) coordinate (T2);

  % \foreach \X in {1, ..., 4} {
  %   \draw[dashed] (B\X) -- (T\X);
  % }

  \path[
    opacity = 0.3,
    left color = blue,
    right color = blue,
    middle color = blue!20,
    shading angle = 72
  ] plot[
    variable = \x,
    domain = 0 : 1.1 * pi,
    smooth
  ] (3.5, \x, { pfft(2 * \x) })
  -- plot[
    variable = \x,
    domain = 1.1 * pi : 0,
    smooth
  ] (0, \x, { pfft(2 * \x) })
  -- cycle;

  \draw plot[
    variable = \x,
    domain = 0 : 1.1 * pi,
    smooth
  ] (3.5, \x, { pfft(2 * \x) })
  -- plot[
    variable = \x,
    domain = 1.1 * pi : 0,
    smooth
  ] (0, \x, { pfft(2 * \x) })
  -- cycle;

  \draw node at (1.9, 1.9, 0) {\(D_{xy}\)};
  \draw[fill = black] (1.8, 1.8, 3) circle (1pt);

  \draw[red, thick, ->] (1.8, 1.8, 3) -- (2.9, 2.9, 5.5)
    node[color = black, above right] {\(\vec{n^{+}}\)};
  \draw[dashed, blue, thick, ->] (1.8, 1.8, 3) -- (0, 0.4, 0.1)
    node[color = black, pos = 1.3] {\(\vec{n^{-}}\)};

  \draw[->] (1.8, 1.8, 3) -- (2.1, 2.1, 5.5)
    node[color = black, above] {\(\vec{u}\)};
\end{tikzpicture}

    \columnbreak

    Будем рассматривать только двусторонние поверхности. Положительной
    нормалью \(\vec{n^{+}}\) будет называть ту нормаль, у которой угол с осью
    \(Oz\) меньше \(90\) градусов. Сторону \(S\) с положительной нормалью
    будет называть \textit{верхней}. Дуально определим отрицательную нормаль и
    \textit{нижнюю} сторону поверхности. Согласуем обходы \(S\) и
    \(D_{xy} = S_{\text{пр. }xy}\)

    Пусть через данный участок поверхности течет жидкость со скоростью
    \(\vec{v}\) и плотностью \(\rho\). Вычислим количество жидкости, проходящей
    через \(t\) за единицу времени \(t\). Будем считать поток положительным,
    если он течет в направлении положительной нормали и отрицательным если в 
    направлении отрицательной.

    \vfill\null
  \end{multicols}
\end{remark}

Рассмотрим более простую задачу: пусть поверхность \(S\) плоская, а
\(v = const\). Тогда жидкость, которая протечет через участок поверхности, может
рассматриваться как наклонный цилиндр. Найдем его объем по известной формуле
\(V = h \cdot S_{\text{осн.}}\), причем высота будет равна проекции скорости на
нормаль, умноженной на время. Таким образом поток \(\Pi\) будет равен
\(\Pi = V = (\vec{v}, \vec{n_{0}}) \Delta t S\).

Теперь вернемся к исходной задаче: поверхность \(S\) криволинейная и через неё
действует некоторая векторная величина \(\vec{F} = (P, Q, R)\). Тогда полученную
ранее формулу можно использовать для вычисления элементарного потока
\(\dd \Pi = (\vec{F} \cdot \vec{n_{0}}) \dd \sigma\). Переход к вычислению всего
потока осуществляется с помощью двойного интеграла:

\begin{align*}
  \Pi
  = \iint_{S} \dd \Pi
  = \iint_{S} \left( \vec{F} \cdot \vec{n_{0}} \right) \dd \sigma 
\end{align*}