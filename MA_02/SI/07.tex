\question{Интегрирование некоторых иррациональных функций, метод тригонометрической подстановки.}

\begin{itemize}
\item Интегралы вида \(\int R(\sqrt{x^2 \pm 1}, x) \dd x\) решаются с помощью
замены \(x\) на гиперболическую функцию:

\begin{align*}
  \sinh u = \frac{e^u - e^{-u}}{2} \qquad
  \cosh u = \frac{e^u + e^{-u}}{2}
\end{align*}
\begin{remark}
  Данные функции называются гиперболическим синусом и гиперболическим косинусом
  соответственно.
\end{remark}

\begin{lemma}
  Основное гиперболическое тождество

  \begin{align*}
    \cosh^2 - \sinh^2 = 1
  \end{align*}
\end{lemma}
\begin{proof}
  \begin{align*}
    \cosh^2 - \sinh^2 =
    \left(\frac{e^{u} + e^{-u}}{2}\right)^2
      - \left(\frac{e^{u} - e^{-u}}{2}\right)^2 =
    \frac{1}{4} \left(e^{2u} + 2 + e^{-2u} - e^{2u} + 2 - e^{-2u} \right) = 1
  \end{align*}
\end{proof}

\begin{remark}\label{hpr-rep}
  Заметим, что

  \begin{align*}
    \ln \abs{\sinh + \cosh}
    = \ln \abs{\frac{e^u - e^{-u}}{2} + \frac{e^u - e^{-u}}{2}}
    = \ln e^{u}
    = u
  \end{align*}
\end{remark}

\begin{example}
  Вычислим 'длинный' логарифм:

  \begin{align*}
    \int \frac{\dd x}{\sqrt{1 + x^2}} = 
    \begin{bmatrix}
      x = \sinh u \implies 1 + x^2 = \cosh^2 u \\
      \dd x = \dd (\sinh u) = \cosh u \dd u \\
      u = \ln \abs{
        \under{x}{\sinh u} + \under{\sqrt{1 + x^2}}{\cosh u}
      } \;\; (\ref{hpr-rep})
    \end{bmatrix} =
    \int \frac{\cosh u}{\cosh u} \dd u =
    u + C =
    \ln \abs{x + \sqrt{1 + x^2}} + C
  \end{align*}
\end{example}

\item Интегралы вида \(\int R(\sqrt{1 - x^2}, x) \dd x\) решаются с помощью
замены \(x\) на синус или косинус.

\item Интегралы вида \(\int R(\sqrt[k_{1}]{x}, \dots, \sqrt[k_{n}]{x}) \dd x\)
решаются с помощью замены \(t = \sqrt[K]{x}\), где \(K\) это НОД для
\(k_{1}, \dotsc, k_{n}\).

\item Интегралы вида \(\int R(\sqrt{ax + b}, x) \dd x\) решаются с помощью
замены \(t = \sqrt{ax + b}\). При этом \(x = \frac{t^2 - b}{a}\),
\(\dd x = \frac{2t}{a} \dd t\).
\end{itemize}
