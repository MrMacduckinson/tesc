\question{Несобственные интегралы 1-го рода (на неограниченном промежутке). Определение и свойства.}

\begin{definition}\label{imp-int-1}
  Интеграл от функции на неограниченном промежутке называет несобственным
  интегралом 1-ого рода.

  \begin{align*}
    \int_{a}^{+\infty} f(x) \dd x
    \transition{def}
    \lim_{\beta \to +\infty} \int_{a}^{\beta} f(x) \dd x
    \\
    \int_{-\infty}^{a} f(x) \dd x
    \transition{def}
    \lim_{\beta \to -\infty} \int_{\beta}^{a} f(x) \dd x
    \\
    \int_{-\infty}^{+\infty} f(x) \dd x
    \transition{def}
    \int_{-\infty}^{c} f(x) \dd x + \int_{c}^{+\infty} f(x) \dd x, \; c \in \RR
  \end{align*}
\end{definition}

\begin{definition}
  Если предел в определении \ref{imp-int-1} существует и конечен, то говорят,
  что интеграл \textit{сходится} (\(\converge\)), в противном случае говорят, что
  интеграл \textit{расходится} (\(\notconverge\)).
\end{definition}

Несобственные интегралы 1-ого рода обладают теми же свойствами, что и
рассмотренные ранее интегралы:
\begin{enumerate}
  \item Линейность
  
  \begin{align*}
    \int_{a}^{b} (\lambda f(x) + \mu g(x)) \dd x =
    \lambda \int_{a}^{b} f(x) \dd x + \mu \int_{a}^{b} g(x) \dd x
  \end{align*}

  \item Аддитивность
  
  \begin{align*}
    \int_{a}^{b} f(x) \dd x =
    \int_{a}^{c} f(x) \dd x + \int_{c}^{b} f(x) \dd x
  \end{align*}

  \item Сравнение
  
  \begin{align*}
    \forall x \in [a; +\infty] \colon f(x) \ge g(x)
    \implies \int_{a}^{+\infty} f(x) \dd x \ge \int_{a}^{+\infty} g(x) \dd x
  \end{align*}
\end{enumerate}

\begin{remark}
  Если 
  \begin{align*}
    \int_{-\infty}^{+\infty} f(x) \dd x
    =
    \int_{-\infty}^{c} f(x) \dd x + \int_{c}^{+\infty} f(x) \dd x
  \end{align*}
  и при этом один из двух полученных интегралов расходится, то расходится и
  изначальный интеграл.
\end{remark}