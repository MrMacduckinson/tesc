\question{Вычисление определенного интеграла. Формула Ньютона-Лейбница.}

\begin{theorem}\label{NL}
  Формула Ньютона-Лейбница

  Пусть \(f(x) \in C_{[a; b]}\), определен \(\int_{a}^{b} f(x) \dd x\) и
  \(F(x)\) это некоторая первообразная для \(f(x)\). Тогда

  \begin{align*}
    \int_{a}^{b} f(x) \dd x = F(b) - F(a) = F(x) \Big\vert_{a}^{b}
  \end{align*}
\end{theorem}
\begin{proof}
  Рассмотрим функцию \(\Phi(x) = \int_{a}^{x} f(t) \dd t\),
  где \(x \in [a ; b]\).
  Тогда по т. Барроу (\ref{Barrow}) \(\Phi(x) = F(x) + C\).

  Найдем значение функции \(\Phi(x)\) в точке \(a\):

  \begin{align*}
    \begin{rcases}
      \Phi(a) = \int_{a}^{a} f(t) \dd t = 0 \\
      \Phi(a) = F(a) + C
    \end{rcases}
    \implies C = -F(a)
  \end{align*}

  Теперь найдем значение функции \(\Phi(x)\) в точке \(b\):

  \begin{align*}
    \begin{rcases}
      \Phi(b) = \int_{a}^{b} f(t) \dd t \\
      \Phi(b) = F(b) + C = F(b) - F(a)
    \end{rcases}
    \implies \int_{a}^{b} f(t) \dd t = F(b) - F(a)
  \end{align*}
\end{proof}

\begin{remark}
  Формула Ньютона-Лейбница работает в тех случаях, когда можно найти \(F(x)\)
  или хотя бы её значения на концах отрезка \([a; b]\).
\end{remark}

\begin{remark}
  Если функция \(f(x)\) кусочно заданная, то используем свойство аддитивности
  и разбиваем отрезок на части.
\end{remark}